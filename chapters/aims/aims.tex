% !TeX root = ../../thesis.tex
\chapter{Objectives}
%
\label{ch:objectives}
%

\epigraphhead[\epipos]{%
\epigraph{%
%
  ``You could find out most things, if you knew the right questions to ask. Even if you didn't, you could
  still find out a lot.''
%
}{%
  \textit{`Gurgeh' in `The Player of Games' by Iain M. Banks}
%
}}


Given their introduction in \cref{ch:nanopores}, it should stand without doubt that nanopores---and
\emph{biological} ones in particular---have become powerful single-molecular sensing tools. Among the key
strengths of nanopores are their ability to interrogate individual molecules label-free, at high sampling
frequencies and with long observation times. Whereas over the past 25 years nanopore research was driven by
genomics, researchers are now recognizing these advantages may be useful also in other fields, including
proteomics, metabolomics, single-molecule enzymology~\cite{Willems-VanMeervelt-2017} and biosensing. This
expanding application space also presents an increasing challenge: translating the `two-dimensional' nanopore
current into a meaningful set of information. Even though complementary experimental techniques may alleviate
the problem, it is here that modeling approaches---be it analytical or computational---can bestow researchers
with a thorough understanding of the complete system itself, rather than its individual components.
Specifically, by elucidating the physical mechanisms that govern the interactions between the nanopore and the
analyte molecules, they may aid in the unambiguous interpretation of the current signals, provide insights
into the prevailing conditions within the pore, and detail guidelines for tailoring the properties of
nanopores. In this dissertation, the focus lies on the development of analytical and computational
methodologies (\textbf{Objective~1}) that explain the physical mechanisms behind the experimentally observed
behavior of nanopore-analyte systems (\textbf{Objectives~2 and 3}), or that can be used as a `computational
microscope' to study \emph{all} the properties of the nanopore itself (\textbf{Objective~4}).

\clearpage
%
\objective{Develop methodologies for accurate modeling of biological nanopores}
%

Due to their proteinaceous nature, the primary computational tool for studying biological nanopores are
\glsfirst{md} simulations, where every atom is modeled explicitly. Unfortunately, the wealth of information
accessible through \gls{md} simulations comes at a heavy computational cost: often necessitating the use of
months of supercomputer time. Even though we will still make use of \gls{md} to construct realistic and well
equilibrated homology models of various biological nanopores, most simulations in this work will be based on
\emph{continuum}, rather than discrete representations of the nanopore systems. In~\cref{ch:electrostatics} we
will show how the \glsfirst{pb} equation can be employed to quantify the equilibrium (\ie~without an external
bias voltage) electrostatic interactions between a nanopore and the surrounding electrolyte on the one hand,
and its analyte molecules on the other. Nanopores are governed by more than just electrostatics however, and
thus in~\cref{ch:epnpns} we develop the \glsfirst{epnp-ns} equations: a simulation framework that introduces
several self-consistent corrections to the electrolyte properties in an attempt to mitigate the shortcomings
of the classical {PNP-NS} equations at the nanoscale and beyond infinite dilution. Because the \gls{epnp-ns}
equations can be solved for continuum systems, they aim to provide a fast yet accurate means to model the
transport of ions and the flow of water through a nanopore. In~\cref{ch:trapping} we will also use an analytic
model---which intrinsically necessitates a reduction to the most essential components---to gain valuable
insights.

%
\objective{Investigate the equilibrium electrostatics of biological nanopores}
%

The interior walls of most biological nanopores are typically lined with charged amino acids. Hence, it is
unsurprising that electrostatic interactions often play a determining role in the overall behavior of a pore.
In~\cref{ch:electrostatics}, we use the \gls{pb} equations to calculate the electrostatic potential within the
several variants of the \glsfirst{plyab}~\cite{Huang-2020}, \glsfirst{frac}~\cite{Wloka-2016,Huang-2017} and
\glsfirst{clya}~\cite{Franceschini-2016} nanopores, and investigate the effect of ionic strength and pH on
their emergent properties, such as the \glsfirst{eof}. Additionally, we map out the electrostatic free
energies associated with \gls{ssdna} and \gls{dsdna} translocation through variants of the \gls{frac} and
\gls{clya} pores, respectively, and link the observed differences with published experimental findings. A
similar methodology will be used in \cref{ch:trapping} for quantifying the electrostatics energy associated
with the trapping of a protein within \gls{clya}~\cite{Soskine-Biesemans-2015}, which is the focus of
\textbf{Objective~3}.

\clearpage

%
\objective{Elucidate the trapping behavior of a protein inside a biological nanopore}
%

In their pioneering work with \gls{clya}, Soskine and Biesemans~\etal{} showed that the dwell time of the
\glsfirst{dhfr} enzyme (\SI{\approx19}{\kDa}) within \gls{clya} could be increased by orders of magnitude by
(1) fusing a positively charged polypeptide tag to the C-terminus of the protein (`\DHFRt'), and (2) allowing
it to bind to \gls{mtx}, a small negatively charged inhibitor
(\SI{454}{\dalton})~\cite{Soskine-Biesemans-2015}. Moreover, the dwell time of \DHFRt was found to depend
strongly and non-monotonously on the magnitude applied bias voltage~\cite{Biesemans-2015}. The physical
mechanisms behind this behavior are elucidated in~\cref{ch:trapping}, using a combination of experiments,
electrostatic energy calculations and an analytical `double energy barrier' model. The latter captures the
essential physics governing the escape of \DHFRt{} from either the \cisi{} or \transi{} sides of \gls{clya},
and its fitting to an extensive set of experimental data will yield quantitative values for magnitude of the
electrophoretic, electro-osmotic, electrostatic and steric forces acting on proteins captured by \gls{clya}.


%
\objective{Mapping the transport properties of a biological nanopore}
%

In their inspiring 2005 publication, Aksimentiev \etal{} made of use \SI{\approx100}{\ns} of all-atom \gls{md}
simulations to map out the electrostatic potential, electro-osmotic flow and ionic concentrations within the
\gls{ahl} nanopore~\cite{Aksimentiev-2005}. Even though the available computational power has increased
\num{\approx1000}-fold over the past 15~years, \gls{md} simulations remain prohibitively expensive compared to
continuum approaches, which are approximately 1000-fold faster at obtaining the same information. Moreover, to
a large extent, they have benefited from the same advances in computational power and algorithms. Hence,
in~\cref{ch:transport} we apply the \gls{epnp-ns} framework developed in~\cref{ch:epnpns} to a 2D-axisymmetric
model of \gls{clya} nanopore to show that continuum simulation can provide the same or more information as
\gls{md} simulations---at a fraction of the (computational) cost. By simulating the \gls{clya} over a wide
range of ionic strengths (\SIrange{0.005}{5}{\Molar}~\ce{NaCl}) and bias voltages
(\SIrange[retain-explicit-plus=true]{-200}{+200}{\mV}), we will be able to gauge the accuracy of the
\gls{epnp-ns} equations for predicting nanoscale conductances. Furthermore, it will allow us to paint a
quantitative picture of nanopore properties that are difficult to access experimentally, including ion
selectivity, ion concentrations, (non)equilibrium electrostatic potentials and the electro-osmotic flow.


%%%%%%%%%%%%%%%%%%%%%%%%%%%%%%%%%%%%%%%%%%%%%%%%%%
% Keep the following \cleardoublepage at the end of this file,
% otherwise \includeonly includes empty pages.
\cleardoublepage

% vim: tw=70 nocindent expandtab foldmethod=marker foldmarker={{{}{,}{}}}
