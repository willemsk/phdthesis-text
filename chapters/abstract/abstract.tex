% !TeX root = ../../thesis.tex
\chapter{Abstract}
%
\label{ch:abstract}
%

Throughout our history, we, humans, have sought to better control and understand our environment. To this end,
we have extended our natural senses with a host of sensors---tools that enable us to detect both the very
large, such as the merging of two black holes at a distance of \num{1.3}~billion light years from earth, and
the very small, such as the identification of individual viral particles from a complex mixture. This
dissertation is devoted to studying the physical mechanisms that govern a tiny, yet highly versatile sensor:
the biological nanopore. Biological nanopores are protein molecules that form nanometer-sized apertures in
lipid membranes. When an individual molecule passes through this aperture (\ie~`it `translocates''), the
temporary disturbance of the ionic current caused by its passage reveals valuable information on its identity
and properties. Despite this seemingly straightforward sensing principle, the complexity of the interactions
between the nanopore and the translocating molecule implies that it is often very challenging to unambiguously
link the changes in the ionic current with the precise physical phenomena that causes them. It is here that
the computational methods employed in this dissertation have the potential to shine, as they are capable of
modeling nearly all aspects of the sensing process with near atomistic precision.

The primary goals of this dissertation are fourfold:
%
\begin{enumerate}
  \item Develop methodologies for accurate modeling of biological nanopores;
  \item Investigate the equilibrium electrostatics of biological nanopores;
  \item Elucidate the trapping behavior of a protein inside a biological nanopore;
  \item Mapping the transport properties of a biological nanopore.
\end{enumerate}
%

In the first part of this thesis (\cref{ch:electrostatics}), we used 3D equilibrium simulations, based on
numerical solutions of the Poisson-Boltzmann equation, to investigate the electrostatic properties of the
pleurotolysin~AB (PlyAB), cytolysin~A (ClyA), and fragaceatoxin~C (FraC) nanopores. In particular, we showed
that a few, or even a single, charge reversal mutations can have a high electrostatic impact, resulting a
strongly reduced or even reversed electro-osmotic flow. Additionally, our simulations indicated that lowering
the electrolyte pH significantly reduced the influence of negatively charged amino acids, whilst leaving that
of the positively charged groups untouched. To elucidate the propensity of the FraC and ClyA pores to
translocate DNA, we computed the electrostatic energy costs associated with ssDNA and dsDNA translocation
through them. This revealed that the precise placement of positive charges can enable translocation of DNA by
either significantly reducing the magnitude of the electrostatic energy barrier, or by allowing the DNA strand
to penetrate deep enough within the pore to build up sufficient force to overcome it. Even though the
simulations performed in this chapter show that some of the key characteristics of biological nanopores can be
derived from their equilibrium electrostatics, it also revealed that a full comprehension requires the
addition of nonequilibrium forces.

The next chapter (\cref{ch:trapping}) revolves around the immobilization (\ie~`trapping') of a single protein
molecule within a nanopore, which is of importance for applications such as single-molecule enzymology. To
this end, we studied the average dwell time \DHFRt{}, a small protein modified with a positively charged
polypeptide at its C-terminus, within ClyA. Concretely, by manipulating its charge distribution, we succeeded
in increasing its average dwell time by several orders of magnitude. Further, we derived an analytical
transport model for the escape of \DHFRt{} from the pore, based on the crossing of the steric, electrostatic,
electrophoretic, and electro-osmotic energy barriers located at either side of pore. A systematic study of the
dwell times as a function of voltage and tag charge, together with extensive equilibrium electrostatic
simulations, allowed us to parameterize this double barrier model, revealing properties that are difficult to
determine experimentally, such as the translocation probabilities and the force exerted by ClyA's
electro-osmotic flow on the protein (\SI{\approx9}{\pN} at \SI{-50}{\mV}). The relative simplicity of the
double barrier model, and the fact that it contains no explicit geometric parameters of \DHFRt{}, suggested
that our approach may be generalizable to other small proteins.

In the final chapters, we developed a novel continuum framework for modeling biological nanopores under
\emph{nonequilibrium} conditions (\cref{ch:epnpns}) and subsequently applied it to the ClyA nanopore
(\cref{ch:transport}). Even though they are often qualitatively useful, the ability of continuum methods to
solve nanoscale transport problems in a quantitative manner is typically poor. To this end, we developed the
extended Poisson-Nernst-Planck-Navier-Stokes ({ePNP-NS}) equations, which self-consistently consider the
finite size of the ions, and the influence of both the ionic strength and the nanoscopic scale of the pore on
the local properties of the electrolyte. By numerically solving the {ePNP-NS} equations for a computationally
efficient model of ClyA, we were able to simulate the nanofluidic properties of the pore for a wide range of
experimentally relevant bias voltages and salt concentrations. We found the simulated ionic conductivities to
be in excellent agreement with their experimentally measured counterparts, suggesting that our model is
physically accurate. Hence, we used our simulations to provide detailed insights into the true ion
selectivity, the ion concentration distributions, the electrostatic potential landscape, the magnitude of the
electro-osmotic flow field, and the internal pressure distribution. As such, the {ePNP-NS} equations provide a
means to obtain fundamental new insights into the nanofluidic properties of biological nanopores and paves the
way towards their rational engineering.

In this dissertation, we showed that simulations, in combination with systematic experiments, can used as
computational `microscopes' to reveal the physical phenomena that underlie nanopore-based sensing. Whereas
simple equilibrium electrostatics are already highly instructive, it is clear that the complex interplay
between the nanopore and the translocating analyte molecule mandates a nonequilibrium approach that is both
rigorous and self-consistent, such as the {ePNP-NS} equations. Further improvements could elevate this
framework from an after-the-fact \emph{analysis} method to a powerful \emph{design} tool for nanopore
researchers, providing a means to automatically screen the properties of novel nanopores, or to predict the
ionic current signal produced by arbitrary molecules.


%%%%%%%%%%%%%%%%%%%%%%%%%%%%%%%%%%%%%%%%%%%%%%%%%%
% Keep the following \cleardoublepage at the end of this file,
% otherwise \includeonly includes empty pages.
\cleardoublepage

% vim: tw=70 nocindent expandtab foldmethod=marker foldmarker={{{}{,}{}}}
