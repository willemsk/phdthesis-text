% !TeX root = ../../thesis.tex
\chapter{An improved PNP-NS framework}
%
\label{ch:epnpns}
%

\epigraphhead[\epipos]{%
\epigraph{%
%
  ``The sciences are monuments devoted to the public good; each citizen owes to them a tribute proportional to
  his talents. While the great men, carried to the summit of the edifice, draw and put up the higher floors,
  the ordinary artists scattered in the lower floors, or hidden in the obscurity of the foundations, must only
  seek to improve what cleverer hands have created.''
%
}{%
  \textit{`Charles-Augustin de Coulomb'}
%
}}

%
%
\definecolor{shadecolor}{gray}{0.85}
\begin{shaded}
This chapter was published as:
%
\begin{itemize}
  \item K. Willems*, D. Rui\'{c}, F. L. R. Lucas, U. Barman, N. Verellen, G. Maglia and P. Van Dorpe.
        \textit{Nanoscale} \textbf{12}, 16775--16795 (2020) %\cite{Willems-2020}
\end{itemize}
%
\newpage
\end{shaded}
%
%


In this chapter we introduce the \glsfirst{epnp-ns} equations, a continuum simulation framework aimed towards
the accurate continuum simulation of ion and water transport through biological nanopores. The \gls{epnp-ns}
equations improve upon the regular {PNP-NS} equations by inclusion of empirical corrections that account for
the influence of the ionic strength and the proximity of protein walls on the properties of the electrolyte.
%

%
To improve readability and to allow for a better integration of the supplementary information into the main
chapter text, the original manuscript was split into \cref{ch:epnpns,ch:transport}, detailing the theory of
the framework and its application to the \glsfirst{clya} nanopore, respectively. The remainder of the
supplementary information can be found in \cref{ch:epnpns_appendix}. The text and figures of this chapter
represent entirely my own work.
%

%
\adapRSC{Willems-2020}
% \cleardoublepage
%
%


\section{Abstract}
%
\label{sec:epnpns:abstract}
%

Despite the broad success of biological nanopores as powerful instruments for the analysis of proteins and
nucleic acids at the single-molecule level, a fast simulation methodology to accurately model their
nanofluidic properties is currently unavailable. This limits the rational engineering of nanopore traits and
makes the unambiguous interpretation of experimental results challenging. Here, we present a continuum
approach that can faithfully reproduce the experimentally measured ionic conductance of the biological
nanopore \glsfirst{clya} over a wide range of ionic strengths and bias potentials. Our model consists of the
\glsfirst{epnp-ns} equations and a computationally efficient 2D-axisymmetric representation for the geometry
and charge distribution of the nanopore. Importantly, the \gls{epnp-ns} equations achieve this accuracy by
self-consistently considering the finite size of the ions and the influence of both the ionic strength and the
nanoscopic scale of the pore on the local properties of the electrolyte. These comprise the mobility and
diffusivity of the ions, and the density, viscosity and relative permittivity of the solvent. Crucially, by
applying our methodology to \gls{clya}, a biological nanopore used for single-molecule enzymology studies, we
could directly quantify several nanofluidic characteristics difficult to determine experimentally. These
include the ion selectivity, the ion concentration distributions, the electrostatic potential landscape, the
magnitude of the electro-osmotic flow field, and the internal pressure distribution. Hence, this work provides
a means to obtain fundamental new insights into the nanofluidic properties of biological nanopores and paves
the way towards their rational engineering.


\section{Introduction}
%
\label{sec:epnp-ns:intro}
%

The transport of ions and molecules through nanoscale geometries is a field of intense study that uses
experimental~\cite{Kasianowicz-1996,Meller-2000,Smeets-2006,Stefureac-2006,Wanunu-2009,Buchsbaum-2013,
Franceschini-2016,Willems-Ruic-Biesemans-2019} theoretical~\cite{Bonthuis-2006,Muthukumar-2010,
Sparreboom-2010,Bocquet-2010,Muthukumar-2014,Liu-2020} and computational methods~\cite{Im-2002,Daiguji-2004,
Aksimentiev-2005,Maffeo-2012,Modi-2012,Thomas-2014,Wang-2014,Kim-2015,Bonome-2017,Basdevant-2019,Cao-2019}.
A primary driving forces behind this research is the development of nanopores as label-free, stochastic
sensors at the ultimate analytical limit (\ie~single molecule)~\cite{Bayley-2001,Dekker-2007,Venkatesan-2011,
Zhang-2016}. These detectors have applications ranging from the analysis of biopolymers such as
DNA~\cite{Deamer-2016,Kasianowicz-1996,Meller-2000,Maglia-2008,Butler-2008,Stoddart-2009,Manrao-2012,
Franceschini-2013,Jain-2018} or proteins~\cite{Soskine-2012,Yusko-2011,Yusko-2017,Houghtaling-2019,
Restrepo-Perez-2018,Talaga-2009,Rodriguez-Larrea-2013,Nivala-2013,Kennedy-2016,Howorka-2020},
to the detection and quantification of biomarkers~\cite{Chen-2013,Niedzwiecki-2013,VanMeervelt-2014,
Huang-2017,Liu-2018,Galenkamp-2018}, or the fundamental study of chemical or enzymatic reactions at the
single molecular level~\cite{Willems-VanMeervelt-2017,Lieberman-2010,Nivala-2013,Ho-2015,Laszlo-2017,
Harrington-2019,Li-2020,Galenkamp-2020}.

Nanopores are typically operated in the resistive-pulse mode, where the changes of their ionic conductance are
monitored over time~\cite{Bayley-2001,Dekker-2007,Maglia-2010,Venkatesan-2011}. Experimentally, this is
achieved by placing the nanopore between two isolated electrolyte compartments and applying a constant DC (or
AC) voltage across them. The $\approx10^8$ difference in ionic resistance of the nanopore
(\SI{\approx1}{\giga\ohm}) compared to the typical electrolyte reservoir
(\SI{\approx10}{\ohm})~\cite{Maglia-2010}, causes the full potential change to occur within (and around) the
pore. The resulting electric field ($10^6$--$10^7$~\si{\V\per\m}) electrophoretically drives a given number of
ions (\ie~the `open-pore' conductance) and water molecules (\ie~the electro-osmotic flow) through the
pore~\cite{Wong-2007,Mao-2014,Haywood-2014,Laohakunakorn-2015}. Similarly, analyte molecules such as DNA and
proteins are subject to the same Coulombic (electrophoretic) and hydrodynamic (electro-osmotic)
forces~\cite{Wong-2007,Grosberg-2010,Muthukumar-2010,Muthukumar-2014}. This guides them towards---and often
through---the nanopore, resulting in a temporary disruption of the baseline ionic conductance. Naively
speaking, for any given analyte molecule, the rate of these resistive pulses is proportional to its
concentration in the reservoir, whereas their duration, magnitude, and intra-pulse fluctuations contain a
wealth of information on its physical properties (\eg~size, shape, and
charge)~\cite{Howorka-2009,Ying-2019,Lu-2020}. Hence, the unique `fingerprints' provided by the
(sub-)nanometer-sized sensing volume of a nanopore can reveal information that is typically inaccessible to
bulk measurements, such hidden intermediate states, dynamic noncovalent interactions, and the presence of
subpopulations~\cite{Ying-2019,Lu-2020}. Notably, nanopores have been used to create ultra-long reads of
individual DNA strands~\cite{Jain-2018}, to determine the shape, volume, and dipole moment of
proteins~\cite{Yusko-2017,Houghtaling-2019}, and to investigate the kinetics of single enzymes tethered
to~\cite{Derrington-2015,Harrington-2019} or trapped within~\cite{Li-2020,Galenkamp-2020} the pore. Because
each current blockade is modulated by the complex interactions between the translocating molecule and the
nanopore itself, they will depend on the properties of both. As a result, despite the successful applications
mentioned above, the unambiguous interpretation of the ionic current signal remains a notoriously difficult
task if a full understanding of the nanofluidic phenomena that underlie them is not available.

The computational approaches most widely used to study nanofluidic transport in ion channels or biological
nanopores comprise discrete methods such as \gls{md}~\cite{Lynden-Bell-1996,Allen-1999,
Aksimentiev-2005,Luan-2008,Bhattacharya-2011,Zhang-2014,DiMarino-2015,Belkin-2016,Basdevant-2019,Cao-2019} and
\gls{bd}~\cite{Schirmer-1999,Im-2002,Noskov-2004,Millar-2008,Egwolf-2010,DeBiase-2015,Pederson-2015}, and
mean-field (continuum) methods based on solving the \gls{pb}~\cite{Grochowski-2008, Baldessari-2008-1} and
\gls{pnp}~\cite{Eisenberg-1996,Gillespie-2002,Simakov-2010} equations. The latter two can be coupled with the
\gls{ns} equation to include electro-osmotically or pressure driven fluid flow~\cite{Lu-2012,Pederson-2015}.
Due to their explicit atomic or particle nature, \gls{md} and \gls{bd} simulations are considered to yield the
most accurate results. However, the large computational cost of simulating a complete biological nanopore
system (100K--1M atoms) for hundreds of nanoseconds still necessitates the use of
supercomputers~\cite{Aksimentiev-2005,Bhattacharya-2011,Wilson-2019,Cao-2019}. The PNP(-NS) equations, on the
other hand, are of particular interest due to their low computational demands and analytical tractability. In
a continuum approach, the simulated system is subdivided in several `structureless' domains, the behavior of
which is parameterized by material properties such as the relative permittivity, diffusion coefficient,
electrophoretic mobility, viscosity or density. Because these properties can only emerge from the collective
behavior or interactions between small groups of atoms (\ie~the mean-field approximation), great care must be
taken when using them to compute fluxes and fields at the nanoscale, where computational elements may only
contain a few molecules~\cite{Corry-2000,Collins-2012}. Nevertheless, even though the PNP equations have been
used extensively for the qualitative simulation of ion channels~\cite{Im-2002,Furini-2006,Liu-2015},
biological nanopores~\cite{Simakov-2010,Pederson-2015, Aguilella-Arzo-2017,Simakov-2018} and their solid-state
counterparts~\cite{Cervera-2005,White-2008, Chaudhry-2014,Laohakunakorn-2015}, the extent to which they are
quantitatively accurate is often challenged~\cite{Corry-2000,Collins-2012,Maffeo-2012,Thomas-2014,Kim-2015}.
To remedy the shortcomings of PNP and NS theory, a number of modifications have been proposed over the years.
These include, among others, (1) steric ion-ion interactions, (2) the effect of protein-ion/water interactions
on their `motility' (\ie~diffusivity and electrophoretic mobility), (3) the concentration dependencies of ion
motility, and solvent relative permittivity, viscosity and density.

The steric ion-ion interactions can be accounted for by computing the excess in chemical potential
($\mu_{i}^\text{ex}$) resulting from the finite size of the ions~\cite{Eisenberg-1996,Bazant-2009,
Daiguji-2010}, or by taking the dielectric-self energy of the ions into account~\cite{Corry-2003,
Bonthuis-2006}. Gillespie \etal{} combined PNP and density functional theory---where $\mu_{i}^\text{ex}$ was
split up in ideal, hard-sphere and electrostatic components---to successfully predict the selectivity and
current of ion channels~\cite{Gillespie-2002}. In the Poisson-Nernst-Planck-Fermi model developed by Liu and
Eisenberg, steric effects are included by treating water molecules as discrete particles, described using
Fermi distributions, and by implementing a mean-field version of the van der Waals
potential~\cite{Liu-2013,Liu-2015,Liu-2020}. In another approach, Kilic \etal{} derived a set of modified PNP
equations based on the free energy functional of the Borukhov's modified \gls{pb} model~\cite{Borukhov-1997}
and observed significantly more realistic concentrations for high surface potentials compared to the classical
PNP equations~\cite{Kilic-2007}. To allow for non-identical ion sizes and more than two ion species, this
model was later extended by Lu \etal{}, who used it to probe the effect of finite ion size on the rate
coefficients of enzymes~\cite{Lu-2011}.

The interaction of ions or small molecules such as water with the heavy atoms of proteins or DNA results in a
strong reduction of their motility, as observed in \gls{md}
simulations~\cite{Makarov-1998,Pronk-2014,Wilson-2019}. Since these effects happen only at distances
\SI{\le1}{\nm}, they can usually be neglected for macroscopic simulations. However, in small nanopores
(\SI{\le10}{\nm} radius), they comprise a significant fraction of the total nanopore radius and hence must be
taken into account~\cite{Noskov-2004,Simakov-2010,Pederson-2015, McMullen-2017}. In continuum simulations,
this can be achieved with the use of positional-dependent ion diffusion coefficients. An example
implementation is the `soft-repulsion PNP' developed by Simakov and
Kurnikova~\cite{Simakov-2010,Simakov-2018}, who used it to predict the ionic conductance of the \glsfirst{ahl}
nanopore. Similar reductions in ion diffusion coefficients have been proposed to improve PNP theory's
estimations of the ionic conductance of ion channels~\cite{Furini-2006,Liu-2015, DeBiase-2015}. The motility
of water molecules is expressed by the NS equations as the fluid's viscosity. Hence, as also observed in
\gls{md} simulations for water molecules near proteins~\cite{Pronk-2014} and confined in hydrophilic
nanopores~\cite{Qiao-Aluru-2003,Vo-2016,Hsu-2017}, the water-solid interaction leads to a viscosity several
times higher compared to the bulk values. Note that this is valid for hydrophilic interfaces only, as the lack
of interaction with hydrophobic interfaces, such as carbon nanotubes, leads to a lower
viscosity~\cite{Ye-2011}.

It is well known that the self-diffusion coefficient $\diffusion_{i}$ and electrophoretic mobility
$\mobility_{i}$ of an ion $i$ depends on the local concentrations of all the ions in the
electrolyte~\cite{ContrerasAburto-2013-1}. Their values typically decrease with increasing salt concentration,
and should not be treated as constants. Moreover, even though the Nernst-Einstein relation
($\mobility_{i}=\diffusion_{i}/\kbt$) is strictly speaking only valid at infinite dilution and a good
approximation at low concentrations (\SI{<10}{\mM}), it significantly overestimates the ionic mobility at
higher salt concentrations~\cite{Mills-1989,Panopoulos-1986,ContrerasAburto-2013-1,ContrerasAburto-2013-2}. In
an empirical approach, Baldessari and Santiago formulated an ionic-strength dependency of the ionic mobility
based on the activity coefficient of the salt~\cite{Baldessari-2008-1} and showed excellent correspondence
between the experimental and simulated ionic conductance of long nanochannels over a wide concentration
range~\cite{Baldessari-2008-2}. Alternatively, Burger \etal{} used a microscopic lattice-based model to derive
a set of PNP equations with non-linear, ion density-dependent mobilities and diffusion coefficients that
provided significantly more realistic results for ion channels~\cite{Burger-2012}. Note that other electrolyte
properties, such as the viscosity~\cite{Hai-Lang-1996}, density~\cite{Hai-Lang-1996} and relative
permittivity~\cite{Gavish-2016}, may also significantly affect the ion and water flux. To better compute the
charge flux in ion channels, Chen derived a new PNP framework~\cite{Chen-2016} that includes water-ion
interactions in the form of a concentration-dependent relative permittivity and an additional ion-water
interaction energy term.

To the best of our knowledge, no attempt has been made to consolidate all of the corrections discussed above
into a single framework. Hence, we propose the \glsfirst{epnp-ns} equations, which improve the predictive
power of the traditional \gls{pnp-ns} equations at the nanoscale and beyond infinite dilution. Our
\gls{epnp-ns} framework takes into account the finite size of the ions using a size-modified PNP
theory~\cite{Borukhov-1997,Lu-2011}, and implements spatial-dependencies for the solvent
viscosity~\cite{Pronk-2014,Vo-2016,Hsu-2017}, the ion diffusion coefficients and their
mobilities~\cite{Makarov-1998,Noskov-2004,Pederson-2015}. It also includes self-consistent
concentration-dependent properties---based on empirical fits to experimental data---for all ions in terms of
diffusion coefficients and mobilities~\cite{Baldessari-2008-1,Mills-1989}, and for the solvent in terms of
density, viscosity~\cite{Hai-Lang-1996} and relative permittivity~\cite{Gavish-2016}. In
\cref{sec:epnp-ns:model} we detail the governing equations, followed by their parameterization with
experimental data from literature in \cref{sec:epnp-ns:parameterization}.


\section{Mathematical model}
%
\label{sec:epnp-ns:model}
%

The use of continuum or mean-field representations for both the nanopore and the electrolyte enables us to
efficiently compute the steady-state ion and water fluxes under almost any condition. The dynamic behavior of
our complete system is described by the coupled Poisson, Nernst-Planck and Navier-Stokes equations, a set of
partial differential equations that describe the electrostatic field, the total ionic flux and the fluid flow,
respectively~\cite{Eisenberg-1996,Cervera-2005,Lu-2012}. To improve upon the quantitative accuracy of the
\gls{pnp-ns} equations for nanopore simulations, we developed an extended version of these equations
({\gls{epnp-ns}}) and implemented it in the commercial finite element solver {COMSOL} Multiphysics (v5.4,
COMSOL Inc., Burlington, MA, USA). 

\subsection{Electrostatic field}
%
\label{sec:epnp-ns:electrostatic_field}
%

We will make use of Poisson's equation to evaluate the electric potential
%
\begin{align}
  \label{eq:poisson}
  \nabla \cdot \left(\absperm \relperm \nabla \potential \right) = -\left( \scdpore + \scdion \right)
  \text{ ,}
\end{align}
%
with $\potential$ the electric potential, $\absperm$ the vacuum permittivity
(\SI{8.85419e-12}{\farad\per\meter}) and $\relperm$ local relative permittivity. The pore's fixed charge
distribution, $\scdpore$,  can be derived directly from the full atom model of the pore
(see~\cref{eq:scdpore}). The ionic charge density in the fluid is given by
%
\begin{align}\label{eq:scdion}
  \scdion = \faraday\sum_{i}\chargen_{i}\concentration_{i}
  \text{ ,}
\end{align}
%
with $\faraday$ Faraday's constant (\SI{96485.33}{\coulomb\per\mole}), and $\concentration_{i}$ the ion
concentration and $\chargen_{i}$ ion charge number of ion $i$. To account for the concentration dependence of
the electrolyte's relative permittivity, we replaced $\relperm$ inside the electrolyte with the expression
%
\begin{align}\label{eq:epnp-ns_relperm_concentration}
  \permittivity_{r,\text{f}}(\avionconc) =
        \permittivity_{r,\text{f}}^0 \permittivity_{r,\text{f}}^c(\avionconc)
  \text{ ,}
\end{align}
%
with $\avionconc=\frac{1}{n}\sum_{i}^{n}\concentration_{i}$ the average ion concentration,
$\permittivity_{r,\text{f}}^0$ the relative permittivity at infinite dilution and
$\permittivity_{r,\text{f}}^c$ a concentration dependent empirical function parameterized with experimental
data
(\cref{eq:epnpns_concentration_solv_epsr,fig:epnpns_concentration_solv_epsr,tab:corrections_equations,tab:corrections_parameters}).

\subsection{Ionic flux}
%

The total ionic flux $\flux_{i}$ of each ion $i$ is given by the size-modified Nernst-Planck
equation~\cite{Lu-2011}, and can be expressed as the sum of diffusive, electrophoretic, convective and steric
fluxes
%
\begin{align}
  \label{eq:sm-nernst-planck}
  \flux_{i} = -\left[
    \diffusion_{i} \nabla \concentration_{i}
    + \chargen_{i} \mobility_{i} \concentration_{i} \nabla \potential
    - \velocity \concentration_{i}
    + \diffusion_{i} \vec{\beta_{i}} \concentration_{i} \right]
  \text{ ,}
\end{align}
%
where $\vec{\beta_{i}}$ is the steric flux vector
%
\begin{align}\label{eq:epnp-ns_steric_flux_vector}
  \vec{\beta_{i}} =
      \frac{ \ionsize_{i}^3 / \ionsize_{0}^3 \dsum_{j} \avogadro \ionsize_{j}^3 \nabla \concentration_{j} }
          { 1 - \dsum_{j} \avogadro \ionsize_{j}^3 \concentration_{j} }
  \text{ ,}
\end{align}
%
and at steady state
%
\begin{align}
  \dfrac{\partial \concentration_{i}}{\partial \timedim} ={}& - \nabla \cdot \flux_{i} = 0
  \text{ ,}
\end{align}
%
with $\diffusion_{i}$ the ion diffusion coefficient, $\concentration_{i}$ the ion concentration,
$\chargen_{i}$ the ion charge number, $\mobility_{i}$ the electrophoretic mobility of ion $i$. $\potential$ is
the electrostatic potential, $\velocity$ the fluid velocity and $\avogadro$ Avogadro's constant
(\SI{6.022e23}{\per\mole}). $\ionsize_{i}$ and $\ionsize_{0}$ are steric cubic diameters of respectively ions
and water molecules. Because currently there are no experimentally verified values available for
$\ionsize_{i}$ and $\ionsize_{0}$, we set them to \SI{0.5}{\nm} (max. \SI{13.3}{\Molar}) and \SI{0.311}{\nm}
(max. \SI{55.2}{\Molar}), respectively~\cite{Bazant-2009}.

The reduction of the ionic motility at increasing salt concentrations and in proximity to the nanopore walls
was implemented self-consistently by replacing $\diffusion_{i}$ and $\mobility_{i}$ with the expressions
%
\begin{align}\label{eq:epnp-ns_diffusion_and_mobility}
  \diffusion_{i}(\avionconc,\walldistance) ={}&
      \diffusion_{i}^0 \diffusion_{i}^c(\avionconc) \diffusion_{i}^w(\walldistance)  \\
  \mobility_{i}(\avionconc,\walldistance) ={}&
      \mobility_{i}^0 \mobility_{i}^c(\avionconc) \mobility_{i}^w(\walldistance)
  \text{ ,}
\end{align}
%
where $\diffusion_{i}^0$ and $\mobility_{i}^0$ represent the values at infinite dilution. The concentration
dependent factors $\diffusion_{i}^c(\avionconc)$ and $\mobility_{i}^c(\avionconc)$ are empirical functions
fitted to experimental data (between \SIrange{0}{5}{\Molar} \ce{NaCl}) of respectively the ion self-diffusion
coefficients~\cite{Mills-1989} (\cref{eq:concentration_d_mu_tna,fig:epnpns_concentration_d}) and the
electrophoretic mobilities~\cite{Bianchi-1989,Currie-1960,Goldsack-1976,DellaMonica-1979}
(\cref{eq:concentration_d_mu_tna,fig:epnpns_concentration_mu}). Likewise, the factors
$\diffusion_{i}^w(\walldistance)$ and $\mobility_{i}^w(\walldistance)$ are empirical functions that introduce
a spatial dependency on the distance from the nanopore wall $\walldistance$, and were parameterized by fitting
to molecular dynamics data
(\cref{eq:epnpns_distance_d,fig:epnpns_distance_d,tab:corrections_equations,tab:corrections_parameters})~\cite{Noskov-2004,Simakov-2010,Makarov-1998,Wilson-2019}.

Based on the observation that the diffusivity of nanometer- to micrometer-sized particles reduces
significantly when confined in pores and slits of comparable dimensions~\cite{Renkin-1954,Deen-1987,
Dechadilok-2006,Muthukumar-2014,Kannam-2017}, Simakov \etal{}~\cite{Simakov-2010} and Pederson
\etal{}~\cite{Pederson-2015} reduced the ion motilities inside the pore as a function of the ratio between the
ion and the nanopore radii. We chose not to include this correction into our model, as extrapolating its
applicability for ions with a hydrodynamic radii comparable to size of the solvent molecules is
questionable~\cite{Anderson-1972,Deen-1987}.


\subsection{Fluid flow}
%

As derived by Axelsson \etal{}~\cite{Axelsson-2015}, the fluid flow and pressure field inside an
incompressible fluid with a variable density and variable viscosity is given by the Navier-Stokes equations:
%
\begin{align}
  \label{eq:navier-stokes-variable}
  \dfrac{\partial}{\partial \timedim} \left( \density \velocity \right) +
  \left( \velocity \cdot \nabla \right) \left( \density\velocity \right)
  + \nabla \cdot \hydrostresstensor = \volumeforce
  \text{ ,}
\end{align}
%
where
%
\begin{align}
  \hydrostresstensor =
  \pressure\identity - \viscosity\left[\nabla\velocity+\left(\nabla\velocity \right)^\mathsf{T}\right]
  \text{ ,}
\end{align}
%
together with the continuity equations for the fluid density
%
\begin{align}
  \label{eq:continuity-density}
  \dfrac{\partial \density}{\partial \timedim} + \velocity \cdot \nabla \density  = 0
  \text{ ,}
\end{align}
%
and the divergence constraint for the momentum
%
\begin{align}
  \nabla \cdot \left( \density\velocity \right) - \velocity \cdot \nabla \density ={}& 0
  \text{ ,}
\end{align}
%
with $\velocity$ the fluid velocity, $\density$ the fluid density, $\hydrostresstensor$ the hydrodynamic
stress tensor, $\viscosity$ the viscosity and $\pressure$ the pressure. The external body force density
$\volumeforce$ that acts on the fluid is given by
%
\begin{align}\label{eq:ion_force_density}
  \volumeforce = \scdion \efield
  \text{ ,}
\end{align}
%
with $\efield = - \nabla \potential$ the electric field vector. At steady-state, the partial derivatives with
respect to time in \cref{eq:navier-stokes-variable,eq:continuity-density} become equal to zero:
%
\begin{align}
  \dfrac{\partial}{\partial \timedim} \left( \density \velocity \right) ={}& 0 \\
  \dfrac{\partial \density}{\partial \timedim} ={}& 0
  \text{ .}
\end{align}
%
As with the previous equations, we introduced a concentration and wall distance dependency for $\viscosity$
and a concentration dependency for the $\density$ by replacing their constant values by
%
\begin{align}\label{eq:epnp-ns_viscosity_density}
  \viscosity(\avionconc,\walldistance) ={}&
    \viscosity^0 \viscosity^c(\avionconc) \viscosity^w(\walldistance) \\
  \density(\avionconc) ={}&
    \density^0 \density^c(\avionconc)
  \text{ ,}
\end{align}
%
where $\viscosity^0$ and $\density^0$ are the values at infinite dilution (\ie~pure water). The empirical
functions $\viscosity^c(\avionconc)$, $\density^c(\avionconc)$ and $\viscosity^w(\walldistance)$ were
parameterized \textit{via} fitting to experimental~\cite{Hai-Lang-1996}
(\cref{eq:epnpns_concentration_solv_eta,eq:epnpns_concentration_solv_rho,fig:epnpns_concentration_solv_eta,fig:epnpns_concentration_solv_rho})
and molecular dynamics~\cite{Pronk-2014} (\cref{eq:epnpns_distance_eta,fig:epnpns_distance_eta}) data obtained
from literature, as summarized in \cref{tab:corrections_equations,tab:corrections_parameters}.


\subsubsection{From {ePNP-NS} to {PNP-NS}}
%

The \gls{epnp-ns} equations revert into the regular \gls{pnp-ns} equations by disabling the steric flux vector
($\vec{\beta}=0$) and all concentration ($\permittivity_{r,\text{f}}^c=1$, $\diffusion_{i}^c=1$,
$\mobility_{i}^c=1$, $\viscosity^c=1$, $\density^c=1$) and wall distance functions ($\diffusion_{i}^w=1$,
$\mobility_{i}^w=1$, $\viscosity^w=1$).


%
\begin{table}[p]
  \begin{threeparttable}
    \centering
    %
    \captionsetup{width=12cm}
    \caption{Parameters and fitting equations used in the {ePNP-NS} equations.}
    \label{tab:corrections_equations}
    %
    \footnotesize
    \renewcommand{\arraystretch}{1.2}
    %
    %
    \begin{tabularx}{12cm}{>{\raggedright\hsize=2.5cm}X >{\hsize=1.5cm}l >{\hsize=5cm}X >{\hsize=2cm}l}
      \toprule
    
      Name & Symbol\tnote{a} & Infinite dilution value\tnote{b}/Function\tnote{c} & Reference \\
    
      \midrule
    
      \multirow{4}{1.5cm}{Relative permittivity}
        & $\permittivity_{r,\text{p}}$ & \num{20} & \cite{Li-2013} \\
        & $\permittivity_{r,\text{m}}$ & \num{3.2} & \cite{Gramse-2013} \\
        & $\permittivity_{r,\text{f}}^0$ & \num{78.15} & \cite{Gavish-2016} \\
        & $\permittivity_{r,\text{f}}^c(\dconc)$ & $1 - \left(1 -	\dfrac{P_1}{P_0}\right) L \left(
          \dfrac{3P_2}{P_0 - P_1} \dconc \right)$ & \cite{Gavish-2016} \vspace{0.5cm} \\
      
      \multirow{4}{1.5cm}{Ion self-diffusion coefficient}
        & $\diffusion_{\Na}^0$ & \SI{1.334e-9}{\square\meter\per\second} & \cite{Mills-1989} \\
        & $\diffusion_{\Cl}^0$ & \SI{2.032e-9}{\square\meter\per\second} & \cite{Mills-1989} \\
        & $\diffusion_{i}^c(\dconc)$ & $\left( 1 + P_1\dconc^{0.5} + P_2\dconc + P_3\dconc^{1.5} + P_4\dconc^2
          \right)^{-1}$ & This work \\
        & $\diffusion_{i}^w(\dwall)$ & $1-\exp{\left(-P_1(\dwall+P_2)\right)}$ &
          \cite{Makarov-1998,Simakov-2010} \vspace{0.25cm} \vspace{0.5cm} \\
      
      \multirow{4}{1.5cm}{Ion electro\hyp{}phoretic mobility} & $\mobility_{\Na}^0$ &
        \SI{5.192e-4}{\square\meter\per\second\per\volt} & \cite{Bianchi-1989} \\
        & $\mobility_{\Cl}^0$ & \SI{7.909e-4}{\square\meter\per\second\per\volt} & \cite{Bianchi-1989} \\
        & $\mobility_{i}^c(\dconc)$ & $\left( 1 + P_1\dconc^{0.5} + P_2\dconc + P_3\dconc^{1.5} + P_4\dconc^2
          \right)^{-1}$ & This work \\
        & $\mobility_{i}^w(\dwall)$ & $1-\exp{\left(-P_1(\dwall+P_2)\right)}$ &
          \cite{Makarov-1998,Simakov-2010} \vspace{0.5cm} \\
    
      \multirow{2}{1.5cm}{Ion transport number}
        & $\transportn_{\Na}^0$ & 0.396 & \cite{Bianchi-1989} \\
        & $\transportn_{\Na}^c(\dconc)$ & $\left( 1 + P_1\dconc^{0.5} + P_2\dconc + P_3\dconc^{1.5} +
          P_4\dconc^2 \right)^{-1}$ & This work \vspace{0.5cm} \\
    
      \multirow{3}{1.5cm}{Dynamic viscosity}
        & $\viscosity^0$ & \SI{8.904e-4}{\pascal\second} & \cite{Hai-Lang-1996} \\
        & $\viscosity^c(\dconc)$ & $ 1 + P_1 \dconc^{0.5} + P_2 \dconc + P_3 \dconc^2 + P_4 \dconc^{3.5}$ &
          This work \\
        & $\viscosity^w(\dwall)$ & $1 + \exp{ \left( -P_1(\dwall-P_2) \right) }$ & \cite{Pronk-2014}
          \vspace{0.5cm} \\
    
      \multirow{2}{1.5cm}{Fluid density}
        & $\density^0$ & \SI{997}{\kilogram\per\cubic\meter} & \cite{Hai-Lang-1996} \\
        & $\density^c(\dconc)$ & $1 + P_1 \dconc + P_2 \dconc^2$ & This work \\
      \bottomrule
    \end{tabularx}
    %
    \begin{tablenotes}
      \item[a] Dependencies on either $\dconc = \avionconc/1$~M (dimensionless average ion concentration) and
      $\dwall = \walldistance/1$~nm (dimensionless distance from the nanopore wall);
      %
      \item[b] Values at infinite dilution for a system temperature of \SI{298.15}{\kelvin};
      %
      \item[c] These functions are empirical and hence have no physical meaning. $L$ is the Langevin function
      $L (x) = \coth(x) - 1/x$. The values of the fitting parameters $P_x$ of each property can be found in
      \cref{tab:corrections_parameters} and graphs of the fits in
      \cref{fig:epnpns_concentration_d,fig:epnpns_concentration_mu,fig:epnpns_concentration_solv,fig:epnpns_distance}.
    \end{tablenotes}
    %
  \end{threeparttable}
\end{table}
%


\section{Empirical parameterization}
%
\label{sec:epnp-ns:parameterization}
%

The properties of the ions and water molecules in the electrolyte depend strongly on the salt concentration
($\concentration$) and the distance from the protein wall ($\walldistance$). To obtain reasonable estimates of
these values for any given \ce{NaCl} concentration or wall distance, we fitted a series of empirical
polynomial functions, using non-linear regression with the \code{lmfit} Python package~\cite{Newville-2014},
to literature data: either bulk electrolyte experimental data for the concentration dependencies and molecular
dynamics data for the wall distance dependencies. In the next sections we will describe the fits in details,
and the ll best-fit parameters are summarized in~\cref{tab:corrections_parameters}.

%
\subsection{Concentration-dependent fitting to bulk electrolyte data}
%

\subsubsection{Diffusivity, mobility and transport number.}
%
The \ce{NaCl} concentration dependency data of ion self-diffusion coefficients, the ion electrophoretic
mobilities and the cation transport number were fitted to
%
\begin{align}\label{eq:concentration_d_mu_tna}
  X_{i} (\concentration) = P_0 X_{i}^c (\concentration) =
  P_0 \left[ 1
           + P_1 \concentration^{0.5}
           + P_2 \concentration
           + P_3 \concentration^{1.5}
           + P_4 \concentration^2 
      \right]^{-1}
  \text{ ,}
\end{align}
%
with $X$ either $\diffusion$ (diffusion), $\mobility$ (mobility) or $\transportn$ (transport number) and $i$
either \Na{} or \Cl{}. $P_0$ is the value at infinite dilution (\ie~$\concentration = \SI{0}{\Molar}$) and
fixed during fitting of parameters $P_1$, $P_2$, $P_3$ and $P_4$.

%
\begin{figure*}[!b]
  \centering

  %
  \begin{subfigure}[t]{11cm}
    \centering
    \caption{}\vspace{-1mm}\label{fig:epnpns_concentration_d_sodium}
    \includegraphics[scale=1]{epnpns_concentration_d_sodium}
  \end{subfigure}
  \\
  \begin{subfigure}[t]{11cm}
    \centering
    \caption{}\vspace{-1mm}\label{fig:epnpns_concentration_d_chloride}
    \includegraphics[scale=1]{epnpns_concentration_d_chloride}
  \end{subfigure}
  %

  \caption[Concentration depend. of ion self-diffusion coefficients in \ce{NaCl}]%
    {%
      \textbf{Concentration dependency of ion self-diffusion coefficients in \ce{NaCl}.}
      %
      (\subref{fig:epnpns_concentration_d_sodium})
      %
      \Na{} and
      %
      (\subref{fig:epnpns_concentration_d_chloride})
      %
      \Cl{} self-diffusion coefficients~\cite{Mills-1989} as a function of the bulk \ce{NaCl} concentration
      (left) and the relative residuals after fitting (right) of \cref{eq:concentration_d_mu_tna}. Circles
      represent the experimental data and solid lines the fitted equation with the gray shading as the
      $3\sigma$ confidence interval.
      %
  }\label{fig:epnpns_concentration_d}
\end{figure*}
%

%
\begin{figure*}[p]
  \centering

  %
  \begin{subfigure}[t]{11cm}
    \centering
    \caption{}\vspace{-1mm}\label{fig:epnpns_concentration_mu_tsodium}
    \includegraphics[scale=1]{epnpns_concentration_mu_tsodium}
  \end{subfigure}
  %
  \\
  %
  \begin{subfigure}[t]{11cm}
    \centering
    \caption{}\vspace{-1mm}\label{fig:epnpns_concentration_mu_sodium}
    \includegraphics[scale=1]{epnpns_concentration_mu_sodium}
  \end{subfigure}
  %
  \\
  %
  \begin{subfigure}[t]{11cm}
    \centering
    \caption{}\vspace{-1mm}\label{fig:epnpns_concentration_mu_chloride}
    \includegraphics[scale=1]{epnpns_concentration_mu_chloride}
  \end{subfigure}
  %

  \caption%
    [Concentration depend. of the \ce{Na+} transport nb. and the ion electroph. mob. in \ce{NaCl}]
    %
    {%
      \textbf{Concentration dependency of the cation transport number and the ion electrophoretic mobilities
      in \ce{NaCl}.}
      %
      (\subref{fig:epnpns_concentration_mu_tsodium})
      %
      \Na{} transport numbers
      and~\cite{Esteso-1976,Haynes-2017,DellaMonica-1979,Panopoulos-1986,Schonert-2013},
      %
      (\subref{fig:epnpns_concentration_mu_sodium})
      %
      \Na{} and
      %
      (\subref{fig:epnpns_concentration_mu_chloride})
      %
      \Cl{} electrophoretic mobilities~\cite{Bianchi-1989,Currie-1960,Goldsack-1976,DellaMonica-1979} as a
      function of the bulk \ce{NaCl} concentration (left) and the relative residuals after fitting (right) of
      \cref{eq:concentration_d_mu_tna}. Circles represent the experimental data and solid lines the fitted
      equation with the gray shading as the $3\sigma$ confidence interval.
      %
  }\label{fig:epnpns_concentration_mu}
\end{figure*}
%

\paragraph{Ion self-diffusion coefficients.}
%
The data for the fitting of the self-diffusion coefficients of the \Na{} ($\diffusion_{\Na}^c (c)$,
\cref{fig:epnpns_concentration_d_sodium}) and \Cl{} ($\diffusion_{\Cl}^c (c)$,
\cref{fig:epnpns_concentration_d_chloride}) ranged from \SIrange{0}{4}{\Molar} and was taken from the
compilation of Mills~\cite{Mills-1989}. Fitting of \cref{eq:concentration_d_mu_tna} yielded relative residuals
\SI{<\pm2}{\percent} (\cref{fig:epnpns_concentration_d}, left panels), indicating an excellent representation
of the data. Because no experimental data was available beyond \SI{4}{\Molar}, the diffusivities between
\SIrange{4}{5.3}{\Molar} had to be extrapolated, whereas for simulated concentrations \SI{>5.3}{\Molar}, their
respective values at \SI{5.3}{\Molar} were used (\ie~$\diffusion_{\Na} (c > \mSI{5.3}{\Molar}) =
\mSI{0.81e-9}{\meter\squared\per\second}$, $\diffusion_{\Cl} (c > \mSI{5.3}{\Molar}) =
\mSI{1.07e-9}{\meter\squared\per\second}$).

\paragraph{Sodium transport number.}
%
The literature data for the concentration dependency of the \Na{} transport number ($\transportn_{\Na}^c (c)$,
\cref{fig:epnpns_concentration_mu_tsodium}) was obtained from five separate
sources~\cite{Esteso-1976,Haynes-2017,DellaMonica-1979,Panopoulos-1986,Schonert-2013} and spanned the entire
experimentally accessible \ce{NaCl} concentration range (\SIrange{0}{5.3}{\Molar}). After fitting
\cref{eq:concentration_d_mu_tna}, we found most relative fitting residuals to be \SI{<\pm1}{\percent}
(\cref{fig:epnpns_concentration_mu_tsodium}, left panel).

\paragraph{Ion electrophoretic mobilities.}
%
The individual data for concentration-dependent ionic mobilities of \Na{} ($\mobility^c_{\Na}$,
\cref{fig:epnpns_concentration_mu_sodium}) and \Cl{} ($\mobility^c_{\Cl}$,
\cref{fig:epnpns_concentration_mu_chloride}) were calculated from literature
values~\cite{Bianchi-1989,Currie-1960,Goldsack-1976,DellaMonica-1979} of the salt's molar conductivity
$\molarconductivity$ (between~\SIrange{0}{5}{\Molar}) using~\cite{ContrerasAburto-2013-1}
%
\begin{align}\label{eq:conductivity-to-mobility}
  \mobility_{i}(\concentration) = \frac{\specmolarconductivity_{i}(\concentration)}{\chargen_{i}\faraday}
  \quad\text{with}\quad \specmolarconductivity_{i}(\concentration) = \molarconductivity(\concentration)
  \transportn_{i}(\concentration)
  \text{ ,}
\end{align}
%
where $\specmolarconductivity_{i}(\concentration)$ is the specific molar conductivity of ion $i$, and
$\faraday$ the Faraday constant (\SI{96485}{\coulomb\per\mole}). The transport number for \Na{} was already
computed in the previous paragraph, and the one for \Cl{} is given by $\transportn_{\Cl}^c (c) = 1 -
\transportn_{\Na}^c (c)$. \Cref{eq:concentration_d_mu_tna} was fitted to both data sets individually, and the
relative residuals were found to be \SI{<\pm0.5}{\percent}. Mobilities for simulated concentrations between
\SIlist{5;5.3}{\Molar} were extrapolated, and for concentrations beyond the values were capped to those at
\SI{5.3}{\Molar} (\ie~$\mobility_{\Na} (c > \mSI{5.3}{\Molar}) =
\mSI{1.74e-8}{\meter\squared\per\volt\per\second}$, $\mobility_{\Cl} (c > \mSI{5.3}{\Molar}) =
\mSI{3.21e-8}{\meter\squared\per\volt\per\second}$).


\subsubsection{Viscosity.}
%

The interpolation function representing the concentration-dependent electrolyte viscosity ($\viscosity
(\concentration)$) is given by
%
\begin{align}\label{eq:epnpns_concentration_solv_eta}
  \viscosity (\concentration) = P_0 \viscosity^c (\concentration) =
  P_0 \left[ 1
            + P_1 \concentration^{0.5}
            + P_2 \concentration
            + P_3 \concentration^{2}
            + P_4 \concentration^{3.5}
      \right]
  \text{ ,}
\end{align}
%
with $P_0$ the value at infinite dilution (fixed during fitting), and $P_1$, $P_2$, $P_3$ and $P_4$ fitting
parameters. Experimental data for parameterization of \cref{eq:epnpns_concentration_solv_eta} was taken from a
single literature source~\cite{Hai-Lang-1996} and covers the entire concentration range
(\SIrange{0}{5.3}{\Molar}). The resulting fit to this data, yielding $\viscosity^c (\concentration)$, is shown
in~\cref{fig:epnpns_concentration_solv_eta}. Relative fitting residuals were found to be very low
(\SI{<\pm0.05}{\percent}, \cref{eq:epnpns_concentration_solv_eta}, left panel). As with the previous
parameters, viscosities for concentrations above saturated concentrations were capped to their values at
\SI{5.3}{\Molar} (\ie~$\viscosity (c > \mSI{5.3}{\Molar}) = \mSI{1.75e-4}{\pascal\second}$).

%
\begin{figure*}[p]
  \centering
  %
  \begin{subfigure}[t]{11cm}
    \centering
    \caption{}\vspace{-1mm}\label{fig:epnpns_concentration_solv_eta}
    \includegraphics[scale=1]{epnpns_concentration_solv_eta}
  \end{subfigure}
  %
  \\
  %
  \begin{subfigure}[t]{11cm}
    \centering
    \caption{}\vspace{-1mm}\label{fig:epnpns_concentration_solv_rho}
    \includegraphics[scale=1]{epnpns_concentration_solv_rho}
  \end{subfigure}
  %
  \\
  %
  \begin{subfigure}[t]{11cm}
    \centering
    \caption{}\vspace{-1mm}\label{fig:epnpns_concentration_solv_epsr}
    \includegraphics[scale=1]{epnpns_concentration_solv_epsr}
  \end{subfigure}
  %

  \caption[Concentration depend. of the electrolyte viscosity, density and rel. perm. in \ce{NaCl}]%
    {%
      \textbf{Concentration dependency of the electrolyte viscosity, density and relative permittivity in
      \ce{NaCl}.}
      %
      (\subref{fig:epnpns_concentration_solv_eta})
      %
      Viscosity~\cite{Hai-Lang-1996},
      %
      (\subref{fig:epnpns_concentration_solv_rho})
      %
      density~\cite{Hai-Lang-1996} and
      %
      (\subref{fig:epnpns_concentration_solv_epsr})
      %
      relative permittivity~\cite{Buchner-1999,Gavish-2016} as a function of the bulk \ce{NaCl} concentration
      (left) and the relative residuals after fitting (right) to
      \cref{eq:epnpns_concentration_solv_eta,,eq:epnpns_concentration_solv_rho,,eq:epnpns_concentration_solv_epsr},
      respectively. Circles represent the experimental data and solid lines the fitted equation with the gray
      shading as the $3\sigma$ confidence interval.
      %
  }\label{fig:epnpns_concentration_solv}
\end{figure*}


\subsubsection{Density.}
%

The interpolation function representing the concentration dependency of the electrolyte density is given by
%
\begin{align}\label{eq:epnpns_concentration_solv_rho}
  \density (\concentration) = P_0 \density^c (\concentration) =
  P_0 \left[ 1
            + P_1 \concentration
            + P_2 \concentration^{2}
      \right]
  \text{ ,}
\end{align}
%
with $P_0$ the value at infinite dilution (fixed during fitting), and $P_1$ and $P_2$ fitting parameters.
Experimental density data was taken from the source as the viscosity~\cite{Hai-Lang-1996} and the fit,
yielding $\density^c (\concentration)$ (\cref{eq:epnpns_concentration_solv_rho}), also showed very low
relative fitting residuals (\SI{<\pm0.05}{\percent}, \cref{eq:epnpns_concentration_solv_rho}, left panel).
Density values for concentrations above \SI{5.3}{\Molar} were capped as usual (\ie~$\density (c >
\mSI{5.3}{\Molar}) = \mSI{1.19e3}{\kilo\gram\per\meter\cubed}$).


\subsubsection{Relative permittivity.}
%

For the concentration dependency of the electrolyte relative permittivity, we used the microfield model
developed by Gavish \etal{}~\cite{Gavish-2016}
%
\begin{align}\label{eq:epnpns_concentration_solv_epsr}
  \relperm (\concentration) = P_0 \permittivity_{r,\text{f}}^c (\concentration) =
  P_0 \left[ 1
              - \left( 1 - \dfrac{P_1}{P_0}\right)
              L \left( \dfrac{3 P_2}{P_0 - P_1} \concentration \right)
      \right]
  \text{ ,}
\end{align}
%
where $L$ is the Langevin function ($L(x) = \coth{(x)} - 1/x$), $P_0$ the value at infinite dilution, $P_1$
the limiting permittivity ($\permittivity_{r,ms}$) and $P_2$ the total excess polarization of the ions
($\alpha$). Even though we fitted the equation to our literature dataset~\cite{Buchner-1999} ($P_1 =
29.50\pm1.32$ and $P_2 = 11.74\pm0.21$), we opted to make use of the parameters given by Gavish for \ce{NaCl}
at \SI{298.15}{\kelvin} ($P_1 = 30.08$ and $P_2 = 11.5$~\cite{Gavish-2016},
\cref{fig:epnpns_concentration_solv_epsr}). We found the relative fitting residuals
(\cref{fig:epnpns_concentration_solv_epsr}, left panel) to be low (\SI{<\pm2}{percent}). Relative
permittivities for concentrations above saturation were capped as before (\ie~$\relperm (c >
\mSI{5.3}{\Molar}) = \num{42.12}$).


\subsubsection{Note on the concentration dependencies.}
%

All concentration dependent parameters use the local ionic strength rather than their individual ion
concentrations. Though valid for electroneutral bulk solutions, this approximation no longer holds inside the
electrical double layer (\ie~near charged surfaces or inside small nanopores), where local electroneutrality
is violated. The main reasons for making this simplification regardless are the lack of non-bulk experimental
data and the absence of a tractable analytical model. Furthermore, we will see in~\cref{ch:transport} that the
current implementation of our concentration dependent functions will lead to an excellent agreement with the
experimental data in all but the most extreme cases, justifying our choice \textit{a posteriori}.


%
\subsection[Wall distance-dependent fitting to {MD} data]%
           {Wall distance-dependent fitting to molecular dynamics data}
%

\subsubsection{Diffusivity and mobility.}
%
The protein wall distance function used for the ion self-diffusion coefficients and the ion electrophoretic
mobilities was taken from Simakov \etal{}~\cite{Simakov-2010}, who fitted it to the molecular dynamics data
from Makarov \etal{}~\cite{Makarov-1998}. The function is a sigmoidal given by
%
\begin{align}\label{eq:epnpns_distance_d}
  X_{i}^w (\walldistance) = 1 - \exp{-P_1 (\walldistance - P_2)}
  \text{ ,}
\end{align}
%
with $X$ either $\diffusion$ (diffusion) or $\mobility$ (mobility) and $i$ either \Na{} or \Cl{} and $P_1$
($a$) and $P_2$ ($r_0$) the fitting parameters. For $P_1$, we used the value determined by Simakov \etal{}
(\SI{6.2}{\per\nm}), but for $P_2$ we opted to use a value of \SI{0.01}{\nm} instead of \SI{0.22}{\nm}
proposed by Simakov~\cite{Simakov-2010} and further corroborated by Wilson \etal{}~\cite{Wilson-2019}. We
chose this smaller offset because \cref{eq:epnpns_distance_d} quickly becomes negative beyond the offset
$P_2$, resulting in nonphysical values. Pederson \etal{} solved this by setting an arbitrary minimum positive
value for the diffusion coefficient~\cite{Pederson-2015}, but this would result in a `dead' zone of nearly
\SI{0.2}{\nm} from each wall with a very low diffusivity (\ie~several orders of magnitude below the bulk
value). This would make the current of a pore with a diameter smaller than \SI{0.5}{\nm} effectively zero,
which has been shown experimentally to be not the case~\cite{Rigo-2019}. In addition, $P_2$ represents the
center-to-center distance between the ion and the nearest heavy atom, meaning that approximately half of the
\SI{0.22}{\nm} would already be inside our geometry (\ie~within the van der Waals radius of the heavy atom),
and the other half would be inaccessible to ions due to excluded volume effects. Because we are not modeling
the latter, we opted to assign the full offset given by $P_2$ to the heavy atom, effectively assuming that all
ion-inaccessible space is already taken into account by our geometry. Our choice of parameters leads to
values of $X_{i}^w (\mSI{0}{\nm}) = 0.06$ (\ie~at the wall boundary) and $X_{i}^w (\mSI{0.75}{\nm}) = 0.99$
(\ie~\SI{0.75}{\nm} within the electrolyte, \cref{fig:epnpns_distance_d}), which roughly corresponds to the
values reported by Wilson \etal{}~\cite{Wilson-2019}.
%

%
\begin{figure*}[b]
  \centering

  %
  \begin{subfigure}[t]{11cm}
    \centering
    \caption{}\vspace{0mm}\label{fig:epnpns_distance_d}
    \includegraphics[scale=1]{epnpns_distance_d}
  \end{subfigure}
  %
  \\
  %
  \begin{subfigure}[t]{11cm}
    \centering
    \caption{}\vspace{0mm}\label{fig:epnpns_distance_eta}
    \includegraphics[scale=1]{epnpns_distance_eta}
  \end{subfigure}
  %

  \caption[Wall distance depend. of ion self-diff. coeff., mobilities and electrolyte viscosity]%
    {%
    \textbf{Wall distance dependency of ion self-diffusion coefficients, mobilities and electrolyte viscosity.}
      %
      (\subref{fig:epnpns_distance_d})
      %
      Relative ion self-diffusion coefficient and
      mobility~\cite{Simakov-2010,Makarov-1998,Pederson-2015,Wilson-2019}, and
      %
      (\subref{fig:epnpns_distance_eta})
      %
      inverse relative viscosity~\cite{Pronk-2014} as a function of the distance from the protein wall (left)
      and the relative residuals after fitting (right) to \cref{eq:epnpns_distance_d,,eq:epnpns_distance_eta},
      respectively. Circles represent the experimental data and solid lines the fitted equation with the gray
      shading as the $3\sigma$ confidence interval.
      %
  }\label{fig:epnpns_distance}
\end{figure*}
%


%
\begin{landscape}
  \hspace{-1.5cm}
  \begin{threeparttable}[p]
    \centering
       \captionsetup{width=19.25cm}
    \caption{Overview of the \ce{NaCl} fitting parameters used for interpolation.}
    \label{tab:corrections_parameters}
    %
    \sisetup{inter-unit-product=\ensuremath{{}\cdot{}}}
    \renewcommand{\arraystretch}{2}
    %
    \scriptsize
    \begin{tabularx}{19.25cm}{@{}
            l l S[table-format=1.3] S[table-format=-1.2(2)e-1] S[table-format=-1.2(2)e-1]
            S[table-format=-1.2(2)e-1] S[table-format=-1.2(2)e-1] S[table-format=>1.2] l @{}}
      \toprule
        & & {Inf. dil.} & \multicolumn{4}{c}{Fitting parameters} & & \\
      \cmidrule(r){3-3} \cmidrule(l){4-7} Property & Eq. & $P_{0}$ & $P_{1}$ & $P_{2}$ & $P_{3}$ & $P_{4}$ &
      $R^{2}$ & References \\
      \midrule
      $\diffusion_{\Na{}}^c(c)$ & \cref{eq:concentration_d_mu_tna} & 1.334 \tnote{a} & 2.02(14)e-1 &
      -3.05(41)e-1 & 2.19(38)e-1 & -3.13(108)e-2 & >0.99 & \cite{Mills-1989} \\
      %
      $\diffusion_{\Cl{}}^c(c)$ & \cref{eq:concentration_d_mu_tna} & 2.032 \tnote{a} & 1.49(30)e-1 &
      -4.94(904)e-2 & 3.40(826)e-2 & 1.43(230)e-2 & >0.99 & \cite{Mills-1989} \\
      %
      $\transportn_{\Na{}}^c(c)$ & \cref{eq:concentration_d_mu_tna} & 0.3963 & 9.38(164)e-2 & 2.86(324)e-3 &
      -1.88(652)e-2 & 4.51(275)e-3 & 0.98 &
      \cite{Esteso-1976,Haynes-2017,DellaMonica-1979,Panopoulos-1986,Schonert-2013}\\
      %
      $\mobility_{\Na{}}^c(c)$ & \cref{eq:concentration_d_mu_tna} & 5.192 \tnote{b} & 7.91(6)e-1 &
      -3.53(17)e-1 & 1.46(15)e-1 & 9.23(389)e-3 & >0.99 &
      \cite{Bianchi-1989,Currie-1960,Goldsack-1976,DellaMonica-1979} \\
      %
      $\mobility_{\Cl{}}^c(c)$ & \cref{eq:concentration_d_mu_tna} & 7.909 \tnote{b} & 6.29(6)e-1 &
      -4.29(17)e-1 & 2.12(14)e-1 & -1.07(37)e-2 & >0.99 &
      \cite{Bianchi-1989,Currie-1960,Goldsack-1976,DellaMonica-1979} \\
      %
      $\viscosity^c(c)$ & \cref{eq:epnpns_concentration_solv_eta} & 0.8904 \tnote{c} & 7.56(27)e-3 &
      7.77(4)e-2 & 1.19(1)e-2 & 5.95(35)e-4 & >0.99 & \cite{Hai-Lang-1996} \\
      %
      $\density^c(c)$ & \cref{eq:epnpns_concentration_solv_rho} & 0.997 \tnote{d} & 4.06(1)e-2 & -6.39(16)e-4
      & & & >0.99 & \cite{Hai-Lang-1996} \\
      %
      $\permittivity_{r,\text{f}}^c(c)$ & \cref{eq:epnpns_concentration_solv_epsr} & 78.15 & 3.08(0)e1 &
      1.15(0)e1 & & & & \cite{Buchner-1999,Gavish-2016} \\
      %
      $\diffusion_{i}^w(d)$ & \cref{eq:epnpns_distance_d} & & 6.2 & 0.01 & & & &
      \cite{Makarov-1998,Simakov-2010,Pederson-2015} \\
      %
      $\mobility_{i}^w(d)$ & \cref{eq:epnpns_distance_d} & & 6.2 & 0.01 & & & &
      \cite{Makarov-1998,Simakov-2010,Pederson-2015} \\
      %
      $\viscosity^w(d)$ & \cref{eq:epnpns_distance_eta} & & 3.36(23) & 1.47(23)e-1 & & & 0.97 &
      \cite{Pronk-2014} \\
      \bottomrule
    \end{tabularx}
    %
    \begin{tablenotes}
      \item[a] Unit and scaling: \SI{e-9}{\square\meter\per\second}
      \item[b] Unit and scaling: \SI{e-8}{\square\meter\per\second\per\volt}
      \item[c] Unit and scaling: \SI{e-4}{\pascal\second}
      \item[d] Unit and scaling: \SI{e3}{\kilo\gram\per\cubic\meter}
    \end{tablenotes}
  \end{threeparttable}
\end{landscape}
%


\subsubsection{Viscosity.}
%
Analogously to the diffusivity of ions, the movement of water molecules near a protein wall is also
hampered~\cite{Makarov-1998}, resulting in substantially higher viscosities~\cite{Pronk-2014}. To model this,
we fitted the following logistic curve to the molecular dynamics data from Pronk \etal{}~\cite{Pronk-2014}
%
\begin{align}\label{eq:epnpns_distance_eta}
  \viscosity^w (\walldistance) = 1 + \exp{ \left( -P_1 (\walldistance - P_2) \right) }
  \text{ ,}
\end{align}
%
with $P_1$ and $P_2$ the fitting parameters, representing the slope of the sigmoidal and the offset of the
sigmoid's center from 0, respectively. Prior to the fitting, the viscosity data for all proteins was offset
using by hydrodynamic radius, such that $\walldistance = \SI{0}{\nm}$ represents the mean distance from the
wall for each protein, regardless of its size. Additionally, we normalized and inverted the relative
viscosities ($\viscosity_0 / \viscosity^w$) allowing us to fit values between 0 and 1. Our parameters result
in values of $\viscosity_0 / \viscosity^w (\SI{0}{\nm}) = 0.37$ (\ie~at the protein wall) and $\viscosity_0
/ \viscosity^w (\SI{1.45}{\nm}) = 0.99$ (\ie~\SI{1.45}{\nm} within the electrolyte,
\cref{fig:epnpns_distance_eta}). The relative fitting residuals (\cref{fig:epnpns_distance_eta}, left panel)
for $\walldistance \ge \mSI{0}{\nm}$ (the utilized portion of the function) are typically
\SI{<\pm5}{\percent}.





\section{Conclusion}
%
\label{sec:epnp-ns:conclusion}
%

In this chapter, we have developed the \glsfirst{epnp-ns} equations, an improved version of the well-known
transport equations that is geared towards the accurate modeling the transport of ions and water through
biological nanopores. Our \gls{epnp-ns} equations combine many of the improvements already available in
literature, in addition to several new corrections. These include the finite size of the ions (\ie~steric
repulsion between ions), self-consistent concentration- and positional-dependent parametrization of the ionic
transport coefficients (\ie~diffusion coefficient and mobility) and of the electrolyte properties
(\ie~density, viscosity, and relative permittivity).

Most electrolyte properties depend on the salt concentration. Typically, ion self-diffusion coefficients,
ionic conductivities, and relative permittivity decrease with concentrations, whereas density and  viscosity
increase. Self-consistent parameterization of the concentration dependencies was based on interpolation
functions, fitted to experimental electrolyte data obtained from literature sources. Note that, given that our
parameters are derived from `bulk' electrolyte data, their accuracy with respect to non-electroneutral
solutions (\eg~the \gls{edl} within a nanopore) is not known. Nevertheless, we expect them to yield
significantly more physical results better compared to infinite dilution values. 

At the nanoscale, the interactions between the nanopore walls on the one hand, and the ions and water
molecules on the other, result in a short-ranged (\SI{<1}{\nm}) reduction of diffusivity and mobility of the
ions~\cite{Makarov-1998}, and an increase of the water viscosity~\cite{Pronk-2014}. We approximated these
distance-dependencies by fitting sigmoid functions to data from \gls{md} simulations, obtained from
literature. While crude, we expect them to reproduce the physically observed effects adequately.

In \cref{ch:transport}, we will apply our \gls{epnp-ns} framework to a 2D-axisymmetric model of a biological
nanopore, yielding a wealth of information that is both qualitatively and quantitatively accurate. 


%%%%%%%%%%%%%%%%%%%%%%%%%%%%%%%%%%%%%%%%%%%%%%%%%%
% Keep the following \cleardoublepage at the end of this file, otherwise \includeonly includes empty pages.
\cleardoublepage

% vim: tw=70 nocindent expandtab foldmethod=marker foldmarker={{{}{,}{}}}
