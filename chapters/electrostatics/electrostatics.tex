\chapter{Modelling the electrostatics of biological nanopores}\label{ch:electrostatics}

\section{Abstract}
\ldots

%
\section{Introduction}

Next to their geometry (\ie~shape, size and length), the fixed charge distribution can strongly influence the
transport properties of a nanopores for molecules of all size: from ions to \gls{dna} to proteins. The reason
for this that, just like ions, most molecules carry a net charge, which will interact electrostatically with
the charges present in the interior walls of the pore. For sub-nanometre pores the precise positioning of
charged residues along the length of the pore not only confers a selectivity towards ions of opposite charge,
but can even allow distinguishing between same charge ions, even if they have the same valence. For example,
the BK channel exhibits a x-fold preference to transport \ce{K+} over \ce{Na+} ions. In larger pores, the
predominant wall charges promotes and hinders the transport of respectively opposing- and same-charge ions.
When coupled with an asymmetric geometry, charged walls typically result in a higher conductance at opposing
bias voltages of the same magnitude, called ionic current rectification.

Next to directly influence the ionic current, fixed charges can also respond to changes in transmembrane
voltage, typically resulting in a closure of the channel. For example, the \gls{kcsa} channel
\cite{Kopec-2019}. This behaviour is not limited to ion channels however, and can even be engineered in larger
pores such as \gls{ahl}. In \gls{7r-ahl} where the addition of 49 ($7\times7$) positively charged residues in
the stem region cause the pore to rapidly and reversibly close under negative bias voltages
\cite{Maglia-2009}.

\section{Computational methods}
%

\subsection{Molecular modelling}
%

\subsubsection{Molecular modelling tools}
%

\subsubsection{Construction of the ClyA homology models}


\paragraph{MODELLER.}

\paragraph{ \gls{vmd} and \gls{namd}}

\subsubsection{Construction of the FraC homology models}
%

\paragraph{Wild-type \gls{frac} homology model from the \pdbid{4tsy} structure.}

\paragraph{\gls{refrac}}

\subsubsection{Construction of the PlyAB homology models}


\subsection{Electrostatic modelling}
%

\subsubsection{Computing protein electrostatic potentials with the Poisson-Boltzmann equation}
%

\gls{apbs}

\begin{figure*}[t]
  \centering
  \medskip
  %
  \begin{minipage}[t]{120mm}
    \centering
    \begin{subfigure}[t]{50mm}
      \centering
      \caption{}\vspace{-5mm}\hspace{1.5mm}\label{fig:apbs_focussing_clya_pqr}
      \includegraphics[scale=1]{apbs_focussing_clya_pqr}
    \end{subfigure}
    %
    \hspace{2mm}
    %
    \begin{subfigure}[t]{55mm}
      \centering
      \caption{}\vspace{-5mm}\hspace{1.5mm}\label{fig:apbs_focussing_clya_apbs_setup}
      \includegraphics[scale=1]{apbs_focussing_clya_apbs_setup}
    \end{subfigure}
  \end{minipage}
  %
  \\ \vspace{4mm}
  %
  \begin{minipage}[t]{120mm}
    \centering
    \begin{subfigure}[t]{50mm}
      \centering
      \caption{}\vspace{-5mm}\hspace{1.5mm}\label{fig:apbs_focussing_frac_pqr}
      \includegraphics[scale=1]{apbs_focussing_frac_pqr}
    \end{subfigure}
    %
    \hspace{2mm}
    %
    \begin{subfigure}[t]{55mm}
      \centering
      \caption{}\vspace{-5mm}\hspace{1.5mm}\label{fig:apbs_focussing_frac_apbs_setup}
      \includegraphics[scale=1]{apbs_focussing_frac_apbs_setup}
    \end{subfigure}
  \end{minipage}
  %
  \\ \vspace{4mm}
  %
  \begin{minipage}[t]{120mm}
    \centering
    \begin{subfigure}[t]{50mm}
      \centering
      \caption{}\vspace{-5mm}\hspace{1.5mm}\label{fig:apbs_focussing_plyab_pqr}
      \includegraphics[scale=1]{apbs_focussing_plyab_pqr}
    \end{subfigure}
    %
    \hspace{2mm}
    %
    \begin{subfigure}[t]{55mm}
      \centering
      \caption{}\vspace{-5mm}\hspace{1.5mm}\label{fig:apbs_focussing_plyab_apbs_setup}
      \includegraphics[scale=1]{apbs_focussing_plyab_apbs_setup}
    \end{subfigure}
  \end{minipage}
  %

\caption[APBS simulation setup.]{%
  \textbf{APBS simulation setup.}
  %
  (\subref{fig:apbs_focussing_clya_pqr})
  %
  Side view and 
  %
  (\subref{fig:apbs_focussing_clya_apbs_setup})
  %
  (\subref{fig:apbs_focussing_frac_pqr})
  %
  %
  (\subref{fig:apbs_focussing_frac_apbs_setup})
  %
  (\subref{fig:apbs_focussing_plyab_pqr})
  %
  %
  (\subref{fig:apbs_focussing_plyab_apbs_setup})
  %
  All images were prepared and rendered using VMD \cite{Humphrey-1996,Stone-1998}.
  }\label{fig:apbs_focussing}
\end{figure*}


\paragraph{Protein electrostatics with the adaptive Poisson-Boltzmann solver.}

\paragraph{Assigning atomic charges and radii with PBD2PQR.}

\paragraph{Properly accounting for solution pH with fractional protonation.}


\subsubsection{Pore-particle electrostatic interaction energies using the Poisson-Boltzmann equation}
%

\paragraph{Electrostatic energies with the adaptive Poisson-Boltzmann solver.}


\section{Results and discussion}

\subsection{Electrostatic analysis of the ClyA nanopore}
%

\begin{figure*}[t]
  \centering
  \medskip
  %

  \begin{subfigure}[t]{120mm}
    \centering
    \caption{}\vspace{-5mm}\hspace{1.5mm}\label{fig:clya_mutants_residues}
    \includegraphics[scale=1]{clya_mutants_residues}
  \end{subfigure}
  %
  \\ \vspace{-2mm}
  %
  \begin{subfigure}[t]{120mm}
    \centering
    \caption{}\vspace{-2.5mm}\hspace{1.5mm}\label{fig:clya_mutants_potential_0150}
    \includegraphics[scale=1]{clya_mutants_potential_0150}
  \end{subfigure}
  %
  \\ \vspace{0mm}
  %
  \begin{subfigure}[t]{120mm}
    \centering
    \caption{}\vspace{-2.5mm}\hspace{1.5mm}\label{fig:clya_mutants_potential_2500}
    \includegraphics[scale=1]{clya_mutants_potential_2500}
  \end{subfigure}
  %

\caption[Electrostatic potential inside ClyA mutants.]{%
  \textbf{Electrostatic potential inside ClyA mutants.}
  %
  (\subref{fig:clya_mutants_residues})
  %
  Side view and 
  %
  (\subref{fig:clya_mutants_potential_0150})
  %
  (\subref{fig:clya_mutants_potential_2500})
  %
  %
  All images were prepared and rendered using VMD \cite{Humphrey-1996,Stone-1998}.
  }\label{fig:clya_mutants}
\end{figure*}


\subsection{Electrostatic analysis of the FraC nanopore}
%

\begin{figure*}[t]
  \centering
  \medskip
  %
  \begin{subfigure}[t]{50mm}
    \centering
    \caption{}\vspace{-2.5mm}\hspace{1.5mm}\label{fig:frac_mutants_pka_scatter}
    \includegraphics[scale=1]{frac_mutants_pka_scatter}
  \end{subfigure}
  %
  \hspace{-4mm}
  %
  \begin{subfigure}[t]{70mm}
    \centering
    \caption{}\vspace{-5mm}\hspace{0.0mm}\label{fig:frac_mutants_pka_residues}
    \includegraphics[scale=1]{frac_mutants_pka_residues}
  \end{subfigure}
  %
  \\ \vspace{2mm}
  %
  \begin{subfigure}[t]{120mm}
    \centering
    \caption{}\vspace{-2.5mm}\hspace{1.5mm}\label{fig:frac_mutants_partialcharge_residues}
    \includegraphics[scale=1]{frac_mutants_partialcharge_residues}
  \end{subfigure}
  %
  \\ \vspace{-2mm}
  %
  \begin{subfigure}[t]{120mm}
    \centering
    \caption{}\vspace{-2.5mm}\hspace{1.5mm}\label{fig:frac_mutants_potential_1000}
    \includegraphics[scale=1]{frac_mutants_potential_1000}
  \end{subfigure}
  %

\caption[Electrostatic potential inside FraC mutants.]{%
  \textbf{Electrostatic potential inside FraC mutants.}
  %
  (\subref{fig:frac_mutants_pka_scatter})
  %
  (\subref{fig:frac_mutants_pka_residues})
  %
  (\subref{fig:frac_mutants_partialcharge_residues})
  %
  (\subref{fig:frac_mutants_potential_1000})
  %
  %
  All images were prepared and rendered using VMD \cite{Humphrey-1996,Stone-1998}.
  }\label{fig:frac_mutants}
\end{figure*}

\subsection{Electrostatic analysis of the PlyAB nanopore}
%


\begin{figure*}[t]
  \centering
  \medskip
  %

  \begin{subfigure}[t]{120mm}
    \centering
    \caption{}\vspace{-5mm}\hspace{1.5mm}\label{fig:plyab_mutants_residues}
    \includegraphics[scale=1]{plyab_mutants_residues}
  \end{subfigure}
  %
  \\ \vspace{-2mm}
  %
  \begin{subfigure}[t]{120mm}
    \centering
    \caption{}\vspace{-2.5mm}\hspace{1.5mm}\label{fig:plyab_mutants_potential_0300}
    \includegraphics[scale=1]{plyab_mutants_potential_0300}
  \end{subfigure}
  %
  \\ \vspace{0mm}
  %
  \begin{subfigure}[t]{120mm}
    \centering
    \caption{}\vspace{-2.5mm}\hspace{1.5mm}\label{fig:plyab_mutants_potential_2500}
    \includegraphics[scale=1]{plyab_mutants_potential_1000}
  \end{subfigure}
  %

\caption[Electrostatic potential inside PlyAB mutants.]{%
  \textbf{Electrostatic potential inside PlyAB mutants.}
  %
  (\subref{fig:plyab_mutants_residues})
  %
  Side view and
  %
  (\subref{fig:plyab_mutants_potential_0300})
  %
  (\subref{fig:plyab_mutants_potential_2500})
  %
  %
  All images were prepared and rendered with VMD \cite{Humphrey-1996,Stone-1998}.
  }\label{fig:plyab_mutants}
\end{figure*}


\subsection{Electrostatic energy analysis of dsDNA translocation through ClyA}
%

%
\subsubsection{DNA translocation at physiological ionic strength.}
%
\cref{fig:clya_mutants_dsdna_0150}

\begin{figure*}[t]
  \centering
  \medskip
  %
  \begin{subfigure}[t]{40mm}
    \centering
    \caption{}\vspace{-5mm}\hspace{1.5mm}\label{fig:clya_mutants_dsdna_0150_energy}
    \includegraphics[scale=1]{clya_mutants_dsdna_0150_energy}
  \end{subfigure}
  %
  \hspace{-3.7mm}
  %
  \begin{subfigure}[t]{81.5mm}
    \centering
    \caption{}\vspace{-5mm}\hspace{1.5mm}\label{fig:clya_mutants_dsdna_0150_potential}
    \includegraphics[scale=1]{clya_mutants_dsdna_0150_potential}
  \end{subfigure}
  %

\caption[Electrostatic energy analysis of dsDNA translocation through ClyA at physiological ionic strength.]{%
  \textbf{Electrostatic energy analysis of dsDNA translocation through ClyA at physiological ionic strength.}
  %
  (\subref{fig:clya_mutants_dsdna_0150_energy})
  %
  Net electrostatic energy ($\energyelec$, \cref{eq:electrostatic_energy}), at physiological ionic strength
  (\SI{150}{\mM}), for a \SI{51}{\bp} piece of \gls{dsdna} (\SI{\approx17.5}{\nm}) as a function of its
  $z$-position ($z_{\mathrm{DNA}}$) inside the \gls{clya} variants AS, R, RR, RR$_{56}$ and RR$_{56}$K. The
  mutation S110R eliminates the \SI{8}{\kT} energy penalty for \gls{dsdna} to enter \lumen{} of \gls{clya}
  from the \cisi{} side up until \SI{150}{\angstrom}. In contrast to D64R, which briefly promotes the
  downwards motion up of \gls{dsdna} starting from \SI{150}{\angstrom}, Q56R only manages to slow the rise of
  the energy barrier. Once inside the constriction, the Q8K mutation results in an energy drop from
  \SIrange{\approx40}{60}{\kT}.
  %
  (\subref{fig:clya_mutants_dsdna_0150_potential})
  %
  Vertical and horizontal cross-sections of the electrostatic potential ($\potential$) inside the \gls{clya}
  variants as the \gls{dsdna} begins to enter the \cisi{} entrance (\SI{200}{\angstrom}, top), reaches the
  center of the \lumen{} (\SI{150}{\angstrom}, middle) and fully entered the \transi{} constriction
  (\SI{65}{\angstrom}, bottom).
  %
  Computations were performed using \gls{apbs} \cite{Baker-2001,Baker-2005}. Molecular images were rendered
  with VMD \cite{Humphrey-1996,Stone-1998}.
  %
  }\label{fig:clya_mutants_dsdna_0150}
\end{figure*}


%
\subsubsection{DNA translocation at high ionic strength.}
%
\cref{fig:clya_mutants_dsdna_2500}

\begin{figure*}[t]
  \centering
  \medskip
  %
  \begin{subfigure}[t]{40mm}
    \centering
    \caption{}\vspace{-5mm}\hspace{1.5mm}\label{fig:clya_mutants_dsdna_2500_energy}
    \includegraphics[scale=1]{clya_mutants_dsdna_2500_energy}
  \end{subfigure}
  %
  \hspace{-3.7mm}
  %
  \begin{subfigure}[t]{81.5mm}
    \centering
    \caption{}\vspace{-5mm}\hspace{1.5mm}\label{fig:clya_mutants_dsdna_2500_potential}
    \includegraphics[scale=1]{clya_mutants_dsdna_2500_potential}
  \end{subfigure}
  %

\caption[Electrostatic energy analysis of dsDNA translocation through ClyA at high ionic strength.]{%
  \textbf{Electrostatic energy analysis of dsDNA translocation through ClyA at high ionic strength.}
  %
  (\subref{fig:clya_mutants_dsdna_2500_energy})
  %
  Net electrostatic energy ($\energyelec$, \cref{eq:electrostatic_energy}), at high ionic strength
  (\SI{2.5}{\Molar}), for a \SI{51}{\bp} piece of \gls{dsdna} (\SI{\approx17.5}{\nm}) as a function of its
  $z$-position ($z_{\mathrm{DNA}}$) inside the \gls{clya} variants AS, R, RR, RR$_{56}$ and RR$_{56}$K. Once
  inside the \transi{} constriction, $\energyelec$ fluctutates periodically with a magnitude of
  \SI{\approx2.5}{\kT} every \SI{36}{\angstrom}, which is close to the pitch of \SI{34}{\angstrom} of B-DNA.
  %
  (\subref{fig:clya_mutants_dsdna_2500_potential})
  %
  Vertical and horizontal cross-sections of the electrostatic potentials ($\potential$) inside the \gls{clya}
  variants as the \gls{dsdna} traverses the \transi{} constriction, showing the first maximum
  (\SI{60}{\angstrom}, top), first mininum (\SI{40}{\angstrom}, middle) and second maximum
  (\SI{25}{\angstrom}, bottom).
  %
  Computations were performed using \gls{apbs} \cite{Baker-2001,Baker-2005}. Molecular images were rendered
  with VMD \cite{Humphrey-1996,Stone-1998}.
  %
  }\label{fig:clya_mutants_dsdna_2500}
\end{figure*}


\ldots
\section{Conclusion}
\ldots

\section{Additional remarks}
\ldots




%%%%%%%%%%%%%%%%%%%%%%%%%%%%%%%%%%%%%%%%%%%%%%%%%%
% Keep the following \cleardoublepage at the end of this file,
% otherwise \includeonly includes empty pages.
\cleardoublepage

% vim: tw=70 nocindent expandtab foldmethod=marker foldmarker={{{}{,}{}}}
