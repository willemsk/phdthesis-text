% !TeX root = ../../thesis.tex
\chapter{The electrostatics of biological nanopores}
%
\label{ch:electrostatics}
%

\epigraphhead[\epipos]{%
\epigraph{%
%
  ``Available energy is the main object at stake in the struggle for existence and the evolution of the
  world.''
%
}{%
  \textit{`Ludwig Eduard Boltzmann'}
%
}}


%
%
\definecolor{shadecolor}{gray}{0.85}
\begin{shaded}
Parts of this chapter were adapted from:
%
\begin{itemize}
  \item K. Willems*, D. Rui\'{c}*, A. Biesemans*, N. S. Galenkamp, P. Van Dorpe and G. Maglia.
        \textit{ACS Nano} \textbf{13 (9)}, 9980--9992 (2019) %\cite{Willems-Ruic-Biesemans-2019}
  \item G. Huang, K. Willems, M. Soskine, C. Wloka and G. Maglia.
        \textit{Nat. Commun.} \textbf{8 (1)}, 1--9 (2017) %\cite{Huang-2017}
  \item L. Franceschini,  T. Brouns, K. Willems, E. Carlon and G. Maglia.
        \textit{ACS Nano} \textbf{10 (9)}, 8394--8402 (2017) %\cite{Franceschini-2016}
  \item G. Huang, K. Willems, M. Bartelds, P. Van Dorpe, M. Soskine and G. Maglia.
        \textit{Nano Lett.} \textbf{20 (50)}, 3819--3827 (2020) %\cite{Huang-2020}
  \item M. Bayoumi, S. Nomidis, K. Willems, E. Carlon and G. Maglia.
        \textit{XXXX} \textbf{X (X)}, XX--XX (2020) %\cite{Bayoumi-2020}
\end{itemize}
%
*equal contributions
%
\newpage
\end{shaded}
%
%

% Glossary reset
\glsresetall
%


In this chapter, the electrostatic properties of the \gls{plyab}, \gls{frac} and \gls{clya} nanopores are
investigated \textit{in silico} using the \gls{pbe}, corroborating results published in previous studies. We
demonstrate the extent with which their electrostatic potentials can be manipulated by modification of
specific amino acids of the pore on the one hand, and by altering the ionic strength and the pH of the
electrolyte on the other. Additionally we use electrostatic energy computations to analyze the translocation
of \gls{ssdna} and \gls{dsdna} through respectively \gls{frac} and \gls{clya} nanopores. \\
%

%
The text and figures of this chapter represent entirely my own work. Several parts were adapted from
co-authored papers.
%
The section
%
``Construction of the PlyAB homology models''
%
was adapted from ref.~\cite{Huang-2020}.
%
%
The sections
%
``Computing the electrostatic potentials within nanopores''
%
and
%
``Computing nanopore-particle electrostatic interaction energies''
%
were adapted from ref.~\cite{Willems-Ruic-Biesemans-2019}.
%
%
The paragraph
%
``Partial protonation: partial charges at pH-values close to the $\pKa$.''
%
and the section
%
``A link between the electrostatic potential, ion-selectivity and electro-osmotic flow.''
%
were adapted from ref.~\cite{Huang-2017}.
%
The section
%
``All charges are important at physiological ionic strength.''
%
was adapted from ref.~\cite{Franceschini-2016}.
%
%
The section
%
``Electrostatic confinement of {dsDNA} within {ClyA}''
%
was adapted from ref.~\cite{Bayoumi-2020}.
%

%
%
% \cleardoublepage
%
%





%
\section{Introduction}
%
\label{sec:elec:intro}
%

As one of the four fundamental forces of nature, electromagnetism dominates the behavior and properties of
virtually all physical phenomena beyond the size of the atomic nucleus (aside from gravity). At the
macroscale, electromagnetism powers almost all of our technology, whereas at the nanoscale it powers life
itself: it is involved in all chemical phenomena, from the properties of salty solutions to intra- and
inter-molecular interactions and chemical reactions. The electromagnetic force is also highly versatile: it
can either be attractive or repulsive, short- or long-ranged, directional or omnidirectional, and weak or
strong~\cite{Walker-2011}. Here, our interest lies with the electrostatic component of the electromagnetic
force (\ie~the static interaction between fixed and mobile charges) and the electrostatic potential and
energies that result from it.

For a biological nanopore, the fixed charge distribution lining its interior walls, next to its geometry,
dominates the transport properties for both ions and many larger molecules. Direct interactions between the
charges of the nanopore and the ions in the surrounding electrolyte lead to observable phenomena such as the
ion selectivity, ionic current rectification and electro-osmotic flow. Indirect interactions (\ie~mediated
through the external electric field) can result in voltage- or pH-dependent gating. For example, the direct
electrostatic interactions between ions and the precisely positioned charges within the sub-nanometer aperture
of the BK ion channel confers it not only with a selectivity towards ions of the opposite charge, but even
allows distinguishing between ions of the \emph{same} charge and valence. In larger pores, the predominant
charges on the wall either promote or hinder the transport of respectively opposing- and same-charge ions.
When coupled with an asymmetric geometry, charged walls typically result in a higher conductance at opposing
bias voltages of the same magnitude, a phenomenon called ionic current rectification. Charges that are not
directly exposed to the electrolyte may also respond to changes in the electrostatic field by inducing
conformational changes. These typically result in either an opening or closure of the channel, as is the case
in the \gls{kcsa} channel~\cite{Kopec-2019}. This behavior is not limited to ion channels however, and can
even be engineered in larger pores. The \gls{ahl} variant \gls{7r-ahl}---in which 49 ($7\times7$) positively
charged residues were added in the stem region---rapidly closes and opens under negative and positive bias
voltages, respectively~\cite{Maglia-2009}.

% TODO: add more here, but what?
% Add section for larger molecules, EOF?
%~\cite{Zhou-2020}

In this chapter, we have investigated the electrostatic properties of the several experimentally relevant
variants of the biological nanopores \gls{plyab}, \gls{frac} and \gls{clya} by numerically solving the
\gls{pbe} for their atomistic models. After describing our computational methods, we begin with an analysis of
the electrostatic potential distribution within the wild-type \gls{plyab}, \gls{plyab-e2} and \gls{plyab-r},
at \SIlist{0.3;1.0}{\Molar} ionic strength. These two pores have oppositely charged constrictions and showed
different protein capture characteristics in experiments. These could be attributed to a significantly reduced
\gls{eof} in \gls{plyab-r}~\cite{Huang-2020}, and are supported by our computed electrostatic potentials.

Next, we switch to the much smaller \gls{wtfrac} (wild-type) and \gls{refrac} pores, who also have oppositely
charged constrictions and were used in two separate studies. In the first~\cite{Huang-2017}, the opposing
response of the \gls{eof} strength to lowering of the electrolyte pH from \numrange{7.5}{4.5} revealed that
the capture of peptides by \gls{frac} is driven primarily by the \gls{eof}. In the second~\cite{Wloka-2016},
it was shown that, unlike its wild-type counterpart, \gls{refrac} was able to translocate both \gls{ssdna} and
\gls{dsdna}. We show that the pH-dependency of the \gls{eof} is in agreement with our computations, and that
the energy barrier for \gls{ssdna} translocation is significantly lowered for \gls{refrac}.

Lastly, we turn to \gls{clya}, for which we analyzed several variants of \gls{clya-as} at ionic strengths of
\SIlist{0.15;2.5}{\Molar}. These mutants showed a varying ability to translocate
\gls{dsdna}~\cite{Franceschini-2016}, a result corroborated by our electrostatic potential and \gls{dsdna}
translocation electrostatic energy calculations. Finally, we show that the strong repulsion between the
negatively charged constriction and the \gls{dna} confines the latter to the center of the
pore~\cite{Bayoumi-2020}.


% Adjusting and finetuning the electrostatic interactions between the nanopore and Enhanced \gls{dna}
% translocation by manipulation of charges\cite{Maglia-2008}

% \cite{Mohammad-2008} Controlling a Single Protein in a Nanopore through Electrostatic Traps 

% Tuning peptides capture dynamics by aHL by changing the electrostatics\cite{Asandei-2015b}

% Enhancement of electro-osmotic flow by lowering pH \cite{Asandei-2016}

% Electrostatic trap\cite{Buchsbaum-2013}

% % at the direct interaction between the nanopore and analyte surface, induced by the electrostatic
% attraction between the two molecules, is essential for protein isoform detection \cite{Fahie-2015b}


%
%
\clearpage
%
%


%
%
\section{Computational methods}
%
\label{sec:elec:methods}
%

\subsection{Molecular modeling}
%
\label{sec:elec:methods:molec}

\subsubsection{Construction of the PlyAB homology models}
%
% Adapted from Huang-2020

\paragraph{Note on the \gls{plyab}, \gls{plya} and \gls{plyb} structures.}
%
As described in \cref{sec:np:plyab}, the \Gls{plyab} nanopore consist of 39 individual protein chains (26
\gls{plya} and 13 \gls{plyb} chains), arranged in an \ce{(BA2)13} configuration with a 13-fold symmetry
(\cref{fig:nanopores_plyab}). Even though a \gls{cryo-em} structure of the full pore complex is available
(\pdbid{4V2T}~\cite{Lukoyanova-Kondos-2015}), it only contains the C\ta{} atoms of the entire protein
backbone. Fortunately, the same authors also released full-atom crystal structures of the \gls{plya}
(\pdbid{4OEB}~\cite{Lukoyanova-Kondos-2015}) and \gls{plyab} (\pdbid{4OEJ}~\cite{Lukoyanova-Kondos-2015})
monomers in their water-soluble conformations, which can be used, together with the \gls{cryo-em} map and the
C\ta{} atoms of the full pore, to create a full atom model of \gls{plyab} \textit{via} constrained homology
modeling and \gls{mdff}.

\paragraph{Construction of an artificial full-atom `prepore' complex.}
%
First, an artificial `prepore' full atom structure was constructed by aligning 13 copies of the soluble
\gls{plyb} monomer and 26 copies of the \gls{plya} monomer to the C\ta{} atoms of the \gls{cryo-em} structure
(with the PyMOL \code{align} command~\cite{PyMOL}), resulting in \gls{rmsd} values of \SI{0.126}{\angstrom}
(\gls{plya} to subunit A1), \SI{0.081}{\angstrom} (\gls{plya} to subunit A2) and \SI{1.371}{\angstrom}
(\gls{plyb} to subunit B). Hence, the backbone conformations of the water-soluble and pore for both \gls{plya}
and \gls{plyb} match very closely, with the exception of the transmembrane region of \gls{plyb} (\ie~residues
\numrange{105}{171} and \numrange{229}{296}).

\paragraph{Homology modeling of the full-atom pore structure.}
%
The prepore complex is a better start for building a full-atom homology model than the \gls{cryo-em}
structure, given that the former contains realistic locations for all of the side-chains in the \gls{plya}
subunits, and for the bulk of the residues in \gls{plyb} protomers (\ie~everything except the pore-forming
region). The full sequence of \gls{plyab}, together with the positional information given by the all the
C\ta{} atoms in the \gls{cryo-em} structure and that of the pre-pore, barring those residues involved in the
transmembrane region, were then used as input for MODELLER \code{automodel} algorithm~\cite{Sali-1993}.
MODELLER produced five full-atom homology models, of which the best one (\ie~the one with the lowest DOPE
score) was fitted to the PlyAB cryo-EM map using simulated annealing molecular dynamics optimization as
implemented in the Flex-EM~\cite{Topf-2008} extension of MODELLER. This entailed a \gls{md} simulated
annealing, constrained by the \gls{cryo-em} density map and a maximum shift of \SI{0.39}{\angstrom}, of 100
steps at \SIlist{150;250;500;1000}{\kelvin}, followed by 200 steps at \SIlist{800;500;250;150;50;0}{\kelvin}
and two conjugate gradients energy minimizations of 200 steps with and without density constraints. Further
refinement of the structure was carried out using symmetry constrained \gls{mdff} (implicit solvent) to the
\gls{cryo-em} density map with \gls{namd}~\cite{Phillips-2005,McGreevy-2014} for \SI{5}{\ns}, followed by
5~cycles of simulated annealing (\ie~heating from \SIrange{25}{350}{\kelvin} and cooling from
\SIrange{350}{300}{\kelvin} in steps of \SI{10}{ps} per \SI{25}{\kelvin}) and a \SI{5}{\ns} equilibration at
\SI{300}{\kelvin} with light (force constant $k = \mSI{0.1}{\kilo\cal\per\mole\per\square\angstrom}$)
harmonic constraints on the C\ta{} atoms.

\paragraph{Construction of the \gls{plyab-e2} and \gls{plyab-r} mutants.}
%
The homology models of \gls{plyab-e2} (\gls{plya}: C62S, C94S; \gls{plyb}: N26D, N107D, G218R, A328T, C441A,
A464V) and \gls{plyab-r} (\gls{plya}: C62S, C94S; \gls{plyb}: N26D, K255E, E260R, E270R, A328T, C441A, A464V)
were created from the homology model of the wild type structure with a custom \gls{tcl} script and \gls{vmd}'s
\code{psfgen} plugin~\cite{Humphrey-1996}. Finally, the energy of each mutant was briefly minimized with an
\gls{md} run (implicit solvent, CHARMM36 force field~\cite{Best-2012}) of \SI{100}{\ps} using \gls{namd}, with
harmonic constraints on the C\ta{} atoms (force constant $k =
\mSI{1}{\kilo\cal\per\mole\per\square\angstrom}$).


\subsubsection{Construction of the FraC homology models}
%

\paragraph{On the \gls{frac} crystal structure.}
%
Next to the protein chains, the crystal structure of \gls{frac} (\pdbid{4TSY}~\cite{Tanaka-2015}) contains
also several fragments of the lipid \gls{sph} that form structural part of the pore itself
(\cref{sec:np:frac,fig:nanopores_frac}). Given that their charged headgroups are positioned outside of the
\lumen{} of the pore, we do not expect these lipids to contribute significantly to its electrostatics.
Therefor, we opted not to include these in our homology models and simulations.

\paragraph{Construction of the \gls{wtfrac} and \gls{refrac} mutants.}
%
Similarly to the \gls{plyab} mutants, the \gls{refrac} mutant (D10R, K159E) was constructed from the crystal
structure using \gls{vmd}'s \code{psfgen} plugin~\cite{Humphrey-1996}. \Gls{wtfrac} required no additional
mutations, but its missing atoms where also added with \code{psfgen}. Both structures were minimized with an
\gls{md} run (implicit solvent, CHARMM36 force field~\cite{Best-2012}) of \SI{100}{\ps} using \gls{namd}, with
harmonic constraints on the C\ta{} atoms (force constant $k =
\mSI{1}{\kilo\cal\per\mole\per\square\angstrom}$).


\subsubsection{Construction of the ClyA homology models}
%

\paragraph{On the \gls{clya} crystal structure.}
%
The dodecameric \textit{E. coli} \gls{clya} crystal structure (\pdbid{2WCD}~\cite{Mueller-2009}) contains all
residues, aside from short fragments at the N- (residues \numrange{1}{7}) and C-termini (residues
\numrange{293}{303}) of each chain. Those at the C-terminus are located outside of the \lumen{} of pore, and
hence not expected to contribute significantly to the electrostatics. The missing residues at the N-terminus
are more import however, given their location at the \transi{} entry (\cref{fig:nanopores_clya}). Hence, we
opted to add a single negatively charged amino acid (Glu 7) in our models.

\paragraph{Construction of the \gls{clya-as} mutant.}
%
As with the previous pore, the wild-type crystal structure was processed with \code{psfgen} to add any missing
atoms and converted to \gls{clya-as} by introduction of the relevant mutations (Q8K, N15S, Q38K, A57E, T67V,
C87A, A90V, A95S, L99Q, E103G, K118R, L119I, I124V, T125K, V136T, F166Y, K172R, V185I, K212N, K214R, S217T,
T224S, N227A, T244A, E276G, C285S and K290Q). Next, a 200-by-\SI{200}{\angstrom} \gls{dphpc} lipid bilayer
patch was created with the CHARMM-GUI membrane builder~\cite{Jo-2008,Lee-2015}, and any lipid residues whose
headgroups were location either inside the pore \lumen{} or protein core were physically removed, followed by
a \SI{100}{\ps} steered \gls{md} run in \gls{namd}, using the mass-density of \gls{clya} as a virtual
repulsive force, to push any lipid tails out of the protein core. The resulting bilayer was then merged with
the \gls{clya-as} structure, solvated with 213364 {TIP3P} water molecules (\gls{vmd}'s \code{solvate} plugin)
and neutralized to \SI{0.15}{\Molar} with 674 \ce{Na+} and 602 \ce{Cl-} ions (\gls{vmd}'s \code{autoionize}
plugin). The resulting system was then equilibrated with \gls{md} (NPT ensemble with a Langevin dynamics
thermostat and a Nos\'{e}-Hoover Langevin barostat, CHARMM36 force field parameters~\cite{Best-2012}) for
\SI{5}{\ns} at \SI{298.15}{\kelvin} while constraining the proteins C\ta{} atoms harmonically (force constant
$k = \mSI{1}{\kilo\cal\per\mole\per\square\angstrom}$).

\paragraph{Construction of the \gls{clya-r}, \gls{clya-rr}, \gls{clya-rr56} and \gls{clya-rr56k} mutants.}
%
The other \gls{clya} mutants where constructed from the \gls{clya-as} system described above by introduction
of the relevant mutants with \code{psfgen}: \gls{clya-r} (S110R), \gls{clya-rr} (S110R/D64R), \gls{clya-rr56}
(S110R/Q56R) and \gls{clya-rr56k} (S110R/Q56R/Q8K). The resulting systems were equilibrated with \gls{md} for
\SI{1}{\ns} using the same conditions are those described above for \gls{clya-as}, and the coordinates of the
pores isolated in individual files.


\subsection{Electrostatic modeling}
%
\label{sec:elec:methods:elec}

\subsubsection{Computing the electrostatic potentials within nanopores}
%
% Adapted from Willems-Ruic-Biesemans-2019

\begin{figure*}[p]
  \centering
  \medskip
  %
  \begin{minipage}[t]{50mm}
    \centering
    \begin{subfigure}[t]{50mm}
      \centering
      \caption{}\vspace{-5mm}\hspace{1.5mm}\label{fig:apbs_focussing_plyab_pqr}
      \includegraphics[scale=1]{apbs_focussing_plyab_pqr}
    \end{subfigure}
    %
    \\ \vspace{2mm}
    %
    \centering
    \begin{subfigure}[t]{50mm}
      \centering
      \caption{}\vspace{-5mm}\hspace{1.5mm}\label{fig:apbs_focussing_frac_pqr}
      \includegraphics[scale=1]{apbs_focussing_frac_pqr}
    \end{subfigure}
    %
    \\ \vspace{2mm}
    %
    \centering
    \begin{subfigure}[t]{50mm}
      \centering
      \caption{}\vspace{-5mm}\hspace{1.5mm}\label{fig:apbs_focussing_clya_pqr}
      \includegraphics[scale=1]{apbs_focussing_clya_pqr}
    \end{subfigure}
    %
  \end{minipage}
  %
  \hspace{2mm}
  %
  \begin{minipage}[t]{55mm}
    \centering
    %
    \begin{subfigure}[t]{55mm}
      \centering
      \caption{}\vspace{-5mm}\hspace{1.5mm}\label{fig:apbs_focussing_plyab_apbs_setup}
      \includegraphics[scale=1]{apbs_focussing_plyab_apbs_setup}
    \end{subfigure}
    %
    \\ \vspace{2mm}
    %
    \begin{subfigure}[t]{55mm}
      \centering
      \caption{}\vspace{-5mm}\hspace{1.5mm}\label{fig:apbs_focussing_frac_apbs_setup}
      \includegraphics[scale=1]{apbs_focussing_frac_apbs_setup}
    \end{subfigure}
    %
    \\ \vspace{2mm}
    %
    \begin{subfigure}[t]{55mm}
      \centering
      \caption{}\vspace{-5mm}\hspace{1.5mm}\label{fig:apbs_focussing_clya_apbs_setup}
      \includegraphics[scale=1]{apbs_focussing_clya_apbs_setup}
    \end{subfigure}
  \end{minipage}

\caption[APBS simulation setup.]{%
  \textbf{APBS simulation setup.}
  Top-side views of the PQR files of  
  %
  (\subref{fig:apbs_focussing_plyab_pqr})
  %
  \gls{plyab},
  %
  (\subref{fig:apbs_focussing_frac_pqr})
  %
  \gls{frac}, and
  %
  (\subref{fig:apbs_focussing_clya_pqr})
  %
  \gls{clya}. Molecules are represented as \gls{vdw} spheres and colored according to their partial charge
  $\partialcharge$. The area of a single subunit is highlighted with a green wedge.
  %
  Axial cross-sections of the relative permittivity (top) and calculated electrostatic potential (bottom)
  grids for
  %
  (\subref{fig:apbs_focussing_plyab_apbs_setup})
  %
  \gls{plyab},
  %
  (\subref{fig:apbs_focussing_frac_apbs_setup})
  %
  \gls{frac}, and
  %
  (\subref{fig:apbs_focussing_clya_apbs_setup})
  %
  \gls{clya}. The \gls{apbs} focussing approach uses two sequential calculations, `coarse' and `fine', to
  obtain a realistic potential distribution.
  %
  Molecular representations were rendered using \gls{vmd}~\cite{Humphrey-1996,Stone-1998}.
  }\label{fig:apbs_focussing}
\end{figure*}

\paragraph{The Poisson-Boltzmann equation.}
%
As described in \cref{sec:np:edl}, the electrostatic potential distribution of a protein dissolved in an
electrolyte medium can be estimated by solving the \gls{pbe}~\cite{Baker-2001,Baker-2005}
%
\begin{equation}\label{eq:pbe}
  - \nabla \cdot \left[ \permittivity (\rpos) \nabla \potential (\rpos) \right] = \scd_{\rm{c}} (\rpos)
  \text{ ,}
\end{equation}
%
with $\rpos$ the location vector, $\potential$ the electrostatic potential (normalized over the thermal
voltage \si{\kbt\per\ec}), $\permittivity$ the local permittivity, and $\scd_{\rm{c}}$ the charge density. For
a molecular system in an electrolyte, $\scd_{\rm{c}}$ can be split into two parts
%
\begin{equation}
  \scd_{\rm{c}} (\rpos) = \scd^{\rm{f}}(\rpos) + \scd^{\rm{m}}(\rpos)
  \text{ ,}
\end{equation}
%
with $\scd^{\rm{f}}$ the fixed charge density---resulting from the distribution of to the atomic partial
charges---and $\scd^{\rm{m}}$ the mobile part charge density---resulting from the distribution of charged ions
in the electrolyte. For $M$ atomic partial charges $\scd^{\rm{f}}$ is given by~\cite{Baker-2001,Baker-2005}
%
\begin{equation}
  \scd^{\rm{f}}(\rpos) = \dfrac{4\pi\ec^2}{\kbt} \sum_{i=1}^{M} Q_i \delta(\rpos - \vec{r_i})
\end{equation}
%
with $\ec$ the elementary charge, $\kbt$ the thermal energy, $\delta$ the Dirac delta function. $Q_i$ and
$\vec{r_i}$ represent the atomic partial charge and the location of atom $i$, respectively. The charge density
due to $N$ mobile charge species can be expressed as~\cite{Baker-2001,Baker-2005}
%
\begin{equation}
  \scd^{\rm{m}}(\rpos) =
    \dfrac{4\pi\ec^2}{\kbt}
      \sum_{j=1}^{N} c_j q_j
      \exp{\left[- q_j \potential(\rpos) - V_j(\rpos) \right]}
\end{equation}
%
with $c_j$ the bulk concentration, $q_j$ the charge number and $V_j(\rpos)$ the steric potential of ion
species $j$. For a monovalent salt such as \ce{NaCl} this expression can be reduced to
%
\begin{equation}
    \scd^{\rm{m}} (\rpos) = - \ionaccess^{-2} (\rpos) \sinh \left[ \potential (\rpos) \right]
\end{equation}
%
with $\ionaccess$ a coefficient that includes both the ion accessibility and the bulk ion concentration.

\paragraph{Solving the \gls{pbe} with \gls{apbs} and \gls{pdb2pqr}.}
%
The \gls{pbe} solver \gls{apbs}~\cite{Jurrus-2018,Baker-2001} requires several inputs regarding the nanopore
(\ie~geometry, fixed charge distribution, relative permittivity) and the electrolyte medium (\ie~ion
concentration, ion size and relative permittivity) and the computational system itself (\ie~grid size, grid
spacing, boundary conditions). These are provided by the user in two input files: (1) a configuration file
that contains all user-defined properties and (2) a ``PQR'' file that contains a coordinate, radius and
partial charge for every atom in the system. The PQR format is similar to the PDB format, and can be generated
from a PDB file using the \gls{pdb2pqr} software package~\cite{Jurrus-2018,Dolinsky-2004,Dolinsky-2007}. We
used the atom radius and partial charge parameters of the \gls{charmm36} force field~\cite{Huang-2013} for all
calculations except the \gls{frac} only pore, where we used the \gls{parse} force
field~\cite{Sitkoff-1994,Sitkoff-1996}. During conversion, \gls{pdb2pqr} assigns a protonation states to all
titratable residues at any user-specified pH value using its \gls{propka} submodule~\cite{Li-2005}.

To improve their accuracy, \gls{apbs} simulations typically consist of a initial `coarse' calculation
(\ie~with grid lengths \SI{>1}{\angstrom}) on a large grid size and with Dirichlet boundary conditions
($\potential = 0$ or $\potential = \potential^{\code{mdh}}$). This is followed by a second `fine' calculation
(\ie~with grid lengths \SI{\approx0.5}{\angstrom}) on a smaller grid size that uses the potentials obtained in
the coarse simulation as a Dirichlet boundary condition ($\potential = \potential^{\mathrm{coarse}}$). The
\code{mdh} superscript refers to the multiple Debye-H\"{u}ckel approach, where approximate values are obtained
analytically with a Debye-H\"{u}ckel model for a multiple, non-interacting spheres containing point
charges~\cite{Baker-2001,Baker-2005,Stone-2010}. The full list of simulation parameters can be found in
\cref{tab:pdb2pqr_apbs_parameters} and the simulation setups in \cref{fig:apbs_focussing}.


\begin{landscape}
\begin{threeparttable}[p]
  \footnotesize
  \centering
  
  %
  \captionsetup{width=18cm}
  \caption{Summary of the {PDB2PQR} and {APBS} input parameters.}
  \label{tab:pdb2pqr_apbs_parameters}
  %

  \renewcommand{\arraystretch}{1.2}
  \scriptsize

  \begin{tabularx}{18cm}{%
      >{\hsize=1cm}l%
      >{\hsize=0.5cm}X%
      >{\hsize=0.5cm}X%
      >{\hsize=2.3cm}X%
      >{\hsize=2.3cm}X%
      >{\hsize=2.3cm}X%
      >{\hsize=2.3cm}X%
      >{\hsize=2.3cm}X%
    }
    %
    \toprule
    \multicolumn{2}{c}{Parameter} & & \multicolumn{5}{c}{System} \\
    \cmidrule(r){1-2}\cmidrule(l){4-8}
    %
    Name/Symbol & Unit & Dom.\tnote{a}
      & PlyAB & FraC & FraC\,+\,ssDNA & ClyA &  ClyA\,+\,dsDNA \\
    \midrule
    %
    %
    \multicolumn{8}{l}{\textbf{PDB2PQR}} \\[1mm]
    %
    pH\tnote{b} & 1 
      & n.a.
      & 7.5 & 4.5, 7.5 & 7.5 & 7.5 & 7.5 \\
    Force-field & n.a. 
      & n.a.
      & CHARMM & PARSE & CHARMM & CHARMM & CHARMM \\
    %
    %
    \midrule
    \multicolumn{8}{l}{\textbf{APBS}} \\[1mm]
    %
    $\temperature$ & \si{\kelvin}
      & G      
      & 298.15  & 298.15 & 298.15 & 298.15 & 298.15  \\
    %
    $\permittivity_{r,\text{electrolyte}}$ & 1 
      & E
      & 80 & 80 & 78.15 & 80 & 78.15 \\
    %
    $\permittivity_{r,\text{protein}}$ & 1
      & P
      & 10 & 10 & 10 & 10 & 10 \\
    %
    $\permittivity_{r,\text{bilayer}}$ & 1
      & B
      & 2  & 2 & n.a. & 2 & n.a. \\
    %
    $\ionsize_{\pm}$ & \si{\angstrom}
      & E
      & 4 & 4 & 4 & 4 & 4 \\
    %
    $\chargen_{\pm}$ & 1
      & E
      & $\pm1$ & $\pm1$ & $\pm1$ & $\pm1$ & $\pm1$\\
    %
    $\concentration_{\pm}$ & \si{\Molar}
      & E
      & 0.3, 1.0 & 1.0 & 0.15, 2.5 & 1.0 & 0.15, 2.0, 2.5 \\
    %
    $g_{x}^{\mathrm{c}} \times g_{y}^{\mathrm{c}} \times g_{z}^{\mathrm{c}}$ & \si{\angstrom}
      & G
      & \apbsgrid{898}{898}{642}{} % PlyAB
      & \apbsgrid{642}{642}{642}{} % FraC
      & \apbsgrid{400}{400}{900}{} % FraC + ssDNA
      & \apbsgrid{834}{837}{898}{} % ClyA
      & \apbsgrid{400}{400}{1100}{} \\ % ClyA + dsDNA
    %
    $\Delta g_{x}^{\mathrm{c}} \times \Delta g_{y}^{\mathrm{c}} \times \Delta g_{z}^{\mathrm{c}}$ & \si{\angstrom}
      & G
      & \apbsgrid{2.00}{2.00}{2.00}{} % PlyAB
      & \apbsgrid{2.00}{2.00}{2.00}{} % FraC
      & \apbsgrid{1.25}{1.25}{0.74}{} % FraC + ssDNA
      & \apbsgrid{2.00}{2.00}{2.00}{} % ClyA
      & \apbsgrid{1.38}{1.38}{0.82}{} \\ % ClyA + dsDNA
    %
    $g_{x}^{\mathrm{f}} \times g_{y}^{\mathrm{f}} \times g_{z}^{\mathrm{f}}$ & \si{\angstrom}
      & G
      & \apbsgrid{225}{225}{209}{} % PlyAB
      & \apbsgrid{161}{161}{161}{} % FraC
      & \apbsgrid{150}{150}{600}{} % FraC + ssDNA
      & \apbsgrid{209}{209}{225}{} % ClyA
      & \apbsgrid{150}{150}{700}{} \\ % ClyA + dsDNA
    %
    $\Delta g_{x}^{\mathrm{f}} \times \Delta g_{y}^{\mathrm{f}} \times \Delta g_{z}^{\mathrm{f}}$ & \si{\angstrom}
      & G
      & \apbsgrid{0.50}{0.50}{0.50}{} % PlyAB
      & \apbsgrid{0.50}{0.50}{0.50}{} % FraC
      & \apbsgrid{0.47}{0.47}{0.49}{} % FraC + ssDNA
      & \apbsgrid{0.50}{0.50}{0.50}{} % ClyA
      & \apbsgrid{0.52}{0.52}{0.52}{} \\ % ClyA + dsDNA
    %
    $\potential_{\mathrm{boundary}}^{\mathrm{coarse}}$ & \si{\volt}
      & G
      & $\potential = 0$
      & $\potential = 0$
      & $\potential = \potential^{\code{mdh}}$
      & $\potential = 0$
      & $\potential = \potential^{\code{mdh}}$ \\
    %
    $\potential_{\mathrm{boundary}}^{\mathrm{fine}}$ & \si{\volt}
      & G
      & $\potential = \potential^{\mathrm{coarse}}$
      & $\potential = \potential^{\mathrm{coarse}}$
      & $\potential = \potential^{\mathrm{coarse}}$
      & $\potential = \potential^{\mathrm{coarse}}$
      & $\potential = \potential^{\mathrm{coarse}}$ \\
    %
    %
    \midrule
    \multicolumn{8}{l}{\textbf{\code{draw\_membrane2}}} \\[1mm]
    %
    $d_{\textrm{bilayer}}$ & \si{\angstrom}
      & B
      & 27 & 27 & 27 & n.a. & n.a. \\
    %
    $r_{\textrm{bilayer}}^{\textrm{top}}$ & \si{\angstrom}
      & B
      & 43 & 23 & 37 & n.a. & n.a. \\
    %
    $r_{\textrm{bilayer}}^{\textrm{bottom}}$ & \si{\angstrom}
      & B
      & 43 & 12 & 18 & n.a. & n.a. \\
    \bottomrule
  \end{tabularx}

  \begin{tablenotes}
   \item[a] Domains: global (G), electrolyte (E), protein/DNA (P), lipid bilayer (B);
   \item[b] pH values used by \gls{propka} during \gls{pdb2pqr} runtime.
  \end{tablenotes}

\end{threeparttable}
\end{landscape}
%

\paragraph{Adding a lipid bilayer to APBS calculations.}
%
Being membrane proteins, active biological nanopores are embedded in a lipid bilayer and hence not surrounded
by electrolyte medium everywhere. From an electrostatic point of view, a lipid bilayer can be modeled as an
uncharged and ion-impermeable dielectric slab with a low permittivity: this is indeed very different from an
ion-containing electrolyte medium with high permittivity. Hence, it is not surprising that the presence of a
lipid bilayer influences the distribution of the electrostatic potential~\cite{Homeyer-2015}. Even though
\Gls{apbs} does not natively support the inclusion of a lipid bilayer, the software package does include an
external program (\code{draw\_membrane2}) that allows adding one to the permittivity, ion-accessibility and
charge maps produced by \gls{apbs} in the form of a dielectric slab with a conical hole at its center. The
relevant parameters (thickness $d_{\textrm{bilayer}}$, top cone radius $r_{\textrm{bilayer}}^{\textrm{top}}$
and bottom cone radius $r_{\textrm{bilayer}}^{\textrm{bottom}}$) are listed in
\cref{tab:pdb2pqr_apbs_parameters}. The use of \code{draw\_membrane2} is a very I/O intensive task, as it
requires \gls{apbs} to write out a set of 5 3D grids (typically \SI{>2}{\gibi\byte} each), for every
individual simulation (\ie~both the coarse \emph{and} fine runs) to disk. Then, they are loaded into memory by
\code{draw\_membrane2}, which adds in the bilayer and finally writes them out to the disk again, where they
can be accessed by \gls{apbs}. This entire process requires large amounts of disk space (when many running
simulations in parallel) and is rapidly I/O bottlenecked, significantly slowing down a single simulation from
\SIrange{20}{200}{\min}. Even though presence of a bilayer has an impact on the electrostatic
energy~\cite{Bonthuis-2006}, its inclusion in \gls{apbs} comes at a significant computational cost and its
absence does not influence the \emph{qualitative} outcome of our energy comparisons. Hence, this approach was
only when computing the electrostatic potential inside the pore-only systems (\num{\pm20} runs), and the
bilayer was omitted entirely for all energy-based calculation of particle translocation (\num{>500} runs).  


\paragraph{Partial protonation: partial charges at pH-values close to the p$\boldsymbol{K}_{\mathbf{a}}$.}
%
% Adapted from Huang-2017
%
When the pH of the solution is close to the $\pKa$ of a titratable residue (\ie~within \num{\pm2}~pH units),
the residue will, on average, be neither fully protonated or deprotonated. The protonated fraction of an acid
($\protfrac_{\mathrm{HA}}$) is given by 
%
\begin{align}\label{eq:pka_protonation}
  \protfrac_{\mathrm{HA}} = \left[ 1 + 10^{\mathrm{pH} - \pKa} \right]^{-1}
  \text{ .}
\end{align}
%
Hence, when $\mathrm{pH} = \pKa$, the residue will be protonated \SI{50}{\percent} of the time and
deprotonated for other \SI{50}{\percent}. This also means that, on average, the partial charge of the residue
($\partialcharge$) lies somewhere between \SIrange{-1}{+1}{\ec}:
%
\begin{align}\label{eq:pka_partial_charge}
  \partialcharge = \partialcharge_{\mathrm{HA}} \protfrac_{\mathrm{HA}}
                    + \partialcharge_{\mathrm{A}^{-}} \left( 1 - \protfrac_{\mathrm{HA}} \right)
  \text{ ,}
\end{align}
%
with $\partialcharge_{\mathrm{HA}}$ and $\partialcharge_{\mathrm{A}^{-}}$ the partial charge of the fully
protonated and deprotonated residue, respectively. \gls{pdb2pqr} does not take into account partial
(de)protonation, and assigns partial charges in a discrete, all-or-nothing manner. This mean that acid
residues (ASP, GLU, TYR, C-terminus) have a charge of \SI{-1}{\ec} if $\mathrm{pH} > \pKa$, and \SI{0}{\ec}
otherwise. Likewise, basic residues (ARG, LYS, HIS, N-terminus) have a charge of \SI{+1}{\ec} if $\mathrm{pH}
< \pKa$, and \SI{0}{\ec} otherwise. Because the side-chain $\pKa$'s of (free) amino acids are either below 6
or above 9, this is a reasonable approximation for simulations at pH values close to 7. However, it does not
adequately represent the equilibrium situation for lower (or higher) pH values. For example, \pH{4.5} is very
close to the $\pKa$ values of glutamate (\num{4.4}) and aspartate (4.0) residues, and most of these residues
would hence be fractionally charged. In addition, the proximity of many charged residues tends to shift their
effective $\pKa$'s up or down, depending on whether the de(protonated) state is energetically favorable or
not, which may even cause residues to be partially (de)protonated at neutral pH.

To calculate the correct equilibrium electrostatic potentials of biological nanopores at lower pH values, we
extended the Python code of \gls{pdb2pqr} (`\code{fractional-PDB2PQR}') with the ability to assign partial
protonation states. The adapted program uses as input (1) a fully protonated PQR molecule (obtained by running
\gls{pdb2pqr} at \pH{1}), (2) the per-residue $\pKa$ values (calculated by either
\gls{propka}~\cite{Olsson-2011} or \gls{delphipka}~\cite{Wang-2015}), (3) a user-defined pH value and (4) the
force-field of choice. This approach was used to compute the electrostatics of the \gls{frac} variants at
\pH{4.5} and \num{7.5}~\cite{Huang-2017}.
%



\subsubsection{Computing nanopore-particle electrostatic interaction energies}
%
\label{sec:elec:methods:elec:energy}
%
% Adapted from Willems-Ruic-Biesemans-2019


When using the \gls{pbe} to evaluate the electrostatic of a system, its electrostatic energy is given
by~\cite{Baker-2005}
%
\begin{equation}\label{eq:pbe_energy}
  \gibbs (\potential, \permittivity) = \int_\Omega \left[
    \scd^{\rm{f}}(\rpos) \potential
    - \dfrac{\permittivity}{2}\left( \nabla \potential \right)^2
    - \ionaccess^{-2} \left( \cosh \potential - 1 \right)
  \right] \,dr
  \text{ ,}
\end{equation}
%
and hence depends not only on the electrostatic potential of the system (\text{via} $\potential$), but also on
the choice of relative permittivities (\textit{via} $\permittivity$) and ionic properties (\textit{via}
$\ionaccess$). \Cref{eq:pbe_energy} can be evaluated directly by \gls{apbs} by including the \code{calcenergy}
keyword its the input file, which enables the rapid evaluation of the electrostatic energies of complex
biomolecular systems. For example, the free electrostatic energy change caused by placing an another molecule
(\ie~`particle') inside a nanopore ($\energyelec$) can be computed using~\cite{Homeyer-2015}
%
\begin{equation}\label{eq:electrostatic_energy}
  \energyelec = \gibbs^{\rm{pore+part}} - \gibbs^{\rm{pore}} - \gibbs^{\rm{part}}
\end{equation}
%
with $\gibbs^{\rm{pore+part}}$, $\gibbs^{\rm{pore}}$ and $\gibbs^{\rm{part}}$ the electrostatic free energies
of systems containing respectively both the nanopore and the particle molecule, the empty nanopore and the
particle molecule alone. A sketch of these different systems is given in \cref{fig:apbs_energy_scheme}. This
approach was used to map out the electrostatic energy landscape of \gls{ssdna} and \gls{dsdna} translocation
through respectively the \gls{frac} and \gls{clya} nanopores in this chapter, and of a tagged \gls{dhfr}
molecule through \gls{clya} in \cref{ch:trapping}.

%
\begin{figure*}[t]
  \centering
  \medskip
  %
  \includegraphics[scale=1]{apbs_energy_scheme}

\caption[Computing net electrostatic energies with APBS.]{%
  \textbf{Computing net electrostatic energies with APBS.}
  %
  The net electrostatic energy difference $\energyelec$ for placing a particle, illustrated here as a
  \gls{dsdna} molecule, inside a nanopore can be computed from the energy of the pore+particle system
  $\gibbs^{\rm{pore+part}}$ (left) by subtracting the energies of the empty pore $\gibbs^{\rm{pore}}$
  (middle) and particle only $\gibbs^{\rm{part}}$ (right) systems.
  %
  }\label{fig:apbs_energy_scheme}
\end{figure*}
%

%
\clearpage
%


\section{Results and discussion}
%
\label{sec:elec:results}
%


\subsection{Electrostatics of the PlyAB nanopore}
%
\label{sec:elec:plyab}
%

\subsubsection{Size matters: large proteins require large nanopores.}
%

To analyze folded proteins with biological nanopores, they must interact in a characteristic manner with the
pore for a time period that is sufficiently long to be properly sampled~\cite{Willems-VanMeervelt-2017}. For
small nanopores such as \gls{ahl} or \gls{mspa} this can be achieved by tethering the protein in close
proximity to the entry of pore~\cite{Movileanu-2000,Fahie-2015,Ho-2015,Laszlo-2016,Thakur-2019}. Larger
nanopores such as \gls{clya}, allow for the immobilization of the entire protein \emph{within} the \lumen{} of
the nanopore
itself~\cite{Soskine-2012,Soskine-2013,Soskine-Biesemans-2015,Biesemans-2015,Wloka-2016,VanMeervelt-2017,Galenkamp-2018,Galenkamp-2020},
which typically requires no modifications to the analyte proteins, even though better signals can sometimes be
obtained with engineered versions~\cite{Soskine-Biesemans-2015,Galenkamp-2020}. The latter approach does
mandate that the protein physically fits inside the pore, which puts a strict upper size limit. Given that
many eukaryotic proteins (average molecular weight of \SI{\approx50}{\kDa}~\cite{Kozlowski-2016}) exceed
\SI{5}{\nm} in diameter, only \gls{clya} has been used successfully so far. Compounding the lack of suitable
nanopores is that, for more than two decades, nanopore research was focused primarily towards \gls{dna}
analysis, resulting in a wide selection of biological nanopores with narrow diameters
(\SIrange{\approx1}{2}{\nm}) entries. With a \lumen{} of \SI{\approx7.2}{\nm} in diameter, the \gls{plyab}
nanopore (see \cref{sec:np:plyab,fig:nanopores_plyab}) is the ideal candidate for analyzing larger globular
proteins. Huang \etal. employed directed evolution to improve the electrophysiological stability of the
wild-type pore, resulting in the variants \gls{plyab-e2} (\gls{plya}: C62S, C94S; \gls{plyb}: N26D, N107D,
G218R, A328T, C441A, A464V; \cref{fig:plyab_mutants_residues}, middle) and \gls{plyab-r} (\gls{plya}: C62S,
C94S; \gls{plyb} N26D, K255E, E260R, E270R, A328T, C441A, A464V; \cref{fig:plyab_mutants_residues},
right)~\cite{Huang-2020}. Notably, \gls{plyab-r} showed significantly altered behavior w.r.t. capture of
negatively charged proteins. In \gls{plyab-e2}, blockades are only observed for the relatively small
(\SI{24}{\kDa}) \tb-casein when it was added at the \transi{} side (at positive bias voltages), and not for
the larger (\SI{66.5}{\kDa}) \gls{bsa}. \gls{plyab-r}, on the other hand, exhibited blockades for both
\tb-casein and \gls{bsa}, and from either the \cisi{} (at positive bias voltages) or \transi{} (at positive
bias voltages) side. In the following section we will describe the electrostatics of these \gls{plyab}
variants, discuss the implications of their differences, and finally link them to the experimentally observed
differences.

%
\begin{figure*}[p]
  \centering
  \medskip
  %

  \begin{subfigure}[t]{120mm}
    \centering
    \caption{}\vspace{-5mm}\hspace{1.5mm}\label{fig:plyab_mutants_residues}
    \includegraphics[scale=1]{plyab_mutants_residues}
  \end{subfigure}
  %
  \\ \vspace{-2mm}
  %
  \begin{subfigure}[t]{120mm}
    \centering
    \caption{}\vspace{-2.5mm}\hspace{1.5mm}\label{fig:plyab_mutants_potential_0300}
    \includegraphics[scale=1]{plyab_mutants_potential_0300}
  \end{subfigure}
  %
  \\ \vspace{0mm}
  %
  \begin{subfigure}[t]{120mm}
    \centering
    \caption{}\vspace{-2.5mm}\hspace{1.5mm}\label{fig:plyab_mutants_potential_1000}
    \includegraphics[scale=1]{plyab_mutants_potential_1000}
  \end{subfigure}
  %

\caption[Electrostatic potential inside PlyAB mutants.]{%
  \textbf{Electrostatic potential inside PlyAB mutants.}
  %
  (\subref{fig:plyab_mutants_residues})
  %
  Cartoon representation of a single subunit of the \gls{plyab-wt} (left), \gls{plyab-e2} (middle), and
  \gls{plyab-r} (right), which were investigated for their ability to capture and analyze large protein in
  Ref.~\cite{Huang-2020}. The labels indicate the mutations w.r.t. the wild-type \gls{plyab}.
  %
  Radially averaged cross-sections ($\radpot$, left) and \SI{25}{\angstrom}-radius cylindrically averaged line
  plots ($\cylpot$, right) of the electrostatic potentials at
  %
  (\subref{fig:plyab_mutants_potential_0300})
  %
  \SI{0.3}{\Molar}
  %
  (\subref{fig:plyab_mutants_potential_1000})
  %
  \SI{1.0}{\Molar}.
  %
  Computations were performed using \gls{apbs}~\cite{Baker-2001,Baker-2005}.
  %
  Molecular representations were rendered using \gls{vmd}~\cite{Humphrey-1996,Stone-1998}.
  %
  }\label{fig:plyab_mutants_potential}
\end{figure*}
%

\subsubsection{Charge reversals induce potential reversals.}
%

Whereas \gls{plyab-e2} (\cref{fig:plyab_mutants_residues}, middle) introduces only one additional negative
charge in the center of the \transi{} \lumen{} (N107D) and a single positive charge at the \transi{} entry
(G218R), \gls{plyab-r} (\cref{fig:plyab_mutants_residues}, right) hosts three negative-to-positive (E260R,
E261R, E270R) and one positive-to-negative (K255E) charge reversals inside the \cisi{} constriction alone. At
an ionic strength of \SI{0.3}{\Molar}, the effect of these mutations is reflected clearly in the radially
averaged ($\radpot$, \cref{fig:plyab_mutants_potential_0300}, left) and the cylindrically averaged ($\cylpot$,
\cref{fig:plyab_mutants_potential_0300}, right) electrostatic potentials of the pores. Whereas $\radpot$
represents the average potential within a cross-section of the pore, $\cylpot$ is the average potential along
the length of the pore and within a the radius of the narrowest part of the channel (here \SI{25}{\angstrom}).
From an electrostatic point-of-view, \gls{plyab-e2} is virtually identical to the wild-type, with the
exception of a slightly more negative constriction (peak $\cylpot$ changes from \SIrange{-0.35}{-0.4}{\kTe} at
$z = \mSI{90}{\angstrom}$) and a lightly more positive \transi{} entry (peak $\cylpot$ changes from
\mSIrange{+0.25}{+0.5}{\kTe} at $z = \mSI{-15}{\angstrom}$) due to the N26D and G218R mutations, respectively.
The many charge reversals in \gls{plyab-r}, on the other hand, also reversed the potential inside the
constriction to strong positive (peak $\cylpot$ changes from \SIrange{-0.35}{+0.35}{\kTe} at $z =
\mSI{90}{\angstrom}$). At higher ionic strengths (\SI{1}{\Molar}, \cref{fig:plyab_mutants_potential_1000}),
the additional screening diminishes the magnitude of the peak $\cylpot$ values in the constriction, and
virtually abolishes any differences in the \transi{} \lumen{} by dropping $\cylpot$ close to zero.

\subsubsection{An electro-osmotic tug of war.}
%

In a (large) nanopore, the electro-osmotic flow is caused by force exerted on the \gls{edl} by the external
electric field, which results in a unidirectional movement of the \gls{edl} that drags on the fluid in the
rest of the pore~\cite{Qiao-Aluru-2003,Mao-2014,Tagliazucchi-2015,Bonome-2017}. The electrostatic potential
along the walls of the wild-type and E2 variants of \gls{plyab} is predominantly negative, which results in a
uniform, positively charged \gls{edl}. Hence, these pores are expected to generate a significant
electro-osmotic flow from \cisi{} to \transi{} at negative voltages (and \textit{vice versa} at positive
voltages). In \gls{plyab-r}, the negative potential inside the \transi{} \lumen{} is reversed abruptly to a
high positive value at the \cisi{} constriction. Indeed, the \gls{edl} inside the constriction of
\gls{plyab-r} is negatively charged, and will exert the opposite force on the fluid as the \gls{edl} in the
\transi{} \lumen{}. The resulting direction of the electro-osmotic flow will depend will depend on whether the
`tug of war' game is won by the \cisi{} or \transi{} side of the pore. Regardless of the final direction of
the \gls{eof} in \gls{plyab-r}, this competition will significantly reduce the magnitude of the \gls{eof}
compared to \gls{plyab-e2}. This may enable the electrophoretic force on the proteins to become stronger than
the electro-osmotic force, and give rise to the capture behavior observed experimentally~\cite{Huang-2020}.

% \clearpage

\subsection{The {pH}-dependent electrostatics of the FraC nanopore}
%
\label{sec:elec:frac}
%

\subsubsection{Two FraC or not two FraC.}
%
As with other nanopores, the narrowest location in \gls{frac}'s hydrophilic channel is expected to dominate
its nanofluidic properties. Hence, it is no surprise that \gls{refrac} (D10R/K159E,
\cref{fig:frac_mutants_pka_residues})~\cite{Wloka-2016}, which contains a negative-to-positive charge reversal
mutation at the bottom of its constriction, showed a much lower conductance (cf.~$\conductance =
\mSI{1.91(17)}{\nS}$ (\gls{wtfrac}) and $\conductance = \mSI{1.19(12)}{\nS}$ (\gls{refrac}), at
\SI{+50}{mV} and \SI{1}{\Molar} \ce{NaCl}~\cite{Wloka-2016}) and a reversed ion selectivity
(cf.~$\permeabilityratio{\ce{K+}}{\ce{Cl-}} = 3.6$ (\gls{wtfrac}), $\permeabilityratio{\ce{K+}}{\ce{Cl-}} =
0.57$ (\gls{refrac}), for \SI{0.467}{\Molar} \ce{KCl} (\cisi) $\|$ \SI{1.960}{\Molar} \ce{KCl} (\transi) at
\pH{7.5}~\cite{Huang-2017}). In addition, counter to its wild-type counterpart, \gls{refrac} showed a positive
aptitude for \gls{dna} translocation, allowing it to differentiate between homopolymeric stretches of A, C and
T polynucleotides~\cite{Wloka-2016}. Finally, in a subsequent study it was shown that \gls{wtfrac}, but not
\gls{refrac}, exhibited different peptide capture dynamics at low pH values~\cite{Huang-2017}.

In this section we aim to rationalize these findings, using electrostatic simulations as a guide. First, the
difference in electrostatic potential between these two variants is investigated when changing the electrolyte
pH from \numrange{7.5}{4.5}. Second, we look into their ability to translocate \gls{dna} by computing the
electrostatic energy landscape of a translocating piece of \gls{ssdna}. For reference, a detailed discussion
of the structure and general properties of \gls{frac} can be found in \cref{sec:np:frac}.

%
\begin{figure*}[t]
  \centering
  \medskip
  %make
  \begin{subfigure}[t]{65mm}
    \centering
    \caption{}\vspace{-5mm}\hspace{1.5mm}\label{fig:frac_mutants_pka_residues}
    \includegraphics[scale=1]{frac_mutants_pka_residues}
  \end{subfigure}
  %
  \begin{subfigure}[t]{50mm}
    \centering
    \caption{}\vspace{-2.5mm}\hspace{1.5mm}\label{fig:frac_mutants_pka_scatter}
    \includegraphics[scale=1]{frac_mutants_pka_scatter}
  \end{subfigure}

\caption[Effective $\pKa$ values of Wt- and ReFraC.]{%
  \textbf{Effective p$\boldsymbol{K}_{\mathbf{a}}$ values of Wt- and ReFraC.}
  %
  (\subref{fig:frac_mutants_pka_residues})
  %
  Molecular representation the most salient titratable residues of \gls{wtfrac} (left) and \gls{refrac}
  (D10R/K159E, right), colored according to their subunit-averaged $\pKa$ values, as computed by
  \gls{propka}~\cite{Olsson-2011,Sondergaard-2011}.
  %
  (\subref{fig:frac_mutants_pka_scatter})
  %
  Scatter plot of the chain-averaged $\pKa$ values of all titratable residues of \gls{wtfrac} (red outlines)
  and \gls{refrac} (blue outlines), as a function of their residue number. Because most $\pKa$ values are
  identical between these two variants, only residues 10 and 159 where plotted for \gls{refrac}. The solid
  lines indicate the baseline $\pKa$ values of the given amino acids used by
  \gls{propka}~\cite{Olsson-2011,Sondergaard-2011}.
  %
  Molecular structures were rendered using \gls{vmd}~\cite{Humphrey-1996,Stone-1998}.
  %
  }\label{fig:frac_mutants_pka}
\end{figure*}
%

\subsubsection{Wt- and ReFraC titrate differently at low pH.}
%

Before proceeding to the electrostatic simulations at \pHlist{7.5;4.5}, it is instructive to analyze the
effective $\pKa$ values of all titratable residues in \gls{wtfrac} and \gls{refrac}
(\cref{fig:frac_mutants_pka}). These values can be calculated self-consistently by software packages such as
\gls{propka}~\cite{Olsson-2011,Sondergaard-2011} or \gls{delphipka}~\cite{Wang-2015}. Of all the titratable
side-chains that face the interior channel wall, only specific residues will be influenced if the pH is
lowered to \num{4.5} (\cref{fig:frac_mutants_pka_residues}). For \gls{wtfrac} these comprise D10 ($\pKa =
4.25$), D17 ($\pKa = 3.76$), E24 ($\pKa = 4.55$), E40 ($\pKa = 5.38$), H67 ($\pKa = 6.28$), H169 ($\pKa =
4.76$), E173 ($\pKa = 4.45$) and H175 ($\pKa = 6.94$); for \gls{refrac} they consist of D17 ($\pKa = 3.80$),
E24 ($\pKa = 4.55$), E40 ($\pKa = 5.38$), H67 ($\pKa = 6.28$), H169 ($\pKa = 4.76$), E173 ($\pKa = 4.45$) and
H175 ($\pKa = 6.94$). Note that these lists are virtually identical, except for the absence of residue 10 in
the case of \gls{refrac}, since it was mutated to an arginine residue ($\pKa = 11.95$). This suggests that the
electrostatic influence of residue 10 will remain unchanged for \gls{refrac} at \pH{4.5}
($\protfrac_{\mathrm{HA}} = 1.00$), whereas the D10 in \gls{wtfrac} will be in a protonated state for a
significantly part of the time ($\protfrac_{\mathrm{HA}} = 0.36$). Inside the confined space of a nanopore
\lumen{}, the close proximity of the like-charged and oppositely-charged residues respectively destabilizes and
stabilizes their (de)protonated states~\cite{Olsson-2011}. This can be observed as significant $\pKa$ shifts
relative to the typical values found inside proteins ($\Delta \pKa = \pKa^{\textrm{effective}} -
\pKa^{\textrm{baseline}}$). In \cref{fig:frac_mutants_pka_scatter}, we plotted the chain-averaged $\pKa$
values of all titratable residues in \gls{wtfrac} and \gls{refrac} together with their conventional baseline
values, as used by \gls{propka}. These values are also summarized \cref{tab:frac_pka_summary}. The effective
$\pKa$ of most charged residues---acidic \emph{and} basic---are lower compared to their baseline values. This
suggests that the deprotonated states of the \num{\approx10} acidic residues (Glu + Asp) are stabilized, at
the cost of the destabilization of their \num{\approx20} basic counterparts (Lys + Arg). Given that the ratio
of positively over negative charged residues is \num{\approx2}, this is a logical result.

%
\begin{threeparttable}
  \footnotesize
  \centering

  %
  \caption{Summary of the WtFraC's and ReFraC's $\pKa$ values.}
  \label{tab:frac_pka_summary}
  %

  \renewcommand{\arraystretch}{1.2}
  \scriptsize

  \begin{tabularx}{9.5cm}{lSSSSSSS}
    %
    \toprule
    %
    \multicolumn{2}{c}{Residue}
    & \multicolumn{2}{c}{$\pKa^{\textrm{effective}}$}
    & \multicolumn{2}{c}{$\Delta \pKa$\tnote{*}}
    & \multicolumn{2}{c}{Count} \\
    \cmidrule(r){1-2} \cmidrule(lr){3-4} \cmidrule(lr){5-6} \cmidrule(lr){7-8}
    {Name} & {$\pKa^{\text{baseline}}$} & {Wt} & {Re} & {Wt} & {Re} & {Wt} & {Re} \\
    %
    \midrule
    %
    C-  & 3.20  & 3.34  & 3.40  & +0.14 & +0.20 & 1   & 1   \\
    ASP & 3.80  & 3.52  & 3.42  & -0.28 & -0.38 & 8   & 7   \\
    GLU & 4.50  & 4.31  & 4.40  & -0.19 & -0.10 & 5   & 6   \\
    HIS & 6.50  & 5.59  & 5.60  & -0.91 & -0.90 & 6   & 6   \\
    N+  & 8.00  & 7.86  & 7.84  & -0.14 & -0.16 & 1   & 1   \\
    TYR & 10.00 & 11.61 & 11.61 & +1.61 & +1.61 & 11  & 11  \\
    LYS & 10.50 & 10.16 & 10.18 & -0.34 & -0.32 & 11  & 10  \\
    ARG & 12.50 & 12.21 & 12.19 & -0.29 & -0.31 & 10  & 11  \\
    %
    \bottomrule
  \end{tabularx}

  \begin{tablenotes}
    \item[*] $\pKa$-shift $\Delta \pKa = \pKa^{\textrm{effective}} - \pKa^{\textrm{baseline}}$.
  \end{tablenotes}

\end{threeparttable}
%

%
\begin{figure*}[t]
  \centering
  \medskip
  %
  \begin{subfigure}[t]{120mm}
    \centering
    \caption{}\vspace{-2.5mm}\hspace{1.5mm}\label{fig:frac_mutants_partialcharge_residues}
    \includegraphics[scale=1]{frac_mutants_partialcharge_residues}
  \end{subfigure}
  %
  \\ \vspace{-2mm}
  %
  \begin{subfigure}[t]{120mm}
    \centering
    \caption{}\vspace{-2.5mm}\hspace{1.5mm}\label{fig:frac_mutants_potential_1000}
    \includegraphics[scale=1]{frac_mutants_potential_1000}
  \end{subfigure}
  %

\caption[Electrostatic potential distribution inside Wt- and ReFraC.]{%
  \textbf{Electrostatic potential distribution inside Wt- and ReFraC.}
  %
  (\subref{fig:frac_mutants_partialcharge_residues})
  %
  Molecular models of the partial charges of the most important titratable residues of \gls{wtfrac} and
  \gls{refrac} at \pHlist{7.5;4.5}, as computed with \cref{eq:pka_partial_charge}.
  %
  (\subref{fig:frac_mutants_potential_1000})
  %
  Radially averaged cross-sections ($\radpot$, left) and \SI{6}{\angstrom}-radius cylindrically averaged line
  plots ($\cylpot$, right) of the electrostatic potentials at at \SI{1}{\Molar} ionic strength and for
  \pHlist{7.5;4.5}, computed from the fractionally charged PQR files with \gls{apbs}.
  %
  Computations were performed using \gls{apbs}~\cite{Baker-2001,Baker-2005}.
  %
  Molecular representations were rendered with \gls{vmd}~\cite{Humphrey-1996,Stone-1998}.
  }\label{fig:frac_mutants_potential}
\end{figure*}

\subsubsection{Partial charges of Wt- and ReFraC at pH 4.5 and 7.5.}
%

Whereas most residues have a unitary charge at \pH{7.5}, this is no longer the case at \pH{4.5}. As discussed
in the methods of this chapter (\cref{sec:elec:methods:elec}), and corroborated by the effective $\pKa$
analysis in the previous paragraph, most of \gls{frac}'s acidic residues will be not be fully (de)protonated
at \pH{4.5}. Translated to electrostatic simulations, this means that at \pH{4.5}, these residues will carry,
on average, a charge between \SIrange{-1}{0}{\ec} (\cref{fig:frac_mutants_partialcharge_residues}), depending
on their exact $\pKa$ value (\cref{fig:frac_mutants_pka,tab:frac_pka_summary}). For \gls{wtfrac}, the most
important negatively charged residues are D10 (\SIrange{-1}{-0.64}{\ec}), D17 (\SIrange{-1}{-0.85}{\ec}), E24
(\SIrange{-1}{-0.47}{\ec}), E40 (\SIrange{-1}{-0.12}{\ec}), and E173 (\SIrange{-1}{-0.53}{\ec}). In the case
of \gls{refrac}, these are D17 (\SIrange{-1}{-0.85}{\ec}), E24 (\SIrange{-1}{-0.47}{\ec}), E40
(\SIrange{-1}{-0.12}{\ec}), E173 (\SIrange{-1}{-0.54}{\ec}), and E159 (\SIrange{-1}{-0.29}{\ec}). Notice that
all basic residues (aside from histidine) do not change their protonation state when going from
\pHrange{7.5}{4.5}. This includes R10 in \gls{refrac}, which, in contrast to D10 in \gls{wtfrac}, keeps its
full charge at low pH.

\subsubsection{Wt- and ReFraC: electrostatic opposites.}
%

As evidenced by the radially averaged cross-sections and the cylindrical averages of \gls{wtfrac}'s and
\gls{refrac}'s electrostatic potentials at \pH{7.5} and \SI{1}{\Molar} ionic strength
(\cref{fig:frac_mutants_potential_1000}), the D10R mutation fully reverses the electrostatic potential inside
the constriction from negative to positive. Because arginine is significantly bulkier compared to glutamate,
the charged groups are packed closer together inside the constriction, resulting in a increase of the peak
cylindrically averaged potential ($\cylpot$) from \SI{\approx-2.5}{\kTe} in \gls{wtfrac} to
\SI[retain-explicit-plus]{\approx+5}{\kTe} in \gls{refrac}. The central part of \gls{frac} (at $z \approx
\mSI{30}{\angstrom}$) is strongly positive, whereas the rest of the \cisi{} side ($z > \mSI{30}{\angstrom}$)
is negative.

Lowering the pH to \num{4.5} has a large effect on the constriction of \gls{wtfrac}, which becomes
significantly less negative. The $\cylpot$ amplitude inside the constriction reduces by \SI{40}{\percent},
from \SIrange{-2.5}{-1.5}{\kTe}). In contrast, maximum $\cylpot$ value inside the constriction of \gls{refrac}
remains steady at \SI[retain-explicit-plus]{+5}{\kTe}. Finally, at low pH, the interior walls of the larger
\cisi{} side ($z > \mSI{30}{\angstrom}$) also reverse their potential, become predominantly positive instead
of negative in both \gls{frac} variants.

\subsubsection{A link between the electrostatic potential, ion-selectivity and electro-osmotic flow.}
%
% Adapted from Huang-2017
%

The electrostatic potentials described above corroborate the experimentally determined ion selectivities at
\pHlist{7.5;4.5}~\cite{Huang-2017}. Whereas \gls{wtfrac} becomes less cation-selective, as evidenced by the
\SI{\approx42}{\percent} drop of its potassium permeability ratio ($\permeabilityratio{\ce{K+}}{\ce{Cl-}}$)
from \num{3.64(37)} (\pH{7.5}) to \num{2.11(23)} (\pH{4.5}), \gls{refrac} becomes \emph{more} anion-selective,
with a \SI{\approx60}{\percent} increase of its chloride permeability ratio
($\permeabilityratio{\ce{Cl-}}{\ce{K+}}$) from \num{1.75(12)} (\pH{7.5}) to \num{2.78(62)} (\pH{4.5}).
Overall, these findings are in agreement with changes observed in the electrostatic potential of these pores,
but the increase in anion selectivity for \gls{refrac} at \pH{4.5} indicates that all charges along the wall
of a nanopore contribute to the ion selectivity, not just those at its narrowest location. If we assume the
preferential transport of a given ion and its hydration shell results in the net directional transport of
water through a nanopore, it follows that the \gls{eof} moves in the direction of the most permeable ion and
the net number of water molecules flowing through the pore ($\flux_{w}$) can be estimated
using~\cite{Piguet-2014}
%
\begin{equation}\label{eq:water_flux_permeability}
  \flux_{w} = N_{w} \dfrac{\current}{\echarge} 
    \left[ \dfrac%
        {1 - \permeabilityratio{\ce{K+}}{\ce{Cl-}}}%
        {1 + \permeabilityratio{\ce{K+}}{\ce{Cl-}}}%
    \right]
    \text{ ,}
\end{equation}
%
with $N_{w}$ the average number of water molecules per ion (\ie~the hydration shell), $\current$ the ionic
current through the pore at a given bias voltage and $\echarge$ the elementary charge. Note that
\cref{eq:water_flux_permeability} likely underestimates the true water flux, as it does not take the
additional drag on the fluid---due to the unidirectional movement of the electrical double layer---into
account.

Assuming a hydration shell of 10 water molecules (\ie~$N_{w} = 10$)~\cite{Boukhet-2016,Piguet-2014} and a bias
voltage of \SI{-50}{\mV}, \gls{wtfrac} transports water molecules from \cisi{} to \transi{} at rates of
\SI{6.1e9}{\per\second} (\pH{7.5}) and \SI{2.5e9}{\per\second} (\pH{4.5}). A reduction of
\SI{\approx59}{\percent} from \pHrange{7.5}{4.5}. Because \Gls{refrac} is anion-selective water will flow in
the opposite direction, or from \cisi{} to \transi{} at positive potentials. Using \gls{refrac}'s current at
\SI{+50}{\mV} in \cref{eq:water_flux_permeability} yields rates of \SI{1.4e9}{\per\second} (\pH{7.5}) and
\SI{2.1e9}{\per\second} (\pH{4.5}). An increase of \SI{\approx51}{\percent} from \pHrange{7.5}{4.5}. The flow
rates can be translated to velocities using
%
\begin{equation}\label{eq:water_flux_to_velocity}
  \velocity = \dfrac{\flux_{w} V_{w}}{\pi R^2}
  \text{ ,}
\end{equation}
%
with $\velocity$ the velocity, $V_{w}$ the volume occupied by a single water molecule
(\SI{\approx30}{\cubic\angstrom}) and $R$ the local radius of the pore. Assuming a radius of \SI{1}{\nm}
(\eg~in the center of \gls{frac}'s channel), the rates above become velocities of \SI{58}{\mmps} (Wt @
\pH{7.5}), \SI{24}{\mmps} (Wt @ \pH{4.5}), \SI{13}{\mmps} (Re @ \pH{7.5}), and \SI{20}{\mmps} (Re @ \pH{4.5}).
These velocities are comparable to literature values for pores of similar
size~\cite{Boukhet-2016,Piguet-2014,Pederson-2015}.

Overall, these data confirm the experimental finding from Huang \etal{} (see~\cite{Huang-2017}), that, at
least for larger proteins such as chymotrypsin, the \gls{eof} drives their interaction with \gls{frac}. The
observed reduction and increase in capture rates for respectively \gls{wtfrac} (at negative bias) and
\gls{refrac} (at positive bias) when lowering the pH from \numrange{7.5}{4.5}, is in agreement with the
respective trends of their electro-osmotic flow velocities at these pH values.


%
\begin{figure*}[p]
  \centering
  \medskip
  %
  \begin{subfigure}[t]{120mm}
    \centering
    \caption{}\vspace{-2.5mm}\hspace{1.5mm}\label{fig:frac_mutants_bead-ssdna_1000_energy}
    \includegraphics[scale=1]{frac_mutants_bead-ssdna_1000_energy}
  \end{subfigure}
  %
  \\ \vspace{3.38mm}
  %
  \begin{subfigure}[t]{120mm}
    \centering
    \caption{}\vspace{-10mm}\hspace{1.5mm}\label{fig:frac_mutants_bead-ssdna_1000_potential}
    \includegraphics[scale=1]{frac_mutants_bead-ssdna_1000_potential}
  \end{subfigure}
  %

\caption[Electrostatic energy of ssDNA translocation through Wt- and ReFraC.]{%
  \textbf{Electrostatic energy of ssDNA translocation through Wt- and ReFraC.}
  %
  (\subref{fig:frac_mutants_bead-ssdna_1000_energy})
  %
  Electrostatic energy landscape ($\energyelec$) for the translocation of \gls{ssdna}, represented here by a
  string of \SI{1}{\nm} diameter beads with \SI{-1}{\ec} charge each, through \gls{wtfrac} and \gls{refrac} at
  \SI{1.0}{\Molar} ionic strength. Due to its positively charged constriction, the energy barrier for
  \gls{refrac} is only \SI{14}{\kT}, almost a third of the \SI{38}{\kT} observed for \gls{wtfrac}.
  %
  (\subref{fig:frac_mutants_bead-ssdna_1000_potential})
  %
  Vertical (top) and horizontal (bottom) cross-sections of the electrostatic potential ($\potential$) inside
  \gls{wtfrac} and \gls{refrac} as the \gls{ssdna} enters the pore from the \cisi{}-side ($\zdna =
  \mSI{150}{\angstrom}$, left) and when it reaches the constriction ($\zdna = \mSI{85}{\angstrom}$, left). The
  location of the horizontal slices are indicated using a gray line (\sliceline).
  %
  Computations were performed using \gls{apbs}~\cite{Baker-2001,Baker-2005}.
  %
  Molecular representations were rendered with \gls{vmd}~\cite{Humphrey-1996,Stone-1998}.
  }\label{fig:frac_mutants_bead-ssdna_1000}
\end{figure*}
%


\subsection{Energetics of {ssDNA} translocation through {Wt}- and {ReFraC}}
%
\label{sec:elec:frac:dna}

\subsubsection{Calculating electrostatic energies using a {ssDNA} bead model.}
%

When observing the contrary electrostatic potentials within the constriction of \gls{wtfrac} and \gls{refrac}
(\cref{fig:frac_mutants_bead-ssdna_1000_potential}), it is unsurprising that only the latter is able to
translocate \gls{dna}. As evinced by the rotaxane experiments performed by Wloka \etal{}, \gls{refrac} is able
to translocate both ss- and \gls{dsdna}~\cite{Wloka-2016}. The fact that the diameter of \gls{dsdna}
(\SI{22}{\angstrom}) is almost twice the width of constriction (\SI{12}{\angstrom}) implies that the
N-terminal \ta-helices of \gls{refrac}'s are pushed apart during \gls{dsdna} translocation. Due to the static
nature of our simulations, we cannot model this without resorting to \gls{md} simulations, and are thus
limited to \gls{ssdna} (\SI{10}{\angstrom} diameter). However, the short persistence length of \gls{ssdna}
(\SI{13}{\angstrom}~\cite{Tinland-1997}) makes it difficult to model atomistically with a reasonable
conformation. Hence, we opted to represent it as a simple linear string of 20 beads with a radius of
\SI{5}{\angstrom} and a charge of \SI{-1}{\ec}, spaced \SI{10}{\angstrom} apart. By placing this bead model at
different heights (from $200 \ge \zdna \ge \mSI{-150}{\angstrom}$) along the center of \gls{wtfrac} and
\gls{refrac}, we were able to map the net electrostatic energy ($\energyelec$, \cref{eq:electrostatic_energy})
caused by the translocation of a \gls{ssdna} molecule through these pores at \SI{1}{\Molar} ionic strength
(\cref{fig:frac_mutants_bead-ssdna_1000}). 

\subsubsection{Reversing the electrostatic potential at the constriction significantly lowers the energy barrier.}
%

When translocating the \gls{ssdna} model from \cisi{} to \transi{}
(\cref{fig:frac_mutants_bead-ssdna_1000_energy}), their is virtually no change in $\energyelec$ observed
between the \gls{dna} entering the pore ($\zdna = \mSI{150}{\angstrom}$,
\cref{fig:frac_mutants_bead-ssdna_1000_potential}, left) up until the moment it reaches the top of the
constriction ($\zdna = \mSI{110}{\angstrom}$). Entering the constriction however, is accompanied by a steep
increase of $\energyelec$ between $ \mSI{110}{\angstrom} > \zdna > \mSI{85}{\angstrom}$
(\cref{fig:frac_mutants_bead-ssdna_1000_potential}, right) from \SI{\approx0}{\kT} to \SI{38}{\kT} for
\gls{wtfrac} and \SI{14}{\kT} for \gls{refrac}. Note that these energy changes are both positive, and hence
energetically \emph{unfavorable}. Even though the interaction between the positively charged R10 residues in
\gls{refrac} and the negatively charged beads in the \gls{ssdna} model reduce the energy barrier in the
constriction by approximately a third, it is still insufficient to make \gls{ssdna} translocation
electrostatically favorable. Once inside the constriction, the energy level remains stable at their respective
maxima until drops rapidly to \SI{0}{\kT} once the \gls{ssdna} exits from the pore at the \transi{} entry
($\mSI{-90}{\angstrom} > \zdna >  \mSI{-120}{\angstrom}$).

\subsubsection{Overcoming the energy barriers inside the constriction.}
%

For the piece of \gls{ssdna} to translocate (at positive bias potentials), it must over come the energy
barrier presented by the constriction: \SI{\approx38}{\kT} for \gls{wtfrac} and \SI{\approx14}{\kT} for
\gls{refrac}. In the case of \gls{wtfrac}, the energy barrier is not only more than twice as high as the one
in \gls{refrac}, the effective force exerted on the \gls{dna} at any given bias voltage will also be less due
to the opposing \gls{eof}. Nevertheless, if we ignore the \gls{eof}, and assume that, in order for it to
translocate, the \gls{dna} only uses the electrophoretic force exerted on it as the first base moves from the
\cisi{} entry ($\zdnac = \mSI{150}{\angstrom}$, \cref{fig:frac_mutants_bead-ssdna_1000_potential}, left) to
the bottom of \transi{} constriction ($\zdnat = \mSI{85}{\angstrom}$,
\cref{fig:frac_mutants_bead-ssdna_1000_potential}, right). The electrophoretic energy ($\Delta
\gibbs^{\textrm{ep}}$) gained by moving a piece of \gls{ssdna} from positions $z_1$ to $z_2$ along length the
pore is given by
%
\begin{align}\label{eq:ep_energy}
  \Delta \gibbs^{\textrm{ep}} ={}& - \int_{z_1}^{z_2} \forceep (z) \,dz \nonumber \\
                              ={}& - \int_{z_1}^{z_2} \chargeq (z) \efield (z) \,dz
  \text{ ,}
\end{align}
%
with $\forceep (z)$ the electrophoretic force, $\chargeq (z)$ the amount of charge inside the pore, and the
$\efield (z)$ the electric field, all a function of \gls{dna} position $z$. Assuming a uniform decay of the
applied potential $\vbias$ along the length of pore ($\length$), $\efield$ can be simplified to
%
\begin{equation}\label{eq:linear_efield}
  \efield (z) = \dfrac{\vbias}{\length}
  \text{ .}
\end{equation}
%
Additionally, $\chargeq (z)$ can be approximated by the linear function
%
\begin{equation}\label{eq:ssdna_charge_in_pore}
  \chargeq (z) = 
        \dfrac{ n_2 - n_1 } { z_2 - z_1 } 
        \left( z - z_1 \right) \chargeq_{\textrm{base}}
  \text{ ,}
\end{equation}
%
with $n_1$ and $n_2$ the number of bases inside the pore when the $z$ is at positions $z_1$ and $z_2$,
respectively, and $q_{\textrm{base}}$ the net charge of a single base.

Finally, plugging \cref{eq:linear_efield,eq:ssdna_charge_in_pore} in \cref{eq:ep_energy} and integrating
yields
%
\begin{align}\label{eq:ep_energy_dna}
  \Delta \gibbs^{\textrm{ep}} %
     ={}& - \int_{z_1}^{z_2} %
          \left(  n_2 - n_1 \right) %
          \dfrac{ z - z_1 }{z_2 - z_1}
          \chargeq_{\textrm{base}} \dfrac{\vbias}{\length} \,dz \nonumber \\
     ={}& - 0.5 \left(  n_2 - n_1 \right)  \chargeq_{\textrm{base}}
            \dfrac{\vbias}{\length} \left(  z_2 - z_1 \right)
  \text{ ,}
\end{align}
%
Using the parameters $n_1 = 0$, $n_2 = 7$, $z_1 = \mSI{150}{\angstrom}$, $z_2 = \mSI{85}{\angstrom}$,
$q_{\textrm{base}} = \mSI{-1}{\ec}$, $\vbias = \mSI{+100}{\mV}$ and $\length = \mSI{65}{\angstrom}$, in
\cref{eq:ep_energy_dna} yields $\Delta \gibbs^{\textrm{ep}} = \mSI{-13.6}{\kT}$. Indeed, this value is very
close to the barrier height of \SI{14}{\kT} observed for \gls{refrac}, but it is far from sufficient for
crossing the \SI{38}{\kT} barrier presented by \gls{wtfrac}. Additionally, because the \gls{eof} works with
the $\forceep$ in \gls{refrac}, but against it in \gls{wtfrac}, the magnitude of $ \left| \Delta
\gibbs^{\textrm{ep}} \right| = \mSI{13.6}{\kT}$ is an underestimation for \gls{refrac}, but an overestimation
for \gls{wtfrac}, which would reduce its ability to translocate \gls{ssdna} even further.


%
%
\subsection{Electrostatics of the ClyA nanopore}
%
\label{sec:elec:clya}
%

\subsubsection{Precise manipulation of charges enables {DNA} translocation at physiological ionic strengths.}
%

As indicated in \cref{sec:np:clya}, the interior walls of \gls{clya} are lined predominantly with negatively
charged residues (cf.~\cref{fig:nanopores_clya_pore_section}), which prevents the pore from translocating
\gls{dna} at salt concentrations lower than \SI{\approx1}{\Molar}~\cite{Franceschini-2013,Franceschini-2016}.
This concentration is far from the physiological ionic strengths (\ie~\SI{0.15}{\Molar}) required for the
(optimal) functioning of many \gls{dna}-processing enzymes. In an extensive mutagenesis study by Franceschini
\etal, more than 20 variants of \gls{clya-as} were tested for their ability to translocate \gls{dsdna} at an
ionic strength of \SI{0.15}{\Molar}~\cite{Franceschini-2016}. These mutants introduced additional positive
charges at key locations throughout the interior walls of the pore (either at the \cisi{} entry, \lumen{} or
\transi{} constriction), typically by the mutagenesis of a neutral (\ie~net \SI{+1}{\ec}) or negatively
charged (\ie~net \SI{+2}{\ec}) residue to arginine or lysine. Interestingly, only the S110R/D64R double mutant
(\gls{clya-rr}) was able to translocate \gls{dsdna}, and only from the \cisi{} side. Because analyzing all 20
variants would be too computationally intensive, we have limited our analysis here to several mutants
representative for the observed electrostatic phenomena: \gls{clya-as}, \gls{clya-r} (S110R), \gls{clya-rr}
(S110R/D64R), \gls{clya-rr56} (S110R/Q56R), and \gls{clya-rr56k} (S110R/Q56R/Q8K). The locations of their
mutations are indicated in \cref{fig:clya_mutants_residues}.

%
\begin{figure*}[p]
  \centering
  \medskip
  %

  \begin{subfigure}[t]{120mm}
    \centering
    \caption{}\vspace{-5mm}\hspace{1.5mm}\label{fig:clya_mutants_residues}
    \includegraphics[scale=1]{clya_mutants_residues}
  \end{subfigure}
  %
  \\ \vspace{-2mm}
  %
  \begin{subfigure}[t]{120mm}
    \centering
    \caption{}\vspace{-2.5mm}\hspace{1.5mm}\label{fig:clya_mutants_potential_0150}
    \includegraphics[scale=1]{clya_mutants_potential_0150}
  \end{subfigure}
  %
  \\ \vspace{0mm}
  %
  \begin{subfigure}[t]{120mm}
    \centering
    \caption{}\vspace{-2.5mm}\hspace{1.5mm}\label{fig:clya_mutants_potential_2500}
    \includegraphics[scale=1]{clya_mutants_potential_2500}
  \end{subfigure}
  %

\caption[Electrostatic potential distribution inside several ClyA variants.]{%
  \textbf{Electrostatic potential distribution inside several ClyA variants.}
  %
  (\subref{fig:clya_mutants_residues})
  %
  Cartoon representation of a single subunit of the ClyA-AS, -R, -RR, RR$_{56}$ and -RR$_{56}$K pores, all of
  which were investigated for their ability to translocate DNA under physiological ionic strength in
  ref.~\cite{Franceschini-2016}. The labels indicate the mutations w.r.t. the \textit{S. typhi} wild-type (for
  AS), or w.r.t. \gls{clya-as} (for the others).
  %
  Radially averaged cross-sections ($\radpot$, left) and \SI{15}{\angstrom}-radius cylindrically averaged line
  plots ($\cylpot$, right) of the electrostatic potentials at
  %
  (\subref{fig:clya_mutants_potential_0150})
  %
  \SI{0.15}{\Molar}
  %
  (\subref{fig:clya_mutants_potential_2500})
  %
  \SI{2.5}{\Molar}.
  %
  Computations were performed using \gls{apbs}~\cite{Baker-2001,Baker-2005}.
  %
  Molecular representations were rendered using \gls{vmd}~\cite{Humphrey-1996,Stone-1998}.
  }\label{fig:clya_mutants_potential}
\end{figure*}
%


\subsubsection{All charges are important at physiological ionic strength.}
%
% Adapted from Franceschine-2016
%

When observing the evolution of the electrostatic potential at an ionic strength of \SI{0.15}{\Molar}
(\cref{fig:clya_mutants_potential_0150,tab:clya_potential_summary}) from \cisi{} to \transi{}, the effect of
the different mutations is clearly visible. The mutation S110R is located at the \cisi{} entry of \gls{clya}
($z \approx \mSI{120}{\angstrom}$), and it switches the potential from negative (\SI{-0.12}{\kTe}) in
\gls{clya-as} to a slightly positive value (\SIrange{+0.02}{0.04}{\kTe}) in the other mutants. Even though
this is a small change, it means that \gls{dna} will now be attracted towards, instead of repelled from, the
\cisi{} entry of the pore. This may help with aligning the \gls{dna} strand with the central axis of the pore,
priming it for entry into the \lumen{}. Next up are mutations D64R and Q56R, both of which are located in the
middle of the \lumen{} ($z \approx \mSI{60}{\angstrom}$). However, whereas D64R introduces two positive
charges, Q56R only yields a single one. This is clearly reflected in the electrostatic potential, where for
Q56R it remains negative (from \SIrange{-0.32}{-0.20}{\kTe}), but D64R manages to push it to be significantly
positive (from \SIrange{-0.32}{+0.20}{\kTe}). This large change towards a positive potential observed inside
the \lumen{} of \gls{clya-rr} may allow the \gls{dna} penetrate sufficiently deep into the pore and hence to
build up enough force from the external electric field to overcome the high negative potential at the
\transi{} constriction. At the bottom of the \lumen{} ($z \approx \mSI{30}{\angstrom}$), the electrostatic
potential falls rapidly to high negative values. The influence of the Q8K mutation, which located in at bottom
of the constriction ($z \approx \mSI{-10}{\angstrom}$), is not clearly visible when comparing \gls{clya-as}
with \gls{clya-rr56k} since the potential decreases from \SIrange{-1.44}{-1.84}{\kTe}. Nevertheless, relative
to the other mutants, \gls{clya-rr56k} does have the least negative constriction. The discrepancies between
the peak values inside the constrictions are likely caused by minor conformational changes in the side chains
the residues that make up \gls{clya}'s \gls{tmd}. Given that they are located at the protein's N-terminus,
they enjoy much more conformational freedom that the other mutated residues, and hence may have moved more
during the \gls{md} equilibration phase of the model creation (see \cref{sec:elec:methods:molec}). Regardless
of its conformation, the Q8K mutation does not significantly impact the negative potential inside the
constriction and as such is not expected to impact \gls{dna} translocation. 


\subsubsection{The constriction dominates at high ionic strengths.}
%

At \SI{2.5}{\Molar}, the overall electrostatic potential inside the pore drops to virtually \SI{0}{\kTe}
everywhere, except inside the constriction
(\cref{fig:clya_mutants_potential_2500,tab:clya_potential_summary}). The magnitude of the negative potential
is significantly reduced for all mutants (\SI{\approx-0.5}{\kTe}) and its influence starts only inside the
constriction itself ($z \approx \mSI{0}{\angstrom}$). This means that at high salt concentrations, the
\gls{dna} can enter into the pore and move down to its constriction unopposed, after which only a small
barrier must be overcome to complete the translocation.


%
\begin{table}
  \footnotesize
  \centering

  %
  \captionsetup{width=8.5cm}
  \caption{Electrostatic potential at key locations for several {ClyA} variants.}
  \label{tab:clya_potential_summary}
  %

  \renewcommand{\arraystretch}{1.2}
  \scriptsize

  \begin{tabularx}{8.5cm}{lSSSSSS}
    %
    \toprule
    %
     & \multicolumn{6}{c}{ {Electrostatic potential ($\cylpot$) [\si{\kTe}]} } \\
    %
    \cmidrule(l){2-7}
    %
     & \multicolumn{2}{c}{$z = \mSI{120}{\angstrom}$}
     & \multicolumn{2}{c}{$z = \mSI{60}{\angstrom}$}
     & \multicolumn{2}{c}{$z = \mSI{-10}{\angstrom}$} \\
    %
    \cmidrule(r){2-3} \cmidrule(lr){4-5} \cmidrule(lr){6-7} 
    %
    {Variant}
      & \SI{0.15}{\Molar} & \SI{2.5}{\Molar}
      & \SI{0.15}{\Molar} & \SI{2.5}{\Molar}
      & \SI{0.15}{\Molar} & \SI{2.5}{\Molar} \\
    %
    \midrule
    %
    AS         & -0.12 & -0.00 & -0.32 & -0.00 & -1.44 & -0.25 \\
    R          & +0.02 & +0.00 & -0.31 & -0.00 & -1.91 & -0.36 \\
    RR         & +0.03 & +0.00 & +0.20 & +0.00 & -1.88 & -0.36 \\
    RR$_{56}$  & +0.04 & +0.00 & -0.20 & -0.00 & -2.14 & -0.57 \\
    RR$_{56}$K & +0.04 & +0.00 & -0.20 & +0.00 & -1.84 & -0.53 \\
    %
    \bottomrule
  \end{tabularx}
\end{table}
%

%
\subsection{Energetics of {dsDNA} translocation through {ClyA}}
%
\label{sec:elec:clya:dna}

\subsubsection{Computing the energy of DNA translocation with a full-atom {dsDNA} model.}
%

Using the same methodology that was used with \gls{ssdna} translocation through \gls{frac} in
\cref{sec:elec:frac}, we used a full atom \gls{dsdna} model to map out the electrostatic energy landscape of
\gls{dna} translocation through the \gls{clya} variants discussed above at \SI{0.15}{\Molar}
(\cref{fig:clya_mutants_dsdna_0150}) and \SI{2.5}{\Molar} (\cref{fig:clya_mutants_dsdna_2500}) ionic
strengths. Here, a bead model is not needed since the long persistence length of \gls{dsdna}
(\SI{390}{\angstrom}~\cite{Gross-2011}) effectively fixes the conformation of a \SI{51}{\bp} strand
(\SI{\approx175}{\angstrom} long) to that of a rigid rod. Additionally, \gls{clya}'s \transi{} constriction is
sufficiently wide (\SI{33}{\angstrom}) to accommodate the full width of a \gls{dsdna} strand
(\SI{22}{\angstrom}).


%
\begin{figure*}[p]
  \centering
  \medskip
  %
  \begin{subfigure}[t]{40mm}
    \centering
    \caption{}\vspace{-5mm}\hspace{1.5mm}\label{fig:clya_mutants_dsdna_0150_energy}
    \includegraphics[scale=1]{clya_mutants_dsdna_0150_energy}
  \end{subfigure}
  %
  \hspace{-3.7mm}
  %
  \begin{subfigure}[t]{81.5mm}
    \centering
    \caption{}\vspace{-5mm}\hspace{1.5mm}\label{fig:clya_mutants_dsdna_0150_potential}
    \includegraphics[scale=1]{clya_mutants_dsdna_0150_potential}
  \end{subfigure}
  %

\caption[Electrostatic energy of {dsDNA} translocation through {ClyA} at physiological ionic strength.]{%
  \textbf{Electrostatic energy of {dsDNA} translocation through {ClyA} at physiological ionic strength.}
  %
  (\subref{fig:clya_mutants_dsdna_0150_energy})
  %
  Net electrostatic energy ($\energyelec$, \cref{eq:electrostatic_energy}), at physiological ionic strength
  (\SI{150}{\mM}), for a \SI{51}{\bp} piece of \gls{dsdna} (\SI{\approx175}{\angstrom}) as a function of its
  $z$-position ($z_{\mathrm{DNA}}$) inside the \gls{clya} variants AS, R, RR, RR$_{56}$ and RR$_{56}$K. The
  mutation S110R eliminates the \SI{8}{\kT} energy penalty for \gls{dsdna} to enter \lumen{} of \gls{clya}
  from the \cisi{} side up until \SI{150}{\angstrom}. In contrast to D64R, which briefly promotes the
  downwards motion up of \gls{dsdna} starting from \SI{150}{\angstrom}, Q56R only manages to slow the rise of
  the energy barrier. Once inside the constriction, the Q8K mutation results in an energy drop from
  \SIrange{\approx40}{60}{\kT}.
  %
  (\subref{fig:clya_mutants_dsdna_0150_potential})
  %
  Vertical and horizontal cross-sections of the electrostatic potential ($\potential$) inside the \gls{clya}
  variants as the \gls{dsdna} begins to enter the \cisi{} entrance (\SI{200}{\angstrom}, top), reaches the
  center of the \lumen{} (\SI{150}{\angstrom}, middle) and fully entered the \transi{} constriction
  (\SI{65}{\angstrom}, bottom). The location of the horizontal slices are indicated using a gray line
  (\sliceline).
  %
  Computations were performed using \gls{apbs}~\cite{Baker-2001,Baker-2005}.
  %
  Molecular representations were rendered with \gls{vmd}~\cite{Humphrey-1996,Stone-1998}.
  %
  }\label{fig:clya_mutants_dsdna_0150}
\end{figure*}
%

\subsubsection{DNA translocation at physiological ionic strengths is a fragile balance of energies.}
%

Upon entry into the pore from the \cisi{} side ($\zdna = \mSI{200}{\angstrom}$,
\cref{fig:clya_mutants_dsdna_0150_potential}, top), the S110R mutation interacts favorably with the
\gls{dna}strand, preventing the gradual \SI{+8}{\kT} rise in energy observed when \gls{clya-as} when the
\gls{dna} moves towards the location of the  D64R (or Q56R) mutation in the middle of the lumen ($\zdna =
\mSI{150}{\angstrom}$, \cref{fig:clya_mutants_dsdna_0150_potential}, middle). When moving further down, the
Q56R mutation in \gls{clya-rr56} and \gls{clya-rr56k} manages to limit the energy increase compared to
\gls{clya-as} and \gls{clya-r}, but the D64R mutation in \gls{clya-rr} \emph{lowers} the energy by
\SI{\approx5}{\kT}. Once the \gls{dna} begins to enter the constriction (from $\zdna = \mSI{100}{\angstrom}$),
however, the energy rises dramatically for all variants by an additional \SI{\approx40}{\kT} until it reaches
a maximum when the \gls{dna} starts to exit the pore from the \transi{} side ($\zdna = \mSI{200}{\angstrom}$,
\cref{fig:clya_mutants_dsdna_0150_potential}, bottom). The influence of the Q8K mutation is also clearly
visible, with \gls{clya-rr56k} having a significantly lower energy (\SI{\approx10}{\kT}) compared to
\gls{clya-rr56}. As the \gls{dna} moves through the constriction, its energy levels appear to fluctuate by a
few \si{\kT} with a period of \SI{\approx30}{\angstrom}. This phenomenon is even more prominent at high salt
concentrations, and will hence be discussed in the next section. 

As we shall see in the next chapter, at positive bias voltages the \gls{eof} in \gls{clya} flows from
\transi{} to \cisi{} and hence exerts a significant amount of upward force on the \gls{dna}, pushing it back
towards \cisi{} entry. If the electro-osmotic force is equal to half the electrophoretic force---a
conservative estimate---any energy gains due to the electrophoretic force will be cut in half. In other words,
this would equate to an effective charge of \SI{-1}{\ec\per\bp} instead of \SI{-2}{\ec\per\bp}. In order to
successfully translocate, the \gls{dsdna} strand must reduce its energy sufficiently to overcome the
\SI{\approx60}{\kT} energy barrier presented by the constriction.  Moving the \gls{dna} strand from the
\cisi{} entry to top of the energy barrier (\cref{eq:ep_energy_dna}, $n_1 = 0$, $n_2 = 41$, $z_1 =
\mSI{200}{\angstrom}$, $z_2 = \mSI{65}{\angstrom}$, $q_{\textrm{base}} = \mSI{-1}{\ec}$, $\vbias =
\mSI{+100}{\mV}$) yields an energy reduction of $\Delta \gibbs^{\textrm{ep}} = \mSI{-76}{\kT}$, which should
indeed be sufficient to overcome the barrier. However, for the \gls{dna} to accrue this much energy reduction,
it must be able to physically reach that location. The S110R mutation helps here by removing the initial
\SI{+8}{\kT} energy increase between $200 > \zdna > \mSI{150}{\angstrom}$ observed only in \gls{clya-as}. Past
this point, the Q56R mutation reduces the rate at which \gls{dna} gains energy compared to \gls{clya-as} and
\gls{clya-r}, but not to same extent as D64R in \gls{clya-rr}, where moving downwards briefly \emph{reduces}
the energy. Likely, at this point in the translocation process, the force balance is precarious in all
mutants. Hence, where the downwards force in \gls{clya-rr} tips the energy balance in favor of the downwards
movement of the \gls{dna}, the opposite is true for the other mutants. In other words, it appears that the
D64R mutation enables the \gls{dna} to move further down into the \lumen{} than the Q56R mutation, allowing it
to sufficiently reduce its energy and eventually overcome the barrier at the constriction.

Note that this analysis presents the situation where the \gls{dsdna} is aligned perfectly with the entry of
the pore, which will often not be the case in reality. Additionally, outside of the pore, the relative
magnitude of the electro-osmotic force is significantly stronger than its electrophoretic counterpart, which
would further compound the chances of \gls{dna} to enter \gls{clya} at all.

%
\begin{figure*}[p]
  \centering
  \medskip
  %
  \begin{subfigure}[t]{40mm}
    \centering
    \caption{}\vspace{-5mm}\hspace{1.5mm}\label{fig:clya_mutants_dsdna_2500_energy}
    \includegraphics[scale=1]{clya_mutants_dsdna_2500_energy}
  \end{subfigure}
  %
  \hspace{-3.7mm}
  %
  \begin{subfigure}[t]{81.5mm}
    \centering
    \caption{}\vspace{-5mm}\hspace{1.5mm}\label{fig:clya_mutants_dsdna_2500_potential}
    \includegraphics[scale=1]{clya_mutants_dsdna_2500_potential}
  \end{subfigure}
  %

\caption[Electrostatic energy of {dsDNA} translocation through {ClyA} at high ionic strength.]{%
  \textbf{Electrostatic energy of {dsDNA} translocation through {ClyA} at high ionic strength.}
  %
  (\subref{fig:clya_mutants_dsdna_2500_energy})
  %
  Net electrostatic energy ($\energyelec$, \cref{eq:electrostatic_energy}), at high ionic strength
  (\SI{2.5}{\Molar}), for a \SI{51}{\bp} piece of \gls{dsdna} (\SI{\approx175}{\angstrom}) as a function of
  its $z$-position ($z_{\mathrm{DNA}}$) inside the \gls{clya} variants AS, R, RR, RR$_{56}$ and RR$_{56}$K.
  Once inside the \transi{} constriction, $\energyelec$ fluctuates periodically with a magnitude of
  \SI{\approx2.5}{\kT} every \SI{36}{\angstrom}, which is close to the pitch of \SI{34}{\angstrom} of B-DNA.
  %
  (\subref{fig:clya_mutants_dsdna_2500_potential})
  %
  Vertical and horizontal cross-sections of the electrostatic potentials ($\potential$) inside the \gls{clya}
  variants as the \gls{dsdna} traverses the \transi{} constriction, showing the first maximum
  (\SI{60}{\angstrom}, top), first minimum (\SI{40}{\angstrom}, middle) and second maximum
  (\SI{25}{\angstrom}, bottom). The location of the horizontal slices are indicated using a gray line
  (\sliceline).
  %
  Computations were performed using \gls{apbs}~\cite{Baker-2001,Baker-2005}.
  %
  Molecular representations were rendered with \gls{vmd}~\cite{Humphrey-1996,Stone-1998}.
  %
  }\label{fig:clya_mutants_dsdna_2500}
\end{figure*}
%

\subsubsection{High ionic strength smooths out and lowers energy barriers.} 
%

At an ionic strengths of \SI{2.5}{\Molar}, no large differences are observed between energy landscapes of the
different \gls{clya} mutants (\cref{fig:clya_mutants_dsdna_2500_energy}). As the \gls{dsdna} enters the pore
from \cisi{} side, its $\energyelec$ remains flat until the it reaches the constriction ($200 > \zdna >
\mSI{100}{\angstrom}$), after which it rises rapidly by \SI{\approx+10}{\kT} inside the constriction. As
discussed in the previous paragraph, at $\zdna = \mSI{100}{\angstrom}$ the \gls{dna} strand will have
decreased its energy by $\Delta \gibbs^{\textrm{ep}} = \mSI{-76}{\kT}$, which exceeds the energy barrier bat
the constriction by almost an order of magnitude. This is in full agreement with the experimental observations
of \gls{dsdna} translocation through \gls{clya-as} at higher ionic
strengths~\cite{Franceschini-2013,Franceschini-2016}.

\subsubsection{The chirality of \gls{dsdna} may induce rotation.}
%

As the \gls{dsdna} traverses the constriction, its energy is observed to fluctuate sinusoidally with an
amplitude of \SI{2.5}{\kT} (\cref{fig:clya_mutants_dsdna_2500_energy}). Likely, the peaks
(\cref{fig:clya_mutants_dsdna_2500_potential}, top and bottom) and valleys
(\cref{fig:clya_mutants_dsdna_2500_potential}, middle) correlate with the precise amount of negative charges
present inside the constriction, which in turn corresponds to how much of the B-\gls{dna}'s minor groove
(\ie~high charge density) or major groove (\ie~low charge density) occupies the constriction. Given that, from
a geometric point of view, moving a B-\gls{dna} strand up or down is identical to rotating it around its
length axis, it is likely that a \SI{2.5}{\kT} energy reduction is sufficient make the \gls{dna} rotate as it
translocates through \gls{clya}.


\subsection{Electrostatic confinement of {dsDNA} within {ClyA}}
%
% Adapted from Bayoumi-2020


\begin{figure*}[t]
  \centering
  \medskip
  %
  \begin{subfigure}[t]{60mm}
    \centering
    \caption{}\vspace{-5mm}\hspace{1.5mm}\label{fig:clya_dsdna_constriction_energy_heatmap}
    \includegraphics[scale=1]{clya_dsdna_constriction_energy_heatmap}
  \end{subfigure}
  %
  \hspace{-3mm}
  %
  \begin{subfigure}[t]{60mm}
    \centering
    \caption{}\vspace{-5mm}\hspace{1.5mm}\label{fig:clya_dsdna_constriction_energy_averaged}
    \includegraphics[scale=1]{clya_dsdna_constriction_energy_averaged}
  \end{subfigure}
  %

\caption[Electrostatic confinement of {dsDNA} within the constriction of {ClyA-AS}.]{%
  \textbf{Electrostatic confinement of {dsDNA} within the constriction of {ClyA-AS}.}
  %
  (\subref{fig:clya_dsdna_constriction_energy_heatmap})
  %
  Contour plot of the electrostatic energy cost/gain ($\Delta \energyelec_{r=0}$) of moving a piece of
  \gls{dsdna} away from the axial center of \gls{clya-as}. The red crosses indicate the location of data
  points.
  %
  (\subref{fig:clya_dsdna_constriction_energy_averaged})
  %
  Energy costs averaged over all angles ($\langle \Delta \energyelec_{r=0} \rangle_{\phi}$), together with the
  corresponding Boltzmann probability distribution $\probability$.
  %
  Computations were performed using \gls{apbs}~\cite{Baker-2001,Baker-2005}.
  %
  Figure adapted with permission from~\cite{Bayoumi-2020}.
  %
  }\label{fig:clya_dsdna_constriction}
\end{figure*}


Given the electrostatic repulsion between \gls{clya-as}'s constriction and the \gls{dsdna}, we expect that the
the latter will be pushed to remain in the center of the pore. To quantify the extent of this confinement, we
computed the electrostatic energy cost of moving a piece of dsDNA away from the center of the pore (\ie~
closer to the walls of \gls{clya}). This information allowed the use of a narrower constriction within the
coarse-grained \gls{md} simulations presented in Bayoumi \etal{}~\cite{Bayoumi-2020}, and was crucial to
properly model the behavior of the \gls{ssdna}- and \gls{dsdna}-rotaxanes within \gls{clya}.

\subsubsection{System setup and energy calculation.}
%

The same piece of \SI{51}{\bp} \gls{dsdna} from previous sections was placed inside \gls{clya-as} at a fixed
height ($\zdna = \mSI{60}{\angstrom}$). Next, the \gls{dna} was moved away from the center of the
constriction ($x = 0$, $y = 0$), using cylindrical coordinates $\rpos_{\mathrm{DNA}} = (r, \phi)$ such that
$x = r \cos{\phi}$ and $y = r \sin{\phi}$:
%
\begin{equation}
  \rpos_{\mathrm{DNA}} = %
  \begin{cases}
    r: [0,5] \msi{\angstrom} &\mbox{with } \Delta r    = \mSI{0.5}{\angstrom} \\
    r: [0,360] \msi{\degree} &\mbox{with } \Delta \phi = \mSI{45}{\degree} \\
  \end{cases}
  \text{ ,}
\end{equation}
%
resulting in 177 unique systems (\ie~1 reference system with only the pore, 88 reference systems with only the
DNA and 88 systems containing the pore and the DNA). The electrostatic potentials and energies at an ionic
strength of \SI{2}{\Molar} where computed as before with \gls{apbs}, using two sequential focussing
calculations (coarse grid: \apbsgrid{500}{500}{700}{\angstrom} size with
\apbsgrid{1.42}{1.42}{0.87}{\angstrom} spacing; fine grid: \apbsgrid{175}{175}{400}{\angstrom} size with
\apbsgrid{0.50}{0.50}{0.50}{\angstrom} spacing). The energy cost (or gain) for moving the \gls{dna} away from
the center of the pore ($r = 0$) was then calculated using
%
\begin{align}
  \Delta \energyelec_{r=0} (r, \phi) = \energyelec (r, \phi) - \energyelec (0,0)
  \text{  .}
\end{align}
%

\subsubsection{The energy minimum is not at the center.}
%

The contour plot of $\Delta \energyelec_{r=0}$ (\cref{fig:clya_dsdna_constriction_energy_heatmap}) reveals
that the energetic minimum for a piece of \gls{dsdna} does not lie at the center of the pore but is slightly
offset, approximately \SI{1}{\angstrom} in both $x$- and $y$-directions. Likely, this is a consequence of the
chirality of the \gls{dna} and the precise location of the major and minor grooves within the constriction of
\gls{clya}, similarly to the fluctuations observed during translocation
(\cref{fig:clya_mutants_dsdna_0150_energy,fig:clya_mutants_dsdna_2500_energy}). 

\subsubsection{Effective pore size.}
%

The angle-averaged relative energy $\langle \Delta \energyelec_{r=0} \rangle_{\phi}$
(\cref{fig:clya_dsdna_constriction_energy_averaged}) reveals quasi exponential increase of energy required to
move a mere \SI{5}{\angstrom} away the center of pore: from \SIrange{0}{7}{\kT}. Even a small offset confers a
significant electrostatic energy penalty, with shifts of \SI{2.0}{\angstrom} and \SI{3.5}{\angstrom} giving
rise to increases of \SI{1}{\kT} and \SI{3}{\kT}, respectively. Assuming Boltzmann statistics, the probability
$\probability$ of the \gls{dna} to be present at a certain offset from the center of the pore
(\cref{fig:clya_dsdna_constriction_energy_averaged}) is given by 
%
\begin{align}
  \probability = \exp{
    \left( \dfrac{\langle \Delta \energyelec_{r=0} \rangle_{\phi}}{\boltzmann\temperature} \right)
    }
    \text{  .}
\end{align}
%
This means that at any given time, \SI{63}{\percent} and \SI{95}{\percent} of \gls{dna} molecules will be
within radius of \SI{2.0}{\angstrom} and \SI{3.5}{\angstrom} from the center of the pore, respectively. Hence,
when placing a cut-off at \SI{3}{\kT}, the electrostatic energy confines the DNA to a circular region with a
radius of \SI{14.5}{\angstrom} ($\mSI{11}{\angstrom} + \mSI{3.5}{\angstrom}$), yielding an effective
constriction size of \SI{29}{\angstrom}, instead of the traditional \SI{33}{\angstrom}.


\section{Conclusion}
%
\label{sec:elec:conclusion}
%

The the electrostatic interactions between nanopores and analyte molecules can play a determining role in the
properties of nanopores. In this chapter we have investigated the influence of ionic strength and solution pH
on the distribution of the equilibrium electrostatic potential (\ie~without an external electric field) within
several variants of the \gls{plyab}, \gls{frac} and \gls{clya} nanopores. This enabled us to provide insights
into experimentally observed properties such a as ion selectivity, and postulate the influence on second order
effects such as the electro-osmotic flow. Self-consistent computation of the electrostatic interaction
energies between a translocating \gls{dna} strand and the \gls{frac} and \gls{clya} nanopores, provided
further insights into the \gls{dna} translocation mechanisms of there pores, confirming experimental
observations.

It is clear that, by studying its electrostatics, one can already reveal a significant fraction a nanopore's
secrets. Nevertheless, as we shall see in the following chapters, equilibrium electrostatics comprise only a
part of a nanopore's. Hence, developing a full understanding requires a thorough quantitative analysis of the
\emph{non-equilibrium} effects (\ie~with an external electric field), such as electrophoresis,
electro-osmosis, and steric hindrance.



%%%%%%%%%%%%%%%%%%%%%%%%%%%%%%%%%%%%%%%%%%%%%%%%%%
% Keep the following \cleardoublepage at the end of this file,
% otherwise \includeonly includes empty pages.
\cleardoublepage

% vim: tw=70 nocindent expandtab foldmethod=marker foldmarker={{{}{,}{}}}
