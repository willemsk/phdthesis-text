\chapter{Computational modeling of biological nanopores}\label{ch:modelling}

\definecolor{shadecolor}{gray}{0.85}
\begin{shaded}
Parts of this chapter were adapted from:\\
\fullcite{Willems-VanMeervelt-2017}
\\
\fullcite{Willems-Ruic-Biesemans-2019}
\\
\fullcite{Willems-2020}
\newpage
\end{shaded}

This chapter serves as a comprehensive introduction to the theoretical and computational approaches that have been developed to model and understand transport phenomena in biological nanopores. \\
\\
The text and figures of this introductory chapter were entirely written and created by myself.

\newpage

\section{Introduction}

% paragraph adapted from introduction of will2018
The computational approaches most widely used to study nanofluidic transport in ion channels or biological
nanopores comprise \emph{discrete} (particle) methods such as molecular dynamics
(MD)\cite{Lynden-Bell-1996,Allen-1999,Aksimentiev-2005,Luan-2008,Bhattacharya-2011,Zhang-2014,DiMarino-2015,Belkin-2016}
and Brownian dynamics
(BD),\cite{Schirmer-1999,Im-2002,Noskov-2004,Millar-2008,Egwolf-2010,DeBiase-2015,Pederson-2015} and
\emph{mean-field} (continuum) methods such as the Poisson-Boltzmann (PB)
equations\cite{Grochowski-2008,Baldessari-2008-1} and Poisson-Nernst-Planck (PNP)
equations,\cite{Eisenberg-1996,Gillespie-2002,Simakov-2010}. The latter can be coupled with the Navier-Stokes
(NS) equation to include the electroosmotic flow.\cite{Lu-2012,Pederson-2015} Due to their explicit atomic or
particle nature, MD and BD simulations are considered to yield the most accurate results. However, the large
computational cost of simulating a complete biological nanopore system (100K--1M atoms) for 100s of
nanoseconds still necessitates the use of supercomputers.\cite{Aksimentiev-2005,Bhattacharya-2011} The
PNP(-NS) equations, on the other hand, are of particular interest due to their low computational cost and
analytical tractability. The aim of the following chapter is two-fold: (1) to equip the reader with deeper
understanding of the different physical phenomena that contribute to molecular transport in nanopores, and
(2) to give the reader a non-exhaustive yet thorough overview of the approaches available to nanopore
researchers investigate these phenomena through computional modeling.

\section{Transport phenomena in nanopores}


\section{Modelling approaches}

\subsection{Discrete particle-based modelling}

\subsubsection{Molecular dynamics}
The most commonly used technique to study the complex behavior and properties of biological (i.e. protein) nanopores is molecular dynamics (\gls{md}).

\subsubsection{Brownian dynamics}




\subsection{Continuum-based modelling}

\subsubsection{Poisson-Boltzmann equation}

\subsubsection{Poisson-Nernst-Planck equation}

\subsubsection{Improvements to the PNP-based modeling}

\subsection{Hybrid modeling approaches}

\section{Outlook}

%%%%%%%%%%%%%%%%%%%%%%%%%%%%%%%%%%%%%%%%%%%%%%%%%%
% Keep the following % otherwise \includeonly includes empty pages.

% vim: tw=70 nocindent expandtab foldmethod=marker foldmarker={{{}{,}{}}}
