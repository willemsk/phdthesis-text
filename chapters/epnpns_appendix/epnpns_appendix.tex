%
\chapter[SI: An improved PNP-NS framework]%
        {Supplementary information: An improved PNP-NS framework}
%
\label{ch:epnpns_appendix}

\definecolor{shadecolor}{gray}{0.85}
\begin{shaded}
  \adapRSC{Willems-2020}
\newpage
\end{shaded}


\section{Weak forms of the {ePNP-NS} equations}
%
\label{sec:epnpns_appendix:weak_forms}
%

To solve partial differential equations with the finite element method, we must derive their weak form.
This is achieved through the multiplication of the equation with an arbitrary test function and integration over
their relevant domains and boundaries (\cref{fig:epnpns_model_boundaries}). The full computational domain of
our model ($\Omega$) is subdivided into domains for the pore ($\Omega_p$), the lipid bilayer ($\Omega_m$), and
the electrolyte reservoir ($\Omega_w$). The relevant boundaries of these domains are also indicated, i.e. the
reservoir's exterior edges at the \cisi{} $\Gamma_{w,c}$) and \transi{} $\Gamma_{w,t}$) sides, the outer edge
of the lipid bilayer ($\Gamma_{m}$) and the interface of the fluid with the nanopore and the bilayer
($\Gamma_{p+m}$).

The following paragraphs detail the weak forms of equations that were used in this
\cref{ch:epnpns,ch:trapping}.

%
\begin{figure*}[t]
  \centering
  %
  \includegraphics[scale=1]{epnpns_model_boundaries}
  %
  
  \caption[Computational domains and boundaries.]%
  {\textbf{Computational domains and boundaries.}
    %
    The full computational domain ($\Omega$) of the model is subdivided into various subdomains---comprising
    the reservoir ($\Omega_w$), the lipid bilayer ($\Omega_m$) and the pore ($\Omega_p$)---and their limiting
    boundaries---the exterior boundary at the \cisi{} ($\Gamma_{w,c}$) \transi{} and the \transi{}
    ($\Gamma_{w,t}$) sides of the reservoir, the exterior boundary of the membrane ($\Gamma_m$) and the
    interior boundary between the reservoir on the one hand, and the pore and membrane on the other
    ($\Gamma_{p+m}$).
    %
  }\label{fig:epnpns_model_boundaries}
\end{figure*}
%


\subsection{Poisson equation}
%

The global potential distribution is described by the Poisson equation (PE)~\cite{Lu-2012}
%
\begin{align}
  \label{eq:epnpns_appendix:poisson}
  \nabla \cdot \displacement = -\left( \scdpore + \scdion \right)
  \text{\quad with \quad}
  \displacement = \absperm \relperm \nabla \potential
  \text{ ,}
\end{align}
%
with $\displacement$ the electrical displacement field, $\potential$ the electric potential, $\absperm$ the
vacuum permittivity (\SI{8.85419e-12}{\farad\per\meter}), and $\relperm$ the local relative permittivity
of the medium. $\scdpore$ and $\scdion$ are the fixed (due to the pore) and mobile (due to the ions) charge
distributions, respectively.

Multiplication of~\cref{eq:epnpns_appendix:poisson} with the potential test function $\psi$ and integration
over the entire model $\Omega=\Omega_w+\Omega_p+\Omega_m$ gives
%
\begin{align}
  \displaystyle\int_{\Omega}
  \left[
    \nabla \cdot \displacement
  \right]
  \psi \,d\Omega
  ={}&
  - \displaystyle\int_{\Omega} \left[ \scdpore + \scdion \right] \psi \,d\Omega \text{ ,}
\end{align}
%
which, after applying the Gauss divergence theorem, yields the final weak formulation
%
\begin{align}
  \displaystyle\int_{\Omega}
  \left[
    \nabla \psi \cdot \displacement
  \right]
  \,d\Omega
  & - \displaystyle\int_{\Gamma_\text{PE}}
  \left[
    \psi \displacement \cdot \vec{n}
  \right]
  \,d\Gamma_\text{PE} \notag \\
  & =
  \displaystyle\int_{\Omega} \left[ \psi \scdpore \right] \,d\Omega
  +
  \displaystyle\int_{\Omega} \left[ \psi \scdion \right] \,d\Omega
  \text{ ,}
\end{align}
%
with boundaries $\Gamma_\text{PE}=\Gamma_{w,c}+\Gamma_{w,t}+\Gamma_m$ and $\vec{n}$ their normal vector. The
boundary integrals at $\Gamma_{w,c}$ and $\Gamma_{w,t}$ are evaluated using the Dirichlet \glspl{bc}
$\potential=0$ and $\potential=\vbias$, respectively. A zero charge \gls{bc}, $\vec{n} \cdot \displacement =
0$, is used for the integral at $\Gamma_{m}$.



\subsection{Size-modified Nernst-Planck equation}
%

The total ionic flux $\flux_{i}$ of ion $i$ at steady-state is expressed by the \gls{smnpe}~\cite{Lu-2012}
%
\begin{equation}\label{eq:smnp}
  \pd{\concentration_{i}}{t} = - \nabla\cdot\flux_{i} = - \nabla \cdot
  \left(
    \diffusion_{i}\nabla\concentration_{i}
    + \chargen_{i}\mobility_{i}\concentration_{i}\nabla\potential
    + \diffusion_{i} \vec{\beta_{i}} \concentration_{i}
    - \velocity\concentration_{i}
  \right)
  \text{ ,}
\end{equation}
%
where
%
\begin{equation}\label{eq:steric_vector}
  \vec{\beta_{i}} =
    \frac{\ionsize_{i}^3/\ionsize_{0}^3 \dsum_{j} \avogadro \ionsize_{j}^3 \nabla \concentration_{j}}
        {1 - \dsum_{j} \avogadro \ionsize_{j}^3 \concentration_{j}}
  \text{ ,}
\end{equation}
%
and with ion diffusion coefficient $\diffusion_{i}$, concentration $\concentration_{i}$, charge number
$\chargen_{i}$, mobility $\mobility_{i}$, electrostatic potential $\potential$, steric saturation factor
$\beta_{i}$ and fluid velocity $\velocity$. $\avogadro$ is Avogadro's constant (\SI{6.022e23}{\per\mole}) and
$\ionsize_{i}$ and $\ionsize_{0}$ are the limiting cubic diameters for ions and water, respectively. Using the
ion concentration test function $d_i$, the weak form of \cref{eq:smnp} becomes
%
\begin{align}
  % Raw integral
  \displaystyle\int_{\Omega_w} \left[ \pd{\concentration_{i}}{t} \right] d_{i} \,d\Omega_w =
  \displaystyle\int_{\Omega_w} \left[ - \nabla \cdot \flux_{i} \right] d_{i} \,d\Omega_w
  \text{, }
\end{align}
%
which can be split into the domain and boundary integrals
%
\begin{align}
  % Gauss divergence theorem
  \displaystyle\int_{\Omega_w} \left[ d_{i} \pd{\concentration_{i}}{t} \right] \,d\Omega_w ={}&
  \displaystyle\int_{\Omega_w} \left[\nabla d_{i} \cdot \flux_{i} \right]\,d\Omega_w
  - \displaystyle\int_{\Gamma_\text{NP}}
  \left[ d_{i} \flux_{i} \cdot \vec{n} \right]\,d\Gamma_\text{NP} \\
  % Fill in flux components
  ={}&
  % Domain
  \displaystyle\int_{\Omega_w}
  \left[
    \nabla d_{i} \cdot
    \left(
      \diffusion_{i}\nabla\concentration_{i}
      + \chargen_{i}\mobility_{i}\concentration_{i}\nabla\potential
      + \diffusion_{i} \vec{\beta_{i}} \concentration_{i}
      - \velocity \concentration_{i}
    \right)
  \right]
  \,d\Omega_w \notag \\
  % Boundary
  & - \displaystyle\int_{\Gamma_\text{NP}}
  \left[
    d_{i}
    \left(
      \diffusion_{i} \nabla \concentration_{i}
      + \chargen_{i} \mobility_{i} \concentration_{i} \nabla \potential
      + \diffusion_{i} \vec{\beta_{i}} \concentration_{i}
      - \velocity \concentration_{i}
    \right)
    \cdot \vec{n}
  \right]
  \,d\Gamma_\text{NP} \notag
  \text{ .}
\end{align}
%
The integrals on the boundaries $\Gamma_\text{NP} = \Gamma_{w,c}+\Gamma_{w,t}+\Gamma_{p+m}$ are evaluated
using the Dirichlet \gls{bc} $\concentration_{i} = \cbulk$ for $\Gamma_{w,c}$ and $\Gamma_{w,t}$, and the no
flux \gls{bc} $\vec{n} \cdot \flux_{i} = 0$ for $\Gamma_{p+m}$


\subsection{Variable density and viscosity Navier-Stokes equation}
%

The steady-state, laminar fluid flow of an incompressible fluid with a variable density and viscosity is given
by the system of equations~\cite{Axelsson-2015}
%
\begin{align}
  \label{eq:ns_density_continuity}
  \velocity \cdot \nabla \density ={}& 0 \\
  \label{eq:ns_conservation}
  \left( \velocity \cdot \nabla \right) \left( \density\velocity \right)
  + \nabla \cdot \hydrostresstensor ={}& \volumeforce \text{\quad with }
  \hydrostresstensor =
    \pressure\identity - \viscosity\left[\nabla\velocity+\left(\nabla\velocity \right)^\mathsf{T}\right]
  \\
  \label{eq:ns_velocity_continuity}
  \nabla \cdot \left( \density\velocity \right) - \velocity \cdot \nabla \density ={}& 0
  \text { ,}
\end{align}
%
with fluid velocity $\velocity$, density $\density$, hydrodynamic stress tensor $\hydrostresstensor$,
viscosity $\viscosity$, pressure $\pressure$ and body force $\volumeforce$. The pressure test function $q$ is
used to derive the weak forms of \cref{eq:ns_density_continuity}
%
\begin{align}
  % Density continuity
  \displaystyle\int_{\Omega_w} \left[ \velocity \cdot \nabla \density \right] q \,d\Omega_w ={}&
  \displaystyle\int_{\Omega_w} \left[ q \velocity \cdot \nabla \density \right] \,d\Omega_w = 0 \text{ ,}
\end{align}
%
and \cref{eq:ns_velocity_continuity}
%
\begin{align}
  % Velocity continuity
  \displaystyle\int_{\Omega_w}
  &\left[\nabla \cdot \left( \density \velocity \right) - \velocity \cdot \nabla \density \right] q
  \,d\Omega_w \\ & =
  \displaystyle\int_{\Omega_w}
  \left[\nabla \cdot \left( \density \velocity \right) \right] q \,d\Omega_w
  -
  \displaystyle\int_{\Omega_w} 
  \left[ q \velocity \cdot \nabla \density \right] \,d\Omega_w \notag \\
  & =
  \displaystyle\int_{\Omega_w}
  \left[\nabla q \cdot \left( \density \velocity \right) \right] \,d\Omega_w
  -
  \displaystyle\int_{\Gamma_\text{NS}}
  \left[ q \left( \density \velocity \right) \cdot \vec{n} \right] \,d\Gamma_\text{NS}
  -
  \displaystyle\int_{\Omega_w}
  \left[ q \velocity \cdot \nabla \density \right] \,d\Omega_w \notag
  \text{ ,}
\end{align}
%
while for \cref{eq:ns_conservation} we use the velocity test function $\vec{v}=\left[v_r, v_\phi, v_z\right]$
%
\begin{align}
  \displaystyle\int_{\Omega_w}
  \left[
    \left( \velocity \cdot \nabla \right) \left( \density\velocity \right) + \nabla \cdot \hydrostresstensor
  \right]
  \cdot \vec{v} \,d\Omega_w
  =
  \displaystyle\int_{\Omega_w} \vec{\force} \cdot \vec{v} \,d\Omega_w 
\end{align}
%
which can be split into the domain and boundary integrals
%
\begin{align}
  % Conservation
  \displaystyle\int_{\Omega_w} \vec{\force} \cdot \vec{v} \,d\Omega_w =
  \displaystyle\int_{\Omega_w} &
  \left[
    \left( \velocity \cdot \nabla \right) \left( \density\velocity \right) \cdot\vec{v}
  \right] 
  \,d\Omega_w 
  -
  \displaystyle\int_{\Omega_w}
  \left[
  \hydrostresstensor \cdot \nabla\vec{v}
  \right]
  \,d\Omega_w \notag \\
  & +
  \displaystyle\int_{\Gamma_\text{NS}}
  \left[
  \vec{v} \cdot
  \hydrostresstensor
  \cdot\vec{n}
  \right]
  \,d\Gamma_\text{NS}
  \text{ ,}
\end{align}
%
with boundaries $\Gamma_\text{NS} = \Gamma_{w,c}+\Gamma_{w,t}+\Gamma_{p+m}$. The no-slip Dirichlet \gls{bc}
$\velocity = 0$ is applied to $\Gamma_{p+m}$, and the no normal stress $\hydrostresstensor\vec{n}=0$ is used
for $\Gamma_{w,c}$ and $\Gamma_{w,t}$.




%%%%%%%%%%%%%%%%%%%%%%%%%%%%%%%%%%%%%%%%%%%%%%%%%%
% Keep the following \cleardoublepage at the end of this file, 
% otherwise \includeonly includes empty pages.
\cleardoublepage

% vim: tw=70 nocindent expandtab foldmethod=marker foldmarker={{{}{,}{}}}
