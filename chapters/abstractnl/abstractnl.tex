% !TeX root = ../../thesis.tex
\chapter{Beknopte samenvatting}
%
\label{ch:abstractnl}
%

Doorheen de hele geschiedenis heeft de mensheid naar middelen gezocht om onze omgeving beter te kunnen
beheersen en te begrijpen. Hiervoor hebben we onze natuurlijke zintuigen uitgebreid met een reeks van
sensoren. Dit zijn werktuigen die ons toelaten om zowel het hele grote, zoals de samenvoeging van twee zwarte
gaten zo'n \num{1.3}~miljard lichtjaren verwijderd van de aarde, als het hele kleine, zoals het identificeren
van een enkel viraal partikel in een oceaan van moleculen, te detecteren. Deze thesis is gewijd aan de studie
van de natuurkundige werkingsprincipes van een erg kleine, maar veelzijdige sensor: de biologische
nanoporie~(BNP). BNPs zijn eiwitten die in staat zijn om openingen ter grootte van enkele nanometers, te
vormen in een lipide dubbellaag. Telkens wanneer er een molecule zich doorheen deze opening beweegt
(`translokeert'), onthult de tijdelijke storing van de ionische stroom kostbare informatie over de identiteit
en de eigenschappen van het translokerende molecule. Ondanks die voor de hand liggende detectiemethode, maakt
de complexiteit van de interacties tussen de nanoporie en het translokerende molecule het erg moeilijk om de
veranderingen in de ionische stroom \'{e}\'{e}n-op-\'{e}\'{e}n te verbinden met hun feitelijke, natuurkundige
oorzaak. Omdat computationele methoden, zoals degene beschreven in deze thesis, bijna alle aspecten van het
detectieproces kunnen modelleren met atomaire precisie, kunnen ze een belangrijke bijdrage leveren ter
oplossing van dit probleem.

Naast het oplijsten van nodige concepten en de huidige stand van zaken in het nanoporie onderzoeksveld
(hoofdstuk~\ref{ch:nanopores}), zijn er vier hoofddoelstellingen in deze thesis:
%
\begin{enumerate}
  \item De ontwikkeling van methodes voor het accuraat modelleren van BNPs;
  \item Het onderzoeken van de evenwichtselektrostatica van BNPs;  
  \item Het verklaren van het gedrag van een eiwit dat gevangen zit in een BNP;
  \item Het in kaart brengen van de transporteigenschappen van een BNP.
\end{enumerate}
%

In het eerste deel van deze thesis (hoofdstuk~\ref{ch:electrostatics}) hebben we gebruik gemaakt van 3D
evenwichtssimulaties, gebaseerd op numerieke oplossingen van de Poisson-Boltzmann vergelijking, om de
elektrostatische eigenschappen van de pleurotolysin~AB (PlyAB), de cytolysin~A (ClyA) en de fragaceatoxin~C
(FraC) nanopori\"{e}n te onderzoeken. In het bijzonder hebben we aangetoond dat het aanbrengen van enkele (of
zelfs \'{e}\'{e}n enkele) mutaties die de lading van het aminozuur omwisselen, een grote elektrostatische
impact kunnen hebben, resulterende in een erg verminderende of zelfs omgekeerde elektro-osmotische stroming.
Bovendien hebben onze simulaties aangetoond dat het verlagen van de pH de invloed van de negatief geladen
aminozuren sterk onderdrukt, terwijl het die van de positief geladen groepen onaangeroerd laat. Om uit te
zoeken in welke mate de FraC en ClyA pori\"{e}n in staat zijn om DNA te translokeren, hebben we de
elektrostatisch gebaseerde energetische kost berekend die gepaard gaat met de translocatie van enkelstrengig
en dubbelstrengig DNA. Hieruit bleek dat de exacte positionering van positieve ladingen in staat is om de
translocatie toe te laten, door ofwel de hoogte van de energetische barri\`{e}re fel te doen afnemen, of door
de DNA streng diep genoeg in de porie te doen binnendringen zodat deze de kracht heeft om de barri\`{e}re te
overwinnen. Hoewel de simulaties in dit hoofdstuk aantonen dat het afleiden van enkele primaire
karakteristieken van biologische nanopori\"{e}n mogelijk is met evenwichtselektrostatica, maken ze ook
duidelijk dat de toevoeging van niet-evenwichtskrachten essentieel is voor de ontwikkeling van een volledig
inzicht.

In het volgende hoofdstuk (hoofdstuk~\ref{ch:trapping}) onderzoeken we de immobilisatie van een enkel eiwit
binnen in een nanoporie, welke van belang is voor toepassingen zoals de studie van enkelvoudige enzymen.
Hiervoor hebben we een studie uitgevoerd van de gemiddelde verblijfstijd van dihydrofolaatreductase
(\DHFRt{}), een klein eiwit aan wiens C-terminus een positief geladen polypeptide werd vastgemaakt, in ClyA.
Meer concreet zijn we erin geslaagd om, door het manipuleren van de ladingsverdeling, de verblijfstijd met
enkele grootteordes te verhogen. Verder hebben we ook een analytisch transportmodel opgesteld, gebaseerd op
het overbruggen van de sterische, elektrostatische en elektro-osmotische energetische barri\`{e}res aan beide
uiteinden van de porie, om de ontsnapping van \DHFRt{} uit ClyA te modelleren. Een systematische studie van de
verblijfstijden in functie van de aangelegde potentiaal, samen met een uitgebreide set van
evenwichtselektrostatica simulaties, liet ons toe om dit dubbele-barri\`{e}re model te parametriseren. Dit
legde op zijn beurt dan weer enkele eigenschappen bloot die moeilijk experimenteel te bepalen zijn, zoals de
translocatie kansen en de kracht die uitgeoefend wordt op het eiwit door de elektro-osmotische stroming van
ClyA (\SI{\approx9}{\pN} bij \SI{-50}{\mV}). De relatieve eenvoud van het dubbele-barri\`{e}re model en het
feit dat het geen expliciete parameters bevat voor de geometrie van \DHFRt{}, suggereren dat deze aanpak ook
voor andere kleine eiwitten van toepassing kan zijn.

In de finale hoofstukken hebben we een nieuw continu\"{u}m kader ontwikkeld voor de modellering van {BNPs}
onderhevig aan niet-evenwichtscondities (hoofdstuk~\ref{ch:epnpns}), die we vervolgens toegepast hebben op de
{ClyA} nanoporie (hoofdstuk~\ref{ch:transport}). Hoewel ze vaak kwalitatief nuttige resultaten genereren, is
de capaciteit van continu\"{u}mmodellen om transportproblemen op nanometerschaal kwantitatief op te lossen
eerder gelimiteerd. Daarom hebben we de ``extended Poisson-Nernst-Planck-Navier-Stokes ({ePNP-NS})''
vergelijkingen ontwikkeld, welke op zelf-consistente wijze de eindige grootte van de ionen in rekening
brengen, alsook de invloed van zowel de ionische sterkte als de nanoscopische schaal van de porie op de lokale
eigenschappen van het elektrolyt. Door de {ePNP-NS} vergelijkingen numeriek op te lossen voor een
computationeel effici\"{e}nt model van {ClyA}, waren we in staat om de nanofluidische karakteristieken van de
porie voor een brede waaier aan experimenteel relevante potentialen en zout concentraties in kaart te brengen.
Hierbij kwamen we tot de vaststelling dat de gesimuleerde ionisch conductiviteiten nagenoeg identiek waren aan
de experimenteel gemeten waardes. Dit getuigt van een natuurkundig nauwkeurig model. Om deze reden hebben we
onze simulaties dan ook gebruikt om gedetailleerde inzichten te verwerven over de ware ionselectiviteit, de
verdeling van de ion concentraties, het landschap van de elektrostatische potentiaal, de sterkte van de
elektro-osmotische stroming en de interne drukverdeling. Zoals ze nu zijn, kunnen de {ePNP-NS} vergelijkingen
al fundamenteel nieuwe inzichten verschaffen over de nanofluidische eigenschappen van BNPs, en maken ze ook
de weg vrij naar het rationeel modificeren van zulke pori\"{e}n.

In deze thesis hebben we aangetoond dat simulaties, in combinatie met systematische experimenten, gebruikt
kunnen worden als computationele `microscopen', die ingezet kunnen worden om de natuurkundige fenomenen te
onderzoeken die aan de basis liggen van nanoporie-gebaseerde sensoren. Eenvoudige evenwichtselektrostatica
berekeningen zijn reeds erg leerrijk gebleken. Desondanks is het duidelijk dat de complexe interacties tussen
de nanoporie en het translokerende analyt molecule een niet-evenwichtsaanpak vereist die zowel rigoureus als
zelf-consistent is, zoals de {ePNP-NS} vergelijkingen. Verdere verbeteringen zouden dit simulatiekader van een
`na-de-feiten' analysemethode kunnen promoveren tot een krachtig ontwerpmiddel. Dit zou nanoporieonderzoekers
in staat stellen om automatisch de eigenschappen van nieuwe nanopori\"{e}n in kaart te brengen, of om het
ionische signaal van het eender welk molecule te voorspellen.


%%%%%%%%%%%%%%%%%%%%%%%%%%%%%%%%%%%%%%%%%%%%%%%%%%
% Keep the following \cleardoublepage at the end of this file,
% otherwise \includeonly includes empty pages.
\cleardoublepage

% vim: tw=70 nocindent expandtab foldmethod=marker foldmarker={{{}{,}{}}}
