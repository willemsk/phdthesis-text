% !TeX root = ../../thesis.tex
\chapter{Trapping of a single protein inside a nanopore}
%
\label{ch:trapping}
%
%
\epigraphhead[\epipos]{%
\epigraph{%
%
  ``Magic is organizing chaos. And while oceans of mystery remain, we have deduced that this requires two
  things. Balance and control.''
%
}{%
  \textit{`Tissaia De Vries'}
%
}
}
%
%

%
%
\definecolor{shadecolor}{gray}{0.85}
\begin{shaded}
This chapter was published as:
%
\begin{itemize}
  \item K. Willems*, D. Rui\'{c}*, A. Biesemans*, N. S. Galenkamp, P. Van Dorpe and G. Maglia.
        \textit{ACS Nano} \textbf{13 (9)}, 9980--9992 (2019) %\cite{Willems-Ruic-Biesemans-2019}
        \\
        *equal contributions
\end{itemize}
% 
\newpage
\end{shaded}
%
%

% Glossary reset
%

In this chapter we used experimental, computational, and theoretical methods to investigate the trapping
behavior of a small protein, \glsfirst{dhfr}, within the \glsfirst{clya} nanopore. It builds upon the
electrostatic energy methodology developed in \cref{ch:electrostatics} to parameterize and verify a
mathematical model that captures the essential physics of the experimentally observed voltage- and
charge-dependent dwell times of a tagged \gls{dhfr} molecule in \gls{clya}. \\
%

%
To the extent possible, the main and supplementary texts and figures of the original manuscript were
integrated into this chapter, and the remainder of the supplementary information can be found in
\cref{ch:trapping_appendix}. The text and figures of this chapter represent entirely my own work. All
experimental work was performed by Dr.~Annemie Biesemans, and the trapping model was derived by
Dr.~rer.~nat.~Dino Rui\'{c} (see~\cref{sec:trapping_appendix:escape_rates}).
%

%
\adapACS{Willems-Ruic-Biesemans-2019}
%

%
Note that further permissions related to the material excerpted should be directed to the ACS.
%
% \cleardoublepage
%
%



%
\section{Abstract}
%
\label{sec:trapping:abstract}
%

The ability to confine and to study single molecules has enabled important advances in natural and applied
sciences. Recently, we have shown that unlabeled proteins can be confined inside the biological nanopore
\glsfirst{clya} and conformational changes monitored by ionic current recordings. However, trapping small
proteins remains a challenge. Here we describe a system where steric, electrostatic, electrophoretic, and
electro-osmotic forces are exploited to immobilize a small protein, \glsfirst{dhfr}, inside \gls{clya}.
Assisted by electrostatic simulations, we show that the dwell time of \gls{dhfr} inside \gls{clya} can be
increased by orders of magnitude (from milliseconds to seconds) by manipulation of the \gls{dhfr} charge
distribution. Further, we describe a physical model that includes a double energy barrier and the main
electrophoretic components for trapping \gls{dhfr} inside the nanopore. Simultaneous fits to the voltage
dependence of the dwell times allowed direct estimates of the \cisi{} and \transi{} translocation
probabilities, the mean dwell time, and the force exerted by the electro-osmotic flow on the protein
(\SI{\approx9}{\pN} at \SI{-50}{\mV}) to be retrieved. The observed binding of \gls{nadph} to the trapped
\gls{dhfr} molecules suggested that the engineered proteins remained folded and functional inside \gls{clya}.
Contact-free confinement of single proteins inside nanopores can be employed for the manipulation and
localized delivery of individual proteins and will have further applications in single-molecule analyte
sensing and enzymology studies.


%
\section{Introduction}
%
\label{sec:trapping:intro}
%

Sensors capable of the label-free interrogation of proteins at the single-molecule level have applications in
biosensing, biophysics, and enzymology~\cite{Gooding-2016,Xie-2001,Willems-VanMeervelt-2017}. In particular,
the ability to observe the behavior of individual proteins allows one to directly retrieve the rates of
kinetic processes and provides a wealth of mechanistic, energetic and structural information,which are not
readily obtained from statistically averaged ensemble (bulk) measurements~\cite{Gooding-2016}. To achieve
single-molecular sensitivity at high signal-to-noise ratios, the observational volume of the sensor should be
similar in size to the object of interest (\ie~zeptoliter-range for a protein with a radius of \SI{2.5}{\nm}).
Moreover, many kinetic processes have relatively long time scales (\eg~\SIrange{e-3}{e0}{\second}) which in
turn necessitate long observational times to obtain a statistically relevant number of events. Hence, the
protein must also remain inside the observational volume for seconds or minutes, a feat that is only possible
if the protein is either physically immobilized or trapped in a local energetic minimum that is significantly
deeper than the thermal energy~\cite{Krishnan-2010,Myers-2015}.

To counteract the random thermal motion of nanoscale objects in solution several optical, microfluidic, and
nanofluidic methodologies have been developed over the years. The optical trapping of nanoscale objects
(\SI{<50}{\nm} radius) requires the sub-diffraction limited confinement of light~\cite{Neuman-2004,Baker-2017,
Bradac-2018} which can be achieved with photonic~\cite{Yang-2009,Mandal-2010} or
plasmonic~\cite{Juan-2009,Chen-2011,
Pang-2011,Bergeron-2013,Kotnala-2014,Kerman-2015,Chen-2018,Verschueren-2019}, nanostructures. Although optical
techniques have been shown be capable of trapping proteins with a radius of
\SI{\approx2.3}{\nm}~\cite{Kotnala-2014}, the high optical intensities required and the solid-state nature of
the devices tend to not only trap the proteins but also unfold them, limiting the scope of their
applicability~\cite{Pang-2011, Verschueren-2019}.

Microfluidic techniques might offer softer alternatives for the immobilization of single molecules. The
\gls{abel} trap makes use of optical tracking to electrophoretically counteract the Brownian motion of
individual dielectric particles~\cite{Cohen-2005,Cohen-2006,Goldsmith-2010, Goldsmith-2011}, enabling the
trapping of proteins down to \SI{\approx2.9}{\nm} radius~\cite{Goldsmith-2011} and even single
fluorophores~\cite{Fields-2011}. Because this technique uses fluorescence microscopy to track the movement of
their targets, the observational time window is ultimately limited by the photobleaching of the
dye~\cite{Cohen-2005,Goldsmith-2010}.

Nanopores, which are nanometer-sized apertures in a membrane separating two electrolyte reservoirs, have been
used extensively to study single
molecules~\cite{Ma-2009,Laszlo-2016,Willems-VanMeervelt-2017,Varongchayakul-2018}. In nanopore analyses, an
electric field is applied across the membrane and information about a molecule passing through the pore is
collected by monitoring the modulations of the ionic charge current. As proteins typically transit the pore at
high velocities (\SIrange{\approx e-3}{e-2}{\meter\per\second})~\cite{Fologea-2007,Plesa-2013,Li-2013B,
Larkin-2014}, the dwell time (\ie~the duration a molecule of interest spends inside the observable volume) is
on the order of \SIrange{e-6}{e-3}{\second}. These timescales have proven sufficient for obtaining structural
information such as protein size, shape, charge, dipole moment, and
rigidity~\cite{Fologea-2007,Larkin-2014,Waduge-2017,Hu-2018,Houghtaling-2019}, but they are too brief to
efficiently study the enzymatic cycle of the majority of human enzymes (turnover numbers between
\SIrange{e-3}{e3}{\per\second}).


%
\begin{figure*}[t]
  %
  \begin{minipage}{6cm}
    %
    \begin{subfigure}[t]{5.5cm}
      \centering
      \caption{}\vspace{-3mm}\label{fig:trapping_concept_clya}
      \includegraphics[scale=1]{trapping_concept_clya}
    \end{subfigure}
    %
  \end{minipage}
  %
  \begin{minipage}{6cm}
    %
    \begin{subfigure}[t]{5.5cm}
      \centering
      \caption{}\vspace{-3mm}\label{fig:trapping_concept_dhfr}
      \includegraphics[scale=1]{trapping_concept_dhfr}
    \end{subfigure}
    %
    \\
    %
    \begin{subfigure}[t]{5.5cm}
      \centering
      \caption{}\vspace{-3mm}\label{fig:trapping_concept_sequence}
      \includegraphics[scale=1]{trapping_concept_sequence}
    \end{subfigure}
    %
  \end{minipage}

\caption[Trapping of proteins inside the {ClyA-AS} nanopore]{%
  \textbf{Trapping of proteins inside the {ClyA-AS} nanopore.}
  %
  (\subref{fig:trapping_concept_clya})
  %
  Surface representation of a type I \gls{clya-as} nanopore embedded in a planar lipid bilayer (derived
  through homology modeling from \pdbid{2WCD}~\cite{Mueller-2009}, see~\cref{sec:elec:methods:molec}), and
  colored according to its electrostatic potential in \SI{150}{\mM} \ce{NaCl} (calculated by
  \gls{apbs}~\cite{Baker-2001,Dolinsky-2004,Dolinsky-2007}, see~\cref{sec:elec:methods:elec}).
  %
  %
  (\subref{fig:trapping_concept_dhfr})
  %
  Depiction of a single \glsfirst{dhfr} molecule extended with a positively charged C-terminal polypeptide tag
  (\DHFR{4}{S}) inside a \gls{clya-as} nanopore. The secondary structure of the tag (primarily
  \textalpha-helical) was predicted by the {PEP-FOLD} server~\cite{Thevenet-2012,Shen-2014}. At negative
  applied bias voltages relative to \transi{}, the electric field ($\protect\overrightarrow{E}$) is expected
  to pull the negatively charged body of \gls{dhfr} upward ($\forceep^{\rm{body}}$) and the positively charged
  fusion tag downward ($\forceep^{\rm{tag}}$), while the electro-osmotic flow pushes the entire protein
  downward ($\forceeo$). Lastly, as the body of \gls{dhfr} is larger than the diameter of the \transi{}
  constriction, the force required to overcome the steric hindrance ($\forcest$) during full
  \cisi{}-to-\transi{} translocation is expected to be significant.
  %
  (\subref{fig:trapping_concept_sequence})
  %
  Sequence of \DHFR{4}{S} fusion tag with its positive and negative residues colored blue and red,
  respectively. The sequence of the Strep-tag starts at residue 183, and the GSS and GSA linkers are shown in
  italicized font. Note that, at \pH{7.5}, the C- and N-termini contribute one negative charge to the body and
  one positive charge to the tag, respectively.
  %
  %
  Images were rendered using \gls{vmd}~\cite{Humphrey-1996,Stone-1998}.
  %
  }\label{fig:clya_dhfr_trapping_concept}
\end{figure*}
%



To increase the observation window of proteins by nanopores, researchers have made extensive use of
noncovalent interactions. By coating solid-state nanopores with \gls{nta} receptors, the dwell time of
His-tagged proteins could be prolonged up to six orders of magnitude~\cite{Wei-2012}. In another account, the
diffusion coefficient of several proteins was reduced 10-fold \textit{via} tethering to a lipid bilayer coated
nanopore~\cite{Yusko-2011,Yusko-2017}. The decoration of biological nanopores with thrombin-specific aptamers
enabled the investigation of the binding kinetics of thrombin to its aptamer~\cite{Rotem-2012} and the
selective detection in the presence of a 100-fold excess of non-cognate proteins~\cite{Soskine-2012}.
Electrophoretic translocation of protein-DNA complexes through small nanopores (\SI{<3}{\nm} diameter)
typically results in the temporary trapping of the entire complex, which has allowed for the study of
polymerase enzymes~\cite{Lieberman-2010,Derrington-2015} and DNA-binding
proteins~\cite{Squires-2015,Yang-2018}. Although promising, none of these approaches could efficiently control
the trapping of the protein inside the nanopore or allow observation of enzyme kinetics or ligand-induced
conformational changes.

The energetic landscape of a protein translocating through a nanopore stems directly from the electrostatic,
electrophoretic, electro-osmotic, and steric forces exerted on it~\cite{Muthukumar-2014}. Given the relatively
high motility of proteins, the creation of a long lasting (\SIrange{10}{100}{\second}), contact-free trap
within a spatial region of a few nanometers mandates the presence of a deep potential energy well within the
nanopore~\cite{Movileanu-2005}. Such a potential profile was achieved by Luchian and co-workers, who showed
that the dwell time of a polypeptide inside the \gls{ahl} pore could be significantly increased by
manipulating the strength of the electro-osmotic flow~\cite{Mereuta-2014,Asandei-2016} or by placement of
oppositely charged amino acids at the polypeptide's termini~\cite{Asandei-2015}. In a similar approach, a
single barnase enzyme was trapped inside \gls{ahl} \textit{via} the addition of a positively charged
N-terminal tag~\cite{Mohammad-2008}.

Previous work in the Maglia group on protein analysis with nanopores was centered around the biological
nanopore \glsfirst{clya}---a protein with a highly negatively charged interior whose shape can best be
described by a large (\SI{\approx 5.5}{\nm} diameter, \SI{\approx 10}{\nm} height, \cisi{} \lumen{}) and a
small (\SI{\approx 3.3}{\nm} diameter, \SI{\approx 4}{\nm} height, \transi{} constriction) cylinder stacked on
top of each other (\cref{fig:trapping_concept_clya})~\cite{Soskine-2012,Soskine-2013}. Upon capture from the
\cisi{} side of the pore, certain proteins exhibited exceptionally long dwell times inside \gls{clya} from
seconds up to tens of
minutes~\cite{Soskine-2012,Soskine-2013,Soskine-Biesemans-2015,Wloka-2017,VanMeervelt-2017,Galenkamp-2018},
enabling the monitoring of conformational changes~\cite{VanMeervelt-2014, Galenkamp-2018,Biesemans-2015} and
even of the orientation~\cite{VanMeervelt-2014} of the proteins inside the nanopore. A subset of the
investigated proteins, such as lysozyme, Dendra2\_M159A and \glsfirst{dhfr}, resided inside the nanopore
\lumen{} only for hundreds of microseconds and hence could not be
studied~\cite{Soskine-2012,Soskine-Biesemans-2015}. It was observed that the size of the nanopore plays a
crucial role in the effectiveness of protein trapping, as a mere \SI{<10}{\percent} increase of \gls{clya}'s
diameter (\ie~by using \gls{clya} nanopores with a higher oligomeric state) is enough to reduce the dwell time
of proteins by almost three orders of magnitude~\cite{Soskine-2013}. Next to pore size, the charge
distribution of proteins can significantly affect their dwell time inside a nanopore. For example, the binding
of the negatively charged (\SI{-2}{\ec}) inhibitor \gls{mtx} to a modified \gls{dhfr} molecule with positively
charged fusion tag at the C-terminus (\DHFRt{}) increased the dwell time of the protein inside the \gls{clya}
nanopore from \SI{\approx3}{\ms} to \SI{\approx3}{\second} at \SI{-90}{\mV}~\cite{Soskine-Biesemans-2015}.

In this work the immobilization of individual \textit{Escherichia coli} \gls{dhfr} molecules
(\cref{fig:trapping_concept_dhfr}) inside the \gls{clya} biological nanopore (specifically type I
\gls{clya}-AS~\cite{Soskine-2013}, \cref{fig:trapping_concept_clya}) is investigated in detail. Using
nanoscale protein electrostatic simulations as a guideline, our results show that the dwell time of
\DHFR{4}{S}---a molecule identical to the above-mentioned \DHFRt{} aside from the insertion of a single
alanine residue its fusion tag (A174\_A175insA, \cref{fig:trapping_concept_sequence})---inside \gls{clya} can
be increased several orders of magnitude by manipulating the distribution of positive and negative charges on
its surface. To elucidate the physical origin of the trapping mechanism, a double energy barrier model was
developed which---by fitting the voltage dependency of the dwell times for various \gls{dhfr} mutants---yields
direct estimates of the \cisi{} and \transi{} translocation rates and the magnitude of force exerted by the
electro-osmotic flow on \gls{dhfr}\@. Our method provides an efficient means to increase the dwell time of the
\gls{dhfr} protein inside the \gls{clya} nanopore and suggests a general mechanism to tune the dwell time of
other proteins, which we believe has significant value for single-molecule sensing and analysis applications.


%
\section{Results and discussion}
%
\label{sec:trapping:results_discussion}
%

\subsection{Phenomenology of {DHFR} trapped inside {ClyA}}
%
\label{sec:trapping:phenomenology}
%

To effectively study the enzymes at the single-molecular level, one must be able to collect a statistically
significant (\ie~typically hundreds) of catalytic cycles from the same enzyme. In the case of the
\textit{E.~coli} \gls{dhfr}, which has a turnover number of \SI{\approx0.08}{\second}~\cite{Kohen-2015}, this
means that the protein must remain trapped inside the pore for tens of seconds. However, as detailed above,
such long dwell times were only achieved for \gls{dhfr} by adding a positively charged polypeptide tag to the
C-terminus of \gls{dhfr}, together with the binding of the negatively charged inhibitor
\gls{mtx}~\cite{Soskine-Biesemans-2015}. Although these long dwell times are encouraging, the requirement for
\gls{mtx} excludes the study of the full enzymatic cycle. Hence, using these previous findings as a starting
point we aim to find out how to prolong the dwell time of a tagged \gls{dhfr} molecule inside the
\gls{clya-as} nanopore without the use of \gls{mtx} and to understand the fundamental physical mechanisms that
determine the escape of \gls{dhfr} from the pore.

The structure of \DHFR{4}{S}, the tagged \gls{dhfr} molecule used as a starting point in this work, can be
roughly divided into a `body', which encompasses the enzyme itself and has a net negative charge,
$\Nbody=\mSI{-10}{\ec}$, and a `tag', which comprises the C-terminal polypeptide extension and bears a net
positive charge, $\Ntag=\mSI{+4}{\ec}$ (\cref{fig:trapping_concept_dhfr,fig:trapping_concept_sequence}). To
capture a tagged \gls{dhfr} molecule, an electric field oriented from \cisi{} to \transi{} (\ie~negative bias
voltage) must be applied across the nanopore which gives rise to an electro-osmotic flow pushing the protein
into the pore ($\forceeo$). The electrophoretic force on the body ($\forceep^{\rm{body}}$) strongly opposes
this electro-osmotic force, but is significantly weakened by the electrophoretic force on the tag
($\forceep^{\rm{tag}}$), allowing the protein to be
captured~\cite{Soskine-2012,Soskine-Biesemans-2015,Biesemans-2015}. As the body and tag of the \gls{dhfr}
molecules bear a significant amount of opposing charges, it is likely that the molecule will align itself with
the electric field, where the tag is oriented toward the \transi{} side. In this configuration the body sits
in the \gls{clya} \lumen{} and the tag is located in or near the narrow constriction. Because the body
(\SI{\approx4}{\nm}) is larger than the diameter of the constriction (\SI{3.3}{\nm}), the steric hindrance
between the body and the pore is expected to strongly disfavor full translocation to the \transi{} reservoir,
giving rise to an apparent `steric hindrance' force ($\forcest$). Finally, Poisson-Boltzmann electrostatic
calculations showed that the negatively charged interior of \gls{clya-as} creates a negative electrostatic
potential within both the \lumen{} (\SI{\approx-0.3}{\kTe}) and the constriction (\SI{\approx-1}{\kTe}) of the
pore~\cite{Franceschini-2016}, which will result in unfavorable and favorable interactions with the body and
the tag, respectively.

%
\begin{figure*}[p]
  \centering
  \begin{minipage}{6cm}
    %
    \begin{subfigure}[t]{5.5cm}
      \caption{}\vspace{-3mm}\label{fig:trapping_apbs_model}
      \includegraphics[scale=1]{trapping_apbs_model}
    \end{subfigure}
    %
    \\
    %
    \begin{subfigure}[t]{5.5cm}
      \caption{}\vspace{-3mm}\label{fig:trapping_apbs_energy}
      \includegraphics[scale=1]{trapping_apbs_pqrs}
      \includegraphics[scale=1]{trapping_apbs_energy_landscape}
    \end{subfigure}
    %
  \end{minipage}
  %
  \hspace{-5mm}
  %
  \begin{minipage}{6cm}
    %
    \begin{subfigure}[t]{5.5cm}
      \caption{}\vspace{-3mm}\label{fig:trapping_apbs_barrier_body}
      \includegraphics[scale=1]{trapping_apbs_barrier_body}
    \end{subfigure}
    %
    \\
    %
    \begin{subfigure}[t]{5.5cm}
      \caption{}\vspace{-3mm}\label{fig:trapping_apbs_barrier_tag}
      \includegraphics[scale=1]{trapping_apbs_barrier_tag}
    \end{subfigure}
    %
  \end{minipage}
  \caption[Energy landscape of \DHFR{4}{S} inside {ClyA-AS}]{%
    \textbf{Energy landscape of \DHFR{4}{S} inside {ClyA-AS}.}
    %
    (\subref{fig:trapping_apbs_model})
    %
    Coarse-grained model of \DHFR{4}{S} used in the electrostatic energy calculations in \gls{apbs}\@. The
    body of \gls{dhfr} consists of seven negatively charged (\SI{-1.43}{\ec}) beads (\SI{1.6}{\nm} diameter)
    in a spherical configuration (\SI{0.8}{\nm} spacing), whereas the tail is represented by a linear string
    of beads (\SI{1}{\nm} diameter, \SI{0.6}{\nm} spacing), each holding the net charge of three amino acids.
    %
    %
    (\subref{fig:trapping_apbs_energy})
    %
    Electrostatic energy ($\energyelec$) resulting from a series of \gls{apbs} energy calculations where the
    coarse-grained \DHFR{4}{S} bead model is moved along the central axis of the pore. The distances $\dxcis$
    and $\dxtrans$ refer to the distances between the energy minimum near the bottom of the \lumen{}
    ($z_{\rm{body}} = \mSI{3}{\nm}$) and the maximum at, respectively, \cisi{} ($z_{\rm{body}} =
    \mSI{5.7}{\nm}$) and \transi{} ($z_{\rm{body}} = \mSI{-0.6}{\nm}$)
    %
    %
    (\subref{fig:trapping_apbs_barrier_body})
    %
    Although every additional negative charge to the body of \gls{dhfr} increases the \transi{} electrostatic
    barrier by \SI{1.46}{\kT}, it has virtually no effect on the \cisi{} barrier, which increases only by
    \SI{0.04}{\kT} per charge.
    %
    %
    (\subref{fig:trapping_apbs_barrier_tag})
    %
    Addition of a single positive charge to \gls{dhfr}'s tag affects the height of the \transi{} and \cisi{}
    much more similarly, with increases of \SI{0.875}{\kT} and \SI{0.621}{\kT} per charge, respectively.
    %
  }\label{fig:apbs_simulation_results}
\end{figure*}
%


\subsection{Energy landscape of {DHFR} in {ClyA}}
%

To increase the dwell time of \gls{dhfr}---and to generalize our findings for other proteins---it is necessary
to understand how the forces exerted on \gls{dhfr} inside the pore behave as a function of the experimental
conditions (\ie~charge distribution and applied bias). In the absence of specific high affinity interactions,
\gls{dhfr}'s trapping behavior should be chiefly determined by its electrostatic interactions with the pore,
whereas the external electrophoretic and electro-osmotic forces can be viewed as modifications thereof. Hence,
we will start by investigating the molecule's electrostatic energy landscape within \gls{clya} in equilibrium
where the externally applied electric field vanishes.

To this end, we used \gls{apbs}~\cite{Baker-2001,Dolinsky-2004,Dolinsky-2007,Li-2005} to compute the
electrostatic energy ($\energyelec$) of a simplified bead-like tagged \gls{dhfr} molecule model as it moves
through the pore (\cref{fig:trapping_apbs_model,fig:trapping_apbs_energy}), similar to the approach used in
\cref{ch:electrostatics} (\cref{sec:elec:methods:elec:energy}) to investigate \gls{ssdna}
(\cref{sec:elec:frac:dna}) and \gls{dsdna} (\cref{sec:elec:clya:dna}) translocation through \gls{frac} and
\gls{clya}, respectively. The tagged \gls{dhfr} molecule was reduced to a coarse-grained `bead' model
(\cref{fig:trapping_apbs_model}), where the bulk of the protein (body) was defined by seven negatively charged
beads ($r = \SI{0.8}{\nm}$, $Q_i = \SI{-1.7143}{\ec}$) in a spherical configuration (\SI{0.8}{\nm} spacing);
and the C-terminal fusion tag (tail) was represented by nine smaller beads ($r = \SI{0.5}{\nm}$,
$\partialcharge{}_i = \num{-3}$ to \SI{+3}{\ec} depending on the amount of charges in their corresponding
amino acids) each representing three amino acids in an alpha-helix (\SI{0.6}{\nm}   spacing). Our reasons for
this simplification were two-fold: (1) the high degree of axial symmetry in the bead model resulted in a free
energy that was independent of the precise orientation of \gls{dhfr}, significantly reducing the number of
required computations; and (2) the reduced body size of the coarse-grained model compared to the full atom
model allowed for the placement of \gls{dhfr} along the entire length of the pore without nonphysical overlaps
between the atoms of \gls{clya} and \gls{dhfr} inside the \transi{} constriction, resulting in more realistic
free energies. This allows the body to pass the constriction without necessitating conformational changes,
which cannot be modeled using \gls{apbs}. Hence, this also means that the magnitude of maxima of the
electrostatic energy landscape, which occur when the charges of the bead model come close to those of the
pore, should be viewed as indicative and not absolute.

Nevertheless, the energy profile of \DHFRt{} (\cref{fig:trapping_apbs_energy}) clearly shows that there is a
significant electrostatic barrier, $\barrier^{\trans}_{\rm{es}}$, to overcome when the body of the \gls{dhfr}
moves through the constriction of the pore. Moreover, we observed a second smaller electrostatic barrier,
$\barrier^{\cis}_{\rm{es}}$, toward the \cisi{} side so that an energetic minimum exists inside \gls{clya} in
which the molecule can reside. The size difference between these two barriers clearly suggests that in the
absence of an external force (\ie~at \SI{0}{\mV} bias) the molecule will exit toward \cisi{} with overwhelming
probability. Nevertheless, the multiple distinct blockade current levels observed experimentally
(see~\cref{fig:trapping_traces}) suggest the presence of multiple energy minima within the pore, of which the
equilibrium energy landscape might provide a potential physical location.

To estimate how the charges on \gls{dhfr} impact its dwell time, we modified the number of charges in the body
from \SIrange{-10}{-13}{\ec} and recomputed the energy landscape using \gls{apbs}
(\cref{fig:trapping_apbs_barrier_body}). We found that the electrostatic energy barrier toward \cisi{} was
largely unaffected (\SI{0.04}{\kT} increase per negative charge) while the barrier for \transi{} exit
increased significantly (\SI{1.46}{\kT} increase per negative charge). The latter is a reflection of the
highly negatively charged and narrow \transi{} constriction of \gls{clya}.

Contrary to the body of \gls{dhfr}, the modification of the charge in the tag from \num{+4} to \SI{+9}{\ec}
influenced the heights of both the \cisi{} and the \transi{} barrier similarly, with increases of
\SI{0.621}{\kT} and \SI{0.875}{\kT} per positive elementary charge, respectively
(\cref{fig:trapping_apbs_barrier_tag}). This behavior can be explained by the fact that, at \gls{dhfr}'s
equilibrium position within the pore, the positively charged tag resides in the highly negatively
electrostatic well present in the \transi{} constriction of the nanopore (\cref{fig:trapping_apbs_energy}).
Moving the molecule from this position into either direction requires this Coulombic attraction to be overcome
which is directly proportional to the number of charges on the tag, irrespective of whether the molecule moves
toward \cisi{} or toward \transi{}.

Note that when an external electric field is applied, the electrophoretic and electro-osmotic forces must be
taken into account. If their net balance is positive (\ie~a net force toward \cisi{}) or negative (\ie~a net
force toward \transi{}), the electrostatic landscape will be tilted upward and downward, respectively
(see~\cref{fig:trapping_apbs_external_biased}). The capture of highly negative charged (\SI{-11}{\ec}) wild
type \gls{dhfr} molecules against the electric field~\cite{Soskine-Biesemans-2015} strongly indicates that the
electro-osmosis outweighs electrophoresis and the energy landscape will be shifted downward at \transi{}
(see~\cref{sec:trapping:biased_landscape}), resulting in higher and lower barrier heights at \cisi{} and
\transi{}, respectively. This effectively deepens the energy minimum, which should manifest as an increase of
\gls{dhfr}'s dwell time.

\subsection{Dwell time measurements}
%

The entry of a single protein into \gls{clya} results in a temporary reduction of the ionic current from the
`open pore' ($\iopen$) to a characteristic `blocked pore' ($\iblock$) level. Previously, we revealed that the
\gls{dhfr} protein shows a main current blockade with $\iresp = \iblock / \iopen \approx \SI{70}{\percent}$
(see~\cref{fig:trapping_traces}). However, occasionally deeper blocks are observed which most likely represent
the transient visit of \gls{dhfr} to multiple locations inside the nanopore. Here we assume that the dwell
time ($\dwelltime$) is simply given by the time from the initial capture to the final release where the
current level returns to the open pore current.

After gathering sufficient statistics for the dwell time events, we computed the expectation value of
$\dwelltime$ by taking the arithmetic mean of all dwell time events. This is because the chance for an escape
can be modeled as the probability of overcoming a potential barrier whose distribution function is exponential
(see~\cref{sec:trapping_appendix:escape_rates}). Note that even if the molecule transitions through multiple
meta-states with individual rates connecting each of them before it exits, the expectation value is still
given by the arithmetic mean (see~\cref{eq:tau_arithmetic_mean}).

%
\begin{table}[t]
  \begin{threeparttable}
    \centering
    \footnotesize
    
    %
    \captionsetup{width=12cm}
    \caption[Mutations and charges of all {DHFR} variants]%
            {Mutations and charges of all {DHFR} variants.}
    \label{tab:dhfr_variants}
    %
  
    \renewcommand{\arraystretch}{1.15}
    \scriptsize
    
    \begin{tabularx}{12cm}{Xllccc}
      \toprule
        & \multicolumn{2}{c}{Mutations\tnote{a}} &
      \multicolumn{3}{c}{$\chargeq_{\rm{DHFR}}$\tnote{b} [\si{\ec}]} \\
      \cmidrule(r){2-3}  \cmidrule(l){4-6}
      Name  & Body & Tag & Body & Tag & Total \\
      \midrule
      \DHFR{4}{S}   &  --- & ---
                    & \num{-10} & \num{+4} & \num{-6} \\
      \DHFR{4}{I}   & V88E P89E & ---
                    & \num{-12} & \num{+4} & \num{-8} \\
      \DHFR{4}{C}   & A82E A83E & ---
                    & \num{-12} & \num{+4} & \num{-8} \\
      \DHFR{4}{O1}  & E71Q & ---
                    & \num{-12} & \num{+4} & \num{-8} \\
      \DHFR{4}{O2}  & T68E R71E & ---
                    & \num{-13} & \num{+4} & \num{-9} \\
      \midrule
      \DHFR{5}{O1}  & E71Q      & A175K
                    & \num{-12} & \num{+5} & \num{-7} \\
      \DHFR{7}{O1}  & E71Q      & A175K A174K A176K
                    & \num{-12} & \num{+7} & \num{-5} \\
      \DHFR{5}{O2}  & T68E R71E & A175K
                    & \num{-13} & \num{+5} & \num{-8} \\
      \DHFR{6}{O2}  & T68E R71E & A175K A174K
                    & \num{-13} & \num{+6} & \num{-7} \\
      \DHFR{7}{O2}  & T68E R71E & A175K A174K A176K
                    & \num{-13} & \num{+7} & \num{-6} \\
      \DHFR{8}{O2}  & T68E R71E & A175K A174K A176K A169K
                    & \num{-13} & \num{+8} & \num{-5} \\
      \DHFR{9}{O2}  & T68E R71E & A175K A174K A176K A169K L177K
                    & \num{-13} & \num{+9} & \num{-4} \\
      \bottomrule
    \end{tabularx}

    \begin{tablenotes}
      \item[a] W.r.t. \DHFR{4}{S}: body residues \numrange{1}{163} and tag residues \numrange{164}{190};
      \item[b] Net charge of \gls{dhfr} at \pH{7.5}.
    \end{tablenotes}

  \end{threeparttable}
\end{table}
%

We observed before that the dwell time of tagged \gls{dhfr} molecules depends strongly on the applied
bias~\cite{Biesemans-2015}. That is, exponentially rising with voltage until a certain bias---which we will
refer to as the \emph{threshold voltage}---followed by an exponential fall. This behavior has also been
observed for charged peptides in \gls{ahl}~\cite{Movileanu-2005}, and is typical for a decay of a bound state
into multiple final states, such as an escape to either \cisi{} or \transi{}
(see~\cref{sec:trapping_appendix:escape_rates}). Therefore, the dwell time of the molecule, as a function of
bias voltage, $\vbias$, can be expressed as the inverse of the sum of two escape rates~\cite{Movileanu-2005}
%
\begin{align}\label{eq:double_barrier_simple}
  \dfrac{1}{\dwelltime} = \rate ={}& \rate^\cis + \rate^\trans \\
    ={}&
    \rate^\cis_0 \exp \left( -\dfrac{\alpha^\cis \ec \vbias}{\kbt} \right) +
    \rate^\trans_0 \exp \left( \dfrac{\alpha^\trans \ec \vbias}{\kbt} \right)
  \text{ ,}
\end{align}
%
where $\rate^{\cis/\trans}$ are the molecule's escape rates toward \cisi{} and \transi{}, respectively. These
can be further decomposed into attempt frequencies $\rate^{\cis/\trans}_0$ and bias dependent barriers in the
exponentials. Although this equation can help to qualitatively describe the experimental data, the reduction
of the entire protein-nanopore system to four parameters does not allow for their physical interpretation.

%
\begin{figure*}[p]
  \centering

  %
	\begin{subfigure}[t]{4cm}
		\centering
		\caption{}\vspace{-5mm}\label{fig:trapping_dwell_times_body_model}
		\includegraphics[scale=1]{trapping_dwell_times_body_model}
  \end{subfigure}
  %
	\begin{subfigure}[t]{6.5cm}
		\centering
		\caption{}\vspace{-3mm}\label{fig:trapping_dwell_times_body_data}
    \includegraphics[scale=1]{trapping_dwell_times_body_data}
  \end{subfigure}
  %

	\caption[Effect of the body charge on the dwell time of tagged {DHFR}]{%
    \textbf{Effect of the body charge on the dwell time of tagged {DHFR}\@.}
    %
    (\subref{fig:trapping_dwell_times_body_model})
    %
    Surface representation of the five tested \DHFR{4}{X} body charge mutants, and the \DHFR{4}{S}+MTX
    complex. The mutated residues are indicated for each variant. The positive charges in the fusion tag are
    colored blue. From top to bottom: \DHFR{4}{S}, \DHFR{4}{I}, \DHFR{4}{C}, \DHFR{4}{O1}, \DHFR{4}{O2}, and
    \DHFR{4}{S}$_{\text{MTX}}$.
    %
    %
    (\subref{fig:trapping_dwell_times_body_data})
    %
    Voltage dependence of the average dwell time ($\dwelltime$) inside \gls{clya-as} for \gls{dhfr} mutants in
    (\subref{fig:trapping_dwell_times_body_model}). The solid lines represent the voltage dependency predicted
    by fitting the double barrier model given by \cref{eq:double_barrier_simple} to the data
    (see~\cref{tab:fitting_parameters_simple}). The dotted lines represent the dwell times due the \cisi{}
    (low to high) and \transi{} (high to low) barriers. The threshold voltages at the maximum dwell time were
    estimated by inserting the fitting parameters into \cref{eq:threshold_voltage_simple}. The error envelope
    represents the minimum and maximum values obtained from repeats at the same condition. All measurements
    were performed at \SI{\approx28}{\celsius} in aqueous buffer at \pH{7.5} containing \SI{150}{\mM}
    \ce{NaCl}, \SI{15}{\mM} \ce{Tris-HCl}. Current traces were sampled at \SI{10}{\kilo\hertz} and filtered
    using a low-pass Bessel filter with a \SI{2}{\kilo\hertz} cutoff.
    %
  }\label{fig:trapping_dwell_times_body}
\end{figure*}
%


\subsection[Engineering {DHFR}'s dwell time by manip. of its charge]%
           {Engineering {DHFR}'s dwell time by manipulation of its charge}
%

The results from the \gls{apbs} simulations, together with the previous work with \DHFRt{} and
\gls{mtx}~\cite{Soskine-Biesemans-2015}, suggest that the dwell time of \gls{dhfr} in \gls{clya} can be
increased by the manipulation of its charge distribution. To achieve the increase in dwell time without the
need for \gls{mtx}, several non-conserved amino acids on the surface of \DHFR{4}{S} were identified and
mutated to negatively charged glutamate residues, resulting in the molecules \DHFR{4}{I}, \DHFR{4}{C},
\DHFR{4}{O1}, and \DHFR{4}{O2} (\cref{tab:dhfr_variants} and \cref{fig:trapping_dwell_times_body_model}).
These mutations modify the number of charges in the body compared to \DHFR{4}{S} and their charges are also in
different locations. For convenience, this series of mutations will be referred to as the \emph{body charge
variations} from here on out.

We performed ionic current measurements for all body charge variations for a wide range of bias voltages
(\SIrange{-40}{-120}{\mV}, see~\cref{fig:trapping_traces,fig:trapping_blockades}) and extracted the dwell
times as shown in \cref{fig:trapping_dwell_times_body_data}. All body charge variations showed the same
increase of the dwell time at low electric fields and decreased at high fields. However, we observed
differences in the threshold voltage and the magnitude of the maximum dwell time. These differences cannot
simply be explained by the total number of charges as \DHFR{4}{I} and \DHFR{4}{C} have the same charge as
\DHFR{4}{O1} but their dwell times are 10-fold lower (\cref{fig:trapping_dwell_times_body_data}). This result
implies that the location of the body charge on \gls{dhfr} plays an important role.

Additional body mutations could potentially compromise the catalytic cycle of \gls{dhfr}\@. Hence, we
proceeded by systematically increasing the number of positive charges to the fusion tag ($\Ntag$) of
\DHFR{4}{O2}, the variant that exhibited the longest dwell time, \textit{via} lysine substitution from
\SIrange{+4}{+9}{\ec} (\cref{tab:dhfr_variants,fig:trapping_dwell_times_tag}). The resulting
\DHFR{$\Ntag$}{O2} mutants  will be referred to as the \emph{tag charge variations}.

Subsequent characterization of their the dwell times revealed that the addition of positive charges to the tag
significantly increased \gls{dhfr}'s dwell time (\cref{fig:trapping_dwell_times_tag_data}). We observed a
similar increase for \DHFR{4}{O1} variants with \num{+5} and \num{+7} tag charge numbers
(see~\cref{fig:trapping_model_comparison_o1_simple}). This behavior is consistent with the tag being trapped
electrostatically inside the negatively charged \transi{}
constriction~\cite{Franceschini-2016,Movileanu-2005,Asandei-2015,Asandei-2016} and it suggests that the tag
plays a crucial role in the trapping of \gls{dhfr}, which was already observed in previous
work~\cite{Soskine-Biesemans-2015}.

%
\begin{figure*}[p]
  \centering

  %
	\begin{subfigure}[t]{4.5cm}
		\centering
		\caption{}\vspace{-5mm}\label{fig:trapping_dwell_times_tag_model}
    \includegraphics[scale=1]{trapping_dwell_times_tag_model}
  \end{subfigure}
  %
  \begin{subfigure}[t]{6.5cm}
		\centering
		\caption{}\vspace{-3mm}\label{fig:trapping_dwell_times_tag_data}
    \includegraphics[scale=1]{trapping_dwell_times_tag_data}
  \end{subfigure}
  %

	\caption[Effect of the tag charge on the dwell time of \DHFR{$\Ntag$}{O2}]{%
    \textbf{Effect of the tag charge on the dwell time of \DHFR{$\Ntag$}{O2}.}
    %
    (\subref{fig:trapping_dwell_times_tag_model})
    %
    Surface representations of all \DHFR{$\Ntag$}{O2} mutants going from $\Ntag=4$ (top) to $\Ntag=9$
    (bottom). The positively charged residues in the tag have been annotated and highlighted in blue.
    %
    %
    (\subref{fig:trapping_dwell_times_tag_data})
    %
    Voltage dependencies of the mean dwell time ($\dwelltime$) for the mutant on the left hand side, fitted
    with the double barrier model of \cref{eq:double_barrier_complex}. The annotated threshold voltages were
    computed by ESI \cref{eq:threshold_voltage_complex}. Solid lines represent the double barrier dwell time,
    and the dotted lines show the dwell times due the \cisi{} (low to high) and \transi{} (high to low)
    barriers. Fitting parameters can be found in \cref{tab:fitting_params_complex}. The error envelope
    represents the minimum and maximum values obtained from repeats at the same condition. Experimental
    conditions are the same as those in \cref{fig:trapping_dwell_times_body}.
    %
  }\label{fig:trapping_dwell_times_tag}
\end{figure*}
%

\subsection{Binding of {NADPH} reveals that {DHFR} remains folded inside the pore}
%

To verify that our \gls{dhfr} variants remained folded inside the nanopore, we measured and analyzed the
binding of \gls{nadph} to the enzyme. The addition of the \gls{nadph} co-factor to the \transi{} solution of
nanopore-entrapped \gls{dhfr} molecules induced reversible ionic current enhancements that reflect the binding
and unbinding of the co-factor to the protein (\cref{fig:trapping_nadph_trace} and
\cref{fig:trapping_nadph_o2_traces,tab:nadph_rates}).

Not all \gls{dhfr} variants were found to be suitable for \gls{nadph}-binding analysis: \DHFR{5}{O2} did not
dwell long enough inside \gls{clya-as} at \SI{-60}{\mV} ($\dwelltime = \SI{0.32\pm0.17}{\second}$) to allow a
detailed characterization of \gls{nadph} binding, whereas \gls{nadph}-binding events to \DHFR{8}{O2} were too
noisy for a proper determination of $\rate_{\rm{on}}$ and $\rate_{\rm{off}}$. No \gls{nadph} binding events to
\DHFR{9}{O2} could be observed.  \gls{nadph}-binding events to the other \gls{dhfr} variants (\DHFR{5}{O2},
\DHFR{6}{O2}, and \DHFR{7}{O2}) showed similar values for $\rate_{\rm{on}}$, $\rate_{\rm{off}}$, and event
amplitude (\cref{tab:nadph_rates}), suggesting that the binding of \gls{nadph} to \gls{dhfr} inside the
\gls{clya-as} nanopore is not affected by the number of positive charges in the C-terminal fusion tag.
Possibly, the inability of \DHFR{8}{O2} and \DHFR{9}{O2} to bind the substrate is due the lodging of
\gls{dhfr} closer to the \transi{} constriction.

%
\begin{figure*}[!t]
  \centering
  %
  \begin{subfigure}[t]{6.25cm}
    \centering
    \caption{}\vspace{-3mm}\label{fig:trapping_nadph_trace}
    \includegraphics[scale=1]{trapping_nadph_trace}
  \end{subfigure}
  %
  \begin{subfigure}[t]{5cm}
    \centering
    \caption{}\vspace{-3mm}\label{fig:trapping_nadph_ires}
    \includegraphics[scale=1]{trapping_nadph_ires}
  \end{subfigure}
  %

  \caption[Binding of NADPH to \DHFR{7}{O2}.]{%
    \textbf{Binding of NADPH to \DHFR{7}{O2}.}
    %
    (\subref{fig:trapping_nadph_trace})
    %
    Top: Typical current trace after the addition of \SI{50}{\nM} \DHFR{7}{O2} to a single \gls{clya-as}
    nanopore added to the \cisi{} reservoir at \SI{-60}{\mV} applied potential. The open-pore current
    (`$\iopen$') and the blocked pore levels (`L1') are highlighted. Bottom: Current trace showing the blocked
    pore current of a single \DHFR{7}{O2} molecule (\SI{50}{\nM}, \cisi) at \SI{-60}{\mV} applied potential
    before (left) and after (right) the addition of \SI{27}{\uM} \gls{nadph} to the \transi{} compartment.
    \Gls{nadph} binding to confined \gls{dhfr} molecule is reflected by current enhancements from the unbound
    `L1' to the \gls{nadph}-bound `L1\textsubscript{NADPH}' current levels, and showed association
    ($\rate_{\rm{on}}$) and dissociation ($\rate_{\rm{off}}$) rate constants of
    \SI{2.03\pm0.58e6}{\per\Molar\per\second} and \SI{71.2\pm20.4}{\per\second}, respectively
    (see~\cref{tab:nadph_rates}).
    %
    (\subref{fig:trapping_nadph_ires})
    %
    Dependence of the $\iresp$ on the applied potential for \DHFR{7}{O2} and \DHFR{7}{O2} bound to
    \gls{nadph}\@. All current traces were collected in \SI{250}{\mM} \ce{NaCl} and \SI{15}{\mM}
    \ce{Tris-HCl}, \pH{7.5}, at \SI{23}{\celsius}, by applying a Bessel low-pass filter with a
    \SI{2}{\kilo\hertz} cutoff and sampled at \SI{10}{\kilo\hertz}.
    %
  }\label{fig:trapping_nadph}

\end{figure*}


Work with solid-state nanopores also previously reported that electric fields inside a nanopore may unfold
proteins during translocation~\cite{Talaga-2009}, suggesting that the high degree of charge separation between
the body and tag of \gls{dhfr} might destabilize its structure. To further investigate the effect of the
applied potential on the protein structure, we analyzed the dependency of the residual current on the applied
potential (\cref{fig:trapping_nadph}). We found that the residual current of both the apo-\gls{dhfr} and the
ligand-bound enzyme increased by \SI{\approx2.5}{\percent} from \SIrange{-60}{-100}{\mV}. A voltage-dependent
change in residual current is compatible with a force-induced stretching of the enzyme. However,
single-molecule force spectroscopy experiments showed that \gls{nadph} binding increases the force required to
unfold the protein by more than 3-fold from \num{27} to \SI{98}{\pN}~\cite{Ainavarapu-2005}. As the change of
residual current over the potential was identical for both apo- and ligand-bound \gls{dhfr}
(\cref{fig:trapping_nadph}), a likely explanation is that, rather than stretching \gls{dhfr}, the applied bias
changes the position of \gls{dhfr} within the nanopore. Hence, our data suggest that, as previously reported
for several other proteins~\cite{VanMeervelt-2017,Galenkamp-2018}, the protein remains folded at different
applied bias.


\subsection{Double barrier model for the trapping of {DHFR}}

Puzzled by the strong dependence of the dwell time on the tag charge, we set out to understand the underlying
trapping mechanism by building a quantitative model. To this end, we will focus on the data set of the dwell
time of \DHFR{$\Ntag$}{O2} shown in \cref{fig:trapping_dwell_times_tag_data}.

We propose a double barrier model that describes the trapping of the molecule as a combination of escape rates
toward \cisi{} and toward \transi{} (see~\cref{sec:trapping_appendix:double_barrier}). Similar to
\cref{eq:double_barrier_simple}, the dwell time is defined in terms of the rate $\rate$ which in turn is given
by the sum of the rate for \cisi{} exit and the rate for \transi{} exit. However, now we define the rates in
terms of energy barriers:
%
\begin{align}\label{eq:double_barrier}
    \frac{1}{\dwelltime} = \rate =
        \rate_0 \exp \left( - \dfrac{\barrier^{\cis}}{\kbt} \right)
        + \rate_0 \exp \left( - \dfrac{\barrier^{\trans}}{\kbt} \right)
    \text{ ,}
\end{align}
%
where $\rate_0$ is the attempt rate and $\barrier^{\cis/\trans}$ are the energy barriers the molecule has to
overcome in order to escape toward \cisi{} and \transi{}, respectively. These can be readily decomposed into
\emph{steric}, \emph{electrostatic}, and \emph{external} contributions:
%
\begin{subequations}\label{eq:decomposition}
\begin{align}
  \barrier^{\cis} ={}&
    \barrier^{\cis}_{\steric,0}
    + \barrier^{\cis}_\static
    + \barrier^{\cis}_\ext \text{ , and,} \\
  \barrier^{\trans} ={}&
    \barrier^{\trans}_{\steric,0}
    + \barrier^{\trans}_\static
    + \barrier^{\trans}_\ext
  \text{ .}
\end{align}
\end{subequations}
%
The steric components $\barrier^{\cis/\trans}_{\steric,0}$ are defined as those interactions of the molecule
with the nanopore that are not electrostatic in nature, such as size- or conformation-related effects as
\gls{dhfr} translocates through the narrow constriction toward \transi{}.

Supported by the \gls{apbs} simulations (\cref{fig:trapping_apbs_energy}) and the corresponding barrier
height to tag charge dependency analyses (\cref{fig:trapping_apbs_barrier_tag}), we infer that the
electrostatic components $\barrier^{\cis/\trans}_{\static}$ can be further decomposed as
%
\begin{subequations}\label{eq:static-barrier}
\begin{align}
	\barrier^{\cis}_\static ={}&
			\barrier^{\cis}_{\static,0}
			+ \Ntag \ec \potbar_{\rm{tag}}^{\cis},\\
	\barrier^{\trans}_\static ={}&
			\barrier^{\trans}_{\static,0}
      + \Ntag \ec \potbar_{\rm{tag}}^{\trans}
  \text{ ,}
\end{align}
\end{subequations}
%
where $\potbar_{\rm{tag}}^{\cis/\trans}$ are the electrostatic potentials associated with the tag charge
$\Ntag$ for the \cisi{} and \transi{} barriers (\ie~the change in barrier height per additional charge in
$\Ntag$) and $\barrier^{\cis/\trans}_{\static,0}$ are two constant terms that combine all electrostatic
interactions between the protein and the pore that do not depend on $\Ntag$ (\eg~body charge related
interactions with the electric fields in the nanopore).


%
\begin{table}[t]
  \centering
  \begin{threeparttable}
    \footnotesize
    \centering
  
    %
    \captionsetup{width=12cm}
    \caption[Fitting parameters for \DHFR{$\Ntag$}{O2}]%
            {Fitting parameters for \DHFR{$\Ntag$}{O2}.}
    \label{tab:fitting_params_complex}
    %
  
    \renewcommand{\arraystretch}{1.5}
    \scriptsize
  
    \begin{tabularx}{12cm}{XXll}
      \toprule
      Parameter   & Description & Type  & Value\tnote{a} \\
      \midrule
      $\vbias$
        & Applied bias voltage
        & independent & \SIrange{40}{120}{\mV} \\
      $\Ntag$
        & Tag charge number
        & independent & \SIrange{4}{9}{} \\
      $\Nbody$
        & Body charge number
        & fixed       & \SI{-13}{} \\
      $\Neo$
        & Equivalent osmotic charge number
        & dependent   & \SI{15.5\pm0.9}{} \\
      $L$
        & Nanopore length
        & fixed       & \SI{14}{\nm} \\
      $\dxtrans$
        & Distance to \transi{} barrier\tnote{b}
        & fixed       & \SI{3.5}{\nm} \\
      $\dxcis$
        & Distance to \cisi{} barrier\tnote{b}
        & dependent   & \SI{5.21\pm1.32}{\nm} \\
      $\potbar_{\rm{tag}}^{\trans}$
        & Change of $\barrier^{\trans}_\static$ with tag charge
        & dependent   & \SI{0.860\pm0.078}{\kbt\per\ec} \\
      $\potbar_{\rm{tag}}^{\cis}$
        & Change of $\barrier^{\cis}_\static$ with tag charge
        & dependent   & \SI{0.218\pm0.167}{\kbt\per\ec} \\
      $\ln (\rateft/\si{\hertz})$
        & Effective attempt rate for the \transi{} barrier
        & dependent   & \num{-3.44\pm1.24} (\SI{3.21e-2}{\hertz}) \\
      $\ln (\ratefc/\si{\hertz})$
        & Effective attempt rate for the \cisi{} barrier
        & dependent   & \num{7.39\pm1.02} (\SI{1.62e3}{\hertz}) \\
      \bottomrule
    \end{tabularx}
  
    \begin{tablenotes}
      \item[a] Errors are confidence intervals for one standard deviation.
      \item[b] Relative to the energetic minimum inside the pore. 
    \end{tablenotes}
  
    \sisetup{table-format=2.4}
  \end{threeparttable}
  \end{table}
  %


The external forces acting on a protein trapped inside \gls{clya} under applied bias voltages manifest in the
barrier contribution $\barrier^{\cis/\trans}_\ext$. They comprise an \emph{electrophoretic} component
$\barrier^{\cis/\trans}_\ep$ and an \emph{electro-osmotic} component $\barrier^{\cis/\trans}_\eo$. The former
results from the strong electric field (\SI{\approx3.5e6}{\volt\per\meter} at \SI{-50}{\mV}) and the nonzero
net charge on the molecule, whereas the latter springs from the force exerted by \gls{clya}'s electro-osmotic
flow, which is strong enough to allow the capture of negatively charged proteins even in opposition to the
electrophoretic force~\cite{Soskine-2012,Soskine-2013,Soskine-Biesemans-2015,Biesemans-2015}. It is assumed
that the bias potential changes linearly over the length of the pore, the external energy barriers are given
by (see~\cref{sec:trapping_appendix:escape_rates})
%
\begin{subequations}\label{eq:external-barrier}
\begin{align}
  \barrier^{\cis}_\ext ={}&
    \barrier^{\cis}_\ep + \barrier^{\cis}_\eo =
    -(\Nnet + \Neo) \ec \dfrac{\dxcis}{L} \vbias \text{ , and,}\\
  \barrier^{\trans}_\ext ={}&
    \barrier^{\trans}_\ep + \barrier^{\trans}_\eo =
    +(\Nnet + \Neo) \ec \dfrac{\dxtrans}{L} \vbias
    \text{ ,}
\end{align}
\end{subequations}
%
where $\Nnet = \Nbody + \Ntag$ is the total number of charges on \gls{dhfr}\@, $L$ is the length of the
nanopore (\SI{14}{\nm}), and $\vbias$ is the negative applied bias.  The strength of the electro-osmotic force
is defined by the \emph{equivalent osmotic charge number} $\Neo$---the number of charges that must be added to
\gls{dhfr} to create an equal electrophoretic force on the molecule. Defining the electro-osmotic force in
terms of an equivalent osmotic charge number reveals its complete analogy to an electrophoretic force, which
has the benefit that the magnitudes of both forces can be readily compared. Moreover, the equivalent osmotic
charge number is an invariant related solely to the size and shape of the molecule.

The quantities $\Delta x^{\cis/\trans}$ are defined as the distances from the electrostatic energy minimum to
the \cisi{} and \transi{} barriers, which depend on the energetic landscape of \gls{clya} and on the precise
location of residence of \gls{dhfr} within the pore. To estimate these values, we can use the APBS simulations
(\cref{fig:trapping_apbs_energy}) from which we can read off that $\dxtrans \approx \SI{3.5}{\nm}$. The
\cisi{} distance is more difficult to define as the \cisi{} electrostatic barrier is much shallower. Without
external fields, it has a distance of about $\SI{\approx2.7}{\nm}$, but as we shall see in
\cref{fig:trapping_apbs_external_biased}, when the energy landscape is tilted by an external force, the
barrier that needs to be overcome is actually located at the \cisi{} entrance of the pore. In practice,
$\dxcis$ will need to be adjusted to a value between these two possibilities to give an adequate estimate and
hence will be left as a fitting parameter.

Inserting \cref{eq:decomposition,eq:static-barrier,eq:external-barrier} into \cref{eq:double_barrier} yields
the final dwell time model:
%
\begin{equation}\label{eq:double_barrier_complex}
	\begin{split}
    \frac{1}{\dwelltime} = \rate ={}&
	\ratefc \exp{%
		\left(
			-\dfrac{%
				\Ntag \ec \potbar_{\rm{tag}}^{\cis}
				- (\Nnet + \Neo) \ec \dfrac{\dxcis}{L} \vbias}{\kbt}
		\right)
		} \\
	&\enspace +
	\rateft \exp{%
		\left(
			-\dfrac{%
				\Ntag \ec \potbar_{\rm{tag}}^{\trans}
				+ (\Nnet + \Neo) \ec \dfrac{\dxtrans}{L} \vbias}{\kbt}
		\right)
    }
    \text{ ,}
	\end{split}
\end{equation}
%
where the static terms are absorbed into the prefactor to form the effective \cisi{} and \transi{} barrier
attempt rates $\rate_{\rm{eff}}^{\cis/\trans}$. The formulation of \cref{eq:double_barrier_complex} offers a
compact description of the most salient features of the molecule-nanopore system, and it enables us to
describe the dwell time of \gls{dhfr} inside \gls{clya} quantitatively as a function of the physical
properties of the system. Fitting this model to all \DHFR{$\Ntag$}{O2} data simultaneously---with both
$\vbias$ and $\Ntag$ as independent variables---leads to the fitting values in
\cref{tab:fitting_params_complex} and the plots in \cref{fig:trapping_dwell_times_tag_data}, which show
excellent accuracy considering the simplicity of our model. This is a strong indication that we captured the
essence of the trapping mechanism within our model.


%
\begin{figure*}[p]
  \centering
  %
  \begin{subfigure}[t]{27.5mm}
		\centering
		\caption{}\vspace{-3mm}\label{fig:trapping_apbs_external_model}
    \includegraphics[scale=1]{trapping_apbs_external_model}
  \end{subfigure}
  %
  \begin{subfigure}[t]{42.5mm}
		\centering
    \caption{}\vspace{-3mm}%
    \label{fig:trapping_apbs_external_equilibrium}
    \includegraphics[scale=1]{trapping_apbs_external_equilibrium}
  \end{subfigure}
  %
  \begin{subfigure}[t]{42.5mm}
		\centering
    \caption{}\vspace{-3mm}%
    \label{fig:trapping_apbs_external_biased}
    \includegraphics[scale=1]{trapping_apbs_external_biased}
  \end{subfigure}
  %
  \caption[Electrostatic landscape of a \DHFRt{} bead model in {ClyA-AS}]%
    {%
      \textbf{Electrostatic landscape of a \DHFRt{} bead model in {ClyA-AS}.}
      %
      (\subref{fig:trapping_apbs_external_model})
      %
      Simplified bead model of all \DHFR{$\Ntag$}{O2} in comparison with their corresponding full-atom
      homology model.
      %
      (\subref{fig:trapping_apbs_external_equilibrium})
      %
      The simulated electrostatic energy landscapes of all \DHFR{$\Ntag$}{O2} variants. Increasing the number
      of positive charges in the tag deepens the energetic minimum at $z \approx \mSI{3}{\nm}$.
      %
      %
      (\subref{fig:trapping_apbs_external_biased})
      %
      Approximation of the `tilting' of the energy landscapes from
      (\subref{fig:trapping_apbs_external_equilibrium}) by an applied bias voltage, which acts on the total
      effective charge of \gls{dhfr} through a constant electric field (\ie~linear potential drop) inside the
      pore (\cref{eq:external_energy}). Because the body is negative, increasing the number of positive
      charges in the tag decreases the electrophoretic force countering the electro-osmotic force, resulting
      in a higher degree of tilting. Down- and upward faces triangles indicate the presence of a local minimum
      and maximum, respectively.
      %
  }\label{fig:trapping_apbs_external}
\end{figure*}
%


\subsection{Effect of tag charge and bias voltage on the energy landscape}
%
\label{sec:trapping:biased_landscape}
%

Our equilibrium electrostatics simulations showed the existence of an electrostatic energy minimum
(\cref{fig:trapping_apbs_external_equilibrium}) at the bottom of \gls{clya}'s \lumen{} ($z = \SI{3}{\nm}$),
flanked by two maxima, located at the \transi{} constriction ($z = \SI{-0.6}{\nm}$), and at the middle of the
\cisi{} \lumen{} ($z = \SI{5.7}{\nm}$). Hence, when \gls{dhfr} resides at the electrostatic potential minimum
inside the nanopore, the largest electrostatic barrier is given by the narrower and negatively charged
\transi{} constriction while the barrier at the \cisi{} side is much shallower. Under the influence of an
external force (\ie~the bias voltage), however, the entire energy landscape will become `tilted', reducing
some barriers and enhancing others. The biased energy $\energyelec_{\vbias}$ is thus
%
\begin{align}\label{eq:external_energy}
  \energyelec_{\vbias} =  \energyelec + \Eext
  \text{ ,}
\end{align}
%
where $\Eext$ is the energy of the \gls{dhfr} molecule due to the external bias voltage. Assuming the bias
voltage changes linearly from \cisi{} to \transi{}, $\Eext$ can be approximated as
%
\begin{align}\label{eq:external_energy_model}
  \Eext =
  \begin{cases}
      \Ntot \dfrac{\vbias}{\SI{14}{\nm}} (z - \SI{11}{\nm})
              & \text{for}\; -3 < z < \SI{11}{\nm} \text{ ,} \\
      0,      & \text{for}\;  z >= \SI{11}{\nm} \text{ ,}\\
      \Ntot \vbias, & \text{for}\;  z <  \SI{-3}{\nm} \text{ ,}
  \end{cases}
\end{align}
%
with $\Ntot = \Nbody + \Ntag + \Neo$ the effective charge of charge of \gls{dhfr}. In the case of
\DHFR{$\Ntag$}{O2} (\cref{fig:trapping_apbs_external_model}), the net charge of the protein is negative
($\Nbody = -13$) and the electro-osmotic flow exerts an opposing force ($\Neo = 15.5$). This means that
increasing the number of positive charges in the tag decreases the net electrophoretic force and thus leads to
a more pronounced tilting of the entire energy landscape, deepening the electrostatic minimum
(cf.~\cref{fig:trapping_apbs_external_equilibrium,fig:trapping_apbs_external_biased}). This increases the
barrier heights at both the \cisi{} and \transi{} sides similarly, which results in the lowering of both the
\cisi{} and \transi{} escape rates and hence longer dwell times. Because the \cisi{} energy barrier in the
\lumen{} is relatively shallow, it disappears at moderate applied voltages (\SI{>-50}{\mV}). This indicates
that, under a negative applied bias voltage, the location of the \cisi{} barrier is voltage dependent and
hence the distance $\dxcis$ is not well defined. The \cisi{} barrier at low and high biases it is located at
respectively the middle of the \lumen{} ($z \approx \mSI{5.7}{\nm}$) and the \cisi{} entry ($z \approx
\mSIrange{10}{13}{\nm}$).

%
\begin{figure*}[b]
  \centering

  %
  \begin{subfigure}[t]{5.5cm}
    \centering
    \caption{}\vspace{-3mm}\label{fig:trapping_translocation_voltage}
    \includegraphics[scale=1]{trapping_translocation_voltage}
  \end{subfigure}
  %
  \begin{subfigure}[t]{5.5cm}
    \centering
    \caption{}\vspace{-3mm}\label{fig:trapping_translocation_probability}
    \includegraphics[scale=1]{trapping_translocation_probability}
  \end{subfigure}
  %

  \caption[Tag charge depend. of the threshold voltage and transl. probability]{%
    \textbf{Tag charge dependence of the threshold voltage and translocation probability.}
    %
    (\subref{fig:trapping_translocation_voltage})
    %
    Every additional positive charge in the fusion tag of the \DHFR{$\Ntag$}{O2} variants increases the
    threshold voltage (\cref{eq:threshold_voltage_complex}) by \SI{\approx5.21}{\mV}. The solid line is a
    linear fit to the data.
    %
    %
    (\subref{fig:trapping_translocation_probability})
    %
    Translocation probability voltage $\vbias^{\probability_{\rm{transl}}}$ plotted against tag charge for
    $\probability_{\rm{transl}} = \mSIlist{0.1;5;50;95;99.9}{\percent}$, shows that variants with high tag
    charge require less bias voltage to fully translocate the pore. Values were obtained through interpolation
    from \cref{eq:ptrans}, using the parameters in \cref{tab:fitting_params_complex}.
    %
  }\label{fig:trapping_translocation}
\end{figure*}
%

%
\begin{table}
  \centering
  %
  \begin{threeparttable}[t]
    \centering
    %
    \captionsetup{width=12cm}
    \caption[Summary of all threshold voltages and their dwell times]%
            {Summary of all threshold voltages and their dwell times.}
    \label{tab:threshold_voltages_and_dwelltimes}
    %
    \renewcommand{\arraystretch}{1.2}
    \footnotesize
    %
    \begin{tabularx}{12cm}{Xllll}
      \toprule
                    & \multicolumn{2}{c}{Simple model\tnote{a}}
                    & \multicolumn{2}{c}{Complex model\tnote{b}} \\
      \cmidrule(r){2-3}\cmidrule(l){4-5}
      {DHFR} variant & $\vthresh$ [mV] & $\tthresh$ [s]
                    & $\vthresh$ [mV] & $\tthresh$ [s] \\
      \midrule
      \DHFR{4}{S}   & \num{56.1} & \num{0.31} & --- & --- \\
      \DHFR{4}{I}   & \num{65.5} & \num{0.25} & --- & --- \\
      \DHFR{4}{C}   & \num{63.7} & \num{0.44} & --- & --- \\
      \DHFR{4}{O1}  & \num{71.5} & \num{2.11} & \num{75.6} & \num{2.28} \\
      \DHFR{5}{O1}  & \num{70.7} & \num{3.14} & \num{69.8} & \num{4.16} \\
      \DHFR{7}{O1}  & \num{65.4} & \num{15.6} & \num{61.6} & \num{13.9} \\
      \DHFR{4}{O2}  & \num{83.0} & \num{2.69} & \num{87.3} & \num{2.28} \\
      \DHFR{5}{O2}  & \num{78.5} & \num{5.68} & \num{79.2} & \num{4.16} \\
      \DHFR{6}{O2}  & \num{70.8} & \num{11.8} & \num{73.0} & \num{7.59} \\
      \DHFR{7}{O2}  & \num{65.4} & \num{10.8} & \num{68.1} & \num{13.9} \\
      \DHFR{8}{O2}  & \num{63.6} & \num{35.1} & \num{64.1} & \num{25.3} \\
      \DHFR{9}{O2}  & \num{59.8} & \num{80.2} & \num{60.4} & \num{46.2} \\
      \bottomrule
    \end{tabularx}
    %
    \begin{tablenotes}
      \item[a] Estimated using \cref{eq:threshold_voltage_simple} after fitting of
      \cref{eq:double_barrier_simple} the individual mean dwell times of each mutant.
      \item[b] Estimated using \cref{eq:threshold_voltage_complex} after fitting of \cref{eq:double_barrier}
      to all \DHFR{$\Ntag$}{O2} mean dwell time data.
    \end{tablenotes}
    %
  \end{threeparttable}
\end{table}
%


\subsection{Characteristics of the trapping}
%

As the double barrier model of \cref{eq:double_barrier_complex} is derived from the underlying physical
interactions of the molecule with the nanopore and with the externally applied field, the fitted parameters of
\cref{tab:fitting_params_complex} are physically relevant quantities that describe the characteristics of the
system.

The sizes of the electrostatic barriers $\potbar_{\rm{tag}}^{\cis/\trans}$ that the tag charges experience are
in direct relation to the gradients of the barrier sizes computed using the \gls{apbs} model
(\cref{fig:trapping_apbs_barrier_tag}). We find that the change of the \transi{} barrier with respect to tag
charge, $\potbar_{\rm{tag}}^{\trans} = \text{\SI{0.860}{\kbt\per\ec}}$, is in excellent agreement with the
simulated gradient of \SI{0.875}{\Vt}. The change observed for the \cisi{} barrier, $\potbar_{\rm tag}^{\cis}
= \text{\SI{0.218}{\kbt\per\ec}}$, is approximately 3-fold smaller compared to its \gls{apbs} value of
\SI{0.621}{\kbt\per\ec}. This deviation likely results from the shallowness of the \cisi{} barrier, causing it
to disappear when the energy landscape is tilted under an applied bias voltage
(see~\cref{fig:trapping_apbs_external_biased}). This gives rise to a \cisi{} barrier that lies at a location
further away from the electrostatic minimum located inside the \transi{} constriction, effectively limiting
the influence of the tag charge number on the barrier height. This claim is further corroborated by the
finding that the fitted value of $\dxcis \approx \SI{5.2}{\nm}$, which is almost twice the distance predicted
by the \gls{apbs} simulations and moves that \cisi{} barrier much closer to the \cisi{} entry.

One of the key insights we obtain from our model is the ability to directly extract information on the
strength of the osmotic flow. However, let us first observe that the equivalent electro-osmotic charge number
$\Neo\approx15.5$ is much bigger than the net charge of all tag charge variations, $\Ntag + \Nbody = -
9,\ldots, -4$, and also has the opposite sign. This is in agreement with the earlier assumption that the
electro-osmotic force is strong enough to overcome the opposing electrophoretic force and is hence responsible
for the capture of the molecule~\cite{Soskine-2012,Soskine-Biesemans-2015}. At a bias of $\vbias =
\SI{-50}{\mV}$ the electro-osmotic force exerted onto the \gls{dhfr} molecule (\cref{eq:osmoticforce}) is
%
\begin{equation}
  \forceeo = \ec \Neo \frac{\vbias}{L} \approx \SI{9}{\pN}
  \text{ .}
\end{equation}
%
The magnitude of this force is in line with those found experimentally for
DNA~\cite{Keyser-2006,vanDorp-2009,Lu-2012} and proteins~\cite{Oukhaled-2011} in solid-state nanopores.

We found the threshold voltages (obtained from the fitted model, see~\cref{eq:threshold_voltage_complex}) to
be roughly linearly dependent on the number of tag charges, with a decrease of \SI{\approx5}{\mV} per
additional positive charge (\cref{fig:trapping_translocation_voltage}). This effect is caused by the increase
of the net external force on the molecule with increasing tag charge, resulting in a simultaneous lowering of
the \transi{} barrier and raising the \cisi{} barrier. The threshold voltages and their corresponding
dwell times for both the simple and complex double barrier models are listed in
\cref{tab:threshold_voltages_and_dwelltimes}.

Another important finding of our model is that the \gls{dhfr} variations are essentially trapped by the
electrostatic forces of the pore on the tag. This can be seen from the direct exponential dependence of the
electrostatics on the tag charge as shown in \cref{eq:static-barrier}. If the molecule was trapped as a whole
between two barriers, we would rather see a dependence on the net charge on the molecule. Indeed, we verified
that such a net charge dependence cannot be fitted to the data. This suggest that the tag acts as an anchor
which is located in the electrostatic minimum created by the \transi{} constriction
(\cref{fig:trapping_apbs_energy}).

We can also determine the probability of a full translocation of \gls{dhfr} using
%
\begin{align}\label{eq:ptrans}
  \probtrans = \dfrac{\rate^{\trans}}{\rate^{\cis} + \rate^{\trans}}
  \text{ ,}
\end{align}
%
where $k^\cis$ and $k^\trans$ can be computed using the individual components given by
\cref{eq:double_barrier_complex} and the parameters in \cref{tab:fitting_params_complex}. At zero bias and
zero tag charge, we find that only \SI{0.002}{\percent} of \gls{dhfr} molecules would exit to the \transi{}
side, indicating that in the absence of an electrophoretic driving force a \cisi{} exit is much more likely
than a \transi{} exit. This is in agreement with our expectations because \gls{dhfr}'s size leads to a
significant steric hindrance when it tries to translocate through the nanopore constriction.

Finally, from \cref{eq:ptrans} we can compute the voltage $\vbias^{\probtrans}$ required to obtain a given
translocation probability (\cref{fig:trapping_translocation_probability}). The number of tag charges
significantly lowers the voltage required to achieve full translocation, for example, $\vbias^{\probtrans}$
for $\probtrans = \SI{99.9}{\percent}$ decreases from \SI{\approx-130}{\mV} to \SI{\approx-85}{\mV} going from
$\Ntag = +4$ to \num{+9}. This effect is mainly due to the lowering of the \transi{} barrier height, as the
\cisi{} escape probability voltage (\SI{0.01}{\percent} line in \cref{fig:trapping_translocation_probability})
only changes from \SI{\approx-40}{\mV} to \SI{\approx-35}{\mV} going from \num{+4} to \num{+9} tag charges.
Hence, the higher the tag charge number, the stronger the net external force which pushes the molecule through
the \transi{} constriction of the pore.

%
\section{Conclusion}
%
\label{sec:trapping:conclusion}
%
We showed previously that neutral or weakly charged proteins larger than the \transi{} constriction
(\SI{>3.3}{\nm}) of \gls{clya} can be trapped inside the nanopore for a relatively long duration (seconds to
minutes) and that their behavior can be sampled by ionic current recordings~\cite{Soskine-2013,
Soskine-2012,Soskine-Biesemans-2015,Biesemans-2015,VanMeervelt-2014,VanMeervelt-2017,Wloka-2017}. In contrast,
small proteins rapidly translocate through the nanopore due to the strong electro-osmotic flow and highly
negatively charged proteins remain inside \gls{clya} only briefly or they do not enter at
all~\cite{Soskine-2012}.

In this work, we use \gls{dhfr} as a model molecule to enhance and investigate the trapping of small and
negatively charged proteins inside the \gls{clya} nanopore~\cite{Biesemans-2015}. \gls{dhfr}
(\SIrange{3.5}{4}{\nm}) is slightly too large to pass through the \transi{} constriction, and its negatively
charged body ($\Nbody = -13$) only allowed trapping the protein inside the nanopore for a few milliseconds.
The introduction of a positively charged C-terminal fusion tag partially counterbalanced the electrophoretic
force and introduced an electrostatic trap in the \transi{} constriction of \gls{clya} that increased the
\gls{dhfr} dwell time up to minutes.

The \gls{dhfr} mutants showed a biphasic voltage dependency which was explained by using a physical model
containing a double energy barrier to account for the exit on either side of the nanopore. The model contained
steric, electrostatic, electrophoretic, and electro-osmotic components and it allowed us to describe the
complex voltage-dependent data for the different \gls{dhfr} constructs. Furthermore, fitting to experimental
data of a series of \DHFR{$\Ntag$}{O2} constructs, in which the positive charge of the tag was systematically
increased, enabled us to deduce meaningful values for \gls{dhfr}'s intrinsic \cisi{} and \transi{}
translocation probabilities, as well as an estimate of the force exerted by the electro-osmotic flow on the
protein of \SI{0.178}{\pico\newton\per\milli\volt} (\eg~\SI{9}{\pN} at \SI{-50}{\mV},
\cref{tab:fitting_params_complex}). We also showed that the \gls{apbs} simulation results of a simple bead
model for the molecule are directly related to the independently fitted parameters of the double barrier
model. In conclusion, this means that it should be possible to predict the dwell times of similar experiments
by obtaining parameters directly from these types of \gls{apbs} simulations. Interestingly, some \DHFRt{}
variants showed several well-defined current blockade levels, indicating the presence of several stable energy
minima within the pore. Even though we did not investigate this phenomenon in detail, we did observe the
presence multiple minima in the electrostatic energy landscape of \DHFRt{}. These findings suggest that the
distinct current blockades observed experimentally might correspond to different physical locations of the
protein along the length of \gls{clya}.

The double barrier model of \cref{eq:double_barrier_complex} in its current form does not adequately describe
mutations that modify the body charge distribution of \gls{dhfr}\@. This is most likely because body charge
variations close to the electrostatically trapped tag will impact the height of the barriers more strongly
than modifications on the far end of the tag. Although a model accounting for this effect could be made, it
would also make the double barrier model significantly more complex without providing any significant
advantages over a more comprehensive atomistic simulation. A more detailed discussion can be found in
\cref{sec:trapping_appendix:body_charge_variations}.

Inside the \lumen{} of \gls{clya}, proteins are able to bind to their specific substrates at all applied
potentials tested (up to \SI{-100}{\mV}), indicating that the electrostatic potential inside the nanopore and
the electrostatic potential originating from the inner surface of the nanopore did not unfold the protein.
Therefore, our results indicate that \gls{clya} nanopores can be used as nanoscale test tubes to investigate
enzyme function at the single-molecule level. Compared to the wide variety of single-molecule techniques based
on fluorescence, nanopore recordings are label-free, which have the advantage of allowing long observation
times.

The electrophoretic trapping of proteins inside nanopores is likely to have practical applications. For
example, arrays of biological or solid-state nanopores will allow the precise alignment of proteins on a
surface. In addition, proteins immobilized inside glass nanopipettes atop a scanning ion conductance
microscope~\cite{Bruckbauer-2007,Babakinejad-2013} can be manipulated with nanometer-scale precision, which
might be used, for instance, for the localized delivery of proteins. Furthermore, ionic current measurements
through the nanopore can be used for the detection of analyte binding to an immobilized protein, which has
applications in single-molecule protein studies and small analyte sensing.


%
\section{Materials and methods}
%
\label{sec:trapping:methods}
%

\subsection{Electrostatic energy landscape computation}
%
The electrostatic energy landscape of a coarse-grained \gls{dhfr} molecule translocating through a full-atom
\gls{clya-as} model was computed using the \gls{apbs}~\cite{Baker-2001}, using the approach described in
\cref{sec:elec:methods:elec:energy}. In summary, a full atom model of \gls{clya-as}~\cite{Franceschini-2016}
was prepared \textit{via} homology modeling with MODELLER software package~\cite{Sali-1993} from the wild-type
\gls{clya} crystal structure (\pdbid{2WCD}~\cite{Mueller-2009}) and its energy was further minimized using the
\gls{vmd}~\cite{Humphrey-1996} and \gls{namd} programs~\cite{Phillips-2005}. A coarse-grained bead model of
\gls{dhfr} was placed at various locations along the central axis of the pore using custom \code{Python} code
and the \code{Biopython} package~\cite{Cock-2009}. The bead model of \DHFRt{} consisted of a `body' of seven
negatively charged beads  (\SI{-1.43}{\ec}) beads (\SI{1.6}{\nm} diameter) in a spherical configuration and a
`tail' of nine smaller beads (\SI{1}{\nm} diameter, \SI{0.6}{\nm} spacing) in a linear configuration with
varying charge (depending on the net charge of the three amino acids they represent). Each atom in the
resulting \gls{clya}-\gls{dhfr} complexes was subsequently assigned a radius and partial charge (according to
the CHARMM36 force-field~\cite{Huang-2013}) with the PDB2PQR program~\cite{Dolinsky-2004,Dolinsky-2007}, and
the electrostatic was energy computed with \gls{apbs}. The net electrostatic energy cost or gain of placing a
\gls{dhfr} molecule (\ie~$\energyelec$) along the central z-axis of \gls{clya} (from
$z_{\rm{body}}=\SI{-12.5}{\nm}$ to $\SI{27.5}{\nm}$ relative to the center of the bilayer, with steps of
\SI{0.5}{\nm} inside the pore) was computed by \cref{eq:electrostatic_energy}. To this end, all systems were
solved using the non-linear \gls{pbe} (\code{npbe}) in two steps with the automatic solver (\code{mg-auto}):
(1) a coarse calculation in a box of \SIgrid{40;40;110}{\nm} with grid lengths of
\SIgrid{0.138;0.138;0.122}{\nm} and multiple Debye-H\"{u}ckel boundary conditions (\code{bcfl mdh}), followed
by (2) a finer focusing calculation in a box of \SIgrid{15;15;70}{\nm} with grid lengths of
\SIgrid{0.052;0.052;0.052}{\nm} that used the values of the coarse calculation at its boundaries. The
monovalent salt concentration was set to \SI{0.150}{\Molar} with a radius of \SI{0.2}{\nm} for both ions. The
solvent and solute relative permittivities were set to \num{78.15} and \num{10}, respectively~\cite{Li-2013}.
Both the charge density and the ion accessibility maps were constructed using cubic B-spline discretization
(\code{chgm spl2} and \code{srfm spl2}).


\subsection{Protein mutagenesis, overexpression, and purification}
%

All \gls{dhfr} variants were constructed, overexpressed and purified using standard molecular biology
techniques~\cite{Soskine-Biesemans-2015,Biesemans-2015}, as described in full detail in the supplementary
information (see~\cref{sec:trapping_appendix:dhfr_cloning}). Briefly, the \DHFR{4}{S} DNA construct was built
from the {pT7-SC1} plasmid containing the \DHFRt{} construct (see ref.~\cite{Soskine-Biesemans-2015}) by
inserting an additional alanine residue at position 175 (located in the fusion tag) with site-directed
mutagenesis. All other variants were derived---again using site-directed mutagenesis---either directly from
\DHFR{4}{S} or from a variant thereof. The plasmids of each \gls{dhfr} variant were used to transform E.
cloni\textsuperscript{\textregistered} EXPRESS BL21(DE3) cells (Lucigen, Middleton, USA), and the \gls{dhfr}
proteins they encode were overexpressed overnight at \SI{25}{\celsius} in a liquid culture. After the
bacterial cells were harvested by centrifugation, the overexpressed proteins were released into solution
through lysis---using a combination of at least a single freeze-thaw cycle, incubation with lysozyme, and
probe tip sonification. Finally, the \gls{dhfr} proteins were purified from the lysate with affinity
chromatography with Strep-Tactin\textsuperscript{\textregistered} Sepharose\textsuperscript{\textregistered}
(IBA Lifesciences, Goettingen, Germany), aliquoted, and stored at \SI{-20}{\celsius} until further use.


\subsection{ClyA-AS overexpression, purification and oligomerization}
%

\gls{clya-as} oligomers were prepared as described previously~\cite{Soskine-2013}, and full details can be
found in \cref{sec:trapping_appendix:dhfr_cloning}. Briefly, the \gls{clya-as} monomers were overexpressed and
purified in a manner similar to that for \gls{dhfr}, with the largest difference being the use of
\ce{Ni-NTA}-based affinity chromatography. After purification, \gls{clya-as} monomers were oligomerized in
\SI{0.5}{\percent} \textbeta-dodecylmaltoside (GLYCON Biochemicals GmbH, Luckenwalde, Germany) at
\SI{37}{\celsius} for \SI{30}{\minute}. The type I oligomer (12-mer) was isolated by gel extraction from a
blue native \gls{page}.


\subsection{Electrical recordings in planar lipid bilayers}
%

Electrical recordings of individual \gls{clya-as} nanopores were carried out using a typical planar lipid
bilayer setup with an AxoPatch 200B (Axon Instruments, San Jose, USA) patch-clamp
amplifier~\cite{Maglia-2010,Soskine-2012}. Briefly, a black lipid membrane consisting of
1,2-diphytanoyl-sn-glycero-3-phosphocholine (Avanti Polar Lipids, Alabaster, USA), was formed inside a
\SI{\approx 100}{\micro\meter} diameter aperture in a thin polytetrafluoroethylene film (Goodfellow Cambridge
Limited, Huntingdon, England) separating two electrolyte compartments. Single nanopores were then made to
insert into the \cisi{}-side chamber (grounded) by addition of \SIrange{0.01}{0.1}{\nano\gram} of
preoligomerized \gls{clya-as} to the buffered electrolyte (\SI{150}{\mM} \ce{NaCl}, \SI{15}{\mM} \ce{Tris-HCl}
\pH{7.5}). All ionic currents were sampled at \SI{10}{\kilo\hertz} and filtered with a \SI{2}{\kilo\hertz}
low-pass Bessel filter. A more detailed description can be found in
\cref{sec:trapping_appendix:nanopore_experiments}.


\subsection{Dwell time analysis and model fitting}
%

The dwell times of the \gls{dhfr} protein blocks were extracted from single-nanopore channel recordings using
the `single-channel search' algorithm of the pCLAMP 10.5 (Molecular Devices, San Jose, USA) software suite.
The process was monitored manually, and any events shorter than \SI{1}{\ms} were discarded. We processed the
dwell time data, and fitted the double barrier model to it, using a custom Python code employing the
\code{NumPy}~\cite{vanderWalt-2011}, \code{pandas}~\cite{McKinney-2010}, and \code{lmfit}~\cite{Newville-2014}
packages. Fitting of the exponential models to the data was performed using non-linear least-squares
minimization with the Levenberg-Marquardt algorithm as implemented by \code{lmfit}. Note that to improve the
robustness and quality of the fitted parameters, we moved the exponential prefactors into the exponential and
we fitted the natural logarithm of the equations to the natural logarithm of the dwell time data.

\subsection{Analytical expression for the threshold voltages}
%

An analytic expression for the threshold voltage $\vthresh$---the bias voltage at maximum dwell time---can
found as the bias voltage for which $\dfrac{d\rate}{d\vbias} = 0$. For the simple double barrier model given
by \cref{eq:double_barrier_simple} this becomes
%
\begin{align}\label{eq:threshold_voltage_simple}
	\vthresh = \dfrac{ \left[
	\frac{%
	\log \left( \rateft / \ratefc \right)+
	\log \left( -N_{\rm{eq}}^\trans / N_{\rm{eq}}^\cis \right)
	}{%
		N_{\rm{eq}}^\trans - N_{\rm{eq}}^\cis
  }\right]}%
  { \ec / \kbt }
  \text{ ,}
\end{align}
%
whereas for the more complex model given by \cref{eq:double_barrier_complex} it becomes
%
\begin{equation}\label{eq:threshold_voltage_complex}
	\vthresh = - \dfrac{\left[
	\frac{%
	\log \left( \rateft / \ratefc \right)
	+ \log \left( \dxtrans / \dxtrans \right)
	+ \left( \potbar_{\rm{tag}}^{\cis} - \potbar_{\rm{tag}}^{\trans} \right)\Ntag
	}{%
		\left( \Nnet + \Neo \right) \left(\dxcis + \dxtrans \right) / L
  }\right]}%
  { \ec / \kbt }
  \text{ .}
\end{equation}
%



%%%%%%%%%%%%%%%%%%%%%%%%%%%%%%%%%%%%%%%%%%%%%%%%%%
% Keep the following \cleardoublepage at the end of this file,
% otherwise \includeonly includes empty pages.
\cleardoublepage

% vim: tw=70 nocindent expandtab foldmethod=marker foldmarker={{{}{,}{}}}
