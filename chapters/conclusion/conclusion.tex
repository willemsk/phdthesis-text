% !TeX root = ../../thesis.tex
\chapter{Conclusions and perspectives}
%
\label{ch:conclusion}
%

\epigraphhead[\epipos]{%
\epigraph{%
%
  ``But just because something does not have an ending doesn't mean it doesn't have a conclusion.''
%
}{%
  \textit{`Ky' in `Use of Weapons' by Iain M. Banks}
%
}}
%
%


In this dissertation, I have demonstrated that computational modeling---in combination with analytical
physical models and systematic experiments---can be a powerful tool to shed light on (1) the physical
mechanisms that drive molecular transport through (biological) nanopores, and (2) the diverse physical
conditions that occur within them. In particular, this work shows that although a great deal of information
can be obtained from \emph{equilibrium} electrostatic simulations, which effectively mimic the nanopore
without an applied bias voltage, the transport of analyte molecules such as \gls{dna} or proteins is governed
by a complex interplay of extrinsic (\ie~(di)electrophoretic and electro-osmotic) and intrinsic
(\ie~electrostatic, steric, and entropic) forces. It is for this reason that a \emph{non-equilibrium}
simulation framework was developed: the \glsfirst{epnp-ns} equations. This framework enables one to accurately
predict the transport of ions, water molecules, and analyte molecules through biological nanopores, providing
a means to paint a complete picture of the sensing process. The following sections summarize the key
conclusions that can be drawn from this dissertation, followed by my perspective on the future of nanopore
sensing, and the role of computational modeling therein.


%
\section{General conclusions}
%
\label{sec:con:conclusions}
%

% Develop methodologies for accurate modeling of biological nanopores
% Investigate the electrostatics of biological nanopores
% Elucidate the trapping behavior of a protein inside a biological nanopore
% Mapping the transport properties of a biological nanopore


\subsubsection{The electrostatics of biological nanopores influence all transport processes.}
%

The importance of electrostatics within biological nanopores cannot be understated. Electrostatic interactions
have an influential hand in defining the nanofluidic properties of most biological nanopores. Not only do they
determine conductance characteristics such as ion selectivity~\cite{Ramirez-2007} and current
rectification~\cite{Wang-2014}, they are also the fundamental enabler of the
\glsfirst{eof}~\cite{Bocquet-2010}, and finally, they strongly mediate the translocation process of analyte
molecules \textit{via} long-range electrostatic forces~\cite{Maglia-2008,Asandei-2015b,Fahie-2015b}.

In~\cref{ch:electrostatics}, we made use of \glsfirst{pb} theory to investigate the influence of charge
reversal mutations, in combination with the ionic strength and pH of the electrolyte, on the equilibrium
electrostatics (\ie~without an applied bias voltage) of the biological nanopores \glsfirst{plyab},
\glsfirst{frac}, and \glsfirst{clya}. We found that introducing charge reversal mutations
(negative-to-positive) within the constrictions of the \gls{plyab} (\gls{plyab-e2} \textit{versus}
\gls{plyab-r}) and \gls{frac} (\gls{wtfrac} \textit{versus} \gls{refrac}) pores translated itself in the full
reversal of the electrostatic potential at those locations, even at high ionic strengths (\SI{1}{\Molar}).
Given the intimate link between the potential and the \gls{eof} (through the \gls{edl}), we hypothesized that
this could have a profound effects on the magnitude of the \gls{eof} for \gls{plyab-r}~\cite{Huang-2020}, and
would even change its directionality for \gls{refrac}~\cite{Huang-2017}, two claims supported by protein
trapping experiments. Additionally, we show that, whereas the magnitude of the electrostatic potential within
the negatively charged wild-type \gls{frac} (\gls{wtfrac}) is strongly reduced upon lowering the pH from
\numrange{7.5}{4.5}, that of the positively charged \gls{refrac} is barely affected. This finding also
confirms the experimental observations~\cite{Huang-2017}. In the case of \gls{clya}, we studied four different
mutants of \gls{clya-as}~\cite{Soskine-2013}: \gls{clya-r} (S110R), \gls{clya-rr} (S110R/D64R),
\gls{clya-rr56} (S110R/Q56R), and \gls{clya-rr56k} (S110R/Q56R/Q8K). Experiments have shown that while all
variants are able to translocate \gls{dsdna} at high salt concentrations
(\SI{>2}{\Molar})~\cite{Franceschini-2013}, only \gls{clya-rr} is able to do so at physiological ionic
strengths (\SI{0.15}{\Molar})~\cite{Franceschini-2016}. This is of importance for applications were the
electrostatic interactions should be as `natural' as possible, such as \gls{dna} mapping or sequencing with
the help of a processive enzyme. We found the S110R mutation at the \cisi{} entry to effectively eliminate the
negative electrostatic potential observed at the \cisi{} entrance of \gls{clya-as}, which in turn is likely to
promote the entry of the \gls{dna} within the \lumen{}. Interestingly, the D64R mutation of \gls{clya-rr}
(\SI{+2}{\ec} per subunit), but not the Q56R mutation of \gls{clya-rr56} (\SI{+1}{\ec} per subunit), resulted
in a reversal the electrostatic potential in the middle of the \lumen{}, the consequences of which will be
discussed in the next paragraph. Even though the Q8K mutation lowered the magnitude of the negative potential
within the constriction of \gls{clya-rr56k}, it remained significantly higher compared to the rest of the
pore. As expected, at high ionic strengths (\SI{2.5}{\Molar}), however, no differences between the mutants
could be observed, in agreement with experiments~\cite{Franceschini-2013,Franceschini-2016}

To gain a better understanding of the electrostatic \emph{energy} barriers that arise during the translocation
of analyte molecules, we computed the electrostatic energy costs (or gains) that result from the movement of a
\gls{dna} molecule (in~\cref{ch:electrostatics}) or a protein (in~\cref{ch:trapping}) along the length of the
pore. In the case of \gls{ssdna} translocation through \gls{refrac}, we found that the D10R charge reversal
lowered the energy barrier for entering the constriction of the pore from \SIrange{38}{14}{\kbt}.
Back-of-the-envelope calculations show that even at moderate applied bias potentials (and even ignoring the
\gls{eof}), the barrier in \gls{refrac} can be readily overcome, whereas the one in \gls{wtfrac} cannot. This
is in agreement with the \gls{dna} translocation experiments~\cite{Wloka-2016}. The situation for \gls{dsdna}
translocation through \gls{clya} (at \SI{0.15}{\Molar} ionic strength) is more complex, as the key
charge-reversal mutation, D64R, is not located within the constriction of the pore, but rather in the middle
of the large \cisi{} \lumen{}. We hypothesize that, just as the S110R mutation promotes the entry of the
\gls{dna} from the \cisi{} side, so does the D64R mutation encourage the strand to penetrate deeper into the
\lumen{} of the pore by \emph{lowering} rather than increasing the electrostatic energy. Under (positive)
applied bias voltages, this would allow the \gls{dna} to accumulate sufficient force to overcome
\SI{\approx40}{\kbt} energy barrier within the constriction. At high salt concentrations (\SI{2.5}{\Molar}),
the energy landscape of all variants is similar, and the energy barrier is reduced to a mere \SI{10}{\kbt},
which can be easily overcome by the electrophoretic force. Interestingly, the energy of the translocating
\gls{dna} molecule appears to fluctuate every \SI{36}{\angstrom} with \SI{\approx2}{\kbt}, suggesting that it
might rotate during translocation. Finally, the strong electrostatic repulsion results in the strong
confinement of the \gls{dsdna} to the center of the pore, even at high salt concentrations.

As an example for the role of electrostatics during protein translocation, we computed the electrostatic
energy of \DHFRt{}, an engineered variant of the \textit{E. coli} \glsfirst{dhfr} enzyme containing a
positively charged C-terminal fusion tag attached to its negative charged core~\cite{Soskine-Biesemans-2015},
as it traverses the \gls{clya} nanopore. Rather than a single barrier at the constriction, the asymmetric
charge distribution of \DHFRt{} gives rise to an energy landscape with an explicit energy minimum, whose depth
is proportional to the number of positive charges in the fusion tag. Hence, this indicates that the protein is
electrostatically trapped within the pore. However, superimposing the externally applied electric field upon
the electrostatic energy of a translocating molecule will `tilt' the energy landscape. In the case of
\DHFRt{}, this tilting caused the magnitude of the \transi{} barrier to decrease, whereas the smaller \cisi{}
barrier was quickly erased and essentially moved to the \cisi{} entry of the pore.

It is clear that the introduction of specific charges, at the right location, can significantly impact the
transport of ions, water molecules, and analytes through biological nanopores. This can be effected either
directly, through electrostatic interactions, or indirectly, through the \gls{eof}, both of which are mediated
by the ionic strength and pH of the electrolyte. Nevertheless, even though equilibrium electrostatic
calculations already provide a wealth of information, applying an external electric field `tilts' the energy
landscape, causing seemingly impassable barriers (at equilibrium) to diminish or even disappear. Hence, a full
understanding of molecular transport through nanopores necessitates the use of non-equilibrium simulations.


\subsubsection{Accurate, non-equilibrium continuum modeling of biological nanopores requires extensive corrections.}
%

To gain even deeper insights into the nanoscale forces involved in nanopore-based sensing, it is necessary to
go beyond equilibrium electrostatics and approximate analytical models, towards modeling the explicit dynamics
of ion and water transport through nanopores. Whereas \gls{md} simulations are still superior in terms of
accuracy~\cite{Aksimentiev-2005,DeBiase-2016,Basdevant-2019}, the heavy computational cost associated with
simulating the motion of a million individual atoms in \SI{1}{\fs} time steps still limits its usefulness for
high-throughput analysis of long-timescale (\ie~\SI{>1}{\us}) phenomena~\cite{Vendruscolo-2011,Phillips-2020}.
Hence, a fair share of literature has been devoted to the development of continuum models, typically based on
\gls{pnp-ns} theory, of
solid-state~\cite{Daiguji-2004,Cervera-2005,White-2008,Lu-2012,Chaudhry-2014,Laohakunakorn-2015,Hulings-2018,Rigo-2019,Melnikov-2020}
and
biological~\cite{Noskov-2004,Cozmuta-2005,OKeeffe-2007,Simakov-2010,Pederson-2015,Simakov-2018,Aguilella-Arzo-2020}
(\ie~mostly on \gls{ahl}) nanopores. Such models are orders of magnitude more efficient, both in terms of
solution time and the ease of data analysis. However their applicability for nanoscale transport problems is
often (rightly) questioned~\cite{Corry-2000,Collins-2012}, which has led to the development of modified
continuum transport equations that attempt to compensate for some of the simplifications introduced by
mean-field
theory~\cite{Noskov-2004,Baldessari-2008-1,Daiguji-2010,Simakov-2010,Lu-2011,Burger-2012,Chen-2016,Liu-2020}.
In~\cref{ch:epnpns}, we implemented a novel continuum framework geared specifically towards the modeling of
biological nanopores: the \glsfirst{epnp-ns} equations. It takes into account steric ion-ion
interactions~\cite{Daiguji-2010,Kilic-2007,Lu-2011,Liu-2020}, the influence of the nanopore walls on the local
ion diffusivity/mobility~\cite{Makarov-1998,Noskov-2004,Pederson-2015,Hulings-2018,Wilson-2019} and the water
viscosity~\cite{Pronk-2014,Vo-2016,Hsu-2017}, as well as the concentration dependencies of ion
diffusivity/mobility~\cite{Baldessari-2008-1,Burger-2012}, and solvent relative permittivity~\cite{Chen-2016},
viscosity and density~\cite{Hai-Lang-1996}. All these dependencies were robustly parameterized using empirical
relations, fitted to experimental or \gls{md} data obtained from literature sources.

Next, in~\cref{ch:transport}, we applied the \gls{epnp-ns} equations to a 2D-axisymmetric model of the
\gls{clya-as} nanopore, whose geometry and charge distribution was derived from series of equilibrated atomic
structures obtained from a \SI{30}{\ns}-long \gls{md} simulation of \gls{clya-as} homology model. The
solutions of the \gls{epnp-ns} equations provided detailed descriptions of the ion concentrations, the
electrostatic potential, the \gls{eof} velocity, and the pressure distributions within the pore, over wide
range of ionic strengths (\SIrange{0.005}{5}{\Molar}) and applied bias voltages (\SIrange{-200}{+200}{\mV}).
Integration of the cation and anion fluxes through the pore gave direct access to the total ionic current, but
also the true ion selectivity, a property that is difficult to pin down experimentally. Notably, the ionic
conductance of \gls{clya} predicted by the \gls{epnp-ns} equations matched closely with the values obtained
from carefully calibrated single-channel recordings, in contrast to the classical \gls{pnp-ns} equations. The
(cat)ion selectivity, although strongly dependent on the ionic strength and bias voltage, also corresponded to
the permeability ratio derived from the reversal potential measurements. As with the transport of \gls{dna}
(see~\cref{ch:electrostatics}) and proteins (see~\cref{ch:trapping}) through \gls{clya}, we found the
transport of ions to subject to the electrostatic energy barriers present within the pore. These barriers did
not only influence the conductive properties of the pore, but also profoundly impacted the availability of
anions with \gls{clya}'s \lumen{}, particularly at negative bias voltages, a condition that may influence the
structure and activity of enzymes trapped within \gls{clya}. Analysis of the \gls{eof} magnitude showed a
non-monotonic dependence on the ionic strength, with a maximum at \SI{0.5}{\Molar} \ce{NaCl}. This knowledge
may be exploited to fine-tune the force balance exerted on analyte molecules and improve their trapping
efficiency. Interestingly, \gls{clya}'s complex charge distribution and the accompanying non-uniform
\gls{edl}, resulted in confined (\SI{\approx1}{\cubic\nm}) zones of high pressure (\SIrange{5}{30}{\atm})
localized at the \cisi{} entry, the middle of the \lumen{}, and within the \transi{} constriction. The
pressure differences induced by these zones may contribute significantly to the net force exerted on
translocating particles, confirming the importance of using the full hydrodynamic stress tensor to calculate
the force, rather than just the Stokes' drag.

The \gls{epnp-ns} equations, together with a suitable model of the pore, constitute a significant advance in
the continuum simulation of biological nanopores, and the modeling of nanoscale transport phenomena in
general. It is an important step towards building a tool that can help to move the field forward on various
fronts, including elucidating the link between the observed variations in ionic current and the physical
events that cause them, evaluating the transport properties of existing nanopores, and guiding the rational
design of novel characteristics.


\subsubsection{Trapping a protein within {ClyA} involves a complex interplay of forces.}
%

The ability to retain proteins, and particularly enzymes, for long periods of time within a nanopore is of
crucial importance for applications such as single-molecule
enzymology~\cite{Willems-VanMeervelt-2017,VanMeervelt-2014,Wloka-2017,Thakur-2019,Galenkamp-2020} and
biomarker sensing~\cite{VanMeervelt-2017,Galenkamp-2018,Zernia-2020}. Previous experiments with \gls{clya}
have shown that, whereas large(r) proteins (that can still enter the pore through the \cisi{} entry) can
remain trapped for seconds to
minutes~\cite{Soskine-2013,Soskine-2012,Soskine-Biesemans-2015,Biesemans-2015,VanMeervelt-2014,VanMeervelt-2017},
smaller proteins (similar in diameter to the \transi{} constriction) translocate through the pore within
milliseconds~\cite{Soskine-2012}. This is too quickly properly sample any conformational changes, enzymatic
reactions or ligand binding. However, other studies revealed that the dwell time of
peptides~\cite{Movileanu-2005,Asandei-2015,Asandei-2016} within biological nanopores could be lengthened by
orders of magnitude by introduction of a large permanent dipole moment.

In~\cref{ch:trapping}, a systematic set of experiments was performed with the \DHFRt{} enzyme. Previous
experiments revealed that he binding of the negatively charged inhibitor molecule \glsfirst{mtx} to \DHFRt{}
extended its peak mean dwell time within the pore from \SIrange{0.3}{9}{\second}, a 30-fold
increase~\cite{Soskine-Biesemans-2015}. We successfully eliminated the need for \gls{mtx} (which would prevent
a study of \gls{dhfr}'s enzymatic properties) by (1) introducing three additional negative charges near the
\gls{mtx} binding site, and (2) by increasing the number of positively charged residues in the fusion tag from
\numrange{+4}{+9}. The former raised the dwell time by an order of magnitude to \SI{2}{\second}, whereas the
latter resulted in dwell times of \SI{4}{\second} (\num{+5}), \SI{8}{\second} (\num{+6}), \SI{14}{\second}
(\num{+7}), \SI{25}{\second} (\num{+8}), and \SI{46}{\second} (\num{+9}). All variants were still able to bind
to their cofactor, \ce{NADH}, indicating that neither our modifications, nor the confinement within the pore,
caused structural disruption or unfolding. Notably, the \num{+7} \DHFRt{} variant recently enabled the
real-time detection \gls{dhfr}'s conformational changes of during its enzymatic cycle~\cite{Galenkamp-2020}.

The dwell times of all \DHFRt{} variants exhibited a biphasic relation with the applied bias voltage, where it
first rose exponentially up until a certain threshold voltage, followed by an exponential
decay~\cite{Biesemans-2015}, similar to the escape rate of an \ta-helical peptide trapped within
\gls{ahl}~\cite{Movileanu-2005}. Such a complex behavior was explained by a shift in \gls{dhfr}'s preference
to exit \gls{clya} from the \cisi{} entry at low voltages, to the \transi{} entry at high applied biases. To
quantify this process, we developed a double energy barrier physical transport model (\ie~one barrier for each
entry), containing steric, electrostatic, electrophoretic, and electro-osmotic energy components. By
meticulously mapping out these voltage dependencies, together with an extensive set of equilibrium
electrostatic energy calculations for all translocating \DHFRt{} variants, we successfully parameterized this
double barrier model, yielding meaningful values for \gls{dhfr}'s intrinsic \cisi{} and \transi{}
translocation probabilities, as well as an estimate of the force exerted by the \gls{eof} on the protein of
\SI{0.178}{\pico\newton\per\milli\volt} (\eg~\SI{9}{\pN} at \SI{-50}{\mV}).

Interestingly, the electrostatic energy barriers obtained from the electrostatic simulations matched closely
with those obtained from the fitting of the double barrier model, indicating that the simulations closely
represent the true physical situation. Moreover, the magnitude of \gls{eof} force on \gls{dhfr} is equivalent
to a net charge of \SI{+15.5}{\ec}---a value significantly larger than that of the typical protein
(\SI{99}{\percent} lies between \SI{\pm10}{\ec}~\cite{Requiao-2017}), \emph{quantitatively} confirming that
the \gls{eof} is the dominating force in protein capture by \gls{clya}. Given the general nature of the double
barrier model (\ie~it includes no explicit structural information of the trapped protein besides its charge),
it may be applicable to other proteins and even other pores, providing a solid approach towards rationally
engineering the dwell time of small proteins within nanopores.



%
\section{Future perspectives}
%
\label{sec:con:perspectives}
%

\subsubsection{Technical improvements to the {ePNP-NS} framework.}
%

Even though the \gls{epnp-ns} framework presented in this dissertation constitutes a solid step forwards for
the accurate and efficient modeling of (biological) nanopores with continuum methods, it is still far from the
being a `user-friendly' tool that can be utilized by nanopore researchers. Currently, the \gls{epnp-ns}
equations have been implemented in the COMSOL Multiphysics software, a versatile, but \emph{commercial} finite
element solver. In the spirit of \emph{open science}, the first (technical) improvement I would endeavor is
porting the current implementation into a free (as in speech, not just beer) and open-source \gls{pde} solver
framework, such as \code{FEniCS} (finite element) or \code{OpenFOAM} (finite volume). Besides making the full
program available, free of charge, to anyone that wishes to use it, this would give full control of the code
the users, enabling them to utilize, adapt, and contribute as they see fit. Additionally, this would provide
clear, straightforward, and uniform paths towards a full integration both the upstream (molecular modeling and
geometry generation) and downstream (data analysis and visualization) workflows. The next challenge to tackle
would be to implement a true 3D model, rather than the 2D-axisymmetric representation. Even though this would
likely result in a significant increase in computational time, the lack of axisymmetric averaging would remove
a layer of abstraction from the simulation that poses an inherent limit on the accuracy of the results that
can be obtained with them. Full 3D models could properly capture the complex, corrugated geometries present in
even the most radially symmetric nanopores. As a result, this would also greatly improve the realism of the
nanopore's fixed charge density, as it eliminates overlapping charges, and charges that fall within the
electrolyte rather the solid dielectric of the pore. Note that the 2D-axisymmetric approach could still remain
a valuable tool for situations were speed, rather than accuracy, is of the essence---as would be the case when
screening many nanopore mutant or conditions. A final technical improvement would be to further automate the
simulation pipeline, at the very least for specific and verified pores, with the goal of providing a
user-friendly tool that is actually usable by less `technically inclined' users.\footnotemark%
%
\footnotetext{
%
Note that unlike me, these are usually the scientists that perform useful experiments and breakthroughs with
actual real-world applications. 
%
}
%
Perhaps this last enhancement is the most important of all, given that any piece of software can only useful
only if it is actually being used.


\subsubsection{Scientific improvements to the {ePNP-NS} framework.}
%

Also from a scientific point-of-view several advances as possible. First, the empirical parameterization that
powers the material properties in the \gls{epnp-ns} should be replaced by a first principles approach that
truly captures the underlying physics, rather mimicking its phenomenological results. In other words, rather
than forcing the approximated physics to be more accurate, it is better to improve or remove the
approximations themselves. To date, perhaps the most powerful example of such a first principles methodology
is the Poisson-Nernst-Planck-Bikerman model developed by Liu and Eisenberg~\cite{Liu-2020}, which, in their
own words ``describes the size, correlation, dielectric, and polarization effects of ions and water in aqueous
electrolytes at equilibrium or nonequilibrium all within a unified framework.'' Nevertheless, despite its
phenomenological approach, the existing \gls{epnp-ns} parameterization remains quantitatively useful, and may
serve as a barometer for which approximations one should address first. On the short term, a more attainable
goal is the explicit implementation of the electrolyte pH, through the inclusion of \ce{H+} and \ce{OH-} ions,
in combination with the water equilibrium reaction and buffer molecules. Besides improving the general realism
of the framework, it would enable the self-consistent modeling of the nanopore's fixed charge distribution
\textit{via} the effective $\pKa$ of all ionizable groups. This is not only relevant for biological nanopores,
but prove even more instrumental for accurately simulating the solid-liquid interfaces present in solid-state
devices.


\subsubsection{Potential applications of the {ePNP-NS} framework.}
%

On the application side, there are several low hanging fruits: (1) simulating nanopore mutants, and (2)
computing the net force exerted on translocating analyte molecules. Recently, we applied our \gls{epnp-ns}
approach to the \gls{plyab-e2} and \gls{plyab-r} pores described in \cref{ch:electrostatics}, and found that
it was indeed able to capture the experimentally observed differences in the \gls{eof} between these two
mutants~\cite{Huang-2020}. The current framework could be applied to numerous other pores and mutants thereof,
to help explain and perhaps even guide experiments. Even more exciting is the self-consistent simulation of
translocating particles of various sizes and charges under \emph{nonequilibrium} conditions. In addition to
the electrostatic force, these computations would yield the full electrophoretic and hydrodynamic force
landscape, together with the residual current values for each individual particle position. By subsequently
feeding these data into a Brownian dynamics simulator, which adds an additional stochastic force to mimic
thermal motion, it would allow one to mimic the actual nanopore sensing process
itself~\cite{Pederson-2015,Hulings-2018}. Effectively, such a system would correlate the dynamics of the
analyte molecule within the pore with the experimentally observed ionic current signal---providing a direct
connection between the experimentalist and the simulant.


\instructionsconclusions


%%%%%%%%%%%%%%%%%%%%%%%%%%%%%%%%%%%%%%%%%%%%%%%%%%
% Keep the following \cleardoublepage at the end of this file,
% otherwise \includeonly includes empty pages.
\cleardoublepage

% vim: tw=70 nocindent expandtab foldmethod=marker foldmarker={{{}{,}{}}}
