%
\chapter[SI: Modeling the transport of ions and water through {ClyA}]%
        {Supplementary information: Modeling the transport of ions and water through {ClyA}}
%
\label{ch:transport_appendix}
%

\definecolor{shadecolor}{gray}{0.85}
\begin{shaded}
  \adapRSC{Willems-2020}
\newpage
\end{shaded}


\section{Comparison of the {ClyA} \lumen{} diameters}
%
\label{sec:transport_appendix:radius_comparision}
%

A comparison of the \gls{clya} structures equilibrated with molecular dynamics with the crystal
(\pdbid{2WCD}~\cite{Mueller-2009}) and \gls{cryo-em} (\pdbid{6MRT}~\cite{Peng-2019}) structures can be found
in \cref{fig:transport_radius_comparison_surface}. The molecular surface was colored according to the internal
radius $r_{\text{int}}$
%
\begin{align}\label{eq:internal_radius}
  r_{\text{int},i} = \sqrt{(x_i^2 + y_i^2)} - r_{\text{vdW},i}
\end{align}
%
with $x_i$, $y_i$ and $r_{\text{vdW},i}$ the $x$-coordinate, $y$-coordinate and van der Waals radius of atom
$i$. All structure were aligned to minimize their RSMD to the initial structure, which has its center of mass
at $(0,0,55)$.

Our comparison shows that the \gls{md} structure does not deviate significantly from the crystal and
\gls{cryo-em} structures in overal geometry. Moreover, the average nanopore diameter in the \cisi{} \lumen{}
is closer to \SI{6}{\nm} than the \SI{5.5}{\nm} reported earlier~\cite{Willems-Ruic-Biesemans-2019} The value
of \SI{5.5}{\nm} should be considered as a measure for the maximum size of a protein that can fit inside
dodecameric \gls{clya} without it touching the walls. The \SI{6}{\nm} diameter value given is the average
\lumen{} diameter, which is slightly larger than \SI{5.5}{\nm} given that it includes---rather than
excludes---all corrugations. 

%
\begin{figure*}[t]
  \centering
  %
  \includegraphics[scale=1]{transport_radius_comparison_surface}
  %

  \caption[\textit{Lumen} diameters of {ClyA}.]%
  {%
    %
    \textbf{\textit{Lumen} diameters of {ClyA}.}
    %
    Side and top view of the molecular surface of the \gls{clya-as} equilibrated with \gls{md} (left), with
    the crystal \pdbid{2WCD}~\cite{Mueller-2009} (middle) and \gls{cryo-em} \pdbid{6MRT}~\cite{Peng-2019}
    (right) structures. Surfaces were colored according to $r_{\text{int}}$, the distance of each atom from
    the central axis of the pore, reduced with its van der Waals radius (see~\cref{eq:internal_radius}). The
    traditional diameter of \SI{5.5}{\nm} and the \SI{6.0}{\nm} are outlined in orange and pink respectively.
    Images were rendered using \gls{vmd}~\cite{Humphrey-1996}.
    %
  }\label{fig:transport_radius_comparison_surface}
\end{figure*}



%
\section{Tabulated ion selectivities}
%
\label{sec:transport_appendix:tab_ion_sel}
%

Tabulated data on the ion selectivity, in terms of cation transport number $\tna$ and cation permeability
ratio $\pna$ for various salt concentrations and voltages can be found in \cref{tab:ion_selectivities}.

%
\begin{table*}[b]
  \centering

  %
  \captionsetup{width=10cm}
  \caption{Tabulated cation transport numbers and permeability ratios.}
  \label{tab:ion_selectivities}
  
  %
  \renewcommand{\arraystretch}{1.2}
  \footnotesize
  %

  \begin{tabularx}{10cm}{XXXXXXX}
    \toprule
     & \multicolumn{6}{c}{$\vbias$ [\si{\mV}]} \\
       \cmidrule{2-7}
     & \multicolumn{2}{c}{$+150$} & \multicolumn{2}{c}{$+100$} & \multicolumn{2}{c}{$+50$} \\
    %
    \cmidrule(r){2-3} \cmidrule(r){4-5} \cmidrule(r){6-7}
    %
    $\cbulk$ [\si{\Molar}]
     & $\tna$$^\text{\emph{a}}$ & $\pna$$^\text{\emph{b}}$
     & $\tna$$^\text{\emph{a}}$ & $\pna$$^\text{\emph{b}}$
     & $\tna$$^\text{\emph{a}}$ & $\pna$$^\text{\emph{b}}$ \\
    %
    \midrule
    0.005 & 0.996 & 235 & 0.997 & 313 & 0.998 & 410 \\
    0.050 & 0.906 & 9.62 & 0.925 & 12.3 & 0.941 & 16.0 \\
    0.150 & 0.786 & 3.68 & 0.805 & 4.13 & 0.824 & 4.69 \\
    0.500 & 0.642 & 1.79 & 0.649 & 1.85 & 0.657 & 1.91 \\
    1.000 & 0.566 & 1.30 & 0.569 & 1.32 & 0.572 & 1.33 \\
    2.000 & 0.502 & 1.00 & 0.503 & 1.01 & 0.503 & 1.01 \\
    5.000 & 0.445 & 0.803 & 0.445 & 0.803 & 0.445 & 0.803 \\
    \midrule
    \midrule
     & \multicolumn{6}{c}{$\vbias$ [\si{\mV}]} \\
       \cmidrule{2-7}
     & \multicolumn{2}{c}{$+150$} & \multicolumn{2}{c}{$+100$} & \multicolumn{2}{c}{$+50$} \\
    %
    \cmidrule(r){2-3} \cmidrule(r){4-5} \cmidrule(r){6-7}
    %
    $\cbulk$ [\si{\Molar}]
      & $\tna$$^\text{\emph{a}}$ & $\pna$$^\text{\emph{b}}$
      & $\tna$$^\text{\emph{a}}$ & $\pna$$^\text{\emph{b}}$
      & $\tna$$^\text{\emph{a}}$ & $\pna$$^\text{\emph{b}}$ \\
    %
    \midrule
    0.005 & 0.998 & 500 & 0.998 & 535 & 0.998 & 542 \\
    0.050 & 0.967 & 29.6 & 0.966 & 28.2 & 0.962 & 25.1 \\
    0.150 & 0.882 & 7.51 & 0.874 & 6.91 & 0.860 & 6.16 \\
    0.500 & 0.687 & 2.20 & 0.680 & 2.12 & 0.672 & 2.05 \\
    1.000 & 0.582 & 1.39 & 0.579 & 1.38 & 0.577 & 1.36 \\
    2.000 & 0.505 & 1.02 & 0.505 & 1.02 & 0.504 & 1.02 \\
    5.000 & 0.445 & 0.802 & 0.445 & 0.802 & 0.445 & 0.802 \\
    %
    \bottomrule
  \end{tabularx}
  \begin{flushleft}
    $^\text{\emph{a}}$Cation transport number $\tna = \conductance_{\Na} / (\conductance_{\Na} + \conductance_{\Cl} )$;
    %
    $^\text{\emph{b}}$Cation permeability ratio $\pna = \conductance_{\Na} / (\conductance_{\Cl} )$.
  \end{flushleft}
\end{table*}
%


\section{Peak values of the radial potential profiles inside {ClyA-AS}}
%

The peak values of the radial electrostatic potential $\radpot$ at the \cisi{} entry, middle of the lumen and
the \transi{} constriction for \SIlist{0.005;0.05;0.15;0.5;5}{\Molar} \ce{NaCl} are summarized in
\cref{tab:radial_potential}.

%
\begin{table}[h]
  \centering
  
  %
  \captionsetup{width=8cm}
  \caption{Peak radial potential.}
  \label{tab:radial_potential}
  %
  \footnotesize
  %
  \begin{tabularx}{8cm}{SSSS}
    \toprule
    & \multicolumn{3}{c}{$\radpot$ [\si{\mV}]} \\
    \cmidrule{2-4}
    & {\cisi{}}      & {\lumen{}}    & {\transi{}}  \\
    {$\cbulk$ [\si{\Molar})]}
    & {$z\approx\mSI{10}{\nm}$} & {$z\approx\mSI{5}{\nm}$} & {$z\approx\mSI{0}{\nm}$} \\
    \midrule
    0.005          & -80         & -108       & -144   \\
    0.05           & -34         &  -50       &  -86   \\
    0.15           & -19         &  -29       &  -57   \\
    0.5            &  -9.3       &  -14       &  -30   \\
    5              &  -1.9       &   -1.7     &   -4.2 \\
    \bottomrule
  \end{tabularx}
  %
\end{table}


\section{Surface integration to compute pore averaged values}
%

The average pore values for quantity of interest $X$ was computed by \cref{eq:pore_surface_integral}
%
\begin{align}\label{eq:pore_surface_integral}
  \left< X \right>_{\alpha} =
    \displaystyle\frac{\displaystyle\iint_{V_{\alpha}} \beta_{\alpha} X \,dr\,dz}
                      {\displaystyle\iint_{V_{\alpha}} \beta_{\alpha} \,dr\,dz}
  \text{ ,}
\end{align}
%
where
%
\begin{equation}
  \alpha=
  \begin{cases}
    \text{PT}, & d \ge 0  \text{~nm} \text{, average over the entire pore} \\
    \text{PB}, & d > 0.5  \text{~nm} \text{, average over the pore `bulk' }  \\
    \text{PS}, & d \le 0.5\text{~nm} \text{, average over the pore `surface' }
  \end{cases}
\end{equation}
%
and
%
\begin{align}
  \beta_{\text{PT}} &=
  \begin{cases}
    1, & \text{if}\ -1.85\le z \le 12.25  \text{ and } r \le r_\text{p}(z) \\
    0, & \text{otherwise}
  \end{cases} \\
  \beta_{\text{PB}} &=
  \begin{cases}
  1, & \text{if}\ -1.85\le z \le 12.25  \text{ and } r \le r_\text{p}(z) \text{ and } d > 0.5 \\
  0, & \text{otherwise}
  \end{cases} \\
  \beta_{\text{PS}} &=
  \begin{cases}
  1, & \text{if}\ -1.85\le z \le 12.25  \text{ and } r \le r_\text{p}(z) \text{ and } d \le 0.5 \\
  0, & \text{otherwise}
  \end{cases}
\end{align}
%
with $d$ the distance from the nanopore wall and $r_\text{p}(z)$ is the radius of the pore at height $z$.

%%%%%%%%%%%%%%%%%%%%%%%%%%%%%%%%%%%%%%%%%%%%%%%%%%
% Keep the following \cleardoublepage at the end of this file, 
% otherwise \includeonly includes empty pages.
\cleardoublepage

% vim: tw=70 nocindent expandtab foldmethod=marker foldmarker={{{}{,}{}}}
