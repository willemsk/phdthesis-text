\chapter{Biological nanopores}\label{ch:nanopores}
\epigraphhead[80]{
  \epigraph{
    ``I don't see how he can ever finish if he doesn't begin.''
  }{
    \textit{`Alice' (by Lewis Carroll)}
  }
}


\definecolor{shadecolor}{gray}{0.85}
\begin{shaded}
Parts of this chapter were adapted from:\\
\fullcite{Willems-VanMeervelt-2017}
\newpage
\end{shaded}

This chapter serves as a comprehensive introduction to the primary concepts required to understand the
objectives and relevance of the work performed in this thesis. In particular, the reader will be familiarized
with biological nanopores as single-molecular sensors, in particular for protein and peptide analysis.
\\
\\
The text and figures of this introductory chapter were entirely written and created by myself.

\newpage

\section{Introduction}



\section{Molecular sensors}

\section{Nanopores as single-molecular sensors}



\section{Nanopore types}

Depending on their constituent material, nanopores are often classified into two distinct groups: the protein-based, \emph{biological} nanopores (\glspl{bnp}) and the semiconductor-based, \emph{solid-state} nanopores (\glspl{ssnp}).\cite{Dekker-2007} This division runs deeper than the material however, as BNPs are are formed in a bottom-up manner through self-assembly, while SSNPs are fabricated top-down using micromachining techniques. Perhaps the most crucial difference between the two classes is the extent to which they can be modified and manipulated. The proteiaceuous

In recent years, many different types of nanopores have been developed that do not necessarily fall within either category. 





\section{Biological nanopores}


Was published \cite{Willems-VanMeervelt-2017}

Pore-forming toxins (\glspl{pft})

\subsection{\textalpha-Hemolysin}

The most widely studied biological nanopore is \textalpha-hemolysin (\gls{ahl}), a homoheptameric porin
secreted by \textit{Staphylococcus aureus}.

\subsubsection{Structure and assembly mechanism}

\subsubsection{Applications}


\subsection{Cytolysin A (ClyA)}

\begin{figure*}[p]
  \centering
  \medskip
    %
  \begin{minipage}{52mm}
    \begin{subfigure}[b]{52mm}
      \centering
      \caption{}\vspace{-8.5mm}\hspace{1.5mm}\label{fig:clya_pore_side}
      \includegraphics[scale=1]{clya_pore_side}
    \end{subfigure}
    %
    \vspace{5mm} \\
    %
    \begin{subfigure}[b]{52mm}
      \centering
      \caption{}\vspace{-8.5mm}\hspace{1.5mm}\label{fig:clya_pore_top}
      \includegraphics[scale=1]{clya_pore_top}
    \end{subfigure}
  \end{minipage}
  %
  \begin{minipage}{66mm}
    \begin{subfigure}[t]{60mm}
      \centering
      \caption{}\vspace{-8.5mm}\hspace{1.5mm}\label{fig:clya_pore_section}
      \includegraphics[scale=1]{clya_pore_section}
    \end{subfigure}
    %
    \vspace{5mm} \\
    %
    \begin{subfigure}[b]{37mm}
      \caption{}\vspace{-8.5mm}\hspace{1.5mm}\label{fig:clya_monomer}
      \includegraphics[scale=1]{clya_monomer}
    \end{subfigure}
    %
    \begin{subfigure}[b]{28mm}
      \centering
      \caption{}\vspace{-8.5mm}\hspace{1.5mm}\label{fig:clya_protomer}
      \includegraphics[scale=1]{clya_protomer}
    \end{subfigure}
  \end{minipage}
\caption{%
  \textbf{Structure of the Cytolysin A (ClyA) porin.}
  %
  (\subref{fig:clya_pore_side})
  %
  Side view and 
  %
  (\subref{fig:clya_pore_top})
  %
  top view of the dodecameric \gls{clya} nanopore (\pdbid{6MRT} \cite{Peng-2019}), with a single
  subunit (protomer) highlighted in color and the location of the lipid bilayer indicated in blue.
  %
  (\subref{fig:clya_pore_section})
  %
  Cross-section view of the interior molecular surface of the pore, colored according to its estimated
  electrostatic potential in physiological conditions as calculated by \gls{apbs}
  \cite{Baker-2001,Baker-2005}.
  %
  (\subref{fig:clya_monomer})
  %
  The \SI{34}{\kilo\dalton} water-soluble \gls{clya} monomer (\pdbid{1QOY} \cite{Wallace-2000}) consists of a
  core bundle of \ta-helices (\ta A1, \ta A2, \ta B, \ta C, \ta F), onto which a hydrophobic \tb-tongue
  motif---flanked by two short \ta-helices (\ta D and \ta E)---and a C-terminal \ta-helix (\ta G) are packed.
  The two glycines in the \tb-tongue confer it with a high degree of flexibility, while the removal of Phe191
  from its hydrophobic pocket triggers pore formation.
  %
  (\subref{fig:clya_protomer})
  %
  A single ClyA protomer (\pdbid{6MRT} \cite{Peng-2019}), with the same coloring as the monomer, showing the
  insertion of the \ta-tongue into the lipid bilayer and the near \SI{14}{\nm} conformational change required
  for the \ta A1 helix to form the transmembrane part of the pore.
  %
  All images were prepared and rendered using VMD \cite{Humphrey-1996,Stone-1998}.
  }\label{fig:clya_pore_structure}
\end{figure*}

In 2009, Mueller \etal{} revealed the crystal structure of Cytolysin A (\gls{clya}) \cite{Mueller-2009}, a
large \textalpha-\gls{pft} secreted by various \textit{Salmonella enterica} and \textit{Esterischia coli}
strains with a high affinity for mammalian cell membranes. The large size of its \lumen---relative to
\gls{ahl} and \gls{mspa}, the most commonly used protein nanopores at the time---makes \gls{clya} particularly
well equipped to fully capture and study larger biopolymers, such as proteins, in their native
three-dimensional configuration (\ie~folded). In the following section, we will discuss the structure and
oligomerization mechanism of \gls{clya} in detail.

\subsubsection{Pore structure}

\paragraph{Pore shape and charge.}
%
The \SI{408}{\kilo\dalton} \gls{clya} pore consists of 12 idential subunits (protomers), arranged in a ring
like quaternary structure to form a cylindrical structure roughly \SI{14}{\nm} in height and \SI{11}{\nm} in
diameter (\cref{fig:clya_pore_side,fig:clya_pore_top}). The pore walls in contact with the solvent are formed
by a core bundle of four tightly packed \ta-helixes per protomer, and the transmembrane region is formed by
the iris-like arrangement of the amphiphatic N-terminal \ta-helices. Virtually all helices contribute to
interprotomer contacts, resulting in the burying of \SI{\approx2400}{\square\angstrom} surface area, and the
formation of 25~H-bonds and 13~salt bridges per protomer-protomer interface. The interior shape of \gls{clya}
can be described as a set of two hollow cylinders placed on top of each other: a large \cisi{} chamber
(`\lumen{}') with an inner diameter of \SI{5.5}{\nm} and a height of \SI{10}{\nm} and a smaller \transi{}
chamber (`\textit{constriction}') with a width of \SI{3.3}{\nm} and a height of \SI{4}{\nm}
(\cref{fig:clya_pore_section}). \Gls{clya} has an excess negative charge (\SI{-60}{\ec} at \pH{7.5}), most of
which riddle the interior walls of the pore. This results in a negative electrostatic potential inside both
the \lumen{} and the constriction of the pore (\cref{fig:clya_pore_section}) and explains the observed cation
selectivity \cite{Soskine-2012,Franceschini-2016}.

\paragraph{Note on the oligomerization state.}
%
Next to the typical 12-subunit pore (dodecamer, `Type I'), \gls{clya} can also form 13-mers (tridecamers,
`Type II') and 14-mers (tetradecamers, `Type III') \cite{Soskine-2013}. Aside from their increased aperture
size, the overal structure and assembly mechanism of these pores is identical to that of the dodecamer, and
hence the following discussion will be limited to the Type I pore. Interested readers are referred to the work
of Peng and coworkers \cite{Peng-2019}, who recently obtained high-resolution structures of the Type II and
III pores using \gls{cryo-em}.


\subsubsection{Pore formation and kinetics}

\paragraph{From water-soluble monomer to oligomeric transmembrane pore.}
%
Like the pore, the water-soluble monomer (\cref{fig:clya_monomer}) consists of predominantly \ta-helices, with
the core of the protein formed by a bundle of four long \ta-helices (\ta A, \ta B, \ta C and \ta F). Packed at
the top and bottom of this bundle are respectively the C-terminal helix (\ta G) and a hydrophobic \tb-hairpin
motif flanked by to short \ta-helices (\ta D and \ta E) \cite{Wallace-2000,Mueller-2009}. Comparing this
structure to that of the protomer (\cref{fig:clya_protomer}), reveals that while three of the core helices are
merely straightened and elongated (at the expense of the \tb-hairpin and the \ta A2, \ta D and \ta E helices),
the \ta A1 helix must move to the opposite side of the monomer, a distance of \SI{\approx14}{\nm} (!). The
mechanism that enables these large conformational changes is similar to that of a spring-loaded lock, which is
opened upon membrane binding. The latter is facilitated---\textit{via} both specific and non-specific
interactions---by the partially solvent-exposed hydrophobic residues of the \tb-hairpin motif, causing it to
swing outwards (with Gly180 and Gly201 acting as hinges) and insert into the lipid bilayer.  This removes a
stabilizing aromatic residue (Phe190) from its hydrophobic pocket and unlatches a two-helix spring-loaded
mechanism that results in 1) the straightening and extension of three of the core helices (\ta B, \ta C and
\ta F), and 2) a \SI{180}{\degree} swinging of the \ta A helix (relative to \ta B, \ta C and \ta F). The
amphiphatic \ta-A1 helix, which will form the \gls{tmd}, now rests on top of the membrane and is connected to
the rest of the protein via a helix-turn-helix motif, with Pro36 acting as a hinge. The subsequent
oligomerization of the membrane-bound monomers further packs and buckles the \ta A1 helices, effectivily
pushing them downwards and allowing them to `wedge' a hole into the lipid bilayer.

\paragraph{Oligomerization kinetics.}
%
Initial far-UV \gls{cd} measurements of the \gls{clya} oligomerization process in the presence of the mild,
non-ionic detergent \gls{ddm} suggested a two-step process, in which the soluble monomers (\ce{M}) undergo a
rapid transition ($t_{1/2} \approx \SI{80}{\second}$) to molten globule-like intermediate (\ce{I}), followed
by a slow assembly  ($t_{1/2} \approx \SI{1000}{\second}$) of the membrane-bound protomers (\ce{P})into the
full dodecameric pore (\ce{P_12}) \cite{Eifler-2006}. However, subsequent investigation of the oligomerization
kinetics \textit{via} \gls{rbc} hemolysis assays \cite{Vaidyanathan-2014} and single-molecule
\gls{fret} \cite{Benke-2015} revealed that the intermediate state is actually an off-pathway by-product that
slows down the pore formation \cite{Roderer-2017}
%
\kineticscheme{arrow coeff=2}{
  I
  \arrow{<=>[\small$k_{\text{IM}} = \SI{5.0e-2}{\per\second}$][\small$k_{\text{MI}} = \SI{3.0e-1}{\per\second}$]} 
  M 
  \arrow{<=>[\small$k_{\text{MP}} = \SI{1.7e-2}{\per\second}$][\small$k_{\text{PM}} = \SI{4.7e-4}{\per\second}$]}
  P
}{eq:clya_protomer_formation}
%
Note that converting from \ce{M -> I} ($k_{\text{MI}}$) is $\approx17~\times$ faster than \ce{M -> P}
($k_{\text{MP}}$), resulting the initial accumulation of molten globule-like intermediates. On the long term,
however, most of the monomers will convert to protomers, as the reverse reaction in the monomer to protomer
transition \ce{P -> M} ($k_{\text{PM}}$) is $\approx36~\times$ slower compared to its forward reaction
\ce{M -> P}.

After sufficient monomers have bound to the membrane, the increased probability of intermolecular collision
between protomers results in the formation of linear mixtures of oligomers ($\ce{P_{n/m}}$) by a process
called protomer elongation \cite{Roderer-2017}
%
\kineticscheme{arrow coeff=2}{
  \chemfig{P_n + P_m}
  \arrow{->[\small$k_{\text{elong}} = \SI{1.0e5}{\per\Molar\per\second}$]}
  \chemfig{P_{n+m}}
  \hspace{0.5cm}(n + m < 12)
}{eq:clya_protomer_elongation}
%
Because virtually all $\ce{P_n}$ species contribute to this process, the effective protomer concentration does
not decrease, making it both rapid and efficient. Formation of the full pore occurs rapidly \textit{via} ring
closure when a pair of oligomers---with a total protomer count of 12---collide and associate
%
\kineticscheme{arrow coeff=2}{
  \chemfig{P_n + P_{12-n}}
  \arrow{->[\small$k_{\text{pore}} = \SI{3.0e7}{\per\Molar\per\second}$]}
  \chemfig{P_{12}}
  \hspace{0.5cm}(n = 1,2,\ldots,11)
}{eq:clya_ring_closure}
%
Even though all $\ce{P_{1\ldots11}}$ species contribute to the formation of $\ce{P_12}$, the relatively high
abundance of $\ce{P_5}$, $\ce{P_6}$ and $\ce{P_7}$ dictate that \SI{>50}{\percent} of all full pores are
formed by these oligomers \cite{Benke-2015}.




\subsubsection{Applications}
Starting \gls{clya}  has since been repurposed by the Maglia group for capture and study of folded proteins, 


% be purified and isolated using native polyacrylamide gel electrophoresis (\gls{page})


\subsection{Fragaceatoxin C (FraC)}

Fragaceatoxin C (\gls{frac})

\subsubsection{Structure and assembly mechanism}


\begin{figure*}[t]
  \centering
  \medskip
    %
  \begin{minipage}{58mm}
    \begin{subfigure}[t]{58mm}
      \centering
      \caption{}\vspace{-8.5mm}\hspace{1.5mm}\label{fig:frac_pore_side}
      \includegraphics[scale=1]{frac_pore_side}
    \end{subfigure}
    %
    \vspace{5mm} \\
    %
    \begin{subfigure}[t]{58mm}
      \centering
      \caption{}\vspace{-8.5mm}\hspace{1.5mm}\label{fig:frac_pore_top}
      \includegraphics[scale=1]{frac_pore_top}
    \end{subfigure}
  \end{minipage}
  %
  \begin{minipage}{58mm}
    \begin{subfigure}[t]{58mm}
      \centering
      \caption{}\vspace{-8.5mm}\hspace{1.5mm}\label{fig:frac_pore_section}
      \includegraphics[scale=1]{frac_pore_section}
    \end{subfigure}
    %
    \vspace{5mm} \\
    %
    \begin{subfigure}[t]{28mm}
      \caption{}\vspace{-8.5mm}\hspace{1.5mm}\label{fig:frac_monomer}
      \includegraphics[scale=1]{frac_monomer}
    \end{subfigure}
    %
    \begin{subfigure}[t]{28mm}
      \centering
      \caption{}\vspace{-8.5mm}\hspace{1.5mm}\label{fig:frac_protomer}
      \includegraphics[scale=1]{frac_protomer}
    \end{subfigure}
  \end{minipage}
\caption{%
  \textbf{Structure of the Fragaceatoxin C (FraC) porin.}
  %
  (\subref{fig:frac_pore_side})
  %
  %
  %
  (\subref{fig:frac_pore_top})
  %
  %
  %
  (\subref{fig:frac_pore_section})
  %
  (\subref{fig:frac_monomer})
  %
  %
  (\subref{fig:frac_protomer})
  %
  Images were rendered using VMD \cite{Humphrey-1996,Stone-1998}.
  }\label{fig:frac_pore_structure}
\end{figure*}


\subsubsection{Applications}



\subsection{Pleurotolysin AB (PlyAB)}

Pleurotolysin AB (\gls{plyab})

\subsubsection{Structure and assembly mechanism}

\begin{figure*}[t]
  \centering
  \medskip
    %
  \begin{minipage}{56mm}
    \begin{subfigure}[t]{56mm}
      \centering
      \caption{}\vspace{-8.5mm}\hspace{1.5mm}\label{fig:plyab_pore_side}
      \includegraphics[scale=1]{plyab_pore_side}
    \end{subfigure}
    %
    \vspace{5mm} \\
    %
    \begin{subfigure}[t]{56mm}
      \centering
      \caption{}\vspace{-8.5mm}\hspace{1.5mm}\label{fig:plyab_pore_top}
      \includegraphics[scale=1]{plyab_pore_top}
    \end{subfigure}
  \end{minipage}
  %
  \begin{minipage}{60mm}
    \begin{subfigure}[t]{56mm}
      \centering
      \caption{}\vspace{-8.5mm}\hspace{1.5mm}\label{fig:plyab_pore_section}
      \includegraphics[scale=1]{plyab_pore_section}
    \end{subfigure}
    %
    \vspace{5mm} \\ 
    %
    \begin{subfigure}[t]{35mm}
      \caption{}\vspace{-8.5mm}\hspace{1.5mm}\label{fig:plyab_monomer}
      \includegraphics[scale=1]{plyab_monomer}
    \end{subfigure}
    %%
    \begin{subfigure}[t]{24mm}
      \centering
      \caption{}\vspace{-8.5mm}\hspace{1.5mm}\label{fig:plyab_protomer}
      \includegraphics[scale=1]{plyab_protomer}
    \end{subfigure}
  \end{minipage}
\caption{%
  \textbf{Structure of the Pleurotolysin AB (PlyAB) porin.}
  %
  (\subref{fig:plyab_pore_side})
  %
  %
  %
  (\subref{fig:plyab_pore_top})
  %
  %
  %
  (\subref{fig:plyab_pore_section})
  %
  (\subref{fig:plyab_monomer})
  %
  %
  (\subref{fig:plyab_protomer})
  %
  Images were rendered using VMD \cite{Humphrey-1996,Stone-1998}.
  }\label{fig:plyab_pore_structure}
\end{figure*}


\subsubsection{Applications}



\subsection{Other noteworthy biological nanopores}

\section{Solid-state nanopores}

\section{Hybrid nanopores}



%%%%%%%%%%%%%%%%%%%%%%%%%%%%%%%%%%%%%%%%%%%%%%%%%%
% Keep the following \cleardoublepage at the end of this file,
% otherwise \includeonly includes empty pages.
\cleardoublepage

% vim: tw=70 nocindent expandtab foldmethod=marker foldmarker={{{}{,}{}}}
