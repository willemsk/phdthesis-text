% !TeX root = ../../thesis.tex
\chapter{Biological nanopores}
%
\label{ch:nanopores}
%


\epigraphhead[\epipos]{%
\epigraph{%
%
  ``There is a place in our minds now that tools---made things, things of value---occupy, and there was a time
  when our ancestors' minds had no such place. Nothing was ever for keeps, and there was no tomorrow for the
  tool-makers.''
%
}{%
  \textit{`Ruth Emerson' in `The Doors of Eden' by Adrian Tchaikovsky}
%
}
}

% \epigraphhead[\epipos]{%
% \epigraph{%
% %
%   ``No lifetime is long enough for those who wish to create, Raul. Or for those who simply wish to understand
%   themselves and their lives. It is, perhaps, the curse of being human, but also a blessing.''
% %
% }{%
%   \textit{`Paul Dur\'e' in `The Rise of Endymion' by Dan Simmons}
% %
% }
% }

%
%
\definecolor{shadecolor}{gray}{0.85}
\begin{shaded}
Parts of this chapter were adapted from:
%
\begin{itemize}
  \item K. Willems*, V. Van Meervelt*, C. Wloka and G. Maglia.
        \textit{Phil. Trans. R. Soc. B} \textbf{372}, 20160230 (2017) %\cite{Willems-VanMeervelt-2017}
\end{itemize}
%
*equal contributions
%
\newpage
\end{shaded}
%
%

%
This chapter serves as a comprehensive introduction to the primary concepts required to understand the
objectives and relevance of the work performed in this dissertation. In particular, the reader will be
familiarized with biological nanopores as single-molecule sensors, the physical mechanisms that govern
nanopore sensing, and the current approaches for the modeling of nanopores.
%

%
The text and figures of this introductory chapter were entirely written and created by me.
%
The section
%
``Confining proteins within biological nanopores''
%
was adapted from ref.~\cite{Willems-VanMeervelt-2017}.



\clearpage

%
%
\section{Use of tools}
%

Some 5 million years ago, our evolutionary ancestors became bipedal in order to better observe and understand
the world around them. Up until the \textit{Homo habilis} introduced the \emph{use of tools}, these ancients
depended solely on their senses to provide them with information, and their hands to shape their environment.
Arguably, this extensive tool usage also set in motion a process of rapid encephalization (\ie~increase of the
brain-to-body mass ratio), further improving the human clade's capability to process, interpret and
communicate information. Even though many other species are active tool-users, to date, only the \textit{Homo}
genus appears to employ tools to fabricate more complex ones. This behavior has been observed as far back as
2.6~million years, but the advances brought forth by the scientific and industrial revolutions truly ushered
in a new era of complexity, which was topped only by the invention of the modern computer and the internet,
which have led us into the present day's information age.

Just as we shaped our tools, so have they shaped us. They have impacted virtually all aspects of our society
and are still doing so at an increasing rate. This is particularly true in the sciences, which, in their quest
to find new ways to probe, analyze, quantify, and understand the world around us, have been both tremendous
benefactors and beneficiaries. To this end, we have succeeded in extending our senses far beyond their natural
capabilities with a host of external `sensors', which, quoting the Cambridge dictionary, are
%
\begin{quote}
  devices that are used to record that something is present or that there are changes in something.
\end{quote}
%
Contemporary technical capabilities aside, both the scale of the sensor and the nature of the `something' it
is supposed to detect, are constrained solely by our ingenuity and imagination. For example, we have built
both gigantic machines---a \SI{27}{\kilo\meter}-long particle accelerator for studying the physical basis of
gravity~\cite{ATLAS-2012}, or a \SI{4}{\kilo\meter}-long laser interferometer for detecting gravitational
waves produced during the merging of two black holes~\cite{Abbott-2016}---as well as tiny
devices---sub-millimeter-sized implantable sensors for monitoring physiological functions~\cite{Dong-2019}, or
\SI{20}{\nm}-wide \glspl{fet} capable of quantifying biomarkers from blood samples~\cite{Krivitsky-2016}.

This dissertation should be framed in the latter category. Specifically, we have used physical, computational,
and mathematical tools---in the form of experiments, numerical simulations, and physical models,
respectively---to improve our fundamental understanding of minuscule sensors, approximately \num{10000} times
smaller than the width of a human hair: \emph{nanopores}.


%
%
\section{Sensing at the single-molecule level}
%

Before delving into the ``what'', ``how'', and ``why'' of nanopores, let us take a step back and put them into
a broader context. Nanopores, and particularly biological ones, belong to the class of single-molecule
biosensors. Here, the `bio' reflects that one or more biological components (\eg~tissues or cells, or be parts
thereof, such as organelles, enzymes, antibodies, or \gls{dna} molecules) comprise an integral part of the
sensing process, usually as the element that interacts (specifically) with the analyte molecule of
interest~\cite{Banica-2012}. The sensor element can be any type of physicochemical detector (\eg~optical,
electrochemical, mechanical, etc.) that quantifies the interactions between the biological element and the
analyte molecule. In this section we will introduce the concept of single-molecule sensing, together with a
non-exhaustive selection of some the most prominent single-molecule techniques.

\subsection{From bulk to single-molecule}

Classical `bulk' (bio)sensors typically measure the ensemble-average of a quantity of interest, meaning that
the final signal originates from a great many (read: from \numrange{10}{e21}) molecules at the same time. For
many applications, such as the enzyme-linked immunosorbent assay or the quantitative polymerase chain
reaction, this information is typically more than sufficient. However, considering that there are many ways to
obtain the same average value, the very nature of the ensemble-average also hides the property of interest's
true distribution, and may obscure rare or transient events. For example, when measuring the activity of an
enzyme in bulk, does the signal stem from a single, `average' sub-population, or are there multiple
sub-populations active at the same time---some working slowly and others faster, balancing each other to yield
the observed ensemble-average? One way to answer this question is to interrogate the individual enzymes
themselves, at the so-called `single-molecule' level. Whereas bulk experiments can be considered deterministic
and continuous due to the sheer number of molecules involved in every measurement, the process of
single-molecule sensing is inherently stochastic in nature, meaning that a property can only be quantified by
`sampling' it multiple times, up until a given statistically relevance is reached and the true distribution is
revealed. Besides the study of enzyme kinetics, single-molecule measurements empower several super-resolution
imaging techniques---for which the 2014 Nobel Prize in Chemistry was awarded~\cite{Weiss-2014}---and they lie
at the heart of the third-generation \gls{dna} sequencing technologies, enabling faster and longer reads,
direct detection of epigenetic markers, improved portability, and lower cost~\cite{Schadt-2010}.

Regardless of its precise application or mode-of-action, any single-molecule technique faces two fundamental
challenges compared to bulk sensors: (1) the amplification of a signal emanating from a single molecule to a
macroscopically measurable signal, and (2) the reduction of the sensing volume such that it contains only a
single analyte molecule. Given that the majority of molecules are intrinsically nanoscopic (\ie~with
characteristic length scales between \SIrange[range-phrase={ and }]{1}{10}{\nm}), it follows that, without
some means of amplification, the signal it generates is equally small, and hence demanding to detect reliably.
The second challenge, on the other hand, is not related to the size of the analyte molecule \textit{per se},
but rather to its concentration. The reason being that, for the duration of at least a single measurement, the
detection volume may contain only a single molecule (and the same one at that). For example, the
inter-molecular distances inside solutions with analyte concentrations of \SI{1}{\nM}, \SI{1}{\uM}, and
\SI{1}{\mM} are, on average, \SIlist{1500;150;15}{\nm}, respectively. Hence, achieving the small sensing
volumes required for physiologically relevant concentrations (\si{\uM} to \si{\mM}) may be difficult to
achieve~\cite{Zhu-2012}. 


\subsection{Fluorescence spectroscopy}
%

One phenomenon that lends itself particularly well for single-molecule techniques is fluorescence: the
emission of light by a molecule (\ie~fluorophore) after absorbing energy from an external electro-magnetic
radiation source. Because some energy is lost to the environment during the fluorescence process, the emitted
light is often redshifted compared to its absorbed counterpart, allowing the two to be easily separated from
one-another. Moreover, considering that a single fluorophore can emit \numrange{1}{100}~million photons every
second, they have intrinsically high signal-to-noise ratios. However, because the diffraction poses a physical
lower limit on how strongly light can be confined, the reduction of the sensing volume is not so easily
solved. For state-of-the-art confocal microscopes, this limit is equal to roughly half the wavelength of the
light used (\ie~\SIrange{\approx150}{350}{\nm} for visible light), which mandates to the use of very low
analyte concentrations (\SI{\le1}{\nM}). Luckily, several solutions to this problem have been developed over
the years, notably those based on the near-field techniques. For example, the exponentially decaying
(`evanescent') near-fields inside nanoscale metal apertures called
\glspl{zmw}~\cite{Levene-2003,Eid-2009,Zhu-2012}, or protruding from the tip of a
\gls{nsom}~\cite{Ambrose-1994,Hosaka-2001}, allow the confinement of light to volumes of
$50\times50\times50$~\si{\cubic\nm}. Even stronger confinement, down to a few \si{\cubic\nm}, can be achieved
with specialized metal structures that generate plasmonic `hot-spots'~\cite{Xin-Lu-2019}. Finally, the
combination of the above-mentioned techniques with another distance-dependent phenomenon such as \gls{fret},
which is the non-radiative transfer of energy between a donor and acceptor fluorophore when they are closer
than \SI{\approx10}{\nm}~\cite{Roy-2008}, allows one to both limit the observation volume and study
conformational dynamics at the nanoscale~\cite{Kim-2013}. A major drawback of fluorescence-based techniques is
that contrary to label-free methods, they often necessitate attaching a fluorescent dye molecule to the
target molecule. This adds additional complexity to the sensing process and may affect the properties of the
analyte. Perhaps more importantly, the inevitable photobleaching of the typical organic fluorophore
(\ie~seconds) imposes strict limitations on the length that any individual molecule can be observed, and hence
restricts its use to the study of relatively fast processes.


\subsection{Force spectroscopy}
%

A different approach is taken by force spectroscopy techniques, where a large `probe' is monitored with a
highly sensitive force-feedback system to measure the properties of individual molecules when they are placed
under some form of mechanical strain~\cite{Neuman-2008}. The type of probe depends on the technique in
question. In \gls{afm}, it consists of a very sharp tip that is either scanned across a surface to determine
its topology (hence the term microscopy) or attached (non)-covalently to the molecule of interest such that
it can exert a precise pulling force on it. In optical and magnetic tweezers, analyte molecules are attached
to respectively dielectric and magnetic probe particles and trapped within optical or magnetic field
gradients. Because the displacement of the particle correlates with the exerted force, the latter can be
easily deduced by monitoring the particle's position over time. Hence, these techniques transduce nanoscale
motions and forces by means of much larger probes and reduce the sensing volume \textit{via} covalent
attachment to the probe or the surface. Force spectroscopy has been used to investigate the (un)folding of
proteins~\cite{Bustamante-2020,Jagannathan-2013} and protein domains~\cite{Rief-1997,Kellermayer-1997}, and to
elucidate protein-\gls{dna} interactions~\cite{Abbondanzieri-2005,Shlyakhtenko-2007,Vanderlinden-2019} and
dynamics~\cite{Lyubchenko-2018,Brouns-2018}. Major drawbacks of force spectroscopy techniques are their
limited throughput (all), the complexity of the instrumentation (all), and the high laser intensities required
(optical tweezers).


\subsection{Ionic current spectroscopy (\ie~nanopores)}
%

%
\begin{figure*}[b]
  \centering
  \medskip
  %
  \includegraphics[scale=1]{nanopores_concept_section}
  \\
  \includegraphics[scale=1]{nanopores_concept_trace}
  %

  %
\caption[Concept of single-molecule sensing with nanopores]{%
  \textbf{Concept of single-molecule sensing with nanopores.}
  %
  (Left panel)
  %
  When a net potential difference ($\Delta V$) is applied across two electrolyte reservoir (\cisi{} and
  \transi{}) separated by a lipid bilayer containing a single nanopore, the electric field drives cations
  (blue) and anions (red) through the pore, resulting in a measurable `open pore' current ($\iopen$).
  %
  (Middle panel)
  %
  After a given time, $\Delta t_{\rm{d,1}}$, the entry of analyte 1 molecule (purple) into the water-filled
  channel of the pore from the \cisi{}-side transiently blocks the ionic current, until the molecule exits
  into the \transi{} reservoir (\ie~it `translocates'). This gives rise to the $\ilevel{1}$ current level,
  which has a characteristic `dwell time' $t_{\rm{d,1}}$.
  %
  (Right panel)
  %
  Similarly, entry of the larger analyte 2 molecule (green) into pore from the \cisi{} side, after inter-event
  time $\Delta t_{\rm{d,2}}$, leads to the deeper current blockade level $\ilevel{2}$ with a dwell time
  $t_{\rm{d,2}}$. Because of its size however, it cannot pass through the narrowest point of the channel and
  hence will either remain trapped inside the pore, or eventually exit back into the \cisi{} reservoir.
  %
  Whereas the duration and depth of the block yield information on the properties of the molecule, the
  inter-event time correlates with its concentration.
  %
  }\label{fig:nanopores_concept}
\end{figure*}
%


This brings us to back to the technique at hand, nanopore-based sensing~\cite{Howorka-2009,Wang-2018}.
Conceptually, single-molecule nanopore measurements (\cref{fig:nanopores_concept}) are perhaps the easiest to
understand of all the techniques discussed in this section. A nanopore device consists of a nanoscopic
aperture (\SIrange{1}{100}{\nm} in diameter) in an otherwise impermeable membrane. This opening serves as the
sole liquid connection between two reservoirs filled with a saline solution (\ie~the electrolyte, typical
ionic strengths vary between \SI{10}{\mM}~and~\SI{3}{\Molar}). When an electrical potential difference is
applied between the two reservoirs (\ie~the \cisi{} and \transi{} sides), a strong directional electric field
is formed within the nanopore that sets into motion the charged ions contained within. This opposing stream of
cations and anions towards their respective electrode polarities results in a steady-state ionic current. For
example, in a \SI{0.15}{\Molar} \ce{KCl} solution, the conductivity of a typical biological nanopore (\eg~with
a diameter of \SI{3}{\nm} and a length of \SI{10}{\nm}) would be approximately \SI{1}{\nS}
(see~\cref{sec:np:potential}). Applying a bias voltage of \SI{0.1}{\volt} across such a nanopore would result
in a current of \SI{100}{\pA}, which is equivalent to the passage of \num{600} million ions through the pore
every second. Hence, the large number of ions enable the real-time measurement of the ionic current at high
sampling frequencies, typically in the \SIrange{1}{50}{\kHz} range~\cite{Maglia-2010}, though experiments up
to \SI{1}{\mega\hertz} have been reported~\cite{Rosenstein-2012}.\footnotemark%
%
\footnotetext{
%
Note that the actual sampling limit is set by the signal-to-noise ratio. Noise in nanopores has many
contributions from a diverse set of physical phenomena, and includes flicker, shot, thermal current,
dielectric, and capacitive noise. A detailed discussion falls beyond the scope of this dissertation, and the
interested reader is referred to the comprehensive overview on current noise in nanopores published recently
by Fragasso \etal{}~\cite{Fragasso-2020}.
%
}
%
If the characteristic diameter of the nanopore is comparable to that of the molecule of interest (\ie~the
analyte), its entry into the pore will temporarily and significantly disrupt the regular `open pore' ionic
current, resulting in a so-called `blocked-pore' signal. Here, the depth of the current blockade and its
duration yield information on the properties of the molecule, whereas the inter-event time correlates with its
concentration. For any given analyte molecule, both the magnitude and the duration of a current blockade
depend on the exact type of nanopore (biological or solid-state, geometry, material, etc.), and the conditions
of the experiment (ionic strength, pH, bias voltage, etc.), making them notoriously difficult to predict and
interpret unambiguously. Typical blocked-pore current values range from \SIrange{10}{95}{\percent} of the open
pore current~\cite{Fragasso-2020,Zernia-2020,Howorka-2009}. The regular dwell times vary from
\SI{10}{\us}~to~\SI{100}{\ms} (\eg~for \gls{dna}~\cite{Maglia-2010} or protein~\cite{Watanabe-2017}
translocation) but can increase to seconds or even minutes in specific cases (\eg~proteins trapped within a
nanopore~\cite{Soskine-2012}). Finally, the fact that the interior dimensions of the pore are typically such
that only a single molecule can fit inside, makes that nanopores are intrinsically well-suited as
single-molecule sensors.

Over the past 24 years, the primary driving force behind the development of nanopores has been, and perhaps
still is, single-molecule \gls{dna} sequencing~\cite{Deamer-2016}. This application has now been successfully
commercialized by the company \gls{ont}~\cite{ONT-2020,Jain-2018}. This has spurred researchers to deploy
nanopores for sensing a wide variety of analytes~\cite{Wang-2018} with both organic chemistries---such as
nucleic acids~\cite{Kasianowicz-1996,Meller-2000,Stoddart-2009,Manrao-2012},
proteins~\cite{Mohammad-2008,Firnkes-2010,Spiering-2011,RodriguezLarrea-2013}, and
polymers~\cite{Robertson-2007,Baaken-2011}---and inorganic ones---such as
ions~\cite{Bezrukov-1993,Kasianowicz-1995,Kasianowicz-1999,Ali-2011,Roozbahani-2020} and metallic
nanoparticles~\cite{Astier-2009,Angevine-2014,Campos-2018}. Evidently, the application space is equally broad,
ranging from proteomics~\cite{Yusko-2017,Houghtaling-2019} and protein
sequencing~\cite{Restrepo-Perez-2018,Huang-2019}, to single-molecule
enzymology~\cite{Willems-VanMeervelt-2017,Ho-2015,Wloka-2017,Harrington-2019,Galenkamp-2020} and
metabolomics~\cite{VanMeervelt-2017,Zernia-2020}.

Nanopores are not only extremely versatile, but their inherently small size and relatively simple read-out
mechanism (\ie~a current amplifier) makes them amenable to both miniaturization and parallelization. Both of
which are exemplified by \gls{ont}'s \code{MinION} device---a handheld, third generation sequencer powered by
\num{512} individual nanopores capable of providing \SI{30}{\giga\bp} (10 times the human genome) of real-time
sequencing data~\cite{ONT-2020}. Next to these advantages, nanopores are also readily combined with the other
single-molecule techniques discussed above, providing both complementary of confirmatory information. Examples
from literature include \gls{afm}~\cite{Aramesh-2019}, optical
tweezers~\cite{Keyser-2006,vanDorp-2009,Hall-2009,Galla-2014}, magnetic tweezers~\cite{Peng-2009},
fluorescence microscopy~\cite{McNally-2010,Anderson-2014,Assad-2014,Huang-2015},
\glspl{zmw}~\cite{Auger-2014,Larkin-2017,Spitzberg-2019}, and plasmonic
structures~\cite{Im-2010,Chen-2018,Verschueren-2018,Garoli-2019}.


%
\begin{figure*}[b]
  \centering
  \medskip
  %
  \begin{subfigure}[b]{115mm}
    \centering
    \caption{}\vspace{-2.5mm}\label{fig:nanopores_types_biological}
    \includegraphics[scale=1]{nanopores_types_biological}
  \end{subfigure}
  %
  \\ \vspace{-1mm}
  %
  \begin{subfigure}[b]{115mm}
    \centering
    \caption{}\vspace{-2.5mm}\label{fig:nanopores_types_solidstate}
    \includegraphics[scale=1]{nanopores_types_solidstate}
  \end{subfigure}
  %
%
\caption[Types of nanopores.]{%
  \textbf{Types of nanopores.}
  %
  Cross-sectional (left) and top (right) views for a fully atomistic representation of 
  %
  (\subref{fig:nanopores_types_biological})
  %
  a Cytolysin A (ClyA) biological nanopore (orange) embedded in a lipid bilayer (green), and
  %
  (\subref{fig:nanopores_types_solidstate})
  %
  a double-conical solid-state nanopore drilled through a \SI{7}{\nm}-thick \ce{Si3N4} membrane (green and
  gray spheres).
  %
  Both pores are surrounded by a \SI{0.15}{\Molar} electrolyte solution containing cations (blue spheres),
  anions (red spheres) and water molecules (light blue spheres). Note that for clarity, only the water
  molecules behind the pore are shown, represented by their oxygen atom.
  %
  Molecular structures were prepared and rendered using \gls{namd}~\cite{Phillips-2005} and
  \gls{vmd}~\cite{Humphrey-1996,Stone-1998}.
  %
  }\label{fig:nanopores_types}
\end{figure*}
%


%
%
\section{Types of nanopores}
%
\label{sec:np:types}
%

Depending on their constituent material, nanopores are typically classified into two distinct groups: the
protein-based \glspl{bnp}~\cite{Willems-VanMeervelt-2017} (\cref{fig:nanopores_types_biological}), and the
semiconductor-based \glspl{ssnp}~\cite{Dekker-2007} (\cref{fig:nanopores_types_solidstate}). This division
runs deeper than just the material however: whereas \glspl{bnp} are formed through \emph{bottom-up}
self-assembly by inserting themselves into the semi-fluid membrane material (\eg~a lipid
bilayer)~\cite{Howorka-2017}, \glspl{ssnp} are fabricated by permanently removing material from the membrane
material itself with \emph{top-down}, micromachining techniques~\cite{Dekker-2007}. In the next sections we
will describe both types in more detail.


\subsection{Biological nanopores}
%

The majority of the \glspl{bnp} in-use today are in fact `repurposed' \glspl{pft}, water-soluble proteins
excreted by pathogenic bacteria that oligomerize and self-assemble onto the membranes of other cells to
permeabilize them, disrupting vital ionic gradients, leaking nutrients into the environment, and potentially
killing the target cell~\cite{Peraro-2015}. \Glspl{pft} belong to either the \ta- or \tb-\gls{pft} class,
depending on whether the architecture of their \gls{tmd} is based on \ta-helices or \tb-sheets, respectively.
In \ta-\glspl{pft}, the \gls{tmd} is composed of amphipathic \ta-helices, typically arranged in a barrel
configuration with titled sidewalls. The hollow interior of the pore is lined with hydrophilic amino-acids and
its exterior surface is covered with hydrophobic residues. This double nature allows the toxin to form a
stable, water-filled channel into an otherwise water-impermeable membrane. \Gls{clya} from \textit{E. coli} or
\textit{S. typhi}~\cite{Mueller-2009} (also known as HlyE) and \gls{frac} from \textit{Actinia
fragacea}~\cite{Tanaka-2015} are the most well-studied nanopores from this class. The \gls{tmd} of
\tb-\glspl{pft} consists of an amphipathic \tb-barrel fold, which is typically a perfect cylinder. The most
well-known \tb-\glspl{pft} that are also used as nanopores are \gls{ahl} from \textit{Staphylococcus
aureus}~\cite{Song-1996}, aerolysin from \textit{Aeromonas hydrophila}~\cite{Iacovache-2016}, and the recently
characterized \gls{plyab} from \textit{Pleurotus ostreatus}~\cite{Lukoyanova-Kondos-2015}. Notable non-toxin
membrane channels with a \tb-barrel fold include \gls{mspa} outer-membrane protein~\cite{Faller-2004} and the
\gls{csgg} amyloid secretion channel from \textit{E. coli}~\cite{Goyal-2014}---both of which have been
successfully employed for nanopore \gls{dna} sequencing~\cite{Manrao-2012,Brown-2016}---and
\gls{fhua}~\cite{Locher-1998}, \gls{ompf}~\cite{Yamashita-2008} and \gls{ompg}~\cite{Subbarao-2006} proteins
from \textit{E. coli}. Besides native membrane proteins, also other channel-like proteins have been engineered
to behave like nanopores. For example, the \gls{phi29p}~\cite{Xu-2019}, an essential component of the
\gls{dna} packaging mechanism of the \textPhi29 phage, was modified to insert (reversibly) into lipid bilayers
and shown to translocate \gls{dsdna}~\cite{Wendell-2009}. Not all membrane-inserting nanopores are
protein-based though, as researchers employed \gls{dna} origami~\cite{Rothemund-2006} to design and create
nanopores---from scratch---using nucleic acids as the primary building
block~\cite{Bell-2011,Langecker-2012,Burns-2013,Bell-2014,Gopfrich-2016,Gopfrich-2019}.


\subsection{Solid-state nanopores}
%

The methodologies for fabrication \glspl{ssnp} are derived from techniques used in the semiconductor industry
to create the nanometer-scale transistors found in virtually any contemporary electronic device. Generally,
the fabrication of a \gls{ssnp} starts with the deposition of a thin dielectric layer (typically
\SIrange{5}{50}{\nm}) on top of a flat silicon substrate. Because this layer will form both the membrane and
the nanopore, it must have both a high mechanical strength and a high chemical resistance. Hence, suitable
materials include \ce{Si3N4}~\cite{Li-2001,Storm-2003}, \ce{SiO2}~\cite{Storm-2005},
\ce{Al2O3}~\cite{Venkatesan-2009}, \ce{HfO2}~\cite{Larkin-2013} or even \ce{Si}
itself~\cite{Malachowski-2013}. Next, a part of the silicon at the backside of the substrate is removed using
wet or dry etching, leaving small window that contains the so-called `free-standing membrane'. The \gls{ssnp}
is then formed by creating an aperture into this membrane, which can be achieved by either \textit{ex-situ}
drilling with a focused ion~\cite{Li-2001} or electron~\cite{Storm-2003} beams, or by electron-beam
lithography patterning in combination with wet/dry etching~\cite{Nam-2009}. With controlled dielectric
breakdown, \glspl{ssnp} can also be fabricated \textit{in situ}, where an intact membrane is mounted into the
measurement cell and locally etched using a high transmembrane voltage~\cite{Kwok-2014}. By the
(non-)conformal deposition of oxides~\cite{Chen-2004}, metals~\cite{Li-2013d,Auger-2014,Spitzberg-2019} or
organic molecules~\cite{Wanunu-2007,Yusko-2011,Wei-2012,Rotem-2012} inside the pores, their size, surface
properties and biocompatibility can be manipulated with relative ease~\cite{Eggenberger-2019}. Additionally,
by transferring or growing quasi-2D materials on top of large, `classical' nanopores, researchers have been
able to fabricate ultra-thin (\ie~monoatomic) nanopores from graphene~\cite{Fischbein-2008} and
\ce{MoS2}~\cite{Feng-2015b} for both \gls{dna}~\cite{Feng-2015,Merchant-2010,Song-2011} and
protein~\cite{Shan-2013} detection. Next to these complex semiconductor-based methods, the nanoscale apertures
obtained by pulling glass pipettes provides a powerful, although intrinsically unscalable, approach for
fabricating \glspl{ssnp}~\cite{Wang-2006}. They allow the low-noise detection of
\gls{dna}~\cite{Steinbock-2013} and proteins~\cite{Li-2013B} and can even be integrated with a field-effect
transistor~\cite{Ren-2020}.


\subsection{Biological or solid-state: choosing is losing?}
%

Because they are made of either proteins or \gls{dna}, \glspl{bnp} can be produced using well-established
bacterial expression systems, making them relatively inexpensive to work with. Using molecular biology
techniques such as site-directed mutagenesis, this same proteinaceous nature also provides a straightforward
method for tailoring the surface properties of \glspl{bnp} with atomic
precision~\cite{Howorka-2001,RinconRestrepo-2011}. Even though \glspl{bnp} typically have superior
signal-to-noise ratios~\cite{Fragasso-2020}, the fragility of the lipid bilayer severely limits the lifetime
of a \gls{bnp}. This prevents the storage of any individual \gls{bnp} for longer than a few hours, its usage
at harsh experimental conditions (\eg~elevated temperatures). The stochastic nature of formation process
dictates that neither the time nor location of the nanopore's insertion into the lipid bilayer can be easily
controlled, making experiments often exceedingly tedious. The top-down approach taken with \glspl{ssnp}, on
the other hand, provides significantly more flexibility regarding the nanopore's size and material properties.
Their solid-state nature makes them both physically and chemically robust, enabling them to withstand even
harsh treatments (\eg~piranha or \ce{O2}-plasma cleaning). However, the expertise and equipment required to
reproducibly fabricate \glspl{ssnp} (\ie~a cleanroom with micromachining tools) can be prohibitively
expensive. Without a proper protective layer, such as \ce{HfO2}, the size of a \ce{Si3N4} pore is known to
`drift' due slow chemical etching, either during measurements or when improperly stored~\cite{Chou-2020}. The
destructive nature of the fabrication process (\eg~\textit{via} ion beam milling, plasma etching, or
dielectric breakdown) makes precise control of the pore's shape challenging to achieve~\cite{vandenHout-2010}.
Finally, without an appropriate, bio-compatible coating, the intrinsic `stickiness' of their surfaces leads to
the physisorption of most biomolecules~\cite{Eggenberger-2019,Awasthi-2020}. This results into the denaturing
of analytes of interest at best, and the rapid clogging of the pore at worst~\cite{Yusko-2011}. Currently, the
biological pores still trump their artificial counterparts both in terms of performance and applications.
However, the field is rapidly working to close this gap, and the intrinsic scalability (\eg~parallelization),
the possibilities of integration with other devices (\eg~transistors), and the mass-manufacturability of
\glspl{ssnp} remain intriguing advantages. As with many competing concepts, perhaps the optimal solution may
lie somewhere in the middle, with the so-called `hybrid nanopores' in which one attempts to combine the best
of both worlds~\cite{Hall-2010,Im-2010,Cai-2018}.

\clearpage

%
%
\section{Biological nanopores of interest}
%
\label{sec:np:interest}

As is evident from \cref{sec:np:types}, there are many \glspl{bnp} worthy of an in-depth description of their
structural characteristics and pore formation mechanism. Nevertheless, we will limit ourselves here to those
that are directly relevant to this dissertation. We will start with \glsfirst{ahl}, the first nanopore used
for the detection of biomolecules, and hence also the one with the largest body of experimental, theoretical,
and computational work. Next, we will discuss \glsfirst{clya}, \glsfirst{frac} and \glsfirst{plyab}, the three
nanopores that will be the subject of interest within the results chapters.


\subsection[Alpha-Hemolysin (aHL)]{\ta-Hemolysin (\ta{}HL)}
%
\label{sec:np:ahl}
%

%
\begin{figure*}[p]
  \centering
  \medskip
  
  %
  \begin{minipage}[t]{53mm}
    \begin{subfigure}[b]{53mm}
      \centering
      \caption{}\vspace{-8.5mm}\hspace{1.5mm}\label{fig:nanopores_ahl_pore_side}
      \includegraphics[scale=1]{nanopores_ahl_pore_side}
    \end{subfigure}
    %
    \vspace{5mm} \\
    %
    \begin{subfigure}[b]{53mm}
      \centering
      \caption{}\vspace{-8.5mm}\hspace{1.5mm}\label{fig:nanopores_ahl_pore_top}
      \includegraphics[scale=1]{nanopores_ahl_pore_top}
    \end{subfigure}
  \end{minipage}
  %
  \begin{minipage}{65mm}
    \begin{subfigure}[t]{53mm}
      \centering
      \caption{}\vspace{-8.5mm}\hspace{1.5mm}\label{fig:nanopores_ahl_pore_section}
      \includegraphics[scale=1]{nanopores_ahl_pore_section}
    \end{subfigure}
    %
    \vspace{5mm} \\
    %
    \begin{subfigure}[b]{32mm}
      \caption{}\vspace{-8.5mm}\hspace{1.5mm}\label{fig:nanopores_ahl_monomer}
      \includegraphics[scale=1]{nanopores_ahl_monomer}
    \end{subfigure}
    %
    \begin{subfigure}[b]{32mm}
      \centering
      \caption{}\vspace{-8.5mm}\hspace{1.5mm}\label{fig:nanopores_ahl_protomer}
      \includegraphics[scale=1]{nanopores_ahl_protomer}
    \end{subfigure}
  \end{minipage}
%

\caption[Structure of the \ta-hemolysin (\ta HL) porin]{%
  \textbf{Structure of the \ta-hemolysin (\ta HL) porin.}
  %
  (\subref{fig:nanopores_ahl_pore_side})
  %
  Side view and 
  %
  (\subref{fig:nanopores_ahl_pore_top})
  %
  top view of the heptameric \glsfirst{ahl} nanopore (\pdbid{7AHL}~\cite{Song-1996}), with a single subunit
  (protomer) highlighted in orange. The location of the lipid bilayer is indicated in blue, together with the
  \cisi{}  (`extracellular') and \transi{} (`intracellular') sides of the membrane. The pore is shaped as
  \SI{11}{\nm}-tall mushroom shaped with a cap (\ie~the extracellular part) and stem (\ie~the transmembrane
  \tb-barrel) widths of \SI{10}{\nm} and \SI{4}{\nm}, respectively.
  %
  (\subref{fig:nanopores_ahl_pore_section})
  %
  Electrostatically colored cross-section of the interior molecular surface of \gls{ahl}. The potential at
  physiological conditions was calculated using \gls{apbs}~\cite{Baker-2001,Baker-2005}. Indicated are the
  sizes of the \cisi{} vestibule, the transmembrane stem domain and the \SI{1.6}{\nm} wide central
  constriction.
  %
  (\subref{fig:nanopores_ahl_monomer})
  %
  Crystal structure of the \SI{33}{\kDa} water-soluble \gls{ahl} monomer (\pdbid{4YHD}~\cite{Sugawara-2015})
  compared to that of
  %
  (\subref{fig:nanopores_ahl_protomer})
  %
  the protomer (\pdbid{7AHL}~\cite{Song-1996}) in the final pore. While the extracellular cap and the
  membrane-binding rim domains remain virtually unaltered during the pore formation process, the N-terminal
  latch and (pre)stem regions undergo considerable conformational changes. 
  %
  All images were prepared and rendered using \gls{vmd}~\cite{Humphrey-1996,Stone-1998}.
  %
  }\label{fig:nanopores_ahl}
\end{figure*}
%

The \gls{ahl} protein (\cref{fig:nanopores_ahl}) is a \tb-\gls{pft} secreted by \textit{Staphylococcus aureus}
in its post-exponential and stationary growth phases~\cite{Bhakdi-1991}. It shows a high affinity for
sphingomyelin-cholesterol microdomains~\cite{Menestrina-2001,Valeva-2006}, but also readily binds
non-specifically to other membranes at higher concentrations~\cite{Hildebrand-1991}. Its high chemical and
structural stability (\eg~temperature, pH, urea, mutations), together with the early availability of its
crystal structure (1996)~\cite{Song-1996}, have made \gls{ahl} the workhorse of choice in the nanopore field
for many years.


\subsubsection{Pore structure, shape, and charge distribution.}
%

The \gls{ahl} pore (\pdbid{7AHL}~\cite{Song-1996}) is composed out of seven identical subunits of 293~amino
acids each, that form a \SI{232}{\kDa} mushroom-shaped protein complex with a height of \SI{11}{\nm} and a
width of \SI{10}{\nm} (\cref{fig:nanopores_ahl_pore_side,fig:nanopores_ahl_pore_top}). Its secondary structure
elements are predominantly \tb-sheets (\SI{\approx50}{\percent}), exemplified by the \SI{4}{\nm}-wide
\tb-barrel transmembrane region (the `stem') and the \tb-sandwich folds of its extracellular domains (the
`cap' and `rim'), supplemented with \SI{\approx35}{\percent} of non-conventional structure~\cite{Song-1996}.
The interior of \gls{ahl} (\cref{fig:nanopores_ahl_pore_section}) consists of a vestibule compartment at the
\cisi{} side, with an entry diameter of \SI{3.5}{\nm} and a height of \SI{5}{\nm}, followed by a
\SI{6}{\nm}-long, \SI{2}{\nm}-wide cylindrical channel. Separating both compartments is a narrow constriction
\SI{1.6}{\nm} in diameter, formed by residues Glu111 and Lys147. The electrostatic surface potential
(see~coloring of \cref{fig:nanopores_ahl_pore_section}) reveals that \gls{ahl}'s internal surface charge
distribution is balanced, with neither positive nor negative charges dominating. However, due to the slight
excess of positive charges at the \cisi{} entry and the constriction, wild type \gls{ahl} pores are found to
be somewhat anion selective~\cite{Menestrina-1986}.


\subsubsection{Oligomerization stoichiometry.}
%

Up until the release of the crystal structure, it was believed that the \gls{ahl} pore complex was a 6-mer
(\ie~hexameric), instead the now widely accepted 7-mer (\ie~heptameric)~\cite{Song-1996}. Even though it is
likely that \gls{ahl} also forms hexamers, the majority of pores (\SI{>90}{\percent}) formed under typical
conditions are heptameric~\cite{Menestrina-1986}, as supported by scientific evidence from a wide variety of
techniques such as crystallography~\cite{Song-1996,Galdiero-2004}, single-molecule fluorescence
photobleaching~\cite{Das-2007} and particle tracking~\cite{Thompson-2011}, and \gls{md}
simulations~\cite{Aksimentiev-2005,Bhattacharya-2011,Basdevant-2019}. 


\subsubsection{Monomer \textit{versus} protomer.}
%

When comparing the structures of the \gls{ahl} water soluble monomer (\cref{fig:nanopores_ahl_monomer},
\pdbid{4YHD}~\cite{Sugawara-2015}) with that of the protomer (\cref{fig:nanopores_ahl_protomer},
\pdbid{7AHL}~\cite{Song-1996}), it is evident that the majority of the secondary structure is maintained
during pore formation. Major conformational changes are observed for the stem domain, which unfolds to form
the transmembrane \tb-barrel, and within the triangle region, which connects it to the core of the pore.
Additionally, the first 20 N-terminal amino acids (\ie~the `N-latch') move away from their monomer positions
to (1) make contacts with the cap domain of its neighboring protomer and (2) destabilize the prestem domain.
Indeed, an intact N-latch was found to be essential for both stable oligomerization and high hemolytic
activity~\cite{Song-1996}. The \tb-sandwich folds of the extracellular `cap' and membrane-binding `rim'
domains remain nearly identical before and after pore formation.

\subsubsection{Mechanism of pore formation.}
%

After secretion by the host cell, the highly water-soluble \gls{ahl} monomers diffusive freely in solution
until they encounter, and bind to, the membrane of a target cell, mediated by both electrostatic and
hydrophobic interactions. It has been shown that \gls{ahl} has both specific and aspecific membrane-binding
modes~\cite{Hildebrand-1991}, with the former likely mediated by sphingomyelin-cholesterol
microdomains~\cite{Valeva-2006}. Once bound, the monomers will collide and reversibly oligomerize,
self-assembling into a heptameric `prepore' that finally transitions (quasi-irreversibly) into the final pore
by inserting its stem domains into the lipid bilayer. Single-molecule fluorescence particle tracking
experiments have revealed this process to be extremely rapid, with single monomers transitioning into
heptameric pores typically within \SI{5}{\ms}, and seemingly without formation of intermediate
states~\cite{Thompson-2011}. The ability of \gls{ahl} (and other \glspl{pft}) to keep the concentration of
intermediates low is thought to limit assembly errors~\cite{Lee-2016b}, mainly by preventing the formation of
higher order oligomers. At best, these may form pores with sub-optimal stoichiometries, at worst, they may
lead to non-functional aggregates, both of which are outcomes that reduce the efficacy of the
toxin~\cite{Fahie-2013,Subburaj-2015}. Energetically speaking, oligomerization is stabilized by both entropic
and enthalpic means, as in the final heptameric pore each protomer buries approximately \SI{33}{\percent} of
its solvent-accessible area, establishes 850 van der Waals contacts, and forms 120 salt bridges and hydrogen
bonds~\cite{Song-1996}. The prepore-to-pore transition is initiated by the destabilization of the prestem due
steric hindrance with its own N-latch and the neighboring prestem~\cite{Sugawara-2015}. Specifically, residues
Asp13--Gly15 of the N-latch compete with residue Tyr118 of the prestem for binding to D45 in the cap domain,
which causes all the prestem regions to release from their respective cores and assemble into the final stem
domain. This is likely to be a two-step process, in which first the extramembrane part of the \tb-barrel is
formed, followed by the transmembrane section~\cite{Sugawara-2015}.



\subsection{Cytolysin A (ClyA)}
%
\label{sec:np:clya}
%

\begin{figure*}[p]
  \centering
  \medskip
    %
  \begin{minipage}[t]{51mm}
    \begin{subfigure}[b]{51mm}
      \centering
      \caption{}\vspace{-8.5mm}\hspace{1.5mm}\label{fig:nanopores_clya_pore_side}
      \includegraphics[scale=1]{nanopores_clya_pore_side}
    \end{subfigure}
    %
    \vspace{5mm} \\
    %
    \begin{subfigure}[b]{51mm}
      \centering
      \caption{}\vspace{-8.5mm}\hspace{1.5mm}\label{fig:nanopores_clya_pore_top}
      \includegraphics[scale=1]{nanopores_clya_pore_top}
    \end{subfigure}
  \end{minipage}
  %
  \begin{minipage}{66mm}
    \begin{subfigure}[t]{60mm}
      \centering
      \caption{}\vspace{-8.5mm}\hspace{1.5mm}\label{fig:nanopores_clya_pore_section}
      \includegraphics[scale=1]{nanopores_clya_pore_section}
    \end{subfigure}
    %
    \vspace{5mm} \\
    %
    \begin{subfigure}[b]{37mm}
      \caption{}\vspace{-8.5mm}\hspace{1.5mm}\label{fig:nanopores_clya_monomer}
      \includegraphics[scale=1]{nanopores_clya_monomer}
    \end{subfigure}
    %
    \begin{subfigure}[b]{28mm}
      \centering
      \caption{}\vspace{-8.5mm}\hspace{1.5mm}\label{fig:nanopores_clya_protomer}
      \includegraphics[scale=1]{nanopores_clya_protomer}
    \end{subfigure}
  \end{minipage}
\caption[Structure of the cytolysin A (ClyA) porin]{%
  \textbf{Structure of the cytolysin A (ClyA) porin.}
  %
  (\subref{fig:nanopores_clya_pore_side})
  %
  Side view and 
  %
  (\subref{fig:nanopores_clya_pore_top})
  %
  top view of the dodecameric \glsfirst{clya} nanopore (\pdbid{6MRT}~\cite{Peng-2019}), with a single
  subunit (protomer) highlighted in color and the location of the lipid bilayer indicated in blue.
  %
  (\subref{fig:nanopores_clya_pore_section})
  %
  Cross-section view of the interior molecular surface of the pore, colored according to its estimated
  electrostatic potential in physiological conditions as calculated by
  \gls{apbs}~\cite{Baker-2001,Baker-2005}.
  %
  (\subref{fig:nanopores_clya_monomer})
  %
  The \SI{34}{\kDa} water-soluble \gls{clya} monomer (\pdbid{1QOY}~\cite{Wallace-2000}) consists of a core
  bundle of \ta-helices (\ta A1, \ta A2, \ta B, \ta C, \ta F), onto which a hydrophobic \tb-tongue
  motif---flanked by two short \ta-helices (\ta D and \ta E)---and a C-terminal \ta-helix (\ta G) are packed.
  The two glycine residues in the \tb-tongue confer it with a high degree of flexibility, while the removal of
  Phe191 from its hydrophobic pocket triggers pore formation.
  %
  (\subref{fig:nanopores_clya_protomer})
  %
  A single \gls{clya} protomer (\pdbid{6MRT}~\cite{Peng-2019}), with the same coloring as the monomer, showing
  the insertion of the \ta-tongue into the lipid bilayer and the near \SI{14}{\nm} conformational change
  required for the \ta A1 helix to form the transmembrane part of the pore.
  %
  All images were prepared and rendered using \gls{vmd}~\cite{Humphrey-1996,Stone-1998}.
  }\label{fig:nanopores_clya}
\end{figure*}

In 2009, Mueller \etal{} resolved the crystal structure of \gls{clya}
(\cref{fig:nanopores_clya})~\cite{Mueller-2009}, a large \ta-\gls{pft}, secreted by various \textit{S.
enterica} and \textit{E. coli} strains, with a high affinity for mammalian cell membranes. The large size of
its \lumen{}---compared to \gls{ahl} and \gls{mspa}, the most commonly used protein nanopores at the
time---makes \gls{clya} particularly well equipped to fully capture and study larger biopolymers, such as
proteins, in their native three-dimensional configuration
(\ie~folded)~\cite{Soskine-2013,Soskine-Biesemans-2015}.


\subsubsection{Pore structure, shape, and charge distribution.}
%

The \SI{408}{\kDa} \gls{clya} pore (\pdbid{6MRT}~\cite{Peng-2019}) consists of 12 identical subunits
(protomers), arranged in a ring like quaternary structure to form a cylindrical complex roughly \SI{14}{\nm}
in height and \SI{11}{\nm} in diameter (\cref{fig:nanopores_clya_pore_side,fig:nanopores_clya_pore_top}). The
pore walls in contact with the solvent are formed by a core bundle of four tightly packed \ta-helices per
protomer, and the transmembrane region is formed by the iris-like arrangement of the amphipathic N-terminal
\ta-helices. Virtually all helices contribute to interprotomer contacts, resulting in the burying of
\SI{\approx2400}{\square\angstrom} surface area, and the formation of 25~H-bonds and 13~salt bridges per
protomer-protomer interface. The interior shape of \gls{clya} can be described as a set of two hollow
cylinders placed on top of each other: a large \cisi{} chamber (`\lumen{}') with an inner diameter of
\SI{\approx5.5}{\nm} and a height of \SI{10}{\nm} and a smaller \transi{} chamber (`\textit{constriction}')
with a width of \SI{\approx3.6}{\nm} and a height of \SI{4}{\nm} (\cref{fig:nanopores_clya_pore_section}).
\Gls{clya} has an excess negative charge (\SI{-60}{\ec} at \pH{7.5}), most of which riddle the interior walls
of the pore. This results in a negative electrostatic potential inside both the \lumen{} and the constriction
of the pore (\cref{fig:nanopores_clya_pore_section}) and explains the observed cation
selectivity~\cite{Soskine-2012,Franceschini-2016}.

\subsubsection{Oligomerization stoichiometry.}
%

Next to the typical 12-subunit pore (dodecamer, `Type I'), \gls{clya} can also form 13-mers (tridecamers,
`Type II') and 14-mers (tetradecamers, `Type III')~\cite{Soskine-2013,Peng-2019}, with constriction diameters
of \SIrange{3.3}{4.0}{\nm}, \SIrange{3.7}{4.4}{\nm} and \SIrange{4.2}{5.2}{\nm}, respectively. Aside from
their increased aperture size, the overall structure and assembly mechanism of these pores is identical to
that of the dodecamer, and hence the following discussion will be limited to the Type I pore. Interested
readers are referred to the work of Peng and coworkers~\cite{Peng-2019}, who recently obtained high-resolution
structures of the Type II and III pores using \gls{cryo-em}.


\subsubsection{Monomer \textit{versus} protomer.}
%

The water-soluble \gls{clya} monomer (\cref{fig:nanopores_clya_monomer}) consists of predominantly
\ta-helices, with the core of the protein formed by a bundle of four long \ta-helices (\ta A, \ta B, \ta C and
\ta F). Packed at the top and bottom of this bundle are respectively the C-terminal helix (\ta G) and a
hydrophobic \tb-hairpin motif flanked by two short \ta-helices (\ta{}D and
\ta{}E)~\cite{Wallace-2000,Mueller-2009}. Comparing this structure to that of the protomer
(\cref{fig:nanopores_clya_protomer}), reveals that while three of the core helices are merely straightened and
elongated (at the expense of the \tb-hairpin and the \ta{}A2, \ta{}D and \ta{}E helices), the \ta{}A1 helix
must move to the opposite side of the monomer, a distance of \SI{\approx14}{\nm}! The mechanism that enables
these large conformational changes is similar to that of a spring-loaded lock, which is opened upon membrane
binding.

\subsubsection{Mechanism of pore formation.}
%

Initial binding of \gls{clya} monomers to the membrane (\textit{via} both specific and non-specific
interactions) is mediated by the partially solvent-exposed, hydrophobic residues of the \tb-hairpin motif.
Using Gly180 and Gly201 as hinges, the \tb-hairpin swings outwards to insert itself into the lipid bilayer.
This removes a stabilizing aromatic residue (Phe190) from its hydrophobic pocket and unlatches a two-helix
spring-loaded mechanism. This results in 1) the straightening and extension of three of the core helices (\ta
B, \ta C and \ta F), and 2) a \SI{180}{\degree} swinging of the \ta A helix (relative to \ta B, \ta C and \ta
F). The amphipathic \ta-A1 helix, which will form the \gls{tmd}, now rests on top of the membrane and is
connected to the rest of the protein \textit{via} a \gls{hth} motif, with Pro36 acting as a hinge. The
subsequent oligomerization of the membrane-bound monomers further packs and buckles the \ta A1 helices,
effectively pushing them downwards and allowing them to `wedge' a hole into the lipid bilayer. In contrast
with \gls{ahl}, \gls{clya} pore formation takes place on a timescale of minutes rather than
milliseconds~\cite{Benke-2015}. Initial far-UV \gls{cd} measurements of the \gls{clya} oligomerization process
in the presence of the mild, non-ionic detergent \gls{ddm} suggested a two-step process, in which the soluble
monomers undergo a rapid transition ($t_{1/2} \approx \SI{80}{\second}$) to a molten globule-like
intermediate, followed by a slow assembly  ($t_{1/2} \approx \SI{1000}{\second}$) of the membrane-bound
protomers into the full dodecameric pore~\cite{Eifler-2006}. However, subsequent investigation of the
oligomerization kinetics using \gls{rbc} hemolysis assays~\cite{Vaidyanathan-2014} and single-molecule
\gls{fret}~\cite{Benke-2015} revealed that the intermediate state is actually an off-pathway by-product that
slows down the pore formation~\cite{Roderer-2017}. Nevertheless, on the long term, most of the monomers will
convert to protomers rather than the molten globule intermediate, and intermolecular collisions between
protomers results in the formation of linear mixtures of oligomers (`protomer
elongation'~\cite{Roderer-2017}). Because virtually all linear oligomer species contribute to this process,
the effective protomer concentration does not decrease, making it both rapid and efficient. Formation of the
full pore then occurs rapidly \textit{via} ring closure when a pair of oligomers---with a total protomer count
of 12---collide and associate. Even though all linear oligomer species contribute to the formation of the
dodecameric pore, the relatively high abundance of the 5-, 6- and 7-mers dictate that \SI{>50}{\percent} of
all full pores are formed by these oligomers~\cite{Benke-2015}.



\subsection{Fragaceatoxin C (FraC)}
%
\label{sec:np:frac}
%

\begin{figure*}[p]
  \centering
  \medskip
    %
  \begin{minipage}[t]{58mm}
    \begin{subfigure}[t]{58mm}
      \centering
      \caption{}\vspace{-8.5mm}\hspace{1.5mm}\label{fig:nanopores_frac_pore_side}
      \includegraphics[scale=1]{nanopores_frac_pore_side}
    \end{subfigure}
    %
    \vspace{5mm} \\
    %
    \begin{subfigure}[t]{58mm}
      \centering
      \caption{}\vspace{-8.5mm}\hspace{1.5mm}\label{fig:nanopores_frac_pore_top}
      \includegraphics[scale=1]{nanopores_frac_pore_top}
    \end{subfigure}
  \end{minipage}
  %
  \begin{minipage}[t]{58mm}
    \begin{subfigure}[t]{58mm}
      \centering
      \caption{}\vspace{-8.5mm}\hspace{1.5mm}\label{fig:nanopores_frac_pore_section}
      \includegraphics[scale=1]{nanopores_frac_pore_section}
    \end{subfigure}
    %
    \vspace{5mm} \\
    %
    \begin{subfigure}[t]{28mm}
      \caption{}\vspace{-8.5mm}\hspace{1.5mm}\label{fig:nanopores_frac_monomer}
      \includegraphics[scale=1]{nanopores_frac_monomer}
    \end{subfigure}
    %
    \begin{subfigure}[t]{28mm}
      \centering
      \caption{}\vspace{-8.5mm}\hspace{1.5mm}\label{fig:nanopores_frac_protomer}
      \includegraphics[scale=1]{nanopores_frac_protomer}
    \end{subfigure}
  \end{minipage}
\caption[Structure of the fragaceatoxin C (FraC) porin]{%
  \textbf{Structure of the fragaceatoxin C (FraC) porin.}
  %
  (\subref{fig:nanopores_frac_pore_side})
  %
  Side and 
  %
  (\subref{fig:nanopores_frac_pore_top})
  %
  top views of the octameric \SI{176}{\kDa} \gls{frac} porin crystal structure
  (\pdbid{4TSY}~\cite{Tanaka-2015}) with a single protein subunit highlighted in color and the co-crystallized
  sphingomyelin lipids indicated in yellow. The overall shape is that of a funnel with a height of
  \SI{7}{\nm}, and top and bottom diameters of \SIlist{11;3.5}{\nm}, respectively.
  %
  (\subref{fig:nanopores_frac_pore_section})
  %
  Cross-sectional view of the internal molecular surface of \gls{frac}, colored according to the local
  electrostatic potential at physiological conditions as calculated by
  \gls{apbs}~\cite{Baker-2001,Baker-2005}. \Gls{frac} has inner diameters of \SIlist{6;1.6}{\nm} on the
  \cisi{} and \transi{} sides, respectively. Note the fenestrations between the subunits, which are partly
  filled by permanently bound lipids and hence expose and the core of membrane bilayer to the solvent.
  %
  (\subref{fig:nanopores_frac_monomer})
  %
  The crystal structure of the \SI{20}{\kDa} \gls{frac} monomer (\pdbid{3VWI}~\cite{Tanaka-2015}) shows that
  in its water-soluble form, the N-terminal \gls{tmd} (blue) is packed tightly against the \tb-sheet rich core
  domain (orange). Displacement of Phe16 (mauve) from its hydrophobic pocket by Val60 (gray) triggers pore
  formation.
  %
  (\subref{fig:nanopores_frac_protomer})
  %
  The N-terminal \ta-helix of each \gls{frac} protomer (\pdbid{4TSY}~\cite{Tanaka-2015}) first elongates and
  then flips \SI{180}{\degree} to span the entire membrane. The conformation of \tb-core remains virtually
  identical to that of the monomer. Where the L2 and L3 lipids are bound to a single subunit, the L1 lipid is
  wedged in between two protomers and contributes structurally to the solvent-exposed pore walls.
  %
  Images were rendered using \gls{vmd}~\cite{Humphrey-1996,Stone-1998}.
  %
  }\label{fig:nanopores_frac}
\end{figure*}

\Gls{frac} (\cref{fig:nanopores_frac}) is an \ta-\gls{pft} belonging to the family of the actinoporins and is
produced by the sea anemone \textit{Actinia fragacea}. Its X-ray structure was resolved in 2015 by Tanaka
\etal{}~\cite{Tanaka-2015}, and surprisingly contained several co-crystallized \gls{sm} lipids per subunit.
This indicates that \gls{frac} does not merely exhibit a high degree of specificity towards
\gls{sm}-containing membranes, but that these lipids comprise an essential structural element of the pore
itself. Recently, \gls{frac} has become a popular for the analysis of peptides~\cite{Huang-2017,Huang-2019}
and amino acids~\cite{Restrepo-Perez-2019a}, making it a promising candidate for protein sequencing
applications~\cite{Restrepo-Perez-2018}.

\subsubsection{Pore structure, shape, and charge distribution.}

The full \SI{176}{\kDa} \Gls{frac} porin (\pdbid{4TSY}~\cite{Tanaka-2015}) consists of eight identical protein
subunits and three \gls{sm} lipids per protomer, bound at highly specific locations
(\cref{fig:nanopores_frac_pore_side,fig:nanopores_frac_pore_top}). It is shaped as a funnel with a height of
\SI{7}{\nm} and \cisi{} and \transi{} outer diameters of \SIlist{11;3.5}{\nm}, respectively. The extracellular
(\cisi) part of the pore comprises the bulk of each protomer and is rich in \tb-sheets (\tb-core). The
\gls{tmd} is formed by the iris-like arrangement of the N-terminal \ta-helices of each subunit, which are long
enough (\SI{3.5}{\nm}) to span the entire lipid bilayer (\SI{\approx2.8}{\nm}). The protomer-protomer
interface buries a surface area of \SI{777}{\square\angstrom}, to which both the \tb-core and the \gls{tmd}
\ta-helices contribute and which is stabilized by protein-lipid H-bonding~\cite{Tanaka-2015}. Close inspection
of the pore walls reveals the presence of eight lateral fenestrations, one between each protomer, that expose
the hydrophobic core of the membrane to the aqueous solvent (\cref{fig:nanopores_frac_pore_section}). These
openings are at least partially occluded by the co-crystallized \gls{sm} lipids, which, as such, contribute
structurally to the pore. Even though the precise biological function of these fenestrations remains unclear,
it is likely that they contribute significantly to the toxicity of sea anemone venom by 1) facilitating the
diffusion of its small hydrophobic components directly into the core of the lipid bilayer, and 2) disrupting
the leaflet composition of the membrane by catalyzing the flip-flop movement of lipids~\cite{Tanaka-2015}. The
interior of \gls{frac} is funnel-shaped, with a \cisi{} diameter of \SI{6}{\nm} and a \trans{} diameter of
only \SI{1.6}{\nm}, the latter also being the pore's narrowest location. Consistent with the observation of
its cation selectivity~\cite{Garcia-Ortega-2011,Wloka-2016}, the interior walls of the pore are predominantly
negatively charged, particularly at the \transi{} constriction (\cref{fig:nanopores_frac_pore_section}). The
extracellular regions of the pore near the lipid headgroups exhibit a strong positive potential however, which
likely aides in the initial adhesion of \gls{frac} to the membrane~\cite{Tanaka-2015}.

%
\subsubsection{Oligomerization stoichiometry.}
%

The lytically active \gls{frac} pore can be obtained as an 8-mer (octamer, `Type~I'), 7-mer (heptamer
`Type~II') and 6-mer (hexamer, `Type~III')~\cite{Huang-2019}, with constriction diameters of
\SIlist{1.6;1.1;0.84}{\nm}, respectively. A non-lytic 9-mer (nonamer) has also been crystallized
(\pdbid{3LIM}~\cite{Mechaly-2011}), and lower oligomeric states also appear to be capable of forming
pores~\cite{Rojko-2016}.

%
\subsubsection{Monomer \textit{versus} protomer.}
%

In the water-soluble monomer of the \gls{frac} pore (\pdbid{3VWI}~\cite{Tanaka-2015}), the N-terminal region
(residues 1--29; contains a short \tb-sheet, a $3_{10}$ helix and an \ta-helix) is packed closely against the
C-terminal \tb-core (residues 30--179; contains 11 \tb-sheets in a \tb-sandwich fold and one \ta-helix) and
held in place by the interaction between Phe16 and a hydrophobic cavity in the \tb-core
(\cref{fig:nanopores_frac_monomer}). The \tb-core remains virtually unchanged in the protomer structure
(\pdbid{4TSY}~\cite{Tanaka-2015}), whereas the N-terminal domain forms a much longer \ta-helix that spans the
entire lipid bilayer (\cref{fig:nanopores_frac_protomer}). Out of the three lipids (`L1', `L2' and `L3') that
are bound at specific locations to the \tb-core, only the L1 position can be considered to be of high affinity
and specific only to \gls{sm}, whereas L2 and L3 are of lower affinity and hence more
promiscuous~\cite{Tanaka-2015}. 

%
\subsubsection{Mechanism of pore formation.}
%

Even though the precise mechanisms with which actinoporins form pores is still under debate, several
experiments have already provided significant insights. Upon initial attachment of the monomer to the
membrane, the binding of a single \gls{sm} lipid to the L1 pocket induces dimerization~\cite{Tanaka-2015}. The
resulting small conformational change (at residues 14--17) displaces Phe16 from its hydrophobic pocket in
favor of Val60, and partially unfolds the N-terminal domain. Because dimerization is only possible in the
presence of \gls{sm}, this lipid acts as both a receptor and an assembly co-factor that contributes to the
final structure of the pore. Sequential addition of other dimeric units \textit{via} random collisions, which
can occur frequently in the small \gls{sm} lipid raft domains, give rise to higher oligomeric complexes that
further destabilize the N-terminal region. Given the limited space available in the \lumen{} of the final pore
(\SI{\approx2}{\nm} diameter), it is likely that the insertion of the N-terminal \ta-helix occurs in a
non-concerted manner prior to full ring closure~\cite{Cosentino-2016}. It is also possible that a pre-pore
structure is formed, similar to the non-lytic \gls{frac} nonamer~\cite{Mechaly-2011}, with the amphipathic
N-termini pushing through the membrane in a concerted fashion~\cite{Tanaka-2015,Rojko-2016} much like the
\gls{clya} porin. Evidence favors the first mechanism, however, given that a sole N-terminal helix of many
actinoporins can exist at the lipid-water interface in a so-called `protein-lipid pore'~\cite{Cosentino-2016}.

%
\clearpage
%

%
\subsection{Pleurotolysin AB (PlyAB)}
%
\label{sec:np:plyab}
%

\Gls{plyab} (\cref{fig:nanopores_plyab}) is a \tb-\gls{pft} secreted by the fungus \textit{Pleurotus
ostreatus} and belongs to the family of \gls{macpf} proteins. Lukoyanova and Kondos \etal{} proposed a
plausible pore structure based on the \gls{cryo-em} image analysis and molecular modeling with a molecular
weight in excess of \SI{1000}{\kDa}~\cite{Lukoyanova-Kondos-2015}, making it one of the largest \glspl{pft}
with a known structure to date.

%
\begin{figure*}[p]
  \centering
  \medskip
  %
  \begin{minipage}[t]{56mm}
    \begin{subfigure}[t]{56mm}
      \centering
      \caption{}\vspace{-8.5mm}\hspace{1.5mm}\label{fig:nanopores_plyab_pore_side}
      \includegraphics[scale=1]{nanopores_plyab_pore_side}
    \end{subfigure}
    %
    \vspace{5mm} \\
    %
    \begin{subfigure}[t]{56mm}
      \centering
      \caption{}\vspace{-8.5mm}\hspace{1.5mm}\label{fig:nanopores_plyab_pore_top}
      \includegraphics[scale=1]{nanopores_plyab_pore_top}
    \end{subfigure}
  \end{minipage}
  %
  \begin{minipage}[t]{60mm}
    \begin{subfigure}[t]{56mm}
      \centering
      \caption{}\vspace{-8.5mm}\hspace{1.5mm}\label{fig:nanopores_plyab_pore_section}
      \includegraphics[scale=1]{nanopores_plyab_pore_section}
    \end{subfigure}
    %
    \vspace{5mm} \\ 
    %
    \begin{subfigure}[t]{35mm}
      \caption{}\vspace{-8.5mm}\hspace{1.5mm}\label{fig:nanopores_plyab_monomer}
      \includegraphics[scale=1]{nanopores_plyab_monomer}
    \end{subfigure}
    %%
    \begin{subfigure}[t]{24mm}
      \centering
      \caption{}\vspace{-8.5mm}\hspace{1.5mm}\label{fig:nanopores_plyab_protomer}
      \includegraphics[scale=1]{nanopores_plyab_protomer}
    \end{subfigure}
  \end{minipage}
\caption[Structure of the pleurotolysin AB (PlyAB) porin]{%
  \textbf{Structure of the pleurotolysin AB (PlyAB) porin.}
  %
  (\subref{fig:nanopores_plyab_pore_side})
  %
  Side and
  %
  (\subref{fig:nanopores_plyab_pore_top})
  %
  top view of the \SI{1067}{\kDa} \gls{plyab} \tb-\gls{pft} porin
  (\pdbid{4V2T}~\cite{Lukoyanova-Kondos-2015}), formed by 13 identical subunits which are each composed of two
  \gls{plya} (pink) and one \gls{plyb} (green) protein chains. The full pore is \SI{13}{\nm} high with
  external diameters of \SI{22}{\nm} and \SI{9}{\nm} at the \cisi{} and \transi{} sides of the  membrane. 
  %
  (\subref{fig:nanopores_plyab_pore_section})
  %
  The cross-section of \gls{plyab}'s molecular surface, colored according to the local electrostatic potential
  at physiological salt concentration (\gls{apbs}~\cite{Baker-2001,Baker-2005}), shows that the interior of
  the pore is predominantly negative. The \lumen{} of the pore is divided by a \SI{5.5}{\nm} wide constriction
  into a conical \cisi{} chamber of \SI{3}{\nm} height and \SI{10.5}{\nm} diameter, and a cylindrical
  \transi{} chamber of \SI{10}{\nm} height and \SI{7.2}{\nm} diameter. 
  %
  (\subref{fig:nanopores_plyab_monomer})
  %
  Cartoon view of the water-soluble monomer configurations of \gls{plya}
  (\pdbid{4OEB}~\cite{Lukoyanova-Kondos-2015}) and \gls{plyb} (\pdbid{4OEJ}~\cite{Lukoyanova-Kondos-2015}),
  aligned according to their position in a single protomer of the full pore (using
  \gls{vmd}~\cite{Humphrey-1996}). Important regions and residues are highlighted in color.
  %
  (\subref{fig:nanopores_plyab_protomer})
  %
  A single \gls{plyab} protomer subunit (\pdbid{4V2T}~\cite{Lukoyanova-Kondos-2015}), showing the extension of
  the transmembrane \ta-helices into \tb-sheets.
  %
  Images were rendered using \gls{vmd}~\cite{Humphrey-1996,Stone-1998}.
  %
  }\label{fig:nanopores_plyab}
\end{figure*}
%

\subsubsection{Pore structure, shape, and charge distribution.}
%

The \gls{plyab} pore (\pdbid{4V2T}~\cite{Lukoyanova-Kondos-2015}) consists of 13 identical `subunits', each of
which is composed of a lipid binding \gls{plya} dimer and a single pore forming \gls{plyb} protein chain
(\cref{fig:nanopores_plyab_pore_side,fig:nanopores_plyab_pore_top}). Similar to \gls{ahl}, the entire pore
complex is mushroom shaped, with a height of \SI{13}{\nm}, a head diameter of \SI{22}{\nm}, and a stem
diameter of \SI{9}{\nm}. When traversing the channel of \gls{plyab} (\cref{fig:nanopores_plyab_pore_section})
from the \cisi{}-side, the \SI{10.5}{\nm}-wide pore entry quickly begins to narrow down to a diameter of
\SI{5.5}{\nm} at a depth of \SI{3}{\nm}. Once past this constriction, the pore opens up into the pore
\lumen{}, a \SI{10}{\nm}-long, \SI{7.2}{\nm}-wide cylindrical chamber formed by the large \tb-barrel
\gls{tmd}. The interior walls of wild-type \gls{plyab} are predominantly negatively charged, particularly at
the constriction and in the middle of the \lumen{} (see~surface coloring in
\cref{fig:nanopores_plyab_pore_section}). Likely, this negatively charged interior enables \gls{plyab} to
employ charge-based facilitated diffusion to preferentially deliver small cationic proteins, such as
granzyme~B, over neutral and anionic ones~\cite{Stewart-2014,Reboul-2016}.


\subsubsection{Oligomerization stoichiometry.}
%

As with the other pores discussed above, \gls{plyab} can form pores with different stoichiometries.
\Gls{cryo-em} analysis revealed a diverse population of 14- (\SI{5}{\percent}), 13- (\SI{75}{\percent}), 12-
(\SI{15}{\percent}) and 11-mers (\SI{5}{\percent})~\cite{Lukoyanova-Kondos-2015}.


\subsubsection{Monomer \textit{versus} protomer.}
%

The water-soluble \gls{plya} crystal structure (\pdbid{4OEB}~\cite{Lukoyanova-Kondos-2015}) contains a
membrane-binding \tb-sandwich fold (\cref{fig:nanopores_plyab_monomer})---similar to that of the actinoporin
family (\eg~\gls{frac}, \cref{fig:nanopores_frac_monomer})---but lacks the typical N-terminal transmembrane
region. The water-soluble \gls{plyb} structure (\pdbid{4OEJ}~\cite{Lukoyanova-Kondos-2015}) can be subdivided
in a \tb-rich globular trefoil domain at the C-terminus, and a \gls{macpf} domain at the N-terminus. At its
core, the latter contains a four-stranded bent and twisted \tb-sheet (typical for the \gls{macpf}
superfamily), a flexible \gls{hth}-motif and two \ta-helical regions ({TMH1} and {TMH2}). In the protomer
structure (\pdbid{4VT2}~\cite{Lukoyanova-Kondos-2015}), the {TMH1} and {TMH2} regions become fully unwound
into \tb-hairpins to become part of the 52-sheet membrane-spanning \tb-barrel
(\cref{fig:nanopores_plyab_protomer}).


\subsubsection{Mechanism of pore formation.}
%

Pore formation is initiated by the binding of the water-soluble, potentially pre-dimerized, \gls{plya}
components to the target membrane. Each \gls{plya} dimer will then recruit a larger \gls{plyb} subunit to form
the membrane-bound \gls{plyab} monomer. High speed imaging of perforin pore assembly has revealed that these
protomers loosely (but irreversibly) assemble into prepore oligomers of up to 8 subunits, which combine
further into larger prepores~\cite{Leung-2017}. When they are sufficiently large, these prepores will close
their ring structure, followed by a set of large conformational changes that ends with the formation of a
large \gls{tmd} \tb-barrel. The \gls{cryo-em} and mutagenesis study performed by
Lukoyanova~\etal{}~\cite{Lukoyanova-Kondos-2015} revealed that the \tb-barrel formation is triggered by the
disruption of the interaction between the \gls{hth}-motif and the {TMH2} helix during oligomerization. This
causes the opening of the central {MACPF} \tb-sheets, resulting in the gradual cooperative refolding of the
{TMH1} and {TMH2} helices towards the membrane and the top-down zippering of the
\tb-barrel~\cite{Reboul-2016}.


%
\clearpage
%

%
%
\section{A physical perspective on nanopore sensing}
%
\label{sec:np:physical_perspective}
%

The apparent simplicity of the principles behind nanopore sensing, together with its scalability and high
sensitivity towards even the smallest of changes within them, have made nanopores into one the most successful
and broadly applied single-molecule sensors. Nanopores have been used as detectors for small molecules, down
to individual ions, as well as biological polymers or particles, such as \gls{dna}, peptides, proteins, and
virions. Equally broad are the number of strategies that have been deployed to improve the selectivity and
sensitivity of nanopores: from the conjugation of binding sites, peptides, aptamers, or even entire proteins,
within the pore or to one of its entries, to the modification of the analyte molecules themselves. An in-depth
overview of the literature describing these strategies can be found in the Ph.D. dissertation of Veerle Van
Meervelt~\cite{VanMeervelt-2017-PhD}. Regardless of the precise details of the sensing approach, before any
analyte molecule can be detected,  it must first be attracted towards the pore, and subsequently captured for
a period of time sufficiently long to be sampled. In the following sections, we will describe the origin of
the forces at play in nanopores and clarify their importance during all stages of the sensing process. We will
start this section with discussing a phenomenon that mediates many of these interactions: the \gls{edl}.


\subsection{The electrical double layer within a nanopore}
%
\label{sec:np:edl}
%

%
\begin{figure*}[ptb]
  \centering

  %
  \begin{subfigure}[t]{115mm}
    \centering
    \caption{}\vspace{-2.5mm}\label{fig:nanopores_edl_overview}
    \includegraphics[scale=1]{nanopores_edl_overview}
  \end{subfigure}
  %
  \\
  %
  \begin{subfigure}[t]{115mm}
    \centering
    \caption{}\vspace{-2.5mm}\label{fig:nanopores_edl_confined}
    \includegraphics[scale=1]{nanopores_edl_confined}
  \end{subfigure}
  %

\caption[Schematic overview of the structure of the electrical double layer]{%
  \textbf{Schematic overview of the structure of the electrical double layer (EDL).}
  %
  (\subref{fig:nanopores_edl_overview})
  %
  When a charged surface is brought into contact with an electrolyte solution, mobile ions of opposite charge
  (\ie~the counterions) will accumulate near the surface, whereas same-charge ions (\ie~the co-ions) will
  deplete. This region of excess charge in the fluid is called the \glsfirst{edl} and has a characteristic
  thickness equal to the Debye length ($\dbl$, see~\cref{eq:nanopores_debye_length}). Counterions in the Stern
  layer are semi-permanently bound to the surface and partially lose their hydration shell. Because the
  surface electrostatic potential drops exponentially inside the electrical double layer, its influence
  becomes negligible after a few Debye lengths, rapidly resulting in an electroneutral bulk ionic
  distribution.
  %
  (\subref{fig:nanopores_edl_confined})
  %
  Effect of geometrical confinement on the composition of the \gls{edl} within a nanopore at high (left) and
  low (right) ionic strengths. At high salt concentrations, $\dbl$ is short enough to allow the center of the
  pore to remain electroneutral. At low salt concentrations on the other hand, $\dbl$ is the same or larger
  than the radius of the pore, the \glspl{edl} begin to overlap and the center of the pore also contains an
  excess of counterions. Regardless of the salt concentration, the \gls{edl} takes up a significant portion of
  the total pore volume, and hence plays an important role in all nanopore transport phenomena.
  %
  }\label{fig:nanopores_edl}
\end{figure*}
%

Virtually all intermolecular interactions are either directly (\eg~charge-charge, H-bonds) or indirectly
(\eg~\gls{vdw} forces) electrostatic in nature. Indeed, most surfaces, including those of nanopores and large
analyte molecules, contain ionizable groups that become charged when brought into contact with an aqueous
solution. Hence, it is instrumental to reflect on both their magnitude and the length-scales at which they
exert their influence. At the level of interactions between two individual charges, a first characteristic
length scale for electrostatics is the Bjerrum length, $\bjl$, which corresponds to the distance at which the
electrostatic interaction energy between two charged species, $\energyelec (\bjl)$, becomes similar to their
thermal energy, $\boltzmann \temperature$. In other words, it represents the distance below which the
electrostatic interactions start to dominate over thermal effects. For electrolyte solutions containing two
ion species, it can be expressed as~\cite{Bocquet-2010}
%
\begin{equation}\label{eq:nanopores_bjerrum_length}
  \bjl = \dfrac{\ec^2}{4 \pi \absperm \relperm \boltzmann \temperature}
  \text{ ,}
\end{equation}
%
with $\ec$ the elementary charge, $\absperm$ the vacuum permittivity, $\relperm$ the relative permittivity of
the solution, $\boltzmann$ the Boltzmann constant, and $\temperature$ the temperature. For a typical
monovalent salt at room temperature $\bjl = \mSI{0.7}{\nm}$, a value that comes close to the radius of many
(biological) nanopores. Hence, it is evident that the transport of ions and other charged molecules through
nanopores will be influenced by electrostatic interactions. However, the typical electrolyte contains many
highly mobile ions in close proximity to each other: for an electrolyte with a physiological salt
concentration (\eg~\SI{0.15}{\Molar}), a volume of \grid{10}{10}{10}{\cubic\nm} contains 180~ions, which
corresponds to a mean ion-ion spacing of only \SI{\approx1.8}{\nm}. Consequently, even though thermal effects
dominate at distances $>\bjl$ for individual ions, the extent with which local perturbations can propagate to
longer length scales will depend on the (local) density of all ions the in solution. This phenomenon leads to
the formation of the \glsfirst{edl} (\cref{fig:nanopores_edl_overview}), which describes the rearrangement of
the ion density in response to a fixed surface charge density or potential. The \gls{edl} is a diffuse layer
that contains an excess of counter-ions (\ie~opposing the wall charge) and a depletion of co-ions (\ie~same as
the wall charge), whose width is inversely proportional to both $\bjl$ and the ionic strength of the
electrolyte, $\ionstr$. A characteristic value is given by the so-called Debye length, $\dbl$, which, for an
electrolyte with $N$ ion species, is given by~\cite{Bocquet-2010}
%
\begin{equation}\label{eq:nanopores_debye_length}
  \dbl = \left(
          \dfrac{ \permittivity \boltzmann \temperature }%
                { \ec^2 \avogadro \dsum_i^N \ci^0 \chargeni^2 } \right)^{1/2}
       \equiv \left( 8 \pi \bjl \avogadro \ionstr \right)^{-1/2}
  \text{ ,}
\end{equation}
%
with $\ci^0$ the molar concentration of ion $i$ (the $0$ superscript indicates that it is the `bulk'
concentration), $\chargeni$ its charge number (\ie~valence), and $\avogadro$ the Avogadro constant. Being the
characteristic length scale of the \gls{edl}, $\dbl$ corresponds to the thickness of the diffuse layer next to
the charged wall. Additionally, it represents the length scale over which the electrolyte screens the bulk
medium from the surface electrostatic potential. For salt concentrations of \SIlist{0.01;0.1;1}{\Molar},
$\dbl$ amounts to \SIlist{3.1;0.97;0.31}{\nm}, respectively. The local violation of electroneutrality, caused
by the imbalance between cations and anions within the diffuse layer, creates a macroscopic zone with a
permanent non-zero volume charge density onto which a tangential electric field can exert a Coulombic force.
Hence, the \gls{edl} will not only mediate electrostatic interactions between the nanopore and the analyte
molecules, it will also result in a drag force on the liquid close to the nanopore walls that gives rise to a
net directional flux of water known as the \gls{eof}. Additionally, in a nanopore with a radius of
\SI{1}{\nm}, the \gls{edl} will begin to overlap with itself even at moderate (\eg~\SI{0.1}{\Molar}) ionic
strengths (\cref{fig:nanopores_edl_confined}). This leads to phenomena such as ion concentration
depletion/enrichment~\cite{Plecis-2005}, surface conductance~\cite{Stein-2004},
permselectivty~\cite{Plecis-2005} or pre-concentration~\cite{Pu-2004}---all of which dramatically impact the
transport properties of these pores.

We conclude this section with a brief discussion of the \glsfirst{pbe}, the most widely used approach to
describe the potential and ion distribution in the \gls{edl}. The \gls{pbe} relates the electrostatic
potential---expressed using Poisson's equation---to the electrochemical thermodynamic equilibrium of an ionic
solution---expressed through Boltzmann statistics---and reads~\cite{Gouy-1910,Chapman-1913,Baker-2005}
%
\begin{equation}\label{eq:nanopores_pbe}
  \nabla^2 \potential(\vec{r}) = - \dfrac{\scd(\vec{r})}{\absperm \relperm}
  = - \dfrac{1}{\absperm \relperm}
  \sum_i \left[ \avogadro \ci^0 \chargen_i \ec
         \exp \left( \dfrac{-\chargen_i \ec \potential(\vec{r}) }{\kbt} \right)
        \right]
  \text{ ,}
\end{equation}
%
where $\potential(\vec{r})$ is the electrostatic potential at position $\vec{r}$ and $\scd(\vec{r})$ is the
charge density. Linearizing the exponential term in \cref{eq:nanopores_pbe} with a truncated Taylor expansion
(valid for $\left| \chargeni \ec \potential \right| (\kbt)^{-1} < 1$) yields the linearized \gls{pbe}
%
\begin{equation}\label{eq:nanopores_pbe_linear}
  \nabla^2 \potential(\vec{r})
  = \dfrac{\ec^2}{\absperm \relperm \kbt} \sum_i \left[ \avogadro \ci^0 \chargeni \right]
      \potential(\vec{r})
  = \invdbl^2 \potential(\vec{r})
  \text{ ,}
\end{equation}
%
where $\invdbl = \dbl^{-1}$ provides a clear connection to the Debye length described above. Because it is
derived using mean-field theory, the \gls{pbe} represents ions as densities of point charges in a
structureless medium and ignores any ion-ion correlations~\cite{Bocquet-2010}. Particular care must be taken
for systems with very high surface potentials (more than a few \si{\kTe}), or for phenomena where specific
ion-ion or water-water interactions must be considered~\cite{Collins-2012}. Nevertheless, \gls{pb} theory
captures many of the essential physical properties of electrical double layers, and while meaningful
analytical solutions to this partial differential equation require considerable simplifications and
linearization, with numerical methods it can be solved with relative ease for complex (bio)molecular
systems~\cite{Baker-2001,Baker-2005}.


\subsection{On the timescale of ion and analyte dynamics}
%
\label{sec:np:dynamics}
%

The measured nanopore signal results from the complex interplay between a few very large molecules (\ie~the
pore and the analyte) and a great many small particles (\ie~water and ions). Hence, before delving further
into the physics of nanopores, it is instructive to reflect on the timescales over which the behavior of ions
and analyte molecules within the pore can be safely averaged. Because the diffusivity of a molecule is roughly
proportional to its size, ions will move \numrange{10}{100}-fold faster relative to the typical analyte
molecule (\ie~\gls{dna} or proteins). This suggests that ions will be able to respond quasi-instantly to a
change of the pore--analyte configuration, such as a different position, orientation, or applied bias voltage.
More concretely, the characteristic timescale at which the \gls{edl} can adapt to change is given by
$\chargedensrelaxtime$, the surface charge density relaxation time~\cite{Bazant-2004}
%
\begin{equation}\label{eq:nanopores_relaxation_time}
  \chargedensrelaxtime \sim \dfrac{\dbl L}{\diffusion} 
                            - \dfrac{\dbl^2}{\diffusion}
                            - \dfrac{\dbl \stl}{\diffusion}
  \text{ ,}
\end{equation}
%
with $\diffusion$ the average ion diffusion coefficient, $L$ the characteristic system size, and $\stl$ the
Stern layer thickness. The value of $\chargedensrelaxtime$ for a nanopore lies in the order of
\SIrange{0.1}{10}{\ns}. For example, in a system with $L = \mSI{10}{\nm}$ (the length of the typical
\gls{bnp}), $\diffusion = \mSI{2}{\square\nm\per\ns}$ (\eg~\ce{K+} and \ce{Cl-} ions), $\dbl = \mSI{1}{\nm}$
(\ie~physiological ionic strength), and $\stl = \mSI{0.1}{\nm}$ (the average ionic radius),
$\chargedensrelaxtime = \mSI{4.5}{\ns}$. In comparison, the dwell time of a large analyte molecule at any
given position within a biological nanopore ranges from \SI{1}{\us}~to~\SI{1}{\second}. For example, at
\SI{+100}{\mV}, \gls{dsdna} translocates through \gls{clya} at a rate of \SI{\approx1}{\bp} every
\SI{7000}{\ns} (\SI{290}{\bp} in \SI{2}{\ms})~\cite{Franceschini-2013}. Hence, given the three
order-of-magnitude separation in timescales of these two phenomena, it is reasonable to assume that the ions
respond instantaneously to any change in within the nanopore--analyte system. In other words, the ionic
atmosphere can always be considered to be at equilibrium with respect to the pore and the analyte. As we shall
see in the results chapters, this assumption will allow us to calculate meaningful energies and forces with
`steady-state' solutions, rather than necessitating the use of complex time-dependent analysis.


\subsection{The electric field: the potentials they are a-changin'}
%
\label{sec:np:potential}
%

Due to their high ionic resistivity, the majority of the potential difference ($\potdiff$) applied between the
\cisi{} and \transi{} electrolyte compartments of a nanopore system will change within, and around, the pore.
The overall profile of the electric potential---and by extension the electric field---can be approximated with
the help of an equivalent circuit model~\cite{Wanunu-2009,Grosberg-2010,Kowalczyk-2011}, where the pore is
represented by a resistor ($\Rpore$) and is surrounded by an access resistance ($\Raccess$) on either side
(\cref{fig:nanopores_efield_circuit,fig:nanopores_efield_cartoon}). $\Raccess$ represents the transition from
the bulk electrolyte to the confinement of the pore. Note that even though such a model does not consider the
electrostatics due to the pore's fixed charge distribution, it has proven to be instrumental in the
development of analytically tractable models for describing the capture dynamics of \gls{dna} in both
solid-state~\cite{Wanunu-2009,Grosberg-2010,Muthukumar-2010} and biological
nanopores~\cite{Chinappi-2015,Nomidis-2018}, or for estimating the conductance of nanopores with complex
shapes~\cite{Wanunu-2009,Kowalczyk-2011}.

%
\begin{figure*}[b]
  \centering

  %
  \begin{subfigure}[t]{15mm}
    \centering
    \caption{}\vspace{-2.5mm}\label{fig:nanopores_efield_circuit}
    \includegraphics[scale=1]{nanopores_efield_circuit}
  \end{subfigure}
  %
  \hspace{-3mm}
  %
  \begin{subfigure}[t]{25mm}
    \centering
    \caption{}\vspace{-2.5mm}\label{fig:nanopores_efield_cartoon}
    \includegraphics[scale=1]{nanopores_efield_cartoon}
  \end{subfigure}
  %
  \hspace{-2.5mm}
  %
  \begin{subfigure}[t]{35mm}
    \centering
    \caption{}\vspace{-2.5mm}\label{fig:nanopores_efield_potential}
    \includegraphics[scale=1]{nanopores_efield_potential}
  \end{subfigure}
  %
  \hspace{-0.5mm}
  %
  \begin{subfigure}[t]{35mm}
    \centering
    \caption{}\vspace{-2.5mm}\label{fig:nanopores_efield_efield}
    \includegraphics[scale=1]{nanopores_efield_efield}
  \end{subfigure}
  %

\caption[Equiv. circuit model for the elec. potential and field in a nanopore]{%
  \textbf{Equivalent circuit model for the electric potential and field in a nanopore.}
  %
  (\subref{fig:nanopores_efield_circuit})
  %
  The potential distribution within a nanopore system can be approximated by pore resistance ($\Rpore$),
  flanked by two access resistances ($\Raccess = \Raccess^{\textit{cis}} =
  \Raccess^{\textit{trans}}$)~\cite{Kowalczyk-2011}. Because of the access resistance, the potential
  difference between the \cisi{} and \transi{} electrodes is not exactly, $\potdiff$, but is given more
  accurately by $\potcis - \pottrans$.
  %
  (\subref{fig:nanopores_efield_cartoon})
  %
  Cartoon model showing the equipotential planes outside and within the pore.
  %
  (\subref{fig:nanopores_efield_potential})
  %
  Potential and 
  %
  (\subref{fig:nanopores_efield_efield})
  %
  electric field profiles
  %
  along the central axis of the pore for a perfectly cylindrical pore (thin red lines,
  \cref{eq:nanopores_potential_profile,eq:nanopores_efield_profile}), or modified to approximately reflect the
  influence of the geometry profile of the cartoon pore (thick green and blue lines). Due to the access
  resistance, a portion of $\potdiff$ decays outside of the pore near its entries, which helps with the
  capture of analytes. However, the bulk of the potential change occurs within the pore itself, which results
  in a drastic increase of the electric field strength. Note that most biological nanopores are not perfect
  cylinders and hence will have non-uniform electric fields along their length.
  %
  Figure adapted with permission from Ref.~\cite{Chinappi-2015}. Copyrighted by the American Physical Society.
  %
  }\label{fig:nanopores_efield}
\end{figure*}
%

Assuming that the pore is a perfect cylinder with length $\length$ and diameter $\diameter$ ($\Rpore = 4
\length / (\pi \sigma \diameter^2)$~\cite{Grosberg-2010}), and the pore entries act as planar disk electrodes
with diameter $\diameter$ ($\Raccess = 1 / (2\sigma\diameter)$~\cite{Hall-1975}), the ionic current $\current$
flowing through the pore is given by Ohm's law~\cite{Kowalczyk-2011}
%
\begin{equation}\label{eq:nanopores_ohms_law}
  \current = \dfrac{\potdiff}{\Rpore + 2 \Raccess} 
           = \potdiff \sigma \left( \dfrac{4 \length}{\pi \diameter^2} + \dfrac{1}{\diameter} \right)^{-1}
           \equiv \potdiff \sigma 2 \pi \dchar
  \text{ ,}
\end{equation}
%
with $\potdiff$ the potential difference between the \cisi{} and \transi{} electrodes and $\sigma$ the
conductivity of the electrolyte. Note that while constant in the bulk solution, $\sigma$ can be, and usually
is, location-dependent within the confined environment of a nanopore~\cite{Chinappi-2015}. As defined in
Nomidis~\etal{}~\cite{Nomidis-2018}, $\dchar$ represents the characteristic length of the system
%
\begin{equation}\label{eq:nanopores_characteristic_length}
  \dchar = \dfrac{1}{2\pi} \left[\dfrac{4 \length}{\pi\diameter^2} + \dfrac{1}{\diameter} \right]^{-1}
  \text{ .}
\end{equation}
%
Because $\current$ is constant, it can also be computed using the differential form of Ohm's
law~\cite{Chinappi-2015}, allowing us to formulate the spatial derivative of the potential
%
\begin{equation}\label{eq:nanopores_current_cross_section}
  \dfrac{d \potential(z)}{dz} = \dfrac{\current}{\sigma S(z)}
  \text{ ,}
\end{equation}
%
where $z$ is either the radial distance from the pore entry or the location along the length of the pore,
$S(z)$ the cross-sectional area through which the current is measured, and $\potential(z)$ the electrostatic
potential. In most nanopores, the diameter of a pore can vary significantly along its length, but for a
perfectly cylindrical pore $S(z) = \pi \diameter^2 / 4$, which does not depend on $z$. Assuming spherical
symmetry, outside of the pore $S(z) = \pi z^2$. Applying Kirchhoff's circuit laws to
\cref{eq:nanopores_ohms_law,eq:nanopores_current_cross_section} yields the potential drops due the access
resistance of the \cisi{} and \transi{} entries of the pore, given by
%
\begin{align}\label{eq:nanopores_potcis}
  \potcis   ={}& \potdiff \dfrac{\dchar}{\diameter}
  \text{ ,}
\end{align}
%
and
%
\begin{align}\label{eq:nanopores_pottrans}
  \pottrans ={}& \potdiff \dfrac{\dchar}{\diameter} \left[ \pi + \dfrac{8 \length}{\diameter} \right]
  \text{ ,}
\end{align}
%
respectively. Finally, an axial profile of the potential ($\potential (z)$,
\cref{fig:nanopores_efield_potential}) can be obtained by (somewhat arbitrarily) positioning the disk
electrode at their respective \cisi{} ($z = -\length / 2$) and \transi{} pore entries ($z = \length /
2$)~\cite{Chinappi-2015}, and integrating \cref{eq:nanopores_current_cross_section} with the boundary
conditions $\potential(-\infty) = 0$ and $\potential(+\infty) = \potdiff$,
%
\begin{equation}\label{eq:nanopores_potential_profile}
  \dfrac{\potential (z)}{\potdiff} =
  \begin{dcases}
    - \dfrac{\dchar}{2z + \length - \diameter / \pi}
    & \quad z < -\dfrac{\length}{2} \text{ ,} \\
    \dfrac{z}{\length} \left[ \dfrac{\pi \diameter}{4 \length} + 1 \right]^{-1} + \dfrac{1}{2}
    & \quad -\dfrac{\length}{2} \le z \le \dfrac{\length}{2} \text{ ,}\\
    1 - \dfrac{\dchar}{2z - \length + \diameter / \pi}
    & \quad z > \dfrac{\length}{2} \text{ .}
  \end{dcases}
\end{equation}
%
Because the potential in the reservoir is inversely proportional to the distance from the pore entry, it
decays rapidly (\ie~over a few \si{\nm}) to its bulk reservoir value. Within the pore, $\potential (z)$ is a
simple linear function of $z$, at least for a perfectly cylindrical nanopore. However, most actual nanopores
are far from perfect cylinders, and their diameters can easily vary by a factor of two or more within a few
nanometers along the channel's length, leading to deviations from linearity. Additionally, the fixed charge
distributions lining the nanopore walls can also significantly influence the overall electrostatic profile,
which will be the main subject matter of \cref{ch:electrostatics}. The electric field $\efield (z)$
(\cref{fig:nanopores_efield_efield}) is then simply given by the gradient of $\potential (z)$,
%
\begin{equation}\label{eq:nanopores_efield_profile}
  \efield (z) = \nabla \potential (z) =
  \begin{dcases}
    \potdiff \dfrac{2 \dchar}{ \left( 2z - \length + \diameter / \pi \right)^2}
    & \quad z < -\dfrac{\length}{2} \text{ ,} \\
    \dfrac{\potdiff}{\length} \left[ \dfrac{\pi \diameter}{4 \length} + 1 \right]^{-1}
    = \left| \efield \right|_{\text{pore}}
    & \quad -\dfrac{\length}{2} \le z \le \dfrac{\length}{2} \text{ ,}\\
    \potdiff \dfrac{2 \dchar}{ \left( 2z + \length - \diameter / \pi \right)^2 }
    & \quad z > \dfrac{\length}{2} \text{ .} \\
  \end{dcases}
\end{equation}
%
Even though in this system, $\efield (z)$ is constant within the pore ($\left| \efield
\right|_{\text{pore}}$), even relatively small changes in diameter along the nanopore's length will result in
large differences in the local electric field strength. This is because $\efield (z)$ depends on the local
cross-sectional area (\ie~quadratic with respect to diameter, see~\cref{eq:nanopores_current_cross_section}),
with narrower sections resulting in higher field strengths.


\subsection{The electro-osmotic flow: ions pushing water}
%
\label{sec:np:eof}
%

As touched upon in~\cref{sec:np:edl}, when a tangential electric field, $\efield$, is applied across the
\gls{edl}, the non-zero charge density, $\scdion$, within the fluid will be subject to a Coulombic force
density (\cref{fig:nanopores_eof_principle})
%
\begin{equation}\label{eq:nanopores_edl_force}
  \forcedensedl = \scdion \efield
  \text{ ,}
\end{equation}
%
where
%
\begin{equation}\label{eq:nanopores_ion_scd}
  \scdion = \faraday \dsum_i^N \chargeni \ci
  \text{ ,}
\end{equation}
%
with $\faraday = \ec \avogadro$ the Faraday constant, and $\ci$ the \emph{local} concentration of ion $i$.
Even though, at an atomic scale, $\forcedensedl$ is exerted on the ions rather than the fluid as a whole, a
portion of the force---which typically assumed to be all of it---is transferred to the water molecules through
friction. In turn, the unidirectional movement of the \gls{edl} will drag along the rest of the fluid, charged
or not, giving rise to a unidirectional fluid flow---the \glsfirst{eof}~\cite{Bocquet-2010}.


%
\begin{figure*}[b]
  \centering

  %
  \begin{subfigure}[t]{35mm}
    \centering
    \caption{}\vspace{0mm}\label{fig:nanopores_eof_principle}
    \includegraphics[scale=1]{nanopores_eof_principle}
  \end{subfigure}
  %
  \hspace{10mm}
  %
  \begin{subfigure}[t]{25mm}
    \centering
    \caption{}\vspace{-2.5mm}\label{fig:nanopores_eof_cartoon}
    \includegraphics[scale=1]{nanopores_eof_cartoon}
  \end{subfigure}
  %
  \hspace{-0.5mm}
  %
  \begin{subfigure}[t]{40mm}
    \centering
    \caption{}\vspace{-2.5mm}\label{fig:nanopores_eof_zprofile}
    \includegraphics[scale=1]{nanopores_eof_zprofile}
  \end{subfigure}
  %

\caption[Electro-osmotic flow in a nanopore]{%
  \textbf{Electro-osmotic flow in a nanopore.}
  %
  (\subref{fig:nanopores_eof_principle})
  %
  When an external electric field ($\efield_z$) is applied along a charged nanochannel, a Coulombic force
  density ($\forcedensedl$) is exerted on the excess charge density ($\scdion$) present in the \gls{edl} near
  the wall. The resulting unidirectional movement of these excess counterions drags on the surrounding water
  molecules, giving rise to an \glsfirst{eof} with a net water velocity in the z-direction ($\velocity_z(y)$).
  Note that while in hydrophilic nanopores, $\velocity_z(y)\approx0$ at the wall, a non-zero velocity will be
  observed in hydrophobic nanopores~\cite{Bocquet-2010,Manghi-2018}.
  %
  (\subref{fig:nanopores_eof_cartoon})
  %
  Cartoon model showing the flow velocity profiles outside, and within, a nanopore.
  %
  (\subref{fig:nanopores_eof_zprofile})
  %
  %
  Approximate profile of the average \gls{eof} velocity, $\velocity (z)$, along the central axis of the pore
  for a perfectly cylindrical pore (\cref{eq:nanopores_eof_velocity}). Similar to the electric field, the
  decay of $\velocity (z)$ is inversely proportional to the square of the distance from the pore.
  %
  }\label{fig:nanopores_eof}
\end{figure*}
%

As with the electrical field, tractable analytical formulas that accurately describe the \gls{eof} generated
by (biological) nanopores are not possible due to their complex geometries and charge distributions.
Nevertheless, considerable insights can be gained by simplifying the pore geometry and its surface charge
density. For low surface charge potentials (\ie~$\left| \ec \potential\right| / (\kbt) \le 1$), the
volumetric flowrate ($\flowrate$) of water molecules through an infinitely long, cylindrical nanopore with
radius $\radiusa$ is given by~\cite{Laohakunakorn-2015}
%
\begin{equation}\label{eq:nanopores_eof_rate}
  \flowrate = - \dfrac{\pi \radiusa^2}{\viscosity \invdbl} 
  \dfrac{ \bessel{2} ( \invdbl \radiusa ) }{ \bessel{1} (\invdbl \radiusa) }
  \electricfield_z \surfcd
  \equiv - \dfrac{\pi \radiusa^2}{\viscosity \invdbl} H(\invdbl \radiusa) \electricfield_z \surfcd
  \text{ ,}
\end{equation}
%
where $\viscosity$ the fluid viscosity, $\invdbl = (\dbl)^{-1}$ the inverse Debye length
(see~\cref{eq:nanopores_debye_length}), $\electricfield_z$ the electric field along the pore axis, $\surfcd$
the surface charge density of the pore, and $\bessel{n}$ the $n$th-order modified Bessel function of the first
kind. Note that the minus sign indicates that the flow direction will be against the electrical field for
positive surface charge densities (\ie~a negative \gls{edl}) and with it for negative surface charge densities
(\ie~a positive \gls{edl}). $H(\invdbl \radiusa)$ can be considered as an ionic strength `correction factor'
that accounts for the overlapping of the \gls{edl} within the pore. Hence, for conditions where $\invdbl
\radiusa \gg 1$ (\eg~at high ionic strengths or for pores with a large radius), $H(\invdbl \radiusa) \approx
1$ and \cref{eq:nanopores_eof_rate} simplifies to
%
\begin{equation}\label{eq:nanopores_eof_rate_simple}
  \flowrate = - \dfrac{\pi \radiusa^2}{\viscosity \invdbl} \electricfield_z \surfcd
  \text{ .}
\end{equation}
%
As with the ionic current (\cref{eq:nanopores_current_cross_section}), $\flowrate$ is a linear function of the
electric field magnitude, but it is inversely proportional to the square root of the ionic strength
($\flowrate \propto \kappa^{-1} \propto \ionstr^{-1/2}$). This means that $\flowrate$ rapidly approaches
infinity at very low the ion concentrations  ($\flowrate \to \infty$ for $\ionstr \to 0$), a non-physical
result that arises from the infinite cylinder assumption~\cite{Mao-2014}. For realistic pores of finite
length, the maximum flowrate is restricted by `end effects' (akin to the access resistance for ion flow), and
analytical models that address this are available~\cite{Sherwood-2014}.

For an incompressible fluid such as water, the conservation of mass dictates that the number of water
molecules passing through any given cross-section must be equal. Thus, at location $z$ along the central axis
of the pore, the average fluid velocity $\velocity(z)$ within a cross-section $S(z)$ for a perfectly
cylindrical pore with radius $\radiusa(z)$ (\cref{fig:nanopores_eof_zprofile}) can be estimated using
%
\begin{equation}\label{eq:nanopores_eof_velocity}
  \velocity (z) = \dfrac{\flowrate}{S(z)} =
  \begin{dcases}
    \dfrac{\flowrate}{2 \pi \left(z - b \right)^2}
    & \quad z < -\dfrac{\length}{2} \text{ ,} \\
    \dfrac{\flowrate}{\pi \radiusa(z)^2}
    & \quad -\dfrac{\length}{2} \le z \le \dfrac{\length}{2} \text{ ,}\\
    \dfrac{\flowrate}{2 \pi \left(z + b \right)^2}
    & \quad z > \dfrac{\length}{2} \text{ ,}
  \end{dcases}
\end{equation}
%
where the offset $b = 2^{-1/2} \radiusa + \length / 2$ was included to match the velocities at the pore
entries. Similar to the electric field, $\velocity$ scales with the cross-sectional area and axial changes in
pore diameter quickly result in large changes in flow velocity.

It is worth mentioning that the water flow will also depend on the degree of interaction between the water
molecules and the atoms in the nanopore walls, a phenomenon that is often expressed in fluid dynamics by the
`slip length' $\sliplength$~\cite{Bocquet-2010}. Walls with a high degree of friction---such as the rough,
hydrophilic surfaces of proteins~\cite{Zhang-2014,Wong-Ekkabut-2016b,Pronk-2014}---interact strongly with the
water molecules, and yield a fluid velocity close to zero at the liquid-solid interface ($\sliplength = 0$,
the `no-slip' boundary condition). In contrast, the water flow through frictionless pores---such as the
smooth, hydrophobic surfaces within carbon nanotubes~\cite{Ye-2011,Manghi-2018,Bocquet-2020}---is not slowed
down to zero at the liquid-solid interface ($\sliplength \to \infty$, the `slip' boundary condition). In most
nanopores, the appropriate value for $\sliplength$ will lie somewhere in between these extremes, although for
the hydrophilic channels of \glspl{bnp}, $\sliplength$ can be expected to be closer to $0$ than to
$\infty$~\cite{Bocquet-2010,Manghi-2018}.


\subsection{External forces: (di)electrophoresis and electro-osmosis}
%

%
\begin{figure*}[b]
  \centering

  %
  \begin{subfigure}[t]{60mm}
    \centering
    \caption{}\vspace{-2.5mm}\label{fig:nanopores_forces_ext_ep_eo}
    \includegraphics[scale=1]{nanopores_forces_ext_ep_eo}
  \end{subfigure}
  %
  \hspace{-5mm}
  %
  \begin{subfigure}[t]{60mm}
    \centering
    \caption{}\vspace{-2.5mm}\label{fig:nanopores_forces_ext_dep}
    \includegraphics[scale=1]{nanopores_forces_ext_dep}
  \end{subfigure}
  %

\caption[Forces acting on a particle due to the external electric field]{%
  \textbf{Forces acting on a particle due to the external electric field.}
  %
  (\subref{fig:nanopores_forces_ext_ep_eo})
  %
  Because the electric field is strongest within the pore, both the electrophoretic and electro-osmotic forces
  are at their strongest during particle translocation. Note that, depending on the charge of the particle and
  the direction of the \gls{eof} (\ie~the charge of the pore walls), these forces can either reinforce, or
  oppose one another.
  %
  (\subref{fig:nanopores_forces_ext_dep})
  %
  At locations where the electric field is not uniform (\ie~$\nabla \efield \neq 0$), such as near the pore
  entries or near a constriction within the pore, dielectric particles with a permanent or induced dipole
  moment will be subjected to a dielectrophoretic force, which pushes them towards increasing electric field
  magnitudes.
  %
  }\label{fig:nanopores_forces_ext}
\end{figure*}
%

The externally applied bias voltage creates a strong electric field around and within the pore. This field
induces several distinct forces that influence the movement of analyte molecules towards and within the pore
(\cref{fig:nanopores_forces_ext}): the \glsdisp{ep}{electrophoretic~(EP)}, \glsdisp{eo}{electro-osmotic~(EO)}
and \glsdisp{dep}{dielectrophoretic~(DEP)} forces. The former two are the strongest, and hence also the most
well-known. For proteins and \gls{dna} molecules, typical magnitudes of these forces within the pore fall in
the range of \SIrange{1}{100}{\pN}~\cite{Keyser-2006,vanDorp-2009}. Besides attracting analytes and inducing
their translocation, external forces may also impact the structure of the molecules themselves. Electric
field-induced forces within solid-state nanopores have been shown to stretch~\cite{Freedman-2013} or even
unfold~\cite{Freedman-2011} proteins, allowing one to probe their mechanical properties during
translocation~\cite{Waduge-2017,Zhou-2020b}. Whereas the \gls{ep} force is dominant for molecules with high
charge densities (such as \gls{dna})~\cite{vanDorp-2009}, the force balance for other biomolecules
(\eg~proteins) is often more evenly matched, or even in favor of the \gls{eo}
force~\cite{Soskine-2013,Zhang-2020}. The \gls{dep} force is typically much weaker, given that it acts on the
(induced) dipole moment of a particle and is active only places with a strong electric field gradient
(\ie~around the pore entries or near a constriction within the pore). Hence, while dielectrophoresis can often
be ignored without repercussions, in specific cases it plays an important role in the capture and
translocation processes of analyte molecules~\cite{Freedman-2016,Asandei-2016,Chinappi-2020}, and hence its
existence should be acknowledged.


\subsubsection{Electrophoretic and dielectrophoretic forces: sailing the field lines.}
%

For a charged particle in an electric field, the \gls{ep} force, $\forceep$ is given by the electric component
of the Lorentz force (\ie~Coulomb's law)~\cite{Lu-2012}
%
\begin{equation}\label{eq:nanopores_lorentz_force}
  \forceep = \oint_{V} \scd \efield dV = \chargeq \efield
  \text{ ,}
\end{equation}
%
with $\scd$ the charge density, $V$ the volume, and $\chargeq$ the net charge of the particle. In principle,
the integral form of \cref{eq:nanopores_lorentz_force} allows one the evaluate $\forceep$ for any dielectric
particle within arbitrary electric field distributions. The integrated form, on the other hand, is technically
only valid for a charged particle in a homogeneous electric field (\ie~far away from the pore). But, as we
shall see in \cref{ch:trapping}, it can be a reasonable approximation even for more complex systems
(\eg~within the pore). The \gls{dep} force, $\forcedep$, is proportional to the spatial change of the electric
field and the dipole moment of the particle~\cite{Hoelzel-2020}
%
\begin{equation}\label{eq:nanopores_dep_force}
  \forcedep = \left( \dipolemoment \cdot \nabla \right) \efield
  \text{ ,}
\end{equation}
%
with $\dipolemoment$ the dipole moment of the particle. Note that the $\dipolemoment$ contains both a
permanent contribution, due to the non-uniform charge distribution within the
particle~\cite{Hoelzel-2020,VanMeervelt-2017}, and an induced component, resulting from the particle's
effective polarizability~\cite{Minerick-2015}. The latter depends on complex permittivity of both the particle
and the surrounding electrolyte (\ie~the Clausius--Mossotti relation), and it is influenced strongly by the
frequency of the electric field and not just the strength of its gradient. A more in-depth discussion on state
of theory and experiment of \gls{dep} for small biomolecules such as proteins can be found in
Ref.~\cite{Hoelzel-2020}.

Both \gls{ep} and \gls{dep} contributions can be accounted for in a single framework if the force is
calculated \textit{via} integration of the electrostatic Maxwell stress tensor, $\maxwellstresstensorvec$,
over the surface ($\Gamma$) of the particle~\cite{Ai-2011}
%
\begin{equation}\label{eq:nanopores_electric_force}
  \forceele = \oint_{\Gamma} \left( \maxwellstresstensorvec \cdot \normvec \right) d\Gamma
  \text{ ,}
\end{equation}
%
where
%
\begin{equation}\label{eq:nanopores_maxwell_stresstensor}
  \maxwellstresstensorvec = \permittivity \efield \efield
                            - \dfrac{1}{2} \permittivity \left( \efield \cdot \efield \right) \identity
  \text{ ,}
\end{equation}
%
with $\permittivity$ the permittivity, $\efield$ the electric field vector, $\normvec$ the normal vector of
the surface and $\identity$ the identity tensor vector. Note that the full expression of
$\maxwellstresstensorvec$ should include several magnetic field terms, but since magnetism does not (yet?)
contribute significantly to the forces landscape within nanopores, these are set to zero. Because
\cref{eq:nanopores_electric_force} makes no assumptions regarding the shape of the electric field or particle,
its integral nature of allows it to be evaluated (numerically) for arbitrary particle shapes, charge densities
and electric field distributions, if those are available (\eg~through simulations).


\subsubsection{Electro-osmotic force: such a drag.}
%

In first order, the \gls{eo} force acting on particle can be calculated using the well-known Stokes' law
%
\begin{equation}\label{eq:nanopores_drag_force}
  \forcedrag = - 6 \pi \viscosity \particleradius \velocity
  \text{ ,}
\end{equation}
%
where $\forcedrag$ is the drag force on the particle with radius $\particleradius$, $\viscosity$ the
electrolyte viscosity, and $\velocity$ the velocity vector of the fluid. Note that
\cref{eq:nanopores_drag_force} assumes that (1) the flow is laminar (which typically the case for nanopore
systems\footnotemark), %
%
\footnotetext{%
%
In a laminar flow, the fluid moves in smooth `layers' next to each other, with little to no intermixing
besides diffusion. It can be recognized by its low Reynolds number
%
\begin{equation}
\reynolds = \dfrac{\density \vel \diameter}{\viscosity} \text{ ,}\notag
\end{equation}
%
with $\diameter$ the nanopore diameter, and $\density$, $\vel$, and $\viscosity$ the fluid density, velocity,
and viscosity, respectively. For example, for a nanopore with $\diameter = \mSI{5}{\nm}$, and a moderate
flow velocity of $\vel = \mSI{0.1}{\meter\per\second}$, $\reynolds\approx\num{5e-4} \ll 1$~\cite{Schoch-2008}.
%
}%
%
(2) the particle is spherical with a perfectly smooth surface, (3) the surrounding environment (\ie~density
and viscosity) is homogeneous, and (4) the particle does not perturb the flow itself. Evidently, given that
most biomolecules are not smooth spheres, this is often compensated for by using a so-called Stokes radius (or
hydrodynamic radius), rather than true particle radius~\cite{Ortega-2011}. This makes $\forcedrag$ useful as a
good first order approximation for computing the electro-osmotic force, particularly outside of the pore.
Within the pore, however, its validity is less clear given that the size of the particle is often similar to
the pore diameter. Hence, its presence is expected to influence the flow itself. Moreover, as we shall discuss
in \cref{ch:epnpns}, the electrolyte properties within the pore are not uniform, particularly in proximity to
the nanopore~\cite{Qiao-Aluru-2003,Vo-2016,Hsu-2017,Ye-2011} and particle~\cite{Pronk-2014,Makarov-1998}
surfaces, leading to further deviations.

As with the (di)electrophoretic force, a more general framework involves the integration of a stress
tensor---in this case the hydrodynamic stress tensor, $\hydrostresstensorvec$---across the entire particle
surface ($\Gamma$)~\cite{Ghosal-2019}
%
\begin{equation}\label{eq:nanopores_hydrodynamic_force}
  \forcehyd = \oint_{\Gamma} \left( \hydrostresstensorvec \cdot \normvec \right) d\Gamma
  \text{ ,}
\end{equation}
%
where
%
\begin{align}\label{eq:nanopores_hydrodynamic_stresstensor}
  \hydrostresstensorvec =
  \pressure\identity - \viscosity\left[\nabla\velocity + \left(\nabla\velocity \right)^\mathsf{T}\right]
  \text{ ,}
\end{align}
%
with $\pressure$ the pressure, $\viscosity$ the viscosity, $\velocity$ the velocity vector, and $\normvec$ the
surface normal vector. The hydrodynamic force, $\forcehyd$, does not only include the viscous force (\ie~the
interaction between the particle and the moving fluid molecules), it also takes into account any pressure
differences that may arise around the particle prior to, or during
translocation~\cite{Hoogerheide-2014,Wilson-2018}. Hence, if a full, self-consistent solution of the flow and
pressure fields around a particle is available, the numerical integration of
\cref{eq:nanopores_hydrodynamic_force} will yield the most accurate results~\cite{Galla-2014}.



\subsection{Intrinsic forces: electrostatics, steric hindrance, and entropy}
%

Next to the forces originating from the externally applied electric field, there are also several forces that
stem from the `intrinsic' interactions between the analyte molecules and the pore itself
(\cref{fig:nanopores_forces_int}). The nature of these forces can either be enthalpic (\ie~due to
intermolecular forces such as ion-ion interactions, hydrogen bonding, \gls{vdw} forces, etc.), or entropic
(\ie~due to a change in the microscopic degrees of freedom during the translocation process). To facilitate
the discussion, we will roughly subdivide the enthalpic forces into an electrostatic
(\cref{fig:nanopores_forces_int_es}) and a steric (\cref{fig:nanopores_forces_int_steric_entropic}, yellow
glow) component, with the latter being a `clumped' term that encompasses all intermolecular forces besides
electrostatics. For the entropic forces (\cref{fig:nanopores_forces_int_steric_entropic}, orange glow), we
will limit ourselves to a qualitative description of its impact on unfolded polymer translocation.


%
\begin{figure*}[b]
  \centering

  %
  \begin{subfigure}[t]{60mm}
    \centering
    \caption{}\vspace{-2.5mm}\label{fig:nanopores_forces_int_es}
    \includegraphics[scale=1]{nanopores_forces_int_es}
  \end{subfigure}
  %
  \hspace{-5mm}
  %
  \begin{subfigure}[t]{60mm}
    \centering
    \caption{}\vspace{-2.5mm}\label{fig:nanopores_forces_int_steric_entropic}
    \includegraphics[scale=1]{nanopores_forces_int_steric_entropic}
  \end{subfigure}
  %

\caption[Forces acting on a particle due to the intrinsic interactions]{%
  \textbf{Forces acting on a particle due to the intrinsic interactions.}
  %
  (\subref{fig:nanopores_forces_int_es})
  %
  The intrinsic electrostatic interactions between charged atoms on the translocating particle and the
  nanopore walls can yield repulsive (like-like charges) or attractive (like-unlike charges) forces. Unlike
  their externally applied counterpart, the electric field generated by these forces are highly local, and
  strongly mediated by the ionic strength.
  %
  (\subref{fig:nanopores_forces_int_steric_entropic})
  %
  If the diameter of the particle closely matches that of the pore, the confinement leads to steric clashes
  (\ie~repulsive intramolecular interactions) between the particle and the nanopore walls. In the case of
  proteins, this may prevent them from moving through the pore in their native state, resulting in unfolded
  translocation, or force them to assume a specific conformation or orientation---both of which induce an
  entropic penalty.  
  %
  }\label{fig:nanopores_forces_int}
\end{figure*}
%

\subsubsection{Electrostatic forces: ionically mediated love and hate.}
%

Electrostatic forces arise when two charged atoms interact with one another, either in a repulsive (for
like-like charges) or attractive (for like-unlike charges) manner (\cref{fig:nanopores_forces_int_es}). For a
static charge distribution in a pure dielectric environment, Coulomb's law provides an excellent estimation
for these forces. However, in an electrolyte, the situation is significantly more complex due to the presence
of mobile charges (\ie~ions) that form an \gls{edl}, minimizing the overall energy of the system and strongly
mediating the distance at which fixed charges can still influence one another. By providing a proper
description of the \gls{edl}, the \gls{pb} theory discussed in~\cref{sec:np:edl} also provides us with a
method to calculate the total electrostatic energy, $\energyelec$, which for a solvated molecular system with
a volume $\Omega$, amounts to~\cite{Gilson-1993,Im-1998,Baker-2005}
%
\begin{align}\label{eq:nanopores_electrostatic_energy}
  \energyelec = 
  \dfrac{1}{4 \pi} \int_{\Omega}
  \bigg[
    \scd^{\rm{f}} \potential
    - \permittivity \dfrac{\left(\nabla \potential \right)^2}{2}
    - \dsum_{i}^{N} \ci^0 \pd{ e^{- \beta V_i} }{\rpos}
                    \left( e^{ - \beta \chargen_i \ec \potential } - 1 \right)
  \bigg] \, d\rpos
  \text{ ,}
\end{align}
%
with $\beta = (\kbt)^{-1}$, $\scd^{\rm{f}}(\rpos)$ the fixed charge density (\ie~from the pore or the
analyte), $\permittivity(\rpos)$ the spatial distribution of the permittivity and $\potential(\rpos)$ the
electrostatic potential. For every ion $i$ in the electrolyte, $\ci^0$ is the bulk concentration, $\chargen_i$
is the charge number, and $V_i(\rpos)$ is a repulsive potential that excludes it from non-solvent accessible
areas (\eg~within the biomolecules). Because force is the (negative) change in energy over distance, the
electrostatic force on atom $m$, $\force_{\es,m}$, can be calculated by differentiating $\energyelec$ with
respect to the atomic coordinates, $\rpos_m$, yielding~\cite{Gilson-1993,Im-1998,Baker-2005}
%
\begin{align}\label{eq:nanopores_electrostatic_force}
  \force_{\es,m} ={}& - \pd{\energyelec}{\rpos_m} \notag \\
    ={}& - \dfrac{1}{4 \pi} \int_{\Omega} \bigg[ 
      \underbrace{ \pd{\scd^{\rm{f}}}{\rpos_m} \potential }_{ \text{RF} }
      - \underbrace{ \dfrac{\left(\nabla \potential \right)^2}{2} 
      \pd{\permittivity}{\rpos_m} }_{ \text{DB} }
      - \underbrace{ \dsum_{i}^{N}
          \ci^0 \pd{ e^{- \beta V_i} }{\rpos_m}
          \left( e^{ - \beta \chargen_i \ec \potential } - 1 \right) }_{ \text{IB} }
          \bigg] \, d\rpos
  \text{ .}
\end{align}
%
As indicated, the components of the integral in \cref{eq:nanopores_electrostatic_force} can be split into
three distinct contributions. The `RF' or `reaction field' term expresses the response of the potential to the
spatial variations of the fixed charge density. It is mathematically equivalent to Coulomb's law
(\cref{eq:nanopores_lorentz_force})~\cite{Im-1998}, and hence yields the bare (\ie~free-space) force between
charges. The `DB' or `dielectric boundary' term arises from spatial variations of the permittivity between the
solute ($\relperm \approx \text{\numrange{2}{20}}$~\cite{Li-2013}) and the solvent ($\relperm \approx 80$) in
a charged, non-uniform environment. In essence, these differences of polarizability induce a net surface
charge at the interface, onto which the electric field will exert a non-negligible force. The DB force is
particularly important when moving charges into a confined environment, such as ion
channels~\cite{Nadler-2003}. The `IB' or `ionic boundary' term is proportional to the ionic concentration and
accounts for the contribution of the \gls{edl} to the force. Note that the ionic boundary term will always
diminish the contributions of the RF force, whether those are positive or negative. This can intuitively be
understood as the screening effect of the \gls{edl}: higher ionic strengths lead to more screening and hence
lower energetic contributions between fixed charges.

Given that a typical nanopore measurement is performed at physiologically relevant ionic strengths
(\eg~\SIrange{50}{500}{\mM}), and that both the pore and the analyte often carry a significant number of
charges, all the electrostatic force components described above are likely to be non-negligible. Moreover, its
manipulation can be used to dramatically alter the behavior of translocating molecules. Examples include
enhancing the translocation of \gls{dna}~\cite{Maglia-2008}, improving the capture of
peptides~\cite{Asandei-2015b,Asandei-2016}, or mediating the recognition between two protein
isoforms~\cite{Fahie-2015b}. The electrostatic force acting on analyte molecules translocating through
\glspl{bnp}---in the form of the electrostatic energy landscape---will be discussed in detail in
\cref{ch:electrostatics} (\gls{dna} through \gls{frac} and \gls{clya}) and \cref{ch:trapping} (proteins
through \gls{clya}).


\subsubsection{Steric forces: come close, but not too close.}
%

Next to the long-range electrostatic interactions, the close confinement of an analyte molecule within the
interior of the pore is bound to lead to steric clashes between the atoms of the analyte and those of the
pore~\cite{Buchsbaum-2013}. For two (neutral) atoms at distances below \SI{\approx1}{\nm}, the \glsfirst{vdw}
forces start to dominate the interatomic potential. These originate from transient fluctuations in the
electron densities of the atoms, leading to Keesom (permanent--permanent dipole interaction), Debye
(permanent--induced dipole interaction), and London dispersion (fluctuating dipole--induced dipole
interaction) forces. The total \gls{vdw} force is typically first attractive, up until the electron clouds of
the atoms begin the overlap, and the Pauli exclusion principle induces a strong repulsive force. The \gls{lj}
potential, $\potLJ$, is perhaps the most famous approach that captures the essential properties of the
\gls{vdw} interaction~\cite{Paquet-2015}. Spatially differentiating $\potLJ$ yields the \gls{lj} force 
%
\begin{equation}\label{eq:nanopores_LJ_force}
  \forcelj = - \nabla \cdot \potLJ( r ) = - \welldepthLJ \nabla \cdot \bigg[ 
    \underbrace{ \left( \dfrac{ \dminLJ }{ r } \right)^{12} }_{\text{repulsive}}
    -
    \underbrace{ 2 \left( \dfrac{ \dminLJ }{ r } \right)^{6} }_{\text{attractive}}
  \bigg]
  \text{ ,}
\end{equation}
% 
with $r$ the distance between the two atoms, $\welldepthLJ$ the depth of potential well, located at distance
$\dminLJ$. The repulsive (`12') term ensure that $\potLJ$ becomes highly repulsive at very short distances.
Even though the mathematical form has no physical meaning, it adequately reproduces the repulsive force caused
by the overlapping of electron orbitals (\ie~Pauli repulsion). The minus sign in front of the second (`6')
term indicates that is attractive, and its mathematical form can be related back to the underlying physics.
Because the attractive term acts at longer distances, the resulting energy landscape contains an energy
minimum, $\welldepthLJ$, when the atoms are at $r = \dminLJ$. In other words, $\dminLJ$ represents the tipping
point after which the interatomic force becomes repulsive rather than attractive. Looking at the {CHARMM36}
force field parameters~\cite{Huang-2016}, typical \gls{lj} values for $\dminLJ$ and $\welldepthLJ$ within
proteins are \SI{4.1}{\angstrom} and \SI{-0.11}{\kbt} (carbon), \SI{3.6}{\angstrom} and \SI{-0.31}{\kbt}
(nitrogen), \SI{3.5}{\angstrom} and \SI{-0.22}{\kbt} (oxygen) and \SI{2.3}{\angstrom} and \SI{-0.06}{\kbt}
(hydrogen). Do note that the \gls{lj} potential is merely an approximation (\eg~ it is not directional) and
exact solutions require expensive quantum mechanical calculations~\cite{Paquet-2015}.

Given their weak nature and limited range, the \gls{vdw} forces acting on individual atoms are easily overcome
by thermal energy. However, the simultaneous close contact of hundreds of atoms can easily induce a very large
repulsive force (\ie~a high steric energy barrier) that will influence the translocation dynamics of analyte
molecules. This could easily occur when a protein is squeezed through a narrow constriction within a nanopore
(\cref{fig:nanopores_forces_int_steric_entropic}, yellow glow). In~\cref{ch:trapping}, such a steric barrier
will be crucial for explaining the escape dynamics of a protein trapped in \gls{clya}.


\subsubsection{Entropic forces: threading the needle.}
%

Up until now, all the discussed forces are enthalpic in nature---they originate from the need of the analyte
molecules to minimize their internal free energy. However, the confinement of proteins and nucleic acids
within a nanoscale cage~\cite{Liu-2015b}, or their (forced) unfolding during nanopore
translocations~\cite{Muthukumar-2010,Cressiot-2012,Cressiot-2015}, induce an additional energy barrier that is
of entropic nature. This means that certain transitions can only occur from a limited subset of analyte/pore
conformations. We will limit the discussion to the threading of a polymer through a nanopore, though the
enthalpic and entropic effects of folded protein confinement within a biological nanopore will be discussed in
more detail in \cref{sec:np:confinement}.

The entropy, $\entropy$, of a thermodynamic system with number of available microstates, $\microstates$, is
given by Boltzmann's equation~\cite{Neumann-1980}
%
\begin{equation}\label{eq:nanopores_entropy}
  \entropy = \boltzmann \ln \microstates \text{ .}
\end{equation}
%
The energy barrier resulting from the entropy difference between a nanopore-analyte configuration where all
microstates are able to translocate (``0''), and one where only a limited subset can traverse the pore (``1'')
is given by
%
\begin{equation}\label{eq:nanopore_entropic_energy}
  - \temperature \Delta \entropy = - \temperature \left( \entropy_1 - \entropy_0\right)
  = - \boltzmann \temperature \ln \dfrac{\microstates_1}{\microstates_0}
  \text{ ,}
\end{equation}
%
with $\microstates_n$ the number of available microstates in configuration $n$. Several intuitive insights can
be gained from \cref{eq:nanopore_entropic_energy}: (1) if $\microstates_1 = \microstates_0$ (\eg~all
conformational configuration can translocate), $- \temperature \Delta \entropy = 0$, and the free energy
landscape is controlled solely by enthalpic forces, and (2) if $\microstates_1 < \microstates_0$ (\eg~a
polymer translocating polymer that can only enter the pore with one of its ends), $- \temperature \Delta
\entropy > 0$, indicating that a reduction in the number of suitable microstates indeed results in an energy
barrier, and \textit{vice versa} for $\microstates_1 > \microstates_0$ (\eg~when exiting the confines of the
pore). Concretely, microstate ratios of $\microstates_1 / \microstates_0 =
\text{\numlist{0.5;0.1;0.01;0.001}}$ would induce energy barriers of \SIlist{0.69;2.3;4.6;6.9}{\kbt},
respectively.

Note that, as before, we can define an entropic force as the gradient of the entropic
energy~\cite{Neumann-1980}
%
\begin{equation}\label{eq:nanopore_entropic_force}
  \forceentr = - \nabla \cdot (- \temperature \Delta \entropy )  
             = \boltzmann \temperature \nabla \ln \dfrac{\microstates_1}{\microstates_0}
  \text{ .}
\end{equation}
%
The magnitude of $\forceentr$ can be difficult to estimate experimentally, as it is often challenging to
differentiate it from the steric forces~\cite{Buchsbaum-2013}. Nevertheless, quantitative insights can be
obtained by either analytically estimating (the ratio of) the number of microstates at the relevant points in
space~\cite{Tian-2003,Muthukumar-2010,Cressiot-2015}, or by measuring the temperature dependence of the total
force exerted on the analyte experimentally~\cite{Meller-2002,Payet-2015} or through
simulations~\cite{Tian-2003,Matysiak-2006,Vaitheeswaran-2014,Luo-2017}.



\subsection{Analyte capture}
%

%
\begin{figure*}[b]
  \centering

  %
  \begin{subfigure}[t]{115mm}
    \centering
    \caption{}\vspace{-2.5mm}\label{fig:nanopores_capture_overview}
    \includegraphics[scale=1]{nanopores_capture_overview}
  \end{subfigure}
  %
  \\
  %
  \begin{subfigure}[t]{115mm}
    \centering
    \caption{}\vspace{-2.5mm}\label{fig:nanopores_capture_energy}
    \includegraphics[scale=1]{nanopores_capture_energy}
  \end{subfigure}
  %

\caption[Polymer capture process by a nanopore]{%
  \textbf{Polymer capture process by a nanopore.}
  %
  (\subref{fig:nanopores_capture_overview})
  %
  Schematic overview of the different steps of the capture process, accompanied by
  %
  (\subref{fig:nanopores_capture_energy})
  %
  a sketch of the electrophoretic, entropic, and total energy profiles.
  %
  (1) Far away from the nanopore, the weak electric field causes charged polymer analyte molecules---such as
  \gls{dna} or proteins---to slowly diffuse or drift towards the pore. (2) Once inside the capture radius, the
  electric field becomes strong enough to readily overcome the random diffusive motion, and it is pinned at an
  energy minimum close to the entry of the pore. (3) The limited set of conformations with which the molecule
  can translocate through the pore (\eg~\gls{dna} must enter with one of its ends and then fully unfold),
  incurs an entropic energy penalty prior to, or during, the translocation process. (4) After fully
  translocating, the molecule is again free to drift away from the pore and diffuse in the reservoir.
  %
  Adapted with permission from~\cite{Nomidis-2018}.
  %
  }\label{fig:nanopores_capture}
\end{figure*}
%

The capture process of a polymer molecule by a nanopore is outlined in~\cref{fig:nanopores_capture}. It is
important to note that while the magnitude of the electric field inside the exterior reservoirs is negligible
compared to the one within nanopore, is not equal to zero. This means that charged molecules in the reservoir
are subject to both diffusion and electrophoretic drift
(\cref{fig:nanopores_capture_overview})~\cite{Muthukumar-2010}. Given that the electric field is inversely
proportional to the square of the distance from the pore
(see~\cref{eq:nanopores_efield_profile})~\cite{Grosberg-2010}, it becomes particularly relevant in close
proximity to the entries of the pore, where the electric field can become significantly stronger than the
thermal energy. Similarly, the \gls{eof} is non-negligible outside of the nanopore~\cite{Wong-2007} and
contributes significantly to the attraction or repulsion of \gls{dna}~\cite{Firnkes-2010} and
proteins~\cite{Soskine-2012,Soskine-2013} towards the pore. Finally, if a polymer requires a specific
configuration in order to translocate (\ie~unfold)---as is typically the case for the analysis of
\gls{dna}~\cite{Muthukumar-2010}, but not folded proteins~\cite{Yusko-2011,Soskine-2012,Plesa-2013}---it must
also overcome a significant entropic energy barrier before it can commence the translocation process
(\cref{fig:nanopores_capture_energy})~\cite{Muthukumar-2010}.

However, because the focus of this dissertation lies with the transport of molecules through (biological)
nanopores after they have been captured, the interested reader is referred to the seminal papers on polymer
capture by nanopores by Wong and Muthukumar (electro-osmotic flow)~\cite{Wong-2007}, Grosberg and Rabin
(electrophoresis)~\cite{Grosberg-2010}, and Muthukumar (entropic energy barrier)~\cite{Muthukumar-2010}. Of
note as well is the work of Nomidis~\etal~\cite{Nomidis-2018}, whose theoretical and experimental analysis
found that the capture rate of both \gls{ssdna} and \gls{dsdna} by \gls{clya} was limited by the entropic
barrier at the entry of the pore, rather than their arrival due to drift and diffusion.


\subsection{Confining proteins within biological nanopores}
%
\label{sec:np:confinement}
%

Given the magnitude of the forces involved, and the degree of nanoscale confinement, it is worthwhile to
reflect on the structural impact the translocation process may have on the analyte molecule itself. This is
particularly relevant for proteins (\eg~enzymes), where strong pulling forces can cause deformation or
unfolding, and strong confinement can alter their function. In the following section, we will briefly discuss
these items in the context of a protein confined within a biological nanopore~\cite{Galenkamp-2020}.


\subsubsection{The role of electrophoresis and electro-osmosis.}
%

The role of the applied potential across the pore---and the resulting electrophoretic and electro-osmotic
forces---on the folding of proteins inside a (biological) nanopore is unclear. To estimate their magnitude,
using the \gls{clya} pore as an example (\cref{fig:nanopores_clya}), we can estimate the electric field within
the \lumen{} of \gls{clya} by describing it as two cylindrical electrolyte chambers in series. The strength of
the electrical field along the central nanopore axis inside the wide \cisi{} chamber of dodecameric \gls{clya}
can then be estimated to be approximately \SI{3}{\mV\per\nm} for a bias voltage of
\SI{-50}{\mV}.\footnotemark%
%
\footnotetext{
%
As calculated with Ohm's law for \SI{50}{\mV} bias voltage from the resistance of the \cisi{} chamber
($R_{\text{\cisi}} \approx \mSI{2.6e8}{\ohm}$) in series with the \transi{} chamber ($R_{\text{\transi}}
\approx \mSI{2.2e8}{\ohm}$), resulting in the corresponding voltage drops of $\Delta V_{\text{\cisi}} \approx
\mSI{27}{\mV}$ (over a distance $l_{\text{\cisi}} \approx \mSI{10}{\nm}$), and $\Delta V_{\text{\transi}}
\approx \mSI{23}{\mV}$ (over a distance $l_{\text{\transi}} \approx \mSI{3}{\nm}$),
respectively~\cite{Soskine-2013}.
%
}
%
Thus, if an immobilized protein carries a significant net charge (\eg~\SI{10}{\ec}) it would be subject to a
Coulomb force of approximately \SI{4}{\pN} (\cref{eq:nanopores_lorentz_force}). Forces of this magnitude have
been shown before to partially unfold or deform proteins by \gls{afm}~\cite{Best-2001}. However, the Pelta
group showed that within the \SIrange{50}{250}{\mV} voltage range, the fractional residual current (defined as
the blocked pore current divided by the open pore current) of maltose binding protein translocating through a
solid-state nanopore remains constant, as expected for a protein that remains folded~\cite{Talaga-2009}. Using
\gls{clya} and sampling several proteins (human thrombin, {AlkB} and \gls{dhfr}), no changes in fractional
residual current up to \SI{-200}{\mV} were observed, confirming that the protein normally should not unfold
under moderate applied potentials~\cite{Soskine-2013}. However, the net force experienced by a protein
immobilized inside a nanopore will also depend on the magnitude and direction of the electro-osmotic flow. The
negative charges lining the interior walls of \gls{clya} make the pore highly cation-selective and induce a
strong electro-osmotic flow~\cite{Franceschini-2016}, enabling the capture of proteins even against the
electrical field~\cite{Soskine-2012} (see~\cref{ch:trapping,ch:transport}). While the precise magnitude of the
water velocity inside \gls{clya} is currently unknown, experimental data using \gls{ahl}
nanopores~\cite{Paula-1999} and simulations performed with solid-state~\cite{vanDorp-2009,Luan-2008} and
\gls{ahl}~\cite{Aksimentiev-2005,Pederson-2015} nanopores, revealed velocities in the order of
\SI{100}{\mmps}. Assuming a similar water velocity inside our \gls{bnp} and applying Stokes' law
(\cref{eq:nanopores_drag_force}), a static and uncharged sphere of \SI{5}{\nm} in diameter would endure a
force of approximately \SI{4}{\pN}. Thus, under negatively applied potentials (\transi{}) the electrophoretic
and electro-osmotic forces acting on negatively charged proteins would oppose and cancel each other out,
resulting in less stress on the protein and a more diffusion-dominated environment~\cite{Firnkes-2010}. For
positively charged molecules, these two forces will add and increase the net force, resulting in additional
stress on the captured protein.

\subsubsection{The role of confined water.}
%

Another important consideration to make when confining proteins in nanopores, for example when measuring
enzymatic reactions, is the role of the nanopore walls and the relatively strong electrostatic forces inside
the nanopore. As described in \cref{sec:np:clya}, the \gls{clya} protein contains a plethora of negatively
charged residues lining its interior walls (\cref{fig:nanopores_clya_pore_section}). Such a negatively charged
nanocavity is reminiscent of the well-studied \textit{E. coli} {GroEL/GroES} chaperone~\cite{Xu-1997}.
Although the precise mechanism by which {GroEL} aids protein refolding remains
controversial~\cite{England-2008,England-2008b,Motojima-2012,Weber-2013}, it is likely that the confinement
prevents misfolded proteins from aggregating. Another explanation could be that the rich electrostatic
environment inside the {GroEL} cavity increases the water density and hence enhances folding due to an
enhanced hydrophobic effect compared with bulk~\cite{England-2008,England-2008b}. The diffusivity and the
structuring of the water molecules confined inside {GroEL} in the absence of substrate protein are likely to
be similar to that in bulk solution~\cite{Franck-2014}. When there is a protein confined inside the chaperonin
cavity, however, these properties might change depending on the interaction strength between the protein and
the internal walls of the chaperonin~\cite{Weber-2013}. Interestingly, the folding pathways inside the
chaperonin chamber have been proposed to be similar~\cite{Horst-2007} or different~\cite{Jewett-2004} compared
with bulk~\cite{Apetri-2008}. Regardless of the precise folding mechanism and the behavior of the confined
solvent, a charged, hydrophilic cage appears to promote the native state of proteins. Therefore, no
detrimental effects on the conformation of proteins trapped inside \gls{clya} are to be expected.


%
%
% \clearpage
%
%


\section{Approaches for computational modeling of nanopores}
%
\label{sec:np:modeling}
%

Barring the use of excessive simplifications, our ability to describe the behavior of any arrangement of atoms
and molecules with tractable analytical approaches quickly disappears with increasing system complexity. It is
for this reason that scientists resort to simulations, where the equations that describe underlying the
physical phenomena are solved numerically, to a certain accuracy and typically with a computer, rather than
analytically, for an appropriate, non-simplified model of the molecular system. In the final section of this
chapter, we aim to familiarize the reader with some of the most commonly used computational approaches for
modeling complex molecular problems.

% ---starting from their basic underlying principles, highlighting their
% strengths and weaknesses, and concluding with some examples from within the nanopore field.

%
\begin{figure*}[b]
  \centering
  
  %
  \includegraphics[scale=1]{nanopores_simulations_types}
  %

\caption[Hierarchy of simulation methodologies]{%
  \textbf{Hierarchy of simulation methodologies.}
  %
  Quantum mechanical methods (left) have the highest precision, as they model the nuclei and electron
  distributions of atoms explicitly. However, they also have the highest computational cost and are often only
  employed to parameterize the interaction potentials and partial charges used for the atomistic modeling
  methods (middle, \eg~molecular dynamics). Further averaging of the interaction potential over multiple atoms
  gives rise to continuum or `mean-field' methods (right), where the discreteness of the atoms is replaced by
  a structureless medium with material properties.
  %
  }\label{fig:nanopores_simulations_types}
\end{figure*}
%



\subsection{Atomistic, `discrete' modeling}
%

\subsubsection{The interaction potential.}
%

One of the most commonly used simulation approaches to study the behavior of individual atoms in complex
(macro)molecular systems, is to model the atoms (or a combination of atoms) as discrete, hard spheres, whose
motions are governed by an \emph{intermolecular interaction potential}, $\energymd$
(\cref{fig:nanopores_simulations_types}, middle). Roughly speaking, computational approaches that make use of
such an interaction potential can be subdivided into three classes: \gls{mc} simulations, \glsfirst{md},
and stochastic dynamics~\cite{Paquet-2015}. Whereas the \gls{mc} methods employ random sampling to map out the
properties for a given conformational state of a molecular assembly, \gls{md} can be used to track the
time-evolution of this system from one state to the next. In stochastic dynamics, the realism of the
(intrinsically approximate) interaction potential is improved \textit{via} the explicit inclusion a friction
term (to account for solvent viscosity) and an additional random energy term (to account the occasional
perturbation by a high velocity collision). In the context of a molecular system, these inclusions lead to the
well-known Langevin dynamics, which, in the overdamped regime (\ie~no inertia) it is known as \glsfirst{bd}.
Because we will only discuss \gls{md} in detail, the interested reader is referred to the comprehensive
comparison compiled by Paquet and Viktor in Ref.~\cite{Paquet-2015}.

\subsubsection{Molecular dynamics: atoms pushing atoms.}
%

In \gls{md}, Newton's equations of motion are solved numerically for every atom as a function of time,
yielding a so-called `trajectory' that contains the time-dependent positions and velocities for every atom
(\ie~a set of microstates). Concretely, the \gls{md} algorithm computes the position, $\rpos(\timedim)$,
velocity, $\velatom(\timedim)$, and acceleration, $\accelatom(\timedim)$, of each atom for a small
timestep\footnotemark, $\timestep$, for a given set of starting positions and velocities at $\timedim = 0$.%
%
\footnotetext{
%
To avoid discretization errors, $\timestep$ must necessarily be smaller than the fastest vibrational frequency
in the system (typically hydrogen bond vibration), leading to the use of time steps in the order of
\SI{1}{\fs} (\SI{e-15}{\second}). This imposes a practical limit of \SIrange{0.1}{1}{\us} on the length
trajectories of large molecular assemblies. For atomistic systems with \num{\approx e6}~atoms (\eg~a large
biological nanopore), the upper limit  for long-timescale trajectories (\ie~computational times within the
duration of the typical doctoral program), using current algorithms and supercomputers, is \SI{\approx
1}{\ms}~\cite{Vendruscolo-2011,Phillips-2020}.
%
}
%
The net force on every atom, $\force$, is computed from the gradient of the potential energy functional
(\ie~the interaction potential), which consists of a summation of `bonded' (\ie~between atoms sharing a
covalent bond: bond length, angle, and dihedral) and `non-bonded' (\ie~between all atoms: \gls{vdw} and
electrostatics) interactions. Hence,
%
\begin{equation}\label{eq:nanopores_md_force}
  \force (\rpos) = - \nabla \energymd (\rpos)
  \text{ ,}
\end{equation}
%
with
%
\begin{equation}\label{eq:nanopores_md_energy}
  \energymd (\rpos) = \dsum \energymd_{\text{b}} (\rpos) + \dsum \energymd_{\text{nb}} (\rpos)
  \text{ ,}
\end{equation}
%
where $\energymd_{\text{b}}$ and $\energymd_{\text{nb}}$ are the bonded and the non-bonded energy terms,
respectively. Because the calculation $\energymd_{\text{nb}}$ involves the summations over all atoms of the
system, it is particularly computationally taxing ($O(N^2)$ scaling). However, the use of approximations
(\eg~distance cut-offs) or clever algorithmic implementations (\eg~neighbor lists) can significantly reduce
the computational cost ($O(N\log N)$ or even $O(N)$ scaling~\cite{Eastman-2010}).

The parametrization of the functions that describe the (non)-bonded interactions are typically based on
\textit{ab initio} quantum-mechanical simulations (\cref{fig:nanopores_simulations_types}, left), combined
with, and verified by, experimental data when possible. The compilation of parameters for a given set of
(bio)molecules is called a molecular mechanics `force-field', of which many variants exists---each one
optimized for a specific set of applications or conditions. In the case of macromolecular simulations
involving proteins, nucleic acids and ions, the {AMBER}~\cite{Ponder-2003}, {CHARMM}~\cite{Huang-2016} and
{GROMOS}~\cite{Oostenbrink-2004} force fields are the most widely used.

Depending on whether one chooses to keep constant the total energy or the temperature of the system, the
time-average of the \gls{md} trajectory will correspond to respectively the canonical (\gls{nvt}: number,
volume, and temperature), or microcanonical (\gls{nve}: number, volume, and energy) ensemble. The
velocity-Verlet algorithm is one of the most well-known approaches for \gls{nve} \gls{md} simulations, as it
conserves the total energy of the system with a high degree of accuracy~\cite{Swope-1982}. In contrast,
\gls{nvt} simulations mandate the addition or removal of energy from the system to keep the temperature fixed.
This can be achieved by adjusting the velocities of atoms with an additional `thermostat' algorithm such as
Langevin dynamics~\cite{Bussi-2008}, or the velocity-rescaling~\cite{Heyes-1983},
Nos\'{e}-Hoover~\cite{Nose-1984,Hoover-1985}, or Andersen~\cite{Andersen-1980} thermostats.

Even though \gls{md} simulations are among the most realistic simulation methodologies for atomistic
systems, the high (computational) cost associated with both running the simulations and analyzing the
resulting trajectories still limits its scalability~\cite{Vendruscolo-2011,Phillips-2020}. Additionally, great
care must be taken when setting up the simulation domain (\ie~boundary conditions), the choice of force-field,
and the simulation parameters, as each of these can lead to nonphysical results---particularly in confined
spaces~\cite{Wong-ekkabut-2016a}.


\subsubsection{Molecular dynamics applied to biological nanopores.}
%

Due to its wide-spread experimental use, and the early availability of its crystal structure, the \gls{ahl}
nanopore has been the primary subject of interest for \gls{md} studies on \glspl{bnp} over the past
20~years~\cite{Aksimentiev-2005,DeBiase-2016,Basdevant-2019}, most notably by the research groups of Aleksei
Aksimentiev and Mauro Chinappi. The former was the first to use \gls{md} to provide a complete `image' of
\gls{ahl} in terms of electrostatic potential distribution, ion concentrations, ionic conductance, and
electro-osmotic permeability~\cite{Aksimentiev-2005}. Since then, many more \gls{md} studies have been
performed on \gls{ahl}, including the translocation of nucleic acids~\cite{Wells-2007}, and unfolded
proteins~\cite{DiMarino-2015}, the cation-type dependence of the ionic current
rectification~\cite{Bhattacharya-2011}, the magnitude of the electro-osmotic flow~\cite{Bonome-2017}, and the
recognition mechanisms of nucleobases~\cite{Manara-2015b,DeBiase-2016} and amino acids~\cite{DiMuccio-2019}.

Beyond \gls{ahl}, the transport properties of several other \glspl{bnp} have been scrutinized with \gls{md}
simulations. These include the \gls{mspa}~\cite{Bhattacharya-2012,Manara-2015,Bhattacharya-2016,Zhou-2020},
\gls{ael}~\cite{Cao-2018,Ouldali-2020,Zhou-2020}, \gls{frac}~\cite{Zhao-2019}, and
\gls{clya}~\cite{Mandal-2016,Wilson-2019,Li-2020} pores, and recent work by Zhou~\etal{} provides a
comprehensive overview of the nanofluidic properties of the \gls{mspa}, \gls{ahl}, \gls{csgg}, and \gls{ael}
nanopores~\cite{Zhou-2020}.

% MD for nanopores overview
%  \cite{Aksimentiev-2005} Complete picture of aHL (ion conc, EOF, potential, conformational fluct) 
%  \cite{Lynden-Bell-1996} mobility and solvation of ions in nanochannels
%  \cite{Allen-1999} Effect of hydrophillic and hydrophobic walls on flow in nanochannel
%  \cite{Luan-2008} Translocation of DNA through ssnp, EOF screening
%  \cite{Bhattacharya-2011} Ion-type dependence of rectification in aHL
%  \cite{Zhang-2014} EOF simulation in rough nanochannels
%  \cite{DiMarino-2015} unfolded protein translocation through aHL
%  \cite{Belkin-2016} Plasmonic heating of nanopore
%  \cite{Wong-Ekkabut2016} aHL. Water transport through nanochannel. Spatial dependence density. Water PMF.
%  \cite{Mandal-2016} ClyA. Dendrimer transport through ClyA. Ion concentration distributions.
%  \cite{Bonone-2017} aHL. Water transport through nanochannel. Ion concentration profiles. Water conductance.
%  \cite{Basdevant-2019} CG MD. aHL. Ionic transport through alpha-hemolysin.
%  \cite{Zhao-2019} electro-mechanical conductance of a gate-modified FraC pore
%  \cite{Zhou-2020} Overview of MspA, aHL, CsgG and aerolysin, inlcuding potential profiles.
%  \cite{DeBiase-2016} aHL. MD. BD. Link between residual current and the accessible volume.


% BD for nanopores overview
%  \cite{Schirmer-1999}
%  \cite{Im-2002}
%  \cite{Noskov-2004}
%  \cite{Millar-2008}
%  \cite{Egwolf-2010}
%  \cite{Comer-2012}
%  \cite{DeBiase-2015}
%  \cite{Pederson-2015}


\subsection{Continuum, `mean-field' modeling}
%

\subsubsection{The mean-field approximation: averaging the interaction potential.}
%

The high computational cost associated with the vast amount of degrees of freedom in atomic systems enforces
harsh limits on both the attainable system size, and the maximal duration of a simulation. Even though this
computational burden can be reduced by `coarse-graining' (\ie~combining multiple real atoms into a single
pseudoatom with pseudo interaction potentials), the fundamental scaling problem remains unaltered.
Fortunately, the stochasticity of individual atoms tends to rapidly average out in groups of atoms in a
predictable manner, allowing their behavior to be condensed into a single material property that replaces the
unique interaction potentials of a group of atoms with a single effective interaction potential, described by
a structureless medium or continuum (\cref{fig:nanopores_simulations_types}, right). From an energetic
point-of-view, this is equivalent to finding the best approximation for the total energy of a system of
interacting particles that does not include any explicit interaction terms, also known as the \emph{mean-field
approximation}. Importantly, the mean-field approximation reduces the many-body problem presented by explicit
molecular simulations to a much more scalable one-body problem. Well-known examples of material properties
include the relative permittivity, the viscosity of a liquid, or the diffusion coefficient of an ion. Even
though mean-field theories excel at upward scaling (\ie~macroscopic system sizes and timescales), they must be
used with caution when scaling downwards (\ie~microscopic system sizes and timescales), as the averaged
property will no longer adequately represent the true physical behavior. For the case of electrostatics, the
review by Collins provides an exhaustive set of reasons why continuum methods are unrealistic at the
nanoscale, with the lack of explicit ion-water interactions being the main culprit~\cite{Collins-2012}.
Nevertheless, the mean-field approximation has been used extensively, and successfully, for the (qualitative)
simulation of ion channels~\cite{Im-2002,Furini-2006,Liu-2015},
\glspl{bnp}~\cite{Simakov-2010,Pederson-2015,Aguilella-Arzo-2017,Simakov-2018} and
\glspl{ssnp}~\cite{Cervera-2005,White-2008,Chaudhry-2014,Laohakunakorn-2015}, providing meaningful physical
insights at a fraction of the computational cost of a \gls{md} simulation.


\subsubsection{Molecular electrostatics and transport using continuum methods.}
%

In \cref{sec:np:edl}, we have already introduced a continuum representation for modeling the electrostatics in
electrolytes: the \glsfirst{pbe} (see~\cref{eq:nanopores_pbe}). Whereas \gls{pb} theory provides a good
approximation for electrostatics at equilibrium, it contains no terms that consider the dynamics of a system
(\eg~diffusion coefficients), making it incapable of modeling nonequilibrium processes, such as the flux of
ions, water, or analyte molecules through a nanopore under an applied bias voltage. A set of equations that
can tackle this problem are the \gls{npe}, a mass conservation equation that extends Fick's law of diffusion
with electrostatic and convective forces, in combination with \gls{pe} to model the electrostatics, and the
\gls{nse} to compute the fluid velocity and pressure distribution. When properly coupled, the \gls{pnp-nse}
provide a self-consistent framework with which one can model the transport of ions and water through
nanopores. In \cref{ch:epnpns} we will introduce a set of corrections for \glsdisp{pnp-ns}{PNP-NS} theory to
improve their validity at the nanoscale and enable accurate, quantitative predictions for the nanofluidic
properties of biological nanopores.

Let us start with the \gls{npe}, which can be used to describe the flux of a charged chemical species in a
fluid medium. It is given by~\cite{Lu-2011}
%
\begin{equation}\label{eq:nanopores_nernst_planck}
  \underbrace{ \pd{\concentration}{\timedim} }_{\text{variation}}
  = - \nabla \cdot \flux = \nabla \cdot (
    \underbrace{ \diffusion \nabla \concentration }_{\text{diffusion}}
    +
    \underbrace { \chargen \mobility \concentration \nabla \potential }_{\text{migration}}
    -
    \underbrace{\velocity \concentration}_{\text{convection}}
    )
  \text{ ,}
\end{equation}
%
with $\flux$ the flux, $\concentration$ the concentration, $\diffusion$ the diffusion coefficient, $\chargen$
the charge number, and $\mobility$ the electrophoretic mobility of the charged molecule, and $\potential$ the
electric potential and $\velocity$ the fluid velocity. As indicated, the \gls{npe} equations consists of
diffusive, (electro)migratory, and convective flux terms. The diffusive flux makes molecules move along their
concentration gradient (from high to low concentration), and it is proportional to the diffusivity of each
individual molecule. Similarly, the electromigratory flux drives the movement of charged molecules with
(positive charge) or against (negative charge) the electric potential gradient (\ie~the electric field),
proportional to their mobility. Note that the Einstein relation,
$\mobility=\frac{\ec}{\kbt}\chargen\diffusion$, which dictates that the diffusion coefficient is
directly proportional to the electric mobility, only holds at infinite dilution (\ie~$\concentration \to 0$),
and hence should not be used at finite concentrations (even though is often is). Finally, the convective flux
represents the molecules that are dragged along by the movement of the fluid itself, irrespective of their
charge or diffusivity. Note that each chemical species present in the system is described by its own
\gls{npe}, each fully independent from the other. This lack of (steric) interactions between the species can
lead to nonphysical results (\ie~very high ion concentrations in the \gls{edl}), indicating that the \gls{npe}
is only applicable to very dilute solutions. However, as we shall see in~\cref{ch:epnpns,ch:transport}, the
inclusion of coupled, size-dependent terms can mitigate these problems~\cite{Lu-2011}.

As with the \gls{pbe}, the spatial distribution of $\potential$ can be obtained from \gls{pe}
%
\begin{equation}\label{eq:nanopores_poisson}
  \underbrace { \nabla \cdot \left(\absperm \relperm \nabla \potential \right) }_{\text{field divergence}}
  = \underbrace{ - \scd }_{\text{charge source}}
  \text{ ,}
\end{equation}
%
where $\scd$ is the charge distribution within the system. In the case of a charged analyte molecule within a
nanopore, $\scd = \scdfix + \scdion$, with the first term representing the fixed ionized groups on the pore
and the analyte, and $\scdion$ the ionic charge imbalance within the \gls{edl}
(see~\cref{eq:nanopores_ion_scd}). In essence, \cref{eq:nanopores_poisson} states that, within a system with a
given permittivity distribution, the electric field must adapt itself, spatially, in such a way (\ie~the field
divergence) as to perfectly cancel out the imposed charge distribution(s). Note that the inclusion of
$\scdion$ in \cref{eq:nanopores_poisson} couples it back to the \gls{npe}.

In continuum mechanics, the velocity of a viscous fluid (\eg~water) can be described using the famous
\gls{nse}, which comprise a mathematical representation for the conservation of mass and momentum. For an
incompressible flow (\ie~water flow through a nanopore), the momentum balance of the \gls{nse} reads
%
\begin{equation}\label{eq:nanopores_navier_stokes}
  \underbrace{ \density \pd{\velocity}{\timedim} }_{\text{variation}}
  +
  \underbrace{ \density \left( \velocity \cdot \nabla \right) \velocity }_{\text{convection}}
  +
  \underbrace{ \viscosity \nabla \cdot \left(
    \nabla \velocity + \left(\nabla\velocity \right)^\mathsf{T} \right) }_{\text{diffusion}}
  =
  \underbrace{\nabla \cdot \left(\pressure \identity \right)}_{\text{int. source}}
  +
  \underbrace{ \forcedensvol }_{\text{ext. source}}
  \text{ ,}
\end{equation}
%
and its mass balance is given by the continuity equation
%
\begin{equation}\label{eq:nanopores_velocity_continuity}
  \density \nabla \cdot \left( \velocity \right) = 0
  \text{ ,}
\end{equation}
%
with $\density$ the fluid density, $\velocity$ the fluid velocity, $\pressure$ the pressure, and
$\forcedensvol$ a volume force density acting on the fluid. The first two terms of
\cref{eq:nanopores_navier_stokes} represent the local inertia of the fluid, with the first one (variation)
being the time-dependent change of the velocity (zero at steady state), and the second one (convection)
expressing the inertia due to the intrinsic velocity of the fluid. The third (diffusion) and fourth (internal
source) terms result from the divergence of the hydrodynamic stress tensor
(see~\cref{eq:nanopores_hydrodynamic_stresstensor}). Intuitively, the diffusion term can be interpreted as
follows: if the fluid velocities at two neighboring location differ, the viscous friction between them will
result in a transfer of the momentum from the high to low velocity location, analogous to the diffusion of a
chemical species from high to low concentration. The internal source term stems from the pressure differences
within a volume of fluid, causing it to flow from high to low pressures. Finally, the last term represents all
forces that act upon the fluid from external forces, such as gravity or an electric field
(\ie~$\forcedensedl$, see~\cref{eq:nanopores_edl_force}), providing a way to couple the \gls{nse} back to both
the \gls{npe} and the \gls{pe}.

The radical simplification of replacing all interatomic interactions by an average mean field brings with it a
massive reduction in computational cost, enabling continuum methods to investigate large problems over long
timescales. When numerically solving \glspl{pde}, such as the \gls{pnp-nse}, the total computational domain is
discretized into a set of sub-domains, each holding the local solution for the variables of interest. The
nature of this discretization, and how it is linked algebraically to the \glspl{pde} themselves, depends on
the numerical algorithm, of which the \glsdisp{fdm}{finite difference~(FDM)}, \glsdisp{fem}{finite
element~(FEM)}, and \glsdisp{fvm}{finite volume~(FVM)} methods are the most widely used. Note that only the
\gls{fvm} is truly conservative (\ie~mass cannot (dis)appear out of nowhere), as its formulation intrinsically
ensures that, for any given volume element, the inbound and outbound fluxes are identical. Nevertheless, the
\gls{fdm} and \gls{fem} are very popular, as their mathematical elegance makes them straightforward to
implement for a variety of numerical problems, whereas using fine-grained discretization can mitigate problems
with conservation.


\subsubsection{Continuum modeling of biological nanopores.}
%

Initial work on predicting the permeation of ions and water through biological transmembrane channels was
focused towards ion channels~\cite{Eisenberg-1996,Chen-1997,Corry-2003}. As was the case with \gls{md}, to
date, virtually all the continuum modeling studies of \glspl{bnp} have been performed with \gls{ahl} as the
subject of
interest~\cite{Noskov-2004,Cozmuta-2005,OKeeffe-2007,Simakov-2010,Pederson-2015,Simakov-2018,Aguilella-Arzo-2020},
though its use in the \gls{ssnp} field is significantly more
prevalent~\cite{Daiguji-2004,Cervera-2005,White-2008,Lu-2012,Chaudhry-2014,Laohakunakorn-2015,Hulings-2018,Rigo-2019,Melnikov-2020}.
However, despite the wide-spread use of continuum transport models, their ability to quantitatively predict
ionic currents---in their unmodified form---is typically poor~\cite{Corry-2000,Collins-2012}. To remedy this,
researchers have sought for ways to mitigate the shortcomings of {PNP}(-{NS}) theory by including steric
ion-ion interactions~\cite{Kilic-2007,Lu-2011,Liu-2020}, the non-uniform spatial distribution of the
diffusivities~\cite{Cozmuta-2005,Furini-2006,Simakov-2010,}, and the concentration dependencies of electrolyte
properties~\cite{Baldessari-2008-1,Burger-2012,Chen-2016}. Such corrections can be based on theory,
experiment, or extracted from other simulations (\eg~\gls{md}). Conversely, the field distributions (electric
field and electro-osmotic flow) can also be used as input for \gls{bd} simulations as to stochastically map
the capture dynamics of individual particles~\cite{Pederson-2015,Hulings-2018}. Given that \cref{ch:epnpns} of
this dissertation is devoted to building an improved \gls{pnp-ns} framework for the predictive modeling of
biological nanopores, a more detailed discussion of this topic can be found there. Additionally,
in~\cref{ch:transport} we have applied this improved framework to \gls{clya}, in an extensive study of its
nanofluidic properties.

% Nosko\cite{Noskov-2004} \cite{Simakov-2010} dependency of the $\pKa$ and conductivity on the membrane
% position~\cite{Simakov-2018}, to quantify the access resistance of \glspl{bnp}~\cite{Aguilella-Arzo-2020}
% \cite{Valisko-2019} Simulation study of transport processes in a solid-state nanopore using MD, BD and PNP.

%
% \vspace{-0.5em}
%

\section{In summary}
%

The field of single-molecule sensing has grown tremendously over the past two decades. Nanopores have
established themselves as a powerful, label-free single-molecule sensing technology, providing both
confirmatory and complementary data relative to the traditional techniques of force spectroscopy and
fluorescence microscopy. Whereas initial research was driven by nanopore-based \gls{dna} sequencing, a goal
that was recently successfully attained and commercialized by Oxford Nanopore Technologies, the application
space of nanopore sensing has now expanded considerably: from proteomics~\cite{Yusko-2017,Houghtaling-2019}
and protein sequencing~\cite{Restrepo-Perez-2018}, to metabolomics~\cite{Zernia-2020} and single-molecule
enzymology~\cite{Galenkamp-2020,Willems-VanMeervelt-2017}. This expanding set of applications has also come
with an equally diverse set of nanopores in terms of size, shape, and material, well beyond the classical
\ta-hemolysin and silicon nitride variants. The choice in biological nanopores now includes pores with large
interiors for whole-protein sensing, such as cytolysin A~\cite{Soskine-2012} or pleurotolysin
AB~\cite{Huang-2020}, or small funnel shaped ones for small-molecule sensing, such as fragaceatoxin
C~\cite{Huang-2017,Restrepo-Perez-2019a}. Likewise, their solid-state counterparts have advanced
significantly, from quasi-2D pores in graphene~\cite{Fischbein-2008} or \ce{MoS2}~\cite{Feng-2015b} for
ultimate constriction sensitivity, to the integration of field-effect transistors~\cite{Ren-2020} that provide
a new mode of sensing, to sub-angstrom pores that provide novel insights into confined molecular
transport~\cite{Rigo-2019}. However, as the arsenal of available nanopores types and the complexity of their
applications, has grown, so has the need to better understand their physical working principles. Even though a
plethora of mathematical frameworks and simulation methodologies currently exist~\cite{Maffeo-2012}, most
notably molecular dynamics and Poisson-Nernst-Planck theory, the intricate physics of nanoscale transport
within nanopores is pushing their capabilities to the limit and is testing the boundaries of their
validity~\cite{Collins-2012}. Nevertheless, it is here that modeling approaches, by providing a physical
understanding of existing phenomena and a guiding hand for novel experiments, can help to drive the nanopore
field forward.


%%%%%%%%%%%%%%%%%%%%%%%%%%%%%%%%%%%%%%%%%%%%%%%%%%
% Keep the following \cleardoublepage at the end of this file,
% otherwise \includeonly includes empty pages.
\cleardoublepage

% vim: tw=70 nocindent expandtab foldmethod=marker foldmarker={{{}{,}{}}}
