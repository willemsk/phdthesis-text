\chapter{Biological nanopores}\label{ch:nanopores}
\epigraphhead[80]{
  \epigraph{
    ``I don't see how he can ever finish if he doesn't begin.''
  }{
    \textit{`Alice' (by Lewis Carroll)}
  }
}


\definecolor{shadecolor}{gray}{0.85}
\begin{shaded}
Parts of this chapter were adapted from:\\
\fullcite{Willems-VanMeervelt-2017}
\newpage
\end{shaded}

This chapter serves as a comprehensive introduction to the primary concepts required to understand the
objectives and relevance of the work performed in this thesis. In particular, the reader will be familiarized
with biological nanopores as single-molecular sensors, in particular for protein and peptide analysis.
\\
\\
The text and figures of this introductory chapter were entirely written and created by myself.

\newpage

%
%
\section{Introduction}

\todo{nanopores chapter general intro}

%
%
\section{Molecular sensors}

\todo{molecular sensors}

%
%
\section{Nanopores as single-molecular sensors}

\todo{nanopores as single-molecular sensors}


%
%
\section{Types of nanopores}

\todo{types of nanopores}

Depending on their constituent material, nanopores are often classified into two distinct groups: the protein-based, \emph{biological} nanopores (\glspl{bnp}) and the semiconductor-based, \emph{solid-state} nanopores (\glspl{ssnp}).\cite{Dekker-2007} This division runs deeper than the material however, as BNPs are are formed in a bottom-up manner through self-assembly, while SSNPs are fabricated top-down using micromachining techniques. Perhaps the most crucial difference between the two classes is the extent to which they can be modified and manipulated. The proteiaceuous

In recent years, many different types of nanopores have been developed that do not necessarily fall within either category. 

\subsection{Solid-state nanopores}

\subsubsection{Dielectric membranes}

\subsubsection{Quasi 2D membranes}

\subsection{Biological nanopores}

\subsubsection{Pore forming toxins (PFTs)}

Pore-forming toxins (\glspl{pft})

\paragraph{\ta-\glspl{pft}}

\paragraph{\tb-\glspl{pft}}


%
%
\section{Biological nanopores of interest}

\todo{alpha-hemolysin general description}

\subsection{\ta-Hemolysin (\ta Hl)}

The most widely studied biological nanopore is \textalpha-hemolysin (\gls{ahl}), a homoheptameric porin
secreted by \textit{Staphylococcus aureus}.

\subsubsection{Pore structure}

\paragraph{Pore shape and charge.}

\paragraph{Oligomerization state.}

\subsubsection{Pore formation and kinetics}

\paragraph{Pore formation.}

\paragraph{Oligomerization kinetics.}


\subsection{Cytolysin A (ClyA)}

\begin{figure*}[p]
  \centering
  \medskip
    %
  \begin{minipage}[t]{51mm}
    \begin{subfigure}[b]{51mm}
      \centering
      \caption{}\vspace{-8.5mm}\hspace{1.5mm}\label{fig:clya_pore_side}
      \includegraphics[scale=1]{clya_pore_side}
    \end{subfigure}
    %
    \vspace{5mm} \\
    %
    \begin{subfigure}[b]{51mm}
      \centering
      \caption{}\vspace{-8.5mm}\hspace{1.5mm}\label{fig:clya_pore_top}
      \includegraphics[scale=1]{clya_pore_top}
    \end{subfigure}
  \end{minipage}
  %
  \begin{minipage}{66mm}
    \begin{subfigure}[t]{60mm}
      \centering
      \caption{}\vspace{-8.5mm}\hspace{1.5mm}\label{fig:clya_pore_section}
      \includegraphics[scale=1]{clya_pore_section}
    \end{subfigure}
    %
    \vspace{5mm} \\
    %
    \begin{subfigure}[b]{37mm}
      \caption{}\vspace{-8.5mm}\hspace{1.5mm}\label{fig:clya_monomer}
      \includegraphics[scale=1]{clya_monomer}
    \end{subfigure}
    %
    \begin{subfigure}[b]{28mm}
      \centering
      \caption{}\vspace{-8.5mm}\hspace{1.5mm}\label{fig:clya_protomer}
      \includegraphics[scale=1]{clya_protomer}
    \end{subfigure}
  \end{minipage}
\caption[Structure of the Cytolysin A (ClyA) porin.]{%
  \textbf{Structure of the Cytolysin A (ClyA) porin.}
  %
  (\subref{fig:clya_pore_side})
  %
  Side view and 
  %
  (\subref{fig:clya_pore_top})
  %
  top view of the dodecameric \gls{clya} nanopore (\pdbid{6MRT} \cite{Peng-2019}), with a single
  subunit (protomer) highlighted in color and the location of the lipid bilayer indicated in blue.
  %
  (\subref{fig:clya_pore_section})
  %
  Cross-section view of the interior molecular surface of the pore, colored according to its estimated
  electrostatic potential in physiological conditions as calculated by \gls{apbs}
  \cite{Baker-2001,Baker-2005}.
  %
  (\subref{fig:clya_monomer})
  %
  The \SI{34}{\kDa} water-soluble \gls{clya} monomer (\pdbid{1QOY} \cite{Wallace-2000}) consists of a core
  bundle of \ta-helices (\ta A1, \ta A2, \ta B, \ta C, \ta F), onto which a hydrophobic \tb-tongue
  motif---flanked by two short \ta-helices (\ta D and \ta E)---and a C-terminal \ta-helix (\ta G) are packed.
  The two glycines in the \tb-tongue confer it with a high degree of flexibility, while the removal of Phe191
  from its hydrophobic pocket triggers pore formation.
  %
  (\subref{fig:clya_protomer})
  %
  A single ClyA protomer (\pdbid{6MRT} \cite{Peng-2019}), with the same coloring as the monomer, showing the
  insertion of the \ta-tongue into the lipid bilayer and the near \SI{14}{\nm} conformational change required
  for the \ta A1 helix to form the transmembrane part of the pore.
  %
  All images were prepared and rendered using VMD \cite{Humphrey-1996,Stone-1998}.
  }\label{fig:clya_pore_structure}
\end{figure*}

In 2009, Mueller \etal{} revealed the crystal structure of Cytolysin A (\gls{clya},
\cref{fig:clya_pore_structure}) \cite{Mueller-2009}, a large \textalpha-\gls{pft} secreted by various
\textit{Salmonella enterica} and \textit{Esterischia coli} strains with a high affinity for mammalian cell
membranes. The large size of its \lumen---compared to \gls{ahl} and \gls{mspa}, the most commonly used protein
nanopores at the time---makes \gls{clya} particularly well equipped to fully capture and study larger
biopolymers, such as proteins, in their native three-dimensional configuration (\ie~folded). In the following
section, we will discuss the structure and oligomerization mechanism of \gls{clya} in detail.

\subsubsection{Pore structure}

\paragraph{Pore shape and charge.}
%
The \SI{408}{\kDa} \gls{clya} pore (\pdbid{6mrt} \cite{Peng-2019}) consists of 12 idential subunits
(protomers), arranged in a ring like quaternary structure to form a cylindrical structure roughly \SI{14}{\nm}
in height and \SI{11}{\nm} in diameter (\cref{fig:clya_pore_side,fig:clya_pore_top}). The pore walls in
contact with the solvent are formed by a core bundle of four tightly packed \ta-helixes per protomer, and the
transmembrane region is formed by the iris-like arrangement of the amphiphatic N-terminal \ta-helices.
Virtually all helices contribute to interprotomer contacts, resulting in the burying of
\SI{\approx2400}{\square\angstrom} surface area, and the formation of 25~H-bonds and 13~salt bridges per
protomer-protomer interface. The interior shape of \gls{clya} can be described as a set of two hollow
cylinders placed on top of each other: a large \cisi{} chamber (`\lumen{}') with an inner diameter of
\SI{\approx5.5}{\nm} and a height of \SI{10}{\nm} and a smaller \transi{} chamber (`\textit{constriction}')
with a width of \SI{\approx3.6}{\nm} and a height of \SI{4}{\nm} (\cref{fig:clya_pore_section}). \Gls{clya}
has an excess negative charge (\SI{-60}{\ec} at \pH{7.5}), most of which riddle the interior walls of the
pore. This results in a negative electrostatic potential inside both the \lumen{} and the constriction of the
pore (\cref{fig:clya_pore_section}) and explains the observed cation selectivity
\cite{Soskine-2012,Franceschini-2016}.

\paragraph{Monomer vs. protomer.}
%
The water-soluble \gls{clya} monomer (\cref{fig:clya_monomer}) consists of predominantly \ta-helices, with the
core of the protein formed by a bundle of four long \ta-helices (\ta A, \ta B, \ta C and \ta F). Packed at the
top and bottom of this bundle are respectively the C-terminal helix (\ta G) and a hydrophobic \tb-hairpin
motif flanked by to short \ta-helices (\ta D and \ta E) \cite{Wallace-2000,Mueller-2009}. Comparing this
structure to that of the protomer (\cref{fig:clya_protomer}), reveals that while three of the core helices are
merely straightened and elongated (at the expense of the \tb-hairpin and the \ta A2, \ta D and \ta E helices),
the \ta A1 helix must move to the opposite side of the monomer, a distance of \SI{\approx14}{\nm} (!). The
mechanism that enables these large conformational changes is similar to that of a spring-loaded lock, which is
opened upon membrane binding.

\paragraph{Oligomerization stoichiometry.}
%
Next to the typical 12-subunit pore (dodecamer, `Type I'), \gls{clya} can also form 13-mers (tridecamers,
`Type II') and 14-mers (tetradecamers, `Type III') \cite{Soskine-2013,Peng-2019}, with constriction diameters
of \SIrange{3.3}{4.0}{\nm}, \SIrange{3.7}{4.4}{\nm} and \SIrange{4.2}{5.2}{\nm}, respectively. Aside from
their increased aperture size, the overal structure and assembly mechanism of these pores is identical to that
of the dodecamer, and hence the following discussion will be limited to the Type I pore. Interested readers
are referred to the work of Peng and coworkers \cite{Peng-2019}, who recently obtained high-resolution
structures of the Type II and III pores using \gls{cryo-em}.

\subsubsection{Pore formation mechanism}

\paragraph{Pore formation.}
%
Initial binding of \gls{clya} monomers to the membrane (\textit{via} both specific and non-specific
interactions) is mediated by the partially solvent-exposed, hydrophobic residues of the \tb-hairpin motif.
Using Gly180 and Gly201 as hinges, the \tb-hairpin swings outwards to insert itself into the lipid bilayer,
which removes a stabilizing aromatic residue (Phe190) from its hydrophobic pocket and unlatches a two-helix
spring-loaded mechanism. This results in 1) the straightening and extension of three of the core helices (\ta
B, \ta C and \ta F), and 2) a \SI{180}{\degree} swinging of the \ta A helix (relative to \ta B, \ta C and \ta
F). The amphiphatic \ta-A1 helix, which will form the \gls{tmd}, now rests on top of the membrane and is
connected to the rest of the protein via a \gls{hth} motif, with Pro36 acting as a hinge. The
subsequent oligomerization of the membrane-bound monomers further packs and buckles the \ta A1 helices,
effectivily pushing them downwards and allowing them to `wedge' a hole into the lipid bilayer.

\paragraph{Oligomerization kinetics.}
%
Initial far-UV \gls{cd} measurements of the \gls{clya} oligomerization process in the presence of the mild,
non-ionic detergent \gls{ddm} suggested a two-step process, in which the soluble monomers (\ce{M}) undergo a
rapid transition ($t_{1/2} \approx \SI{80}{\second}$) to molten globule-like intermediate (\ce{I}), followed
by a slow assembly  ($t_{1/2} \approx \SI{1000}{\second}$) of the membrane-bound protomers (\ce{P})into the
full dodecameric pore (\ce{P_12}) \cite{Eifler-2006}. However, subsequent investigation of the oligomerization
kinetics \textit{via} \gls{rbc} hemolysis assays \cite{Vaidyanathan-2014} and single-molecule
\gls{fret} \cite{Benke-2015} revealed that the intermediate state is actually an off-pathway by-product that
slows down the pore formation \cite{Roderer-2017}
%
\kineticscheme{arrow coeff=2}{
  I
  \arrow{<=>[\small$k_{\text{IM}} = \SI{5.0e-2}{\per\second}$][\small$k_{\text{MI}} = \SI{3.0e-1}{\per\second}$]} 
  M 
  \arrow{<=>[\small$k_{\text{MP}} = \SI{1.7e-2}{\per\second}$][\small$k_{\text{PM}} = \SI{4.7e-4}{\per\second}$]}
  P
}{eq:clya_protomer_formation}
%
Note that converting from \ce{M -> I} ($k_{\text{MI}}$) is $\approx17~\times$ faster than \ce{M -> P}
($k_{\text{MP}}$), resulting the initial accumulation of molten globule-like intermediates. On the long term,
however, most of the monomers will convert to protomers, as the reverse reaction in the monomer to protomer
transition \ce{P -> M} ($k_{\text{PM}}$) is $\approx36~\times$ slower compared to its forward reaction
\ce{M -> P} ($k_{\text{MP}}$).

After sufficient monomers have bound to the membrane, the increased probability of intermolecular collision
between protomers results in the formation of linear mixtures of oligomers ($\ce{P_{n/m}}$) by a process
called protomer elongation \cite{Roderer-2017}
%
\kineticscheme{arrow coeff=2}{
  \chemfig{P_n + P_m}
  \arrow{->[\small$k_{\text{elong}} = \SI{1.0e5}{\per\Molar\per\second}$]}
  \chemfig{P_{n+m}}
  \hspace{0.5cm}(n + m < 12)
}{eq:clya_protomer_elongation}
%
Because virtually all $\ce{P_n}$ species contribute to this process, the effective protomer concentration does
not decrease, making it both rapid and efficient. Formation of the full pore occurs rapidly \textit{via} ring
closure when a pair of oligomers---with a total protomer count of 12---collide and associate
%
\kineticscheme{arrow coeff=2}{
  \chemfig{P_n + P_{12-n}}
  \arrow{->[\small$k_{\text{pore}} = \SI{3.0e7}{\per\Molar\per\second}$]}
  \chemfig{P_{12}}
  \hspace{0.5cm}(n = 1,2,\ldots,11)
}{eq:clya_ring_closure}
%
Even though all $\ce{P_{1\ldots11}}$ species contribute to the formation of $\ce{P_12}$, the relatively high
abundance of $\ce{P_5}$, $\ce{P_6}$ and $\ce{P_7}$ dictate that \SI{>50}{\percent} of all full pores are
formed by these oligomers \cite{Benke-2015}.


\subsection{Fragaceatoxin C (FraC)}

\begin{figure*}[p]
  \centering
  \medskip
    %
  \begin{minipage}[t]{58mm}
    \begin{subfigure}[t]{58mm}
      \centering
      \caption{}\vspace{-8.5mm}\hspace{1.5mm}\label{fig:frac_pore_side}
      \includegraphics[scale=1]{frac_pore_side}
    \end{subfigure}
    %
    \vspace{5mm} \\
    %
    \begin{subfigure}[t]{58mm}
      \centering
      \caption{}\vspace{-8.5mm}\hspace{1.5mm}\label{fig:frac_pore_top}
      \includegraphics[scale=1]{frac_pore_top}
    \end{subfigure}
  \end{minipage}
  %
  \begin{minipage}[t]{58mm}
    \begin{subfigure}[t]{58mm}
      \centering
      \caption{}\vspace{-8.5mm}\hspace{1.5mm}\label{fig:frac_pore_section}
      \includegraphics[scale=1]{frac_pore_section}
    \end{subfigure}
    %
    \vspace{5mm} \\
    %
    \begin{subfigure}[t]{28mm}
      \caption{}\vspace{-8.5mm}\hspace{1.5mm}\label{fig:frac_monomer}
      \includegraphics[scale=1]{frac_monomer}
    \end{subfigure}
    %
    \begin{subfigure}[t]{28mm}
      \centering
      \caption{}\vspace{-8.5mm}\hspace{1.5mm}\label{fig:frac_protomer}
      \includegraphics[scale=1]{frac_protomer}
    \end{subfigure}
  \end{minipage}
\caption[Structure of the Fragaceatoxin C (FraC) porin.]{%
  \textbf{Structure of the Fragaceatoxin C (FraC) porin.}
  %
  (\subref{fig:frac_pore_side})
  %
  Side and 
  %
  (\subref{fig:frac_pore_top})
  %
  top views of the octameric \SI{176}{\kDa} \gls{frac} porin crystal structure (\pdbid{4TSY}
  \cite{Tanaka-2015}) with single protein subunit highlighted in color and the co-crystalized sphingomyelin
  lipids indicated in yellow. The overall shape is that of a funnel with a height of \SI{7}{\nm}, and top and
  bottom diameters of \SIlist{11;3.5}{\nm}, respectively.
  %
  (\subref{fig:frac_pore_section})
  %
  Cross-sectional view of the internal molecular surface of \gls{frac}, colored according to the local
  electrostatic potential at physiological conditions as calculated by \gls{apbs}
  \cite{Baker-2001,Baker-2005}. \Gls{frac} has inner diameters of \SIlist{6;1.6}{\nm} on the \cisi{} and
  \transi{} sides, respectively. Note the fenestrations between the subunits, which are partly filled
  by permanently bound lipids and hence expose and the core of membrane bilayer to the solvent.
  %
  (\subref{fig:frac_monomer})
  %
  The crystal structure of the \SI{20}{\kDa} \gls{frac} monomer (\pdbid{3VWI} \cite{Tanaka-2015}) shows that
  in its water-soluble form the N-terminal \gls{tmd} (blue) is packed tightly against the \tb-sheet rich core
  domain (orange). Displacement of Phe16 (mauve) from its hydrophobic pocket by Val60 (grey) trigger pore
  formation.
  %
  (\subref{fig:frac_protomer})
  %
  The N-terminal \ta-helix of each \gls{frac} protomer (\pdbid{4TSY} \cite{Tanaka-2015}) first elongates and
  then flips \SI{180}{\degree} to span the entire membrane. The conformation of \tb-core remains virtually
  identical to that of the monomer. Where the L2 and L3 lipids are bound to a single subunits, the L1 lipid is
  wedged in between two protomers and contributes \emph{structurally} to the solvent-exposed pore walls.
  Images were rendered using VMD \cite{Humphrey-1996,Stone-1998}.}\label{fig:frac_pore_structure}
\end{figure*}

Fragaceatoxin C (\gls{frac}, \cref{fig:frac_pore_structure}) is an \ta-\gls{pft} belonging to the family of
the actinoporins, and is produced by the sea anemone \textit{Actinia fragacea}. Its X-ray structure was
resolved in 2015 by Tanaka \etal{} \cite{Tanaka-2015}, and suprisingly contained several co-crystalized
sphingomyelin (\gls{sm}) lipids per subunit. This indicates that \gls{frac} does not merely exhibit a high
degree of specificity towards \gls{sm}-containing membranes, these lipids comprise an essential structural
element of the pore itself. In the following section we will describe and discuss the structure and assembly
mechanism of \gls{frac}.

\subsubsection{Pore structure}

\paragraph{Pore shape and charge.}
%
The full \SI{176}{\kDa} \Gls{frac} porin (\pdbid{4TSY} \cite{Tanaka-2015}) consists of eight identical protein
subunits and three \gls{sm} lipids per protomer, bound at highly specific locations
(\cref{fig:frac_pore_side,fig:frac_pore_top}). It is shaped as a funnel with a height of \SI{7}{\nm} and
\cisi{} and \transi{} outer diameters of \SIlist{11;3.5}{\nm}, respectively. The extracellular (\cisi) part of
the pore comprises the bulk of each protomer and is rich in \tb-sheets (\tb-core). The \gls{tmd} is formed by
the iris-like arrangement of the N-terminal \ta-helices of each subunit, which are long enough (\SI{3.5}{\nm})
to span the entire lipid bilayer. The protomer-protomer interface buries a surface area of
\SI{777}{\square\angstrom}, to which both the \tb-core and the \gls{tmd} \ta-helices contribute, and is
further stabilized by protein-lipid H-bonding \cite{Tanaka-2015}. Close inspection of the pore walls reveal
the presence of eight lateral fenestrations, one between each protomer, that expose the hydrophobic core of
the membrane to the aqueous solvent (\cref{fig:frac_pore_section}). These openings are at least partially
occluded by the co-crystallized \gls{sm} lipids, which as such contribute \emph{structurally} to the pore.
Even though the precise biological function of these fenestrations remains unclear, it is likely that they
contribute significantly to the toxicity of sea anemone venom by 1) facilitating the diffusion of its small
hydrophobic components directly into the core of the lipid bilayer, and 2) disrupting the leaflet composition
of the membrane by catalyzing the flip-flip movement of lipids \cite{Tanaka-2015}. The interior of \gls{frac}
is also funnel-shaped, with a \cisi{} diameter of \SI{6}{\nm} and a \trans{} diameter, which is also the
pore's narrowest constriction, of only \SI{1.6}{\nm}. Consistent with the observation of its cation
selectivity \cite{Garcia-Ortega-2011,Wloka-2016}, the interior walls of the pore are predominantly negatively
charged, particularly at the \transi{} constriction (\cref{fig:frac_pore_section}). The extracellular regions
of the pore near the lipid headgroups exhibit a strong positive potential however, which likely aides in the
initial adhesion of \gls{frac} to the membrane \cite{Tanaka-2015}.

\paragraph{Monomer vs. protomer.}
%
In the water-soluble \gls{frac} monomer (\pdbid{3VWI} \cite{Tanaka-2015}) the N-terminal region (residues
1--29; contains a short \tb-sheet, a 3$_10$ helix and an \ta-helix) is packed closely against the C-terminal
\tb-core (residues 30--179; contains 11 \tb-sheets in a \tb-sandwich fold and one \ta-helix) and held in place
by the interaction between Phe16 and a hydrophobic cavity in the \tb-core (\cref{fig:frac_monomer}). The
\tb-core remains virtually unchanged in the protomer structure (\pdbid{4TSY} \cite{Tanaka-2015}), while the
N-terminal domain forms a much longer \ta-helix that spans the entire lipid bilayer
(\cref{fig:frac_protomer}). In addition, three lipids (`L1', `L2' and `L3') are bound at specific locations to
the \tb-core. Of these, the L1 position can be considered to be of high affinity and specific only to
\gls{sm}, whereas L2 and L3 are of lower affinity and hence more promiscuous \cite{Tanaka-2015}. 

\paragraph{Oligomerization stoichiometry.}
%
The a lytically active \gls{frac} pore can be obtained as an 8-mer (octamer, `Type I'), 7-mer (heptamer `Type
II') and 6-mer (hexamer, `Type III') \cite{Huang-2019}, with constriction diameters of
\SIlist{1.6;1.1;0.84}{\nm}, respectively. A non-lytic 9-mer (nonamer) has also been crystalized (\pdbid{3LIM}
\cite{Mechaly-2011}), and lower oligomeric states also appear to be capable of forming pores
\cite{Rojko-2016}.

\subsubsection{Pore formation mechanism}

\paragraph{Pore formation.}
%
Upon initial attachment of the monomer to the membrane, the binding of a single \gls{sm} lipid to the L1
pocket induces dimerization \cite{Tanaka-2015}. The resulting small conformational change (at residues 14--17)
displaces Phe16 from its hydrophobic pocket in favor of Val60 and partially unfolds the N-terminal domain.
Because dimerization is only possible in the presence of \gls{sm}, this lipid acts as both a receptor and an
assembly co-factor that contributes to the final structure of the pore. Sequential addition of other dimeric
units \textit{via} random collisions---which can occur frequently in the small \gls{sm} lipid raft
domains---give rise to higher oligomeric complexes that further destabilize the N-terminal region. Given the
limited space avaible in the lumen of the final pore (\SI{\approx2}{\nm} diameter), it is likely that the
insertion of the N-terminal \ta-helix occurs in a non-concerted manner prior to full ring closure
\cite{Cosentino-2016}. It is also possible that a pre-pore structure is formed, similar to the non-lytic
\gls{frac} nonamer \cite{Mechaly-2011}, with the amphipatic N-termini pushing through the membrane in a
concerted fashion \cite{Tanaka-2015,Rojko-2016} much like the \gls{clya} porin. Evidence favors the first
mechanism however, given that a sole N-terminal helix of many actinoporins can exist at the lipid-water
interface in a so-called `protein-lipid pore' \cite{Cosentino-2016}.


\paragraph{Oligomerization scheme.}
%
\kineticscheme{arrow coeff=0.5}{
  M
  \arrow{->} 
  \chemfig{M^{*}}
  \arrow{->[\chemfig{+ M^{*} + SM}]}
  \chemfig{M^{**}_2}
  \arrow{->[\chemfig{+ M^{**}_2}]}
  \chemfig{(M^{**}_2)_2}
}{eq:frac_pore_formation}

\subsection{Pleurotolysin AB (PlyAB)}

\begin{figure*}[p]
  \centering
  \medskip
    %
  \begin{minipage}[t]{56mm}
    \begin{subfigure}[t]{56mm}
      \centering
      \caption{}\vspace{-8.5mm}\hspace{1.5mm}\label{fig:plyab_pore_side}
      \includegraphics[scale=1]{plyab_pore_side}
    \end{subfigure}
    %
    \vspace{5mm} \\
    %
    \begin{subfigure}[t]{56mm}
      \centering
      \caption{}\vspace{-8.5mm}\hspace{1.5mm}\label{fig:plyab_pore_top}
      \includegraphics[scale=1]{plyab_pore_top}
    \end{subfigure}
  \end{minipage}
  %
  \begin{minipage}[t]{60mm}
    \begin{subfigure}[t]{56mm}
      \centering
      \caption{}\vspace{-8.5mm}\hspace{1.5mm}\label{fig:plyab_pore_section}
      \includegraphics[scale=1]{plyab_pore_section}
    \end{subfigure}
    %
    \vspace{5mm} \\ 
    %
    \begin{subfigure}[t]{35mm}
      \caption{}\vspace{-8.5mm}\hspace{1.5mm}\label{fig:plyab_monomer}
      \includegraphics[scale=1]{plyab_monomer}
    \end{subfigure}
    %%
    \begin{subfigure}[t]{24mm}
      \centering
      \caption{}\vspace{-8.5mm}\hspace{1.5mm}\label{fig:plyab_protomer}
      \includegraphics[scale=1]{plyab_protomer}
    \end{subfigure}
  \end{minipage}
\caption[Structure of the Pleurotolysin AB (PlyAB) porin.]{%
  \textbf{Structure of the Pleurotolysin AB (PlyAB) porin.}
  %
  (\subref{fig:plyab_pore_side})
  %
  Side and
  %
  (\subref{fig:plyab_pore_top})
  %
  top view of the \SI{1067}{\kDa} \gls{plyab} \tb-\gls{pft} porin (\pdbid{4V2T}
  \cite{Lukoyanova-Kondos-2015}), formed by 13 identical subunits which are each composed of two PlyA (pink)
  one PlyB (green) protein chains. The full pore is \SI{13}{\nm} high with external diameters of \SI{22}{\nm}
  and \SI{9}{\nm} at the \cisi{} and \transi{} sides of the membrane. 
  %
  (\subref{fig:plyab_pore_section})
  %
  The cross-section of \gls{plyab}'s molecular surface, colored according to the local electrostatic potential
  at physiological salt concentration (\gls{apbs} \cite{Baker-2001,Baker-2005}), shows that the interior of
  the pore is predominantly negative. The lumen of the pore is divided by a \SI{5.5}{\nm} wide constriction
  into a conical \cisi{} chamber of \SI{3}{\nm} height and \SI{10.5}{\nm} diameter, and a cylindrical
  \transi{} chamber of \SI{10}{\nm} height and \SI{7.2}{\nm} diameter. 
  %
  (\subref{fig:plyab_monomer})
  %
  Cartoon view of the water-soluble monomer configurations of PlyA (\pdbid{4OEB}
  \cite{Lukoyanova-Kondos-2015}) and PlyB (\pdbid{4OEJ} \cite{Lukoyanova-Kondos-2015}), aligned according to
  their position in a single protomer of the full pore (using \gls{vmd} \cite{Humphrey-1996}). Important
  regions and residues are highlighted in color.
  %
  (\subref{fig:plyab_protomer})
  %
  A single \gls{plyab} protomer subunit (\pdbid{4V2T} \cite{Lukoyanova-Kondos-2015}), showing the extension of
  the transmembrane \ta-helices into \tb-sheets.
  Images were rendered using \gls{vmd} \cite{Humphrey-1996,Stone-1998}.
  }\label{fig:plyab_pore_structure}
\end{figure*}

Pleurotolysin AB (\gls{plyab}, \cref{fig:plyab_pore_structure}) is a \tb-\gls{pft}, secreted by the fungus
\textit{Pleurotus ostreatus}, that belongs to the family of membrane attack complex/perforin-like
(\gls{macpf}) proteins. Lukoyanova and Kondos \etal{} proposed a plausible pore structure based on the
\gls{cryo-em} image analysis and molecular modelling with a molecular weight in excess of \SI{1000}{\kDa}
\cite{Lukoyanova-Kondos-2015}, making it one of the largest \glspl{pft} with a known structure to date.


\subsubsection{Pore structure}

\paragraph{Pore shape and charge.}
%
The \gls{plyab} pore (\pdbid{4V2T} \cite{Lukoyanova-Kondos-2015}) consists of 13 identical `subunits', each of
which is composed of two lipid binding PlyA and a single pore forming PlyB protein chain (\cref{fig:plyab_pore_side,fig:plyab_pore_top}).

\paragraph{Monomer vs. protomer.}
%
The water-soluble PlyA crystal structure (\pdbid{4OEB} \cite{Lukoyanova-Kondos-2015}) contains a
membrane-binding \tb-sandwich fold (\cref{fig:plyab_monomer})---similar to that of the actinoporin family
(\eg~\gls{frac}, \cref{fig:frac_monomer})---but lacks the typical N-terminal transmembrane region. The
water-soluble PlyB structure (\pdbid{4OEJ} \cite{Lukoyanova-Kondos-2015}) can be subdivided in a \tb-rich
globular trefoil domain at the a C-terminus, and a \gls{macpf} domain at the N-terminus. At its core, the
latter contains a four-stranded bent and twisted \tb-sheet (typical for the \gls{macpf} superfamily), a
flexible \gls{hth}-motif and two \ta-helical regions (TMH1 and TMH2). In the protomer structure (\pdbid{4VT2}
\cite{Lukoyanova-Kondos-2015}), the TMH1 and TMH2 regions become fully unwound into \tb-hairpins to become
part of the 52-sheet membrane-spanning \tb-barrel (\cref{fig:plyab_protomer}).

\paragraph{Oligomerization stoichiometry}
%
As with the other pores discussed above, \gls{plyab} is capable of forming pores with different
stoichiometries. \Gls{cryo-em} analysis revealed a diverse population of 14- (\SI{5}{\percent}), 13-
(\SI{75}{\percent}), 12- (SI{15}{\percent}) and 11-mers (SI{5}{\percent}) \cite{Lukoyanova-Kondos-2015}.


\subsubsection{Pore formation mechanism}

\paragraph{Pore formation.}
%
% neither TMH region can enter the membrane without the other (cooperative folding and assembly

\paragraph{Oligomerization scheme.}
%


%
%
\section{Nanopore applications}

\subsection{Small molecules}

\subsection{Nucleic acids }

\subsection{Proteins and peptides}


\todo{frac nanopore applications}






%%%%%%%%%%%%%%%%%%%%%%%%%%%%%%%%%%%%%%%%%%%%%%%%%%
% Keep the following \cleardoublepage at the end of this file,
% otherwise \includeonly includes empty pages.
\cleardoublepage

% vim: tw=70 nocindent expandtab foldmethod=marker foldmarker={{{}{,}{}}}
