\chapter{Three decades of experimental and theoretical work on nanopores}\label{ch:introduction}

\definecolor{shadecolor}{gray}{0.85}
\begin{shaded}
Parts of this chapter were adapted from:\\
\fullcite{Willems-VanMeervelt-2017}
\newpage
\end{shaded}

This chapter serves as a comprehensive introduction to the primary concepts required to understand the objectives and relevance of the work performed in this thesis. In particular, the reader will be familiarized with nanopores as single-molecular sensors, and will be given an overview of the relevant experimental and computational approaches used to investigate their properties.
\\
\\
The text and figures of this introductory chapter were entirely written and created by myself.

\newpage



\section{Molecular sensors}
\section{Nanopores as single-molecular sensors}



\section{Nanopore types}
\gls{bnp}
\gls{ssnp}

\subsection{Biological nanopores}

Was published \cite{Willems-VanMeervelt-2017}

\subsubsection{\textalpha-HL\textalpha-Hemolysin}

The most widely studied biological nanopore is \textalpha-hemolysin (\gls{ahl}), a homoheptameric porin secreted by \textit{Staphylococcus aureus}.

\subsubsection{\textit{Mycobacterium smegmatis} porin A}

\gls{mspa}

\subsubsection{Cytolysin A}

\gls{clya}

\subsubsection{Fragaceatoxin C}

\gls{frac}

\subsubsection{Pleurotolysin AB}

\gls{plyab}

\subsubsection{Other noteworthy biological nanopores}

\subsection{Solid-state nanopores}

\subsection{Hybrid nanopores}



\section{Modelling of biological and solid-state nanopores}
The most commonly used technique to study the complex behavior and properties of biological (i.e. protein) nanopores is molecular dynamics (\gls{md}).

\section{High-throughput nanopore read-out strategies}



% Some dummy code show how to include images.
\begin{figure}
  \centering
  \medskip
  \includegraphics[width=.9\textwidth]{sine}
  \caption{Illustration of how to include a figure. }
  \label{fig:sine}
\end{figure}




%%%%%%%%%%%%%%%%%%%%%%%%%%%%%%%%%%%%%%%%%%%%%%%%%%
% Keep the following \cleardoublepage at the end of this file,
% otherwise \includeonly includes empty pages.
\cleardoublepage

% vim: tw=70 nocindent expandtab foldmethod=marker foldmarker={{{}{,}{}}}
