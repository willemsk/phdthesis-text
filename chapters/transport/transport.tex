% !TeX root = ../../thesis.tex
\chapter{Modeling the transport of ions and water through {ClyA}}
%
\label{ch:transport}
%

% \epigraphhead[\epipos]{%
% \epigraph{%
% %
%   ``Not the possession of truth but the effort in struggling to attain to it brings joy to the researcher.''
% %
% }{%
%   \textit{Gotthold Ephraim Lessing}
% %
% }}

\epigraphhead[\epipos]{%
\epigraph{%
%
  ``That was one of the odd, persistent defects of simulations: no matter how precise they became, the
  participant remained aware that they were not reality.''
%
}{%
  \textit{`Sylveste' in `Revelation Space' by Alastair Reynolds}
%
}}




%
%
\definecolor{shadecolor}{gray}{0.85}
\begin{shaded}
This chapter was published as:
%
\begin{itemize}
  \item K. Willems*, D. Rui\'{c}, F. L. R. Lucas, U. Barman, N. Verellen, G. Maglia and P. Van Dorpe.
        \textit{Nanoscale} \textbf{12}, 16775--16795 (2020) %\cite{Willems-2020}
\end{itemize}
%
\newpage
\end{shaded}
%
%

% Glossary reset
%

In this chapter we apply the \glsfirst{epnp-ns} equations introduced in \cref{ch:epnpns} to the biological
nanopore \glsfirst{clya}. To this end, we set-up a 2D-axisymmetric representation of the geometry and charge
distribution of the pore, and numerically solved the \gls{epnp-ns} equations using \glsfirst{fea} for a wide
range of bias voltages and ionic strengths. This allowed us to validate our model against experimental data,
and investigate \gls{clya}'s transport properties in detail.
%

%
To improve readability and to allow for a better integration of the supplementary information into the main
chapter text, the original manuscript was split into \cref{ch:epnpns,ch:transport}, detailing the theory of
the framework and its application to the \gls{clya} nanopore, respectively. The remainder of the supplementary
information can be found in \cref{ch:transport_appendix}. The text and figures of this chapter represent
entirely my own work. All experimental work was performed by Florian L. R. Lucas.
%

%
\adapRSC{Willems-2020}
%
% \cleardoublepage
%
%


\section{Introduction}
%
\label{sec:transport:intro}
%




To validate our new \gls{epnp-ns} framework, we applied it to a 2D-axisymmetric model of
\gls{clya}~\cite{Mueller-2009}, a large protein nanopore~\cite{Soskine-2012} that typically contains 12 or
more identical subunits~\cite{Soskine-2013,Peng-2019}. We choose to model the \gls{clya-as} type I, a
dodecameric variant of the wild type \gls{clya} from \textit{S. Typhii} that was artificially evolved for
improved stability~\cite{Soskine-2013}. \Gls{clya-as} has been extensively used in experimental studies of
proteins~\cite{Soskine-2013,VanMeervelt-2014,Soskine-Biesemans-2015,Biesemans-2015,Wloka-2017,
VanMeervelt-2017,Galenkamp-2018,Willems-Ruic-Biesemans-2019,Zernia-2020,Galenkamp-2020} and
\gls{dna}~\cite{Franceschini-2013,Franceschini-2016,Nomidis-2018} and has even been employed as a targeted
immunotoxin~\cite{Mutter-2018}. This allowed us to gauge the qualitative and quantitative performance of the
\gls{epnp-ns} equations and simultaneously elucidate previously inaccessible details about the environment
inside the pore.


%
\begin{figure*}[b]
  \centering
  %
  \begin{subfigure}[t]{5.5cm}
    \centering
    \caption{}\hspace{0mm}\label{fig:transport_clya_side}
    \includegraphics[scale=1]{transport_clya_side}
  \end{subfigure}
  %
  %
  \begin{subfigure}[t]{5.5cm}
    \centering
    \caption{}\hspace{0mm}\label{fig:transport_clya_top}
    \includegraphics[scale=1]{transport_clya_top}
  \end{subfigure}
  %

  \caption[All-atom model of {ClyA-AS}]
  {%
    %
    \textbf{All-atom models of {ClyA-AS}.}
    %
    (\subref{fig:transport_clya_side})
    %
    Axial cross-sectional and
    %
    (\subref{fig:transport_clya_top})
    %
    top views of the dodecameric nanopore \gls{clya-as}~\cite{Soskine-2013}, derived through homology
    modeling from the \textit{E. coli} \gls{clya} crystal structure (\pdbid{2WCD}~\cite{Mueller-2009}).
    Measurements indicate the average `cylindrical' sizes of the \cisi{} \lumen{} (\SI{6}{\nm} in diameter,
    \SI{10}{\nm} in height) and the \transi{} constriction (\SI{3.3}{\nm} in diameter, \SI{4}{\nm} in height).
    Images were rendered with \glsxtrshort{vmd}~\cite{Humphrey-1996,Stone-1998}.
  %
  }\label{fig:transport_clya}
\end{figure*}
%


\Gls{clya} is a relatively large protein nanopore that self-assembles on lipid bilayers to form \SI{14}{\nm}
long hydrophilic channels (\cref{fig:transport_clya_side}). The interior of the pore can be divided into
roughly two cylindrical compartments: the \cisi{} \lumen{} (\SI{\approx6}{\nm} diameter, \SI{\approx10}{\nm}
height), and the \transi{} constriction (\SI{\approx3.3}{\nm} diameter, \SI{\approx4}{\nm} height). Because
\gls{clya} consists of 12 identical subunits (\cref{fig:transport_clya_top}), it exhibits a high degree of
radial symmetry, a geometrical feature that can be exploited to obtain meaningful results at a much lower
computational cost~\cite{Cervera-2005,Lu-2012,Pederson-2015}. However, this requires the reduction of the full
3D atomic structure and charge distribution to a realistic 2D-axisymmetric model.

We will start this chapter by describing the process of creating such a 2D-axisymmetric model of a nanopore
from a 3D full atom homology model (\cref{sec:transport:model}). This followed by the validation of the model
by means of direct comparison of simulated and experimentally measured ionic conductances, and a thorough
characterization of the influence of the bulk ionic strength and the applied bias voltage on cation and anion
concentrations inside the pore, the electrostatic potential distribution and magnitude of the electro-osmotic
flow (\cref{sec:transport:results}). Finally, we touch upon our key findings and their impact
(\cref{sec:transport:conclusion}).


%
\begin{figure*}[t]
  \centering
  %
  \begin{minipage}[t]{6.75cm}
    %
    \begin{subfigure}[t]{6.75cm}
      \centering
      \caption{}\label{fig:transport_model_zoom}
      \includegraphics[scale=1]{transport_model_zoom}
    \end{subfigure}
    %
    \\
    %
    \begin{minipage}[t]{5.5cm}
      %
      \addtocounter{subfigure}{1}
      \begin{subfigure}[t]{1.5cm}
        \centering
        \caption{}\label{fig:transport_model_outline}
        \vspace{-5mm}
        \includegraphics[scale=1]{transport_model_outline}
      \end{subfigure}
      %
      \begin{subfigure}[t]{3cm}
        \centering
        \caption{}\hspace{5mm}\label{fig:transport_model_scd}
        \vspace{-5mm}
        \includegraphics[scale=1]{transport_model_scd}
      \end{subfigure}
      %
    \end{minipage}
    %
  \end{minipage}
  %
  \hspace{-1.0cm}
  %
  \addtocounter{subfigure}{-3}
  \begin{subfigure}[t]{5.65cm}
    \centering
    \caption{}\label{fig:transport_model_full}
    \includegraphics[scale=1]{transport_model_full}
  \end{subfigure}
  %

  \caption[A 2D-axisymmetric model of {ClyA-AS}]
  {%
    %
    \textbf{A 2D-axisymmetric model of {ClyA-AS}.}
    %
    (\subref{fig:transport_model_zoom}+\subref{fig:transport_model_full})
    %
    The 2D-axisymmetric simulation geometry of \gls{clya-as} (gray) embedded in a lipid bilayer (green) and
    surrounded by a spherical water reservoir (blue). Note that all electrolyte parameters depend on the local
    average ion concentration $\avionconc=\frac{1}{n}\sum_{i}^{n}\concentration_{i}$ and that some are also
    influenced by the distance from the nanopore wall $\walldistance$.
    %
    (\subref{fig:transport_model_outline})
    %
    The 2D-axisymmetric geometry was derived directly from the all-atom model using a the radially averaged
    atomic density (see methods for details). Hence, it closely follows the outline of a \ang{30} wedge out of
    the homology model.
    %
    (\subref{fig:transport_model_scd})
    %
    The fixed space charge density ($\scdpore$) map of \gls{clya-as}, obtained by Gaussian projection of each
    atom's partial charge onto a 2D plane (see methods for details).
    %
  %
  }\label{fig:transport_model}
\end{figure*}
%



\section{A 2D-axisymmetric model of {ClyA}}
%
\label{sec:transport:model}
%

\subsection{Simulation system geometry and boundary conditions}
%
\label{sec:transport:global_geometry}
%

\subsubsection{Simulation geometry}
%

The complete system (\cref{fig:transport_model_zoom,fig:transport_model_full}) consists of a large
hemispherical electrolyte reservoir ($R=\SI{250}{\nm}$), split through the middle into a \cisi{} and a
\transi{} compartment by a lipid bilayer ($h=\SI{2.8}{\nm}$), which contains the nanopore at its center. Both
the bilayer and the nanopore are represented by dielectric blocks (see~\cref{tab:corrections_equations} for
parameters) that are impermeable to ions and water.

\subsubsection{Boundary conditions}
%

The reservoir boundaries were set up, using Dirichlet \glspl{bc}, to act as electrodes: the \cisi{} side was
grounded ($\potential = 0$) and a fixed bias potential was applied along the \transi{} edge ($\potential =
\vbias$). To simulate the presence of an endless reservoir, the ion concentration at both external boundaries
were fixed to the bulk salt concentration ($\concentration_i = \cbulk$) and the unconstrained flow in and out
of the computational domain was enabled by means of a `no normal stress' condition ($\hydrostresstensor\vec{n}
= 0$). The boundary conditions on the edges of the reservoir shared with the nanopore and bilayer were set to
no-flux ($-\vec{n}\cdot\vec{\flux}_{i} = 0$) and no-slip ($\vec{\velocity} = 0$), preventing the flux of ions
through them and mimicking a sticky hydrophilic surface, respectively. Finally, a Neumann boundary condition
was applied at the bilayer's external boundary ($-\vec{n}\cdot\left(\absperm \relperm \nabla \potential
\right) = 0$). More details can be found in~\cref{sec:epnpns_appendix:weak_forms}.



\subsection{From full atom to a 2D-axisymmetric}
%
\label{sec:transport:pore_geometry}
%

\subsubsection{Molecular dynamics simulations.}
%

To sample the conformational space off the flexible side-chains of the nanopore, we used \gls{md} to
equilibrate the full-atom homology model of \gls{clya-as} type I~\cite{Soskine-2013} from
\cref{ch:electrostatics} (see~\cref{sec:elec:methods:molec}) for \SI{30}{\ns} in an explicit solvent with
harmonic constraints on the protein backbone atoms. The \SI{30}{\ns} production run was performed using a
\gls{nvt} ensemble at \SI{298.15}{\kelvin} and the atomic coordinates saved every \SI{5}{\ps}. Accidental
structural deterioration of the pore was prevented by harmonically restraining the C\ta{} atoms to their
original positions in the crystal structure with a spring constant of \SI{695}{\pN\per\nm} during the entire
\gls{md} run~\cite{Bhattacharya-2011}. All other atoms were allowed to move freely. This allowed us to sample
the conformational landscape of all the \gls{clya} side chains without disrupting the overall structure of the
pore and to generate a well-averaged geometry and charge distribution. The analysis in the next sections were
performed with 50 sets of atomic coordinates---extracted from the final \SI{5}{\ns} of the coordinates of the
\gls{md} production run (\ie~every \SI{100}{\fs})---and aligned by minimizing the \gls{rmsd} between their
backbone atoms (\gls{vmd}'s \code{RMSD tool}~\cite{Humphrey-1996}).

\subsubsection{Side-chain B-factors.}
%

The thermal fluctuations of the side chains can be quantified with the temperature factor or `B-factor'
($\bfactor_i$)~\cite{Bahar-1997}
%
\begin{align}\label{eq:bfactor}
  \bfactor_i = \dfrac{8}{3} \pi^2 \msd \text{ ,}
\end{align}
%
which can be computed for each atom $i$ from its \gls{msd} $\msd$. Starting from the last \SI{10}{\ns} of our
\SI{30}{\ns} \gls{md} trajectory, we computed the $\msd$ using the fast \gls{qcp}
algorithm~\cite{Theobald-2005,Liu-2010}, as implemented in the \code{MDAnalysis} Python
package~\cite{MichaudAgrawal-2011}. A heatmap of the resulting B-factors for each residue, averaged over all
12 monomers of \gls{clya}, can be found in~\cref{fig:transport_bfactor_heatmap}. In addition, we plotted the
frame-averaged B-factor (\cref{fig:transport_bfactor_scatter}), allowing us to compare the flexibility
observed in the \gls{md} trajectory with the B-factors given by the dodecamer \gls{clya} crystal
(\pdbid{2WCD}~\cite{Mueller-2009}) and \gls{cryo-em} (\pdbid{6MRT}~\cite{Peng-2019}) structures.
Interestingly, the $\bfactor_i$ of our \gls{md} simulation does not show much agreement with the original
\pdbid{2WCD} crystal structure, but it reproduces the more flexible regions observed in the (newer)
\pdbid{6MRT} \gls{cryo-em} structure (\ie~residue numbers \numrange{7}{20}, \numrange{95}{105},
\numrange{180}{200} and \numrange{260}{280}). Peng and coworkers also started from the \pdbid{2WCD} crystal
structure to generate their refined \gls{clya} model, albeit using a high resolution \gls{cryo-em} map as a
restraint. Hence, the similarities in flexibility between our model and \pdbid{6MRT} are encouraging that the
\gls{md} relaxes the structure properly. Finally, we also visualized the \gls{md} $\bfactor_i$ on the interior
and exterior molecular surface of \gls{clya-as} (\cref{fig:transport_bfactor_surface}). This shows that the
most flexible regions are the \cisi{} and \transi{} entries of the pore. A cross-section of the final \gls{md}
structure, in comparison with the \gls{clya} dodecamer crystal \pdbid{2WCD}~\cite{Mueller-2009} and
\gls{cryo-em} \pdbid{6MRT}~\cite{Peng-2019} structures can be found
in~\cref{sec:transport_appendix:radius_comparision} (\cref{fig:transport_radius_comparison_surface}).


%
\begin{figure*}[p]
  \centering
  
  %
  \begin{subfigure}[t]{11.5cm}
    \centering
    \caption{}\vspace{0mm}\label{fig:transport_bfactor_heatmap}
    \includegraphics[scale=1]{transport_bfactor_heatmap}
  \end{subfigure}
  %
  \\
  %
  \begin{subfigure}[t]{11.5cm}
    \centering
    \caption{}\vspace{0mm}\label{fig:transport_bfactor_scatter}
    \includegraphics[scale=1]{transport_bfactor_scatter}
  \end{subfigure}
  %
  \\
  %
  \begin{subfigure}[t]{11.5cm}
    \centering
    \hspace{-0.5cm}
    \caption{}\vspace{0mm}\label{fig:transport_bfactor_surface}
    \includegraphics[scale=1]{transport_bfactor_surface}
  \end{subfigure}
  %

  \caption[Per-residue B-factors for the last 10~ns of the {MD} run]%
  {%
    \textbf{Per-residue B-factors for the last 10~ns of the {MD} run.}
    %
    (\subref{fig:transport_bfactor_heatmap})
    %
    Heatmap of the per-residue B-factor ($\bfactor$, \cref{eq:bfactor}) of \gls{clya-as}, averaged over all 12
    monomers, as determined by our \gls{md} simulation.
    %
    (\subref{fig:transport_bfactor_scatter})
    %
    The data from the heatmap, averaged over all frames and plotted together with the B-factors given by the
    crystal (\pdbid{2WCD}~\cite{Mueller-2009} and \gls{cryo-em} \pdbid{6MRT}~\cite{Peng-2019}) structures of
    the wild-type \gls{clya} dodecamer.
    %
    (\subref{fig:transport_bfactor_surface})
    %
    Molecular surface plots of the interior (left) and exterior (right) walls of \gls{clya-as}, colored
    according to the chain-averaged per-residue B-factor. Images were rendered using
    \gls{vmd}~\cite{Humphrey-1996}.
    %
  }\label{fig:transport_bfactor}
\end{figure*}
%

\subsubsection{Computing an axially symmetric geometry from the atomic density.}
%

The 2D-axisymmetric geometry of the \gls{clya-as} nanopore (\cref{fig:transport_model_outline}) was derived
directly from its full atom model by means of the molecular density ($\rho_\text{mol}$). For each of the 50
structures mentioned above, we computed the 3D-dimensional molecular density map on a \SI{0.5}{\angstrom}
resolution grid using the Gaussian function~\cite{Li-2013}
%
\begin{align}\label{eq:denspore}
  \rho_\text{mol} = 1 - \displaystyle\prod_{i} \left[ 1 - 
    \exp{\left(-\dfrac{d_i^2}{(\stdev\atomradius_{i})^2}\right)} \right]
    \text{ ,}
\end{align}
%
with
%
\begin{align}
  d_i = \sqrt{(x-x_i)^2 + (y-y_i)^2 + (z-z_i)^2}
  \text{ ,}
\end{align}
%
where for each atom $i$, $R_i$ is its Van der Waals radius, $d_i$ is the distance of grid coordinates $(x, y,
z)$ from the atom center $(x_i, y_i, z_i)$ and $\stdev = 0.93$ is a width factor. After averaging all 50
density maps in 3D, the resulting map was then radially averaged along the $z$-axis, relative to the center of
the pore, to obtain a 2D-axisymmetric density map. The contour line at \SI{25}{\percent} density was used as
the nanopore simulation geometry, after manual removal of overlapping and superfluous vertices to improve the
quality of the final computational mesh. This 2D-axisymmetric geometry closely follows the outline of a
\SI{30}{\degree} `wedge' of the full atom structure (\cref{fig:transport_model_outline}).

\subsubsection{Computing an axially symmetric charge density from the atomic charge density.}
%

The 2D-axially symmetric charge distribution (\cref{fig:transport_model_scd}) was also derived directly from
the 50 sets of aligned nanopore coordinates that were used for the geometry. Inspired by how charges are
represented in the \gls{pme} method~\cite{Aksimentiev-2005}, we computed the fixed charge distribution of the
nanopore $\scdpore(r,z)$ by assuming that an atom $i$ of partial charge $\partialcharge_{i}$, at the location
$(x_i, y_i, z_i)$ in the full 3D atomistic pore model, contributes an amount $\partialcharge_{i}/2\pi r_i$ to
the partial charge at a point $(r_i,z_i)$ with $r_i = \sqrt{x_i^2 + y_i^2}$ in the averaged 2D-axisymmetric
model. This effectively spreads the charge over all angles to achieve axial symmetry. We assumed a Gaussian
distribution of the space charge density of each atom $i$ around its respective 2D-axisymmetric coordinates
$(r_i,z_i)$ such that
%
\begin{align}\label{eq:scdpore}
  \scdpore(r,z) = \dsum_{i} \dfrac{\echarge\partialcharge_{i}}{\pi(\stdev\atomradius_{i})^2}
            \exp{\left(-\dfrac{(r-r_i)^2 + (z-z_i)^2}{(\stdev\atomradius_{i})^2}\right)}
  \text{ ,}
\end{align}
%
where $\atomradius_{i}$ is the atom radius, $\sigma = 0.5$ is the sharpness factor and $\echarge$ is the
elementary charge. To embed $\scdpore$ with sufficient detail, yet efficiently, into a numeric solver, the
spatial coordinates were discretized with a grid spacing of \SI{0.005}{\nm} in the domain of $\scdpore$ and
precomputed values were used during the solver runtime. All partial charges (at \pH{7.5}) and radii were taken
from the CHARMM36 force field~\cite{Best-2012} and assigned using \code{PROPKA}~\cite{Olsson-2011} and
\code{PBD2PQR}~\cite{Jurrus-2018}. Within \code{COMSOL}, the surface charge density $\scdpore(r,z)$ was
imported as a 2D linear interpolation function and converted into a pseudo-3D volumetric charge density by
normalization over the local circumference of each element ($2 \pi r$). Integration of the resulting charge
density over the final mesh yielded a net charge of \SI{-72.9}{\ec}, which is indeed very close to the
\SI{-72}{\ec} of the original atomistic model. The equilibrium 2D-axisymmetric charge map
(\cref{fig:transport_model_scd}) was loaded directly into our solver as a linear interpolation function
($\scdpore$) and applied across all computational domains.



\subsection{On the extension to non-axisymmetric nanopores}
%

The possibility to use a computationally efficient 2D-axisymmetric model is enabled by \gls{clya}'s high
degree of rotational symmetry (\cref{fig:transport_clya_top}). To successfully extend our \gls{epnp-ns}
framework to non-axisymmetric pores---such as \gls{ompf}~\cite{Yamashita-2008} or
\gls{ompg}~\cite{Subbarao-2006}---and very narrow nanopores---such as \gls{ahl}~\cite{Song-1996},
\gls{frac}~\cite{Tanaka-2015}, or \gls{mspa}~\cite{Faller-2004}---these equations should be solved for a full
3D representation. For narrow pores the radial averaging of the geometry might remove features essential for
correctly modeling the fluidic properties, such as internal corrugations or opposing charges within the same
radial plane. Even though this will significantly increase the computational cost, it would still be
significantly faster than either \gls{bd} or \gls{md} approaches (\ie~hours per data point instead of days).

Describing a methodology for setting up a 3D model is beyond of the scope of this chapter. We expect it to
follow along the same lines as the approach outlined above, with the exception that the actual molecular
surface and charge distribution of the pore (see the methodology used by the
\gls{apbs}~\cite{Baker-2001,Jurrus-2018} for a potential approach) could be used instead of a radially
averaged one. The \gls{epnp-ns} equations themselves would require discretization in Cartesian ($x$,~$y$,~$z$)
rather than cylindrical ($r$,~$\phi$,~$z$) coordinates, but can otherwise remain unchanged.


\section{Results and discussion}
%
\label{sec:transport:results}
%

The \gls{iv} relationships of many nanopores, \gls{clya} included, often deviate significantly from Ohm's law.
This is because the ionic flux arises from a complex interplay between the pore's geometry (\eg~size, shape
and charge distribution), the properties of the surrounding electrolyte (\eg~salt concentration viscosity and
relative permittivity) and the externally applied conditions (\eg~bias voltage, temperature and pressure). The
ability of a computational model to quantitatively predict the ionic current of a nanopore over a wide range
of bias voltages and salt concentrations strongly indicates that it captures the essential physics governing
the nanofluidic transport. Hence, to validate our model, we experimentally measured the single channel ionic
conductance of \gls{clya} at a wide range of experimentally relevant salt concentrations ($\cbulk$) and bias
voltages ($\vbias$). We compared these experimental data with the simulated ionic transport properties in
terms of current, conductance, rectification and ion selectivity, of both the classical \gls{pnp-ns} and the
newly developed \gls{epnp-ns} equations---finding a quantitative match between the experiment and the
simulations.

%
\begin{figure*}[p]
  \centering
  %
  \includegraphics[scale=1]{transport_validation_ivs}
  %
  \caption[Meas. and sim. ionic current through single {ClyA-AS} nanopores]%
  {%
    %
    \textbf{Measured and simulated ionic current through single {ClyA-AS} nanopores.}
    %
    Comparison between the experimentally (expt.) measured, simple resistor pore model (bulk) and the
    simulated (\gls{pnp-ns} and \gls{epnp-ns}) \gls{iv} curves of \gls{clya-as} at \SI{25\pm1}{\celsius}
    between $\vbias=\mSI{\pm200}{\mV}$, and for $\cbulk=\mSIlist{0.05;0.15;0.5;1;3}{\Molar}$ \ce{NaCl}. The
    simulated currents were computed from the total ionic fluxes using \cref{eq:currentsim}. The bulk current
    was calculated by \cref{eq:bulk_nanopore_current} by modeling \gls{clya} as two series resistors
    (\cref{eq:bulk_nanopore_current}~\cite{Soskine-2013,Kowalczyk-2011}), using the bulk \ce{NaCl}
    conductivity at the given concentrations. The gray envelopes represent the experimental errors (standard
    deviation, $n=3$).
    %
  }\label{fig:transport_validation_ivs}
\end{figure*}
%

\subsection{Transport of ions through {ClyA}}
%
\label{sec:iont}
%

\subsubsection{Ionic current and conductance.}
%

The ability of our model to reproduce the ionic current of a biological nanopore over a wide range of
experimentally relevant conditions (between $\vbias = \mSIrange{-150}{+150}{\mV}$ and for $\cbulk =
\mSIlist{0.05;0.15;0.5;1;3}{\Molar}$ \ce{NaCl}) can be seen in \cref{fig:transport_validation_ivs}. Here, we
compare IV relationships of \gls{clya-as} as measured experimentally (`expt.'), simulated using our
2D-axisymmetric model (`\gls{pnp-ns}' and `\gls{epnp-ns}') and naively analytically estimated (`bulk') using a
resistor model of the pore (\cref{eq:bulk_nanopore_current})~\cite{Soskine-2013,Kowalczyk-2011}. The simulated
ionic current $\currentsim$ at steady-state was computed by
%
\begin{align}\label{eq:currentsim}
  \currentsim = \faraday\int_{S}\left(\dsum_{i}\chargen_{i}\normvec\cdot\vec{\flux}_{i}\right)dS
  \text{ ,}
\end{align}
%
with $\chargen_{i}$ the charge number and $\vec{\flux}_{i}$ the total flux of each ion $i$ across \cisi{}
reservoir boundary $S$, and $\normvec$ the unit vector normal to $S$. The current given by the bulk model is
%
\begin{align}\label{eq:bulk_nanopore_current}
  \current(\cbulk,\vbias) = 
      \dfrac{\sigma (\cbulk) \pi}{4}
      \left( 
          \dfrac{ l_{\text{cis}} }{ d_{\text{cis}}^2 } +
          \dfrac{ l_{\text{trans}} }{ d_{\text{trans}}^2 }
      \right)^{-1}
      \vbias
      \text{ ,}
\end{align}
%
with $\sigma$ the (concentration dependent) bulk electrolyte conductivity. The \cisi{} and \transi{} chambers
have have heights and diameters of $l_{\text{cis}}=\text{\SI{10}{\nm}}$ and
$l_{\text{trans}}=\text{\SI{4}{\nm}}$, and $d_{\text{cis}}=\text{\SI{6}{\nm}}$ and
$d_{\text{trans}}=\text{\SI{3.3}{\nm}}$, respectively~\cite{Soskine-2013}.


%
\begin{figure*}[p]
  \centering

  \begin{minipage}[t]{3.75cm}
    %
    \begin{subfigure}[t]{3.75cm}
      \centering
      \caption{}\vspace{0mm}\label{fig:transport_ioncurrent_conductance_heatmap}
      \includegraphics[scale=1]{transport_ioncurrent_conductance_heatmap}
    \end{subfigure}
    %
    \\
    %
    \begin{subfigure}[t]{3.75cm}
      \centering
      \caption{}\vspace{0mm}\label{fig:transport_ioncurrent_conductance_concentration}
      \includegraphics[scale=1]{transport_ioncurrent_conductance_concentration}
    \end{subfigure}
    %
  \end{minipage}
  %
  \hspace{-2.5mm}
  %
  \begin{minipage}[t]{3.75cm}
    %
    \begin{subfigure}[t]{3.75cm}
      \centering
      \caption{}\vspace{0mm}\label{fig:transport_ioncurrent_icr_heatmap}
      \includegraphics[scale=1]{transport_ioncurrent_icr_heatmap}
    \end{subfigure}
    %
    \\
    %
    \begin{subfigure}[t]{3.75cm}
      \centering
      \caption{}\vspace{0mm}\label{fig:transport_ioncurrent_icr_concentration}
      \includegraphics[scale=1]{transport_ioncurrent_icr_concentration}
    \end{subfigure}
    %
  \end{minipage}
  %
  \hspace{-2.5mm}
  %
  \begin{minipage}[t]{3.75cm}
    %
    \begin{subfigure}[t]{3.75cm}
      \centering
      \caption{}\vspace{0mm}\label{fig:transport_ioncurrent_tsodium_heatmap}
      \includegraphics[scale=1]{transport_ioncurrent_tsodium_heatmap}
    \end{subfigure}
    %
    \\
    %
    \begin{subfigure}[t]{3.75cm}
      \centering
      \caption{}\vspace{0mm}\label{fig:transport_ioncurrent_tsodium_concentration}
      \includegraphics[scale=1]{transport_ioncurrent_tsodium_concentration}
    \end{subfigure}
    %
  \end{minipage}

  \caption[Meas. and sim. ionic cond., rectification, and \ce{Na+} sel. of single {ClyA-AS} nanopores.]%
  {%
    %
    \textbf{Measured and simulated ionic conductance, rectification, and cation selectivity of single {ClyA-AS}
    nanopores.}
    %
    (\subref{fig:transport_ioncurrent_conductance_heatmap})
    %
    Contour plot of the simulated (\gls{epnp-ns}) ionic conductance $\conductance = \current /\vbias$ as a
    function of $\vbias$ and $\cbulk$.
    %
    (\subref{fig:transport_ioncurrent_conductance_concentration})
    %
    Log-log plots of $\conductance$ as a function of $\cbulk$ at \SI{+150}{\mV} (top) and \SI{-150}{\mV}
    (bottom)---comparing results obtained through experiments, \gls{pnp-ns} and \gls{epnp-ns} simulations, and
    the simple resistor pore model. The gray envelopes represent the experimental errors (standard deviation,
    $n=3$).
    %
    (\subref{fig:transport_ioncurrent_icr_heatmap})
    %
    Contour plot of the ionic current rectification $\icr = \conductance(+\vbias)/\conductance (-\vbias)$,
    computed from the ionic conductances in the \gls{epnp-ns} simulation, as a function of $\vbias$ and
    $\cbulk$.
    %
    (\subref{fig:transport_ioncurrent_icr_concentration})
    %
    Comparison between the simulated ('ePNP-NS' and 'PNP-NS') and measured (expt.) $\icr$ as a function of
    $\cbulk$ for \SIlist{50;150}{\mV}.
    %
    (\subref{fig:transport_ioncurrent_tsodium_heatmap})
    %
    Contour plot of the \Na{} transport number $\tna = \gna / \conductance$, computed from the individual
    ionic conductances in the \gls{epnp-ns} simulation, as a function of $\vbias$ and $\cbulk$. The $\tna$
    expresses the fraction of the ionic current is carried by \Na{} ions (\ie~the cation selectivity).
    %
    (\subref{fig:transport_ioncurrent_tsodium_concentration})
    %
    Simulated (\gls{pnp-ns} and \gls{epnp-ns}) values of $\tna$ as a function of $\cbulk$ for \SI{+150}{\mV}
    (top) and \SI{-150}{\mV} (bottom). Here, the `bulk' line indicates the bulk \ce{NaCl} cation transport
    number, represented by its empirical function $\tna(\cbulk)$ (see~\cref{tab:corrections_equations} and
    \cref{tab:corrections_parameters}). The solid gray line represents $\tna=0.5$.
    %
  }\label{fig:transport_ioncurrent}
\end{figure*}
%


Whereas the classical \gls{pnp-ns} equations consistently overestimated the ionic current, particularly at
high salt concentrations, the predictions of the \gls{epnp-ns} equations corresponded closely to the measured
values, especially at high ionic strengths ($\cbulk > \mSI{0.5}{\Molar}$). The inability of the classical
\gls{pnp-ns} equations to correctly estimate the current is expected however, as in this regime the model
parameters (\eg~diffusivity, mobility and viscosity, \ldots) already deviate significantly from their
`infinite dilution' values (see~\cref{sec:epnp-ns:parameterization},
\cref{fig:epnpns_concentration_d,fig:epnpns_concentration_mu,fig:epnpns_concentration_solv,fig:epnpns_distance}).
At $\cbulk < \mSI{0.15}{\Molar}$ the \gls{epnp-ns} equations tended to minorly overestimate the ionic current,
but the discrepancies were much smaller than those observed for \gls{pnp-ns}. Further, these ionic strengths
are not usually tested experimentally. Finally, the bulk model managed to capture the currents surprisingly
well at high salt concentrations and positive bias voltages, indicating that, under these conditions, the
distribution of ions inside the pore is similar to the bulk electrolyte. In contrast, this simple model
faltered in the negative voltage regime, indicating that the resistance of the pore is not only determined by
its geometry (\ie~the diameter and length of the `resistor'), but that it strongly modulated by its
electrostatic properties.

The ability of a nanopore to conduct ions can be best expressed by its conductance: $\conductance = \current /
\vbias$. We computed \gls{clya}'s conductance with the \gls{epnp-ns} equations as a function of bias voltage
($\vbias = \mSIrange{-200}{+200}{\mV}$) and bulk \ce{NaCl} concentration ($\cbulk =
\mSIrange{0.005}{5}{\Molar}$), of which a contour plot can be found in
\cref{fig:transport_ioncurrent_conductance_heatmap}. The near horizontal contour lines in the upper part of
the plot show that, at high ionic strengths ($\cbulk > \mSI{1}{\Molar}$), \gls{clya} maintains the same
conductance regardless of the applied bias voltage. This behavior changes at intermediate concentrations
($\mSI{0.1}{\Molar} < \cbulk < \mSI{1}{\Molar}$, typical experimental conditions), where maintaining the same
conductance level with increasing negative bias amplitudes requires increasing salt concentrations. Finally,
at low salt concentrations ($\cbulk < \mSI{0.1}{\Molar}$), the ionic conductance increases when reducing the
negative voltage amplitude but subsequently levels out at positive bias voltages.

The cross-sections of the ionic conductance as a function of concentration at high positive and negative bias
voltages (\cref{fig:transport_ioncurrent_conductance_concentration}), serve to demonstrate the differences
between these respective regimes. At high positive and negative bias voltages ($\vbias=\mSI{\pm150}{\mV}$),
the slopes of the conductance log-log plots with respect to the bulk salt concentration show linear and
bi-linear behavior, respectively. This could be indicative of a different mode of ion conduction of positive
and negative bias voltages, at least for low concentrations ($\cbulk<\mSI{0.15}{\Molar}$). Even though the
bulk model and the \gls{pnp-ns} equations manage to capture the conductance at respectively high and low ionic
strengths, only the \gls{epnp-ns} equations perform well over the entire concentration range, particularly at
experimentally relevant concentrations (\SIrange{0.1}{2}{\Molar})~\cite{Willems-Ruic-Biesemans-2019,
Galenkamp-2020,Franceschini-2013,Franceschini-2016}. Overall, the predictions made using \gls{pnp-ns}
overestimate the conductance over the entire concentration range, but they do converge with those computed
with \gls{epnp-ns} when approaching infinite dilution ($\cbulk<\mSI{0.01}{\Molar}$). A more in-depth
discussion on the effect of the concentration, wall distance and steric corrections on the ionic conductance
can be found in the in~\cref{sec:transport:comparison_corrections} (\cref{fig:transport_comparison_ionic}).

The difference in ionic conduction at opposing bias voltages is also known as the ionic current rectification
(\cref{fig:transport_ioncurrent_icr_heatmap,fig:transport_ioncurrent_icr_concentration}), given by $\icr$:
%
\begin{align}\label{eq:icr}
  \icr(\vbias) = \conductance(+\vbias) / \conductance(-\vbias)
  \text{ ,}
\end{align}
%
with $\conductance(+\vbias)$ and $\conductance(-\vbias)$ the conductance of the pore at opposing bias voltages
of the same magnitude. The ionic current rectification is a phenomenon often observed in nanopores that are
both charged, and contain a degree of geometrical asymmetry along the central axis of the
pore~\cite{Constantin-2007,White-2008,Wang-2014}. The latter can be either intrinsic to geometry of nanopore,
or due to deformations in response to the electric field (\ie~gating or electrostriction). As can be seen in
\cref{fig:transport_ioncurrent_icr_heatmap}, \gls{clya} exhibits a strong degree of rectification, which is to
be expected given its predominantly negatively charged interior and asymmetric \cisi{} (\SI{\approx6}{\nm})
and \transi{} (\SI{\approx3.3}{\nm}) entry diameters (\cref{fig:transport_clya_side}). We found $\icr$ to
increase monotonously with the bias voltage magnitude, at least over the investigated range
(\cref{fig:transport_ioncurrent_icr_concentration}). We found the dependence of $\icr$ on the ionic strength
not to be monotonous, but rather rising rapidly to a peak value at $\cbulk=\mSI{0.15}{\Molar}$, followed by a
gradual decline towards unity at saturating salt concentrations. This concentration is within the transition
zone observed in the conductance at negative bias voltages
(\cref{fig:transport_ioncurrent_conductance_concentration}) and provides further evidence for a change in the
conductive properties of the pore in this regime. Even though the values produced by the \gls{epnp-ns}
simulations do not fully match quantitatively to the experimental results, at least for $\cbulk <
\text{\mSI{0.5}{\Molar}}$, they do (1) reproduce the observed trends qualitatively and (2) exhibit much
smaller errors relative to the \gls{pnp-ns} results (cf.~green and purple lines in
\cref{fig:transport_ioncurrent_icr_concentration}).

The results and comparisons discussed above indicate that \gls{clya}'s conductivity is dominated by the bulk
electrolyte conductivity above physiological salt concentrations ($\cbulk > \mSI{0.15}{\Molar}$). The
breakdown of this simple dependency at lower ionic strengths is particularly evident at negative bias voltages
and is likely caused by the overlapping of the electrical double layer (EDL) inside the pore (\ie~the Debye
length is \SI{\approx1.4}{\nm} at $\cbulk=\mSI{0.05}{\Molar}$). This effectively excludes the co-ions (\Cl)
from the interior of the pore and attracts as many counter-ions (\Na) as needed to screen the fixed charges of
the pore. As a result, \Cl{} ions do not contribute to ionic current and the conductance is dominated by \Na{}
ions attracted by \gls{clya}'s `surface' charges~\cite{Uematsu-2018}. The presence of only a single ion type
inside the pore at low ionic strength may also offer an explanation as to why the \gls{epnp-ns} equations are
more accurate at higher ionic strengths ($\cbulk\ge\mSI{0.15}{\Molar}$). Because our ionic mobilities are
derived from bulk ionic conductances (\ie~for unconfined ions in a locally electroneutral environment) it is
likely that our mobility model begins to break down under conditions where only a single ion type is
present~\cite{Duan-2010}. Another cause of the discrepancies could be a slight narrowing of the nanopore at
low salt concentrations, which cannot be captured by our simulation due to the static nature of its geometry
and charge distributions. Nevertheless, our simplified 2D-axisymmetric model, in conjunction with the
\gls{epnp-ns} equations, is able to accurately predict the ionic current flowing through \gls{clya} for a wide
range of experimentally relevant ionic strengths and bias voltages. This suggests that our continuum system
can accurately capture the essential physical phenomena that drive the ion and water transport through the
nanopore both \emph{qualitatively} and \emph{quantitatively}. Hence, we expect the distribution of the
resulting properties (\eg~ion concentrations, ion fluxes, electric field and water velocity) to closely
correspond to their true values.

\subsubsection{Cation selectivity.}
%
The ion selectivity of a nanopore determines the preference with which it transports one ion type over the
other. Experimentally, it is often determined by placing the pore in a salt gradient (\ie~different salt
concentrations in the \cisi{} and \transi{} reservoirs) and measuring the reversal potential ($\revpot$),
\ie~the bias voltage at which the nanopore current is zero~\cite{Soskine-2013,Franceschini-2016}. The
\gls{ghk} equation can then be used to convert $\revpot$ into the permeability ratio $\pna = \gna / \gcl$.
Here, we represent the \gls{clya}'s ion selectivity
%
(\cref{fig:transport_ioncurrent_tsodium_heatmap,fig:transport_ioncurrent_tsodium_concentration}) 
%
by the fraction of the total current that is carried by \Na{} ions: the apparent \Na{} transport number
%
\begin{align}\label{eq:tna}
  \tna = \dfrac{\pna}{\pna + 1} = \dfrac{\gna}{\gna + \gcl}
  \text{ ,}
\end{align}
%
with $\pna$ the cation permeability ratio, and $\gna$ and $\gcl$ the cation and anion contributions to the
total conductance. As expected from its negatively charged interior, we found \gls{clya} to be cation
selective (\ie~$\tna > 0.5$) for all investigated voltages up to a bulk salt concentration of $\cbulk \approx
\mSI{2}{\Molar}$ \ce{NaCl} (0.5 contour line in \cref{fig:transport_ioncurrent_tsodium_heatmap}). Above this
concentration, $\tna$ falls to a minimum of value of 0.45 at $\cbulk\approx\mSI{5}{\Molar}$, which is still
\num{\approx1.27} times its bulk electrolyte value of 0.35. This shows that---even at saturating
concentrations where the Debye length is \SI{<0.2}{\nm}---\gls{clya} enhances the transport of cations. Below
\SI{2}{\Molar}, the ion selectivity increases logarithmically with decreasing salt concentrations, but it also
becomes more sensitive to the direction and magnitude of the electric field:  with negative bias voltages
yielding higher ion selectivities (\cref{fig:transport_ioncurrent_tsodium_concentration}). For example, to
reach a selectivity of $\tna \approx 0.9$, the salt concentration must fall to \SI{0.05}{\Molar} at
\SI{+150}{\mV}, but only to \SI{0.125}{\Molar} at \SI{-150}{\mV}.

Using the reversal potential method, Franceschini \etal{}~\cite{Franceschini-2016} found \gls{clya}'s ion
selectivity to be $\tna=0.66$ ($\pna=1.9$). This corresponds well to the average between the \cisi{}
($\cbulk=\mSI{1}{\Molar}$, $\tna=0.57$, $\pna=1.3$) and \transi{} ($\cbulk=\mSI{0.15}{\Molar}$, $\tna=0.84$,
$\pna=5.4$) reservoir concentrations used in their experiment. Therefore, this suggests that although
measuring the reversal potential gives valuable insights into the selectivity ion channels and small
nanopores, it does not describe the ion selectivity under symmetric conditions. In addition, the \gls{ghk}
equation does not consider the ionic flux due to the electro-osmotic flow and assumes that the Nernst-Einstein
relation holds for all used concentrations. These two effects should not be ignored as they contribute
significantly to the total conductance of the pore. Furthermore, because the ion selectivity depends strongly
on the ionic strength and often the applied bias voltage, the measured reversal potential will necessarily be
influenced by the chosen salt gradient and represents the selectivity at an undetermined intermediate
concentration. Tabulated data of the ion selectivity for selected voltages and concentrations can be found in
\cref{tab:ion_selectivities}.


\subsection{Ion concentration distribution}\label{sec:ionc}
%

%
\begin{figure*}[p]
  \centering

  %
  \begin{subfigure}[t]{9cm}
    \centering
    \caption{}\vspace{-3mm}\label{fig:transport_concentration_average}
    \includegraphics[scale=1]{transport_concentration_average}
  \end{subfigure}
  %
  \\
  %
  \begin{subfigure}[t]{11.5cm}
    \centering
    \caption{}\vspace{-3mm}\label{fig:transport_concentration_contours}
    \includegraphics[scale=1]{transport_concentration_contours}
  \end{subfigure}
  %
  \\
  %
  \begin{subfigure}[t]{11.5cm}
    \centering
    \caption{}\vspace{-3mm}\label{fig:transport_concentration_profiles}
    \includegraphics[scale=1]{transport_concentration_profiles}
  \end{subfigure}
  %

  \caption[Ion concentration distribution inside {ClyA-AS}]%
  {%
    %
    \textbf{Ion concentration distribution inside {ClyA-AS}.}
    %
    (\subref{fig:transport_concentration_average})
    %
    Relative \Na{} and \Cl{} concentrations averaged over the entire pore volume ($\pavi$,
    \cref{eq:pore_surface_integral}) as a function of the reservoir salt concentration ($\cbulk =
    \mSIrange{0.005}{5}{\Molar}$) and bias voltage ($\vbias = \mSIrange{-200}{+200}{\mV}$).
    %
    (\subref{fig:transport_concentration_contours})
    %
    Contour plots of the relative ion concentration ($\ci/\cbulk$) for both \Na{} and \Cl{} for $\cbulk =
    \mSI{0.15}{\Molar}$ and at $\vbias = \mSIlist{-150;+150}{\mV}$.
    %
    (\subref{fig:transport_concentration_profiles})
    %
    The relative \Na{} and \Cl{} concentration profiles along the radius of the pore, through the middle of
    the constriction ($z = \mSI{-0.3}{\nm}$) and the \lumen{} ($z = \mSI{5}{\nm}$), as indicated by the arrows
    in (\subref{fig:transport_concentration_contours}).
    %
    }\label{fig:transport_concentration}
\end{figure*}
%


Following the validation of the model in previous section, we now proceed by describing the local ionic
concentrations inside \gls{clya}. Detailed knowledge of the ionic environment can be valuable to
experimentalists who seek to trap and study single enzymes with
\gls{clya}~\cite{Soskine-Biesemans-2015,VanMeervelt-2017,Galenkamp-2018}. Moreover, it gives insight into the
origin of the ion current rectification, ion selectivity and the electro-osmotic flow. Note that the figures
below were obtained from a nanoscale continuum steady-state simulation, they represent a time-averaged
situation on the order of \SIrange{10}{100}{\ns}~\cite{Im-2002}.


\subsubsection{Relative cation and anion concentrations.}
%

We use the relative ion concentration averaged over the total pore (`PT') volume ($\pavi$), as a measure for
global ionic conditions inside the pore (\cref{fig:transport_concentration_average}). At low ionic strengths
($\cbulk < \mSI{0.05}{\Molar}$), our simulation predicts a strong enhancement of the \Na{} concentration
($\pavNa$) and a clear depletion of the \Cl{} concentration ($\pavCl$) inside the pore relative to the
reservoir concentration, irrespective of the bias voltage. This effect diminishes rapidly with increasing
ionic strengths, which can be explained by the electrolytic screening of the negative charges lining the walls
of \gls{clya} (\ie~the electrical double layer). At low reservoir concentrations ($\cbulk <
\mSI{0.05}{\Molar}$) the number of ions in the bulk is sparse, leading to the attraction and repulsion of
respectively as many \Na{} and \Cl{} ions as the chemical potential allows. As the concentration increases,
the overall availability of ions improves and the extreme concentration differences between the pore and the
bulk are no longer required to offset the fixed charges lining \gls{clya}'s interior walls. For example,
increasing the reservoir concentration at equilibrium ($\vbias = \mSI{0}{\mV}$) from
\SIrange{0.005}{0.05}{\Molar} causes $\pavNa$ to fall an order of magnitude (\numrange{34}{4.4}) and $\pavCl$
to rise an order of magnitude (\numrange{0.05}{0.31}). Even though  at physiological ionic strength ($\cbulk =
\mSI{0.15}{\Molar}$) their concentrations still differ significantly from those in the reservoir ($\pavNa =
2.1$ and $\pavCl = 0.58$), they do approach bulk-like values ($1.14 \ge \pavNa \ge 1$ and $0.89 \le \pavCl \le
1$) at higher concentrations ($\cbulk \ge \mSI{1}{\Molar}$).

We also observed a significant difference between their sensitivities to the applied bias voltage,
particularly at low salt concentrations (\cref{fig:transport_concentration_average}, left of
\SI{<0.15}{\Molar} line). Whereas the \Na{} concentration shows only a limited response, the \Cl{}
concentration changes much more dramatically. For example, at $\cbulk = \mSI{0.15}{\Molar}$ and when changing
the bias voltage from \SIrange{-150}{+150}{\mV}, $\pavNa$ rises \num{\approx 1.7}-fold (\numrange{1.6}{2.7})
and $\pavCl$ increases \num{\approx 3.8}-fold (\numrange{0.28}{1.06}). This difference is clearly visualized
by the contour plots of the relative ion concentrations ($\ci/\cbulk$) at $\cbulk = \mSI{0.15}{\Molar}$ and
for $\vbias = \mSIlist{-150;+150}{\mV}$ (\cref{fig:transport_concentration_contours}). They reveal that the
\transi{} constriction ($\num{-1.85} < z < \mSI{1.6}{\nm}$) remains depleted of \Cl{} and enhanced in \Na{}
for both $\vbias = \mSI{-150}{\mV}$ and $\vbias = \mSI{+150}{\mV}$. This is not the case in the \lumen{}
($\num{1.6} < z < \mSI{12.25}{\nm}$), in which the \Na{} concentration is bulk-like for $\vbias <
\mSI{0}{\mV}$ and enhanced for $\vbias > \mSI{0}{\mV}$. Conversely, the number of \Cl{} ions becomes more and
more depleted in the \lumen{} for increasing negative bias magnitudes, and it is virtually bulk-like at higher
positive bias voltages. This is further exemplified by the radial profiles of the ion concentrations
(\cref{fig:transport_concentration_profiles}) through the middle of the constriction ($z = \mSI{-0.3}{\nm}$)
and the \lumen{} ($ z = \mSI{5}{\nm}$), which also clearly shows the extent of the electrical double layer.

%
\begin{figure*}[p]
  \centering

  %
  \begin{subfigure}[t]{9cm}
    \centering
    \caption{}\vspace{-3mm}\label{fig:transport_charge_total}
    \includegraphics[scale=1]{transport_charge_total}
  \end{subfigure}
  %
  \\
  %
  \begin{subfigure}[t]{11.5cm}
    \centering
    \caption{}\vspace{-3mm}\label{fig:transport_charge_contours}
    \includegraphics[scale=1]{transport_charge_contours}
  \end{subfigure}
  %
  \\
  %
  \begin{subfigure}[t]{11.5cm}
    \centering
    \caption{}\vspace{-3mm}\label{fig:transport_charge_profiles}
    \includegraphics[scale=1]{transport_charge_profiles}
  \end{subfigure}
  %

  \caption[Ion charge density distribution inside {ClyA-AS}]%
  {%
    %
    \textbf{Ion charge density distribution inside {ClyA-AS}.}
    %
    (\subref{fig:transport_charge_total})
    %
    The average number of ionic charges inside the pore $\pav{\Qion}{\text{PT}}$, is distributed between those
    close to the pore's surface $\pav{\Qion}{\text{PS}}$ (\ie~within \SI{0.5}{\nm} of the wall), and those in
    the `bulk' of the pore's interior $\pav{\Qion}{\text{PB}}$.
    %
    (\subref{fig:transport_charge_contours})
    %
    Cross-section contour plots of the ion space charge density ($\scdion$), expressed as number of elementary
    charges per \si{\cubic\nano\meter}, at $\vbias = \mSI{0}{\mV}$ and for $\cbulk =
    \mSIlist{0.05;0.15;0.5;5}{\Molar}$.
    %
    (\subref{fig:transport_charge_profiles})
    %
    Radial cross-sections of the $\scdion$ at the center of the constriction ($z = \mSI{-0.3}{\nm}$) and the
    \lumen{} ($z = \mSI{5}{\nm}$) of \gls{clya}. The vertical line represents the division between ions in the
    `bulk' ($\walldistance > \mSI{0.5}{\nm}$) of the pore and those located near its surface ($\walldistance
    \le \mSI{0.5}{\nm}$).
    %
  }\label{fig:transport_charge}
\end{figure*}
%

\subsubsection{Ion charge density.}
%

The formation of an electrical double layer inside the pore, and the resulting asymmetry in the cation and
anion concentrations, gives rise to a net charge density inside the pore ($\scdion$, \cref{eq:scdion}). To
investigate the distribution of these charges within \gls{clya}, we divided the total interior volume of the
pore into a `pore surface' (PS) and a `pore bulk' (PB) region. The PB region encompasses a cylindrical volume
at the entry of the pore up until \SI{0.5}{\nm} from the wall ($\walldistance \ge \mSI{0.5}{\nm}$), the
approximate distance from which the wall begins to exert a significant influence on the properties of the
electrolyte (\cref{fig:epnpns_distance}). The PS region includes the remaining volume between the PB domain
and the nanopore wall ($\walldistance < \mSI{0.5}{\nm}$). Integration of $\scdion$ over the PS and PB regions
yields the average number of mobile charges present inside those locations
(\cref{fig:transport_charge_total}): $\pav{\Qion}{\text{PB}}$ and $\pav{\Qion}{\text{PS}}$, respectively.
Although the total number of mobile charges inside the pore, $\pav{\Qion}{\text{PT}} = \pav{\Qion}{\text{PB}}
+ \pav{\Qion}{\text{PS}}$, rises appreciatively with increasing reservoir concentrations, the majority of
these additional charges are confined to the walls of the pore. Up until a reservoir concentration
$\cbulk=\mSI{\approx0.15}{\Molar}$, we found $\pav{\Qion}{\text{PT}}$ to be distributed equally between the
surface (\SI{\approx+27}{\ec}) and bulk (\SI{\approx+22}{\ec}) layers. At high salt concentrations ($\cbulk >
\mSI{1}{\Molar}$), the number of charges in the PS region more than doubles (towards
$\pav{\Qion}{\text{PS}}=\mSI{+58}{\ec}$ at \SI{5}{\Molar}), and those in the PB region diminish (towards
$\pav{\Qion}{\text{PB}} \approx \mSI{0}{\ec}$ at \SI{5}{\Molar}). The bias voltage also influences the total
number of mobile charges in the pore. As can be seen from our simulation results for three different voltages
(\cref{fig:transport_charge_total}), $\pav{\Qion}{\text{PT}}$ is approximately \SIrange{+10}{+15}{\ec} higher
at $\vbias = \mSI{+150}{\mV}$ as compared to $\vbias = \mSI{-150}{\mV}$ for the full range of ion
concentrations. Interestingly, at reservoir concentration \SI{>0.15}{\Molar}, $\pav{\Qion}{\text{PB}}$ becomes
independent of applied voltage and the changes in $\pav{\Qion}{\text{PT}}$ can be attributed to
$\pav{\Qion}{\text{PS}}$.

The cross-section contour plots of $\scdion$ inside \gls{clya} for four different bulk concentrations ($\cbulk
= \mSIlist{0.05;0.15;0.5;5}{\Molar}$) reveal the redistribution of the mobile charges with increasing ionic
strength in more detail. Up until a bulk concentration of $\cbulk = \mSI{\le0.5}{\Molar}$, the EDL inside the
pore overlaps significantly with itself, as evidenced by the net positive charge density found throughout the
interior of the pore (\cref{fig:transport_charge_contours}). Moreover, the absence of \Cl{} ions effectively
prevents the formation of a negatively charged EDL next to the few positively charged residues lining the pore
walls. The situation at high salt concentrations (\eg~\SI{5}{\Molar}) is very different, with almost no charge
density within the PB region of the pore ($\walldistance \ge \mSI{0.5}{\nm}$), but with pockets of highly
charged and alternating positive and negative charge densities close to the nanopore wall
(\cref{fig:transport_charge_contours}, rightmost panel). This sharp confinement is shown clearly by the radial
density profiles (\cref{fig:transport_charge_profiles}) drawn through the constriction ($z = \mSI{-0.3}{\nm}$,
purple triangles) and the \lumen{} ($z = \mSI{5}{\nm}$, green triangles).

It is well known that the activity of an enzyme depends on the composition of the electrolyte that surrounds
it~\cite{Purich-2010-7}. Hence, we expect that the interpretation of kinetic data obtained from enzymes
trapped inside the nanopore~\cite{VanMeervelt-2017,Galenkamp-2018} will benefit from the precise
quantification of the ionic conditions inside the pore, including the concentration difference with the
reservoir but also the significant imbalance between cations and anions~\cite{Warren-1966}.


\subsection{Electrostatic potential and energy}\label{sect:esp}
%

%
\begin{figure*}[p]
  \centering

  %
  \begin{subfigure}[t]{2.5cm}
    \centering
    \caption{}\vspace{-3mm}\label{fig:transport_potential_residues}
    \includegraphics[scale=1]{transport_potential_residues}
  \end{subfigure}
  %
  \hspace{-0.5cm}
  %
  \begin{subfigure}[t]{9cm}
    \centering
    \caption{}\vspace{-3mm}\label{fig:transport_potential_contours}
    \includegraphics[scale=1]{transport_potential_contours}
  \end{subfigure}
  %
  \\
  %
  \begin{subfigure}[t]{11cm}
    \centering
    \caption{}\vspace{-3mm}\label{fig:transport_potential_profiles}
    \includegraphics[scale=1]{transport_potential_profiles}
  \end{subfigure}

  \caption[Equilibrium electrostatic potential inside {ClyA-AS}]%
  {%
    %
    \textbf{Equilibrium electrostatic potential inside {ClyA-AS}}.
    %
    (\subref{fig:transport_potential_residues})
    %
    A single subunit of \gls{clya} in which all amino acids with a net charge and whose side chains face the
    inside of the pore (\ie~those contribute the most to the electrostatic potential) are highlighted.
    Negatively (Asp+Glu) and positively and positively (Lys+Arg) charged residues are colored in red and blue,
    respectively.
    %
    (\subref{fig:transport_potential_contours})
    %
    As a result of these fixed charges \gls{clya} exhibits a complex electrostatic potential ($\potential$)
    landscape at equilibrium (\ie~at $\vbias = \mSI{0}{\mV}$), and whose values inside the pore we have
    plotted for several key concentrations ($\cbulk = \mSIlist{0.05;0.15;0.5;5}{\Molar}$). Note that even at
    physiological salt concentrations ($\cbulk = \mSI{0.15}{\Molar}$), the negative electrostatic potential
    extends significantly inside the \lumen{} ($\num{1.6} < z < \mSI{12.25}{\nm}$), and even more so inside
    the \transi{} constriction ($\num{1.85} < z < \mSI{1.6}{\nm}$). For the former, localized influential
    negative `hotspots' can be found in the middle ($\num{4} < z < \mSI{6}{\nm}$) and at the \cisi{} entry
    ($\num{10} < z < \mSI{12}{\nm}$).
    %
    (\subref{fig:transport_potential_profiles})
    %
    Radial average of the equilibrium electrostatic potential along the length of the pore ($\radpot$) for the
    same concentrations as in (\subref{fig:transport_potential_contours}). Even though the \lumen{} of the
    pore is almost fully screened for $\cbulk > \mSI{0.5}{\Molar}$, the constriction still retains some of its
    negative influence even at \SI{5}{\Molar}.
    %
  }\label{fig:transport_potential}
\end{figure*}
%

The electrostatic potential, or rather the spatial change thereof in the form of an electric field, is one of
the primary driving forces within a nanopore. Typically, the potential can be split into an external,
`non-equilibrium' contribution, resulting from the bias voltage applied between the \transi{} and the \cisi{}
reservoirs, and an intrinsic, `equilibrium' component, caused by the fixed charge distribution of the
pore~\cite{Willems-Ruic-Biesemans-2019}. To accurately describe and understand the nanopore transport
processes both contributions to the net electric field inside the pore are essential, as their relative
magnitudes and directions can significantly influence the transport of
ions~\cite{Aksimentiev-2005,Bhattacharya-2011,DeBiase-2015,Basdevant-2019}, water
molecules~\cite{Laohakunakorn-2015,Bhadauria-2017}, and
biopolymers~\cite{Buchsbaum-2013,Muthukumar-2014,Willems-Ruic-Biesemans-2019}.


\subsubsection{A few important charged residues.}
%

The interior walls of the \gls{clya} nanopore (\cref{fig:transport_potential_residues}) are riddled with
negatively charged amino acids (\ie~aspartate or glutamate), interspaced by a few positively charged residues
(\ie~lysine or arginine). When grouping these charges by proximity, we found three clusters with significantly
more negative than positive residues: inside the \transi{} constriction ($-1.85 < z < \mSI{1.6}{\nm}$; E7,
E11, K14, E18, D21, D25), in the middle of the \cisi{} \lumen{} ($4 < z < \mSI{6}{\nm}$; E53, E57, D64, K147)
and at the top of the pore ($10 < z < \mSI{12}{\nm}$; D114, R118, D121, D122). As we shall see, these clusters
leave strong negative fingerprints in the global electrostatic potential.

\subsubsection{Distribution of the equilibrium electrostatic potential.}
%

The electrostatic potential at equilibrium ($\eqpot$, \ie~at $\vbias = \mSI{0}{\mV}$) reveals the effect of
\gls{clya}'s fixed charges on the potential inside the pore (\cref{fig:transport_potential_contours}). Due to
electric screening by the mobile charge carriers in the electrolyte, however, the extent of their influence
strongly depends on the bulk ionic strength. The contour plot cross-sections of $\eqpot$ for $\cbulk =
\mSIlist{0.05;0.15;0.5;5}{\Molar}$ (\cref{fig:transport_potential_contours}) and their corresponding radial
averages (\cref{fig:transport_potential_profiles}) demonstrate this effect aptly. The radial average
($\radpot$) represents the mean value along the longitudinal axis of the pore and can be computed using
%
\begin{align}\label{eq:epnp-ns_radpot}
  \radpot = \dfrac{1}{\pi R(z)^2}\int_{0}^{R(z)}\potential(r,z) \;2 \pi r \; dr
  \text{ ,}
\end{align}
%
where $R(z)$ is taken as the \gls{clya}'s radius inside the pore ($-1.85 \le z \le \mSI{12.25}{\nm}$), and as
fixed values of \SI{4}{\nm} and \SI{2}{\nm} inside the \cisi{} ($z > \mSI{12.25}{\nm}$) and \transi{} (($z <
\mSI{-1.85}{\nm}$) reservoirs, respectively. Starting from the \cisi{} entry ($z \approx \SI{10}{\nm}$), the
electrostatic potential is dominated by the acidic residues D114, D121 and D122, resulting in a rapid
reduction of $\radeqpot$ upon entering the pore. Next, $\radeqpot$ slowly decreases up until the middle of the
\lumen{} ($z \approx \mSI{5}{\nm}$), where the next set of negative residues, namely E53, E57 and D64, lower
it even further. After a brief increase, $\radeqpot$ attains its maximum amplitude inside the \transi{}
constriction ($z \approx \mSI{0}{\nm}$) due to the close proximity of the amino acids E7, E11, E18, D21 and
D25, and then quickly falls to \num{0} inside the \transi{} reservoir.

At low ionic strengths ($\cbulk < \mSI{0.05}{\Molar}$), the lack of sufficient ionic screening results in
relatively high negative potentials throughout the entire pore. For example, at low concentrations ($\cbulk =
\mSI{0.05}{\Molar}$), the $\radeqpot$ inside the constriction ramps up to a value of \SI{-86}{\mV}
(\SI{-3.35}{\kTe}), which significantly exceeds the single ion thermal voltage $\si{\kTe} = \mSI{25.7}{\mV}$.
Hence, on the one hand they prohibit anions such as \Cl{} from entering the pore, and on the other they
attract cations such as \Na{} and trap them inside the pore. For intermediate concentrations ($0.05 \le \cbulk
< \mSI{0.5}{\Molar}$) the influence of the negative charges becomes increasingly confined to several
`hotspots' near the nanopore walls, most notably at entry of the pore ($10 < z < \mSI{12}{\nm}$), in the
middle of the \lumen{} ($4 < z < \mSI{6}{\nm}$), and in the constriction ($-1.85 < z < \mSI{1.6}{\nm}$), in
accordance with the charge groups discussed in the previous section. Even though the magnitude of $\radeqpot$
at $\cbulk = \mSI{0.15}{\Molar}$ drops below \SI{1}{\kTe} inside the \lumen{} ($\radeqpot \approx
\mSI{-14}{\mV}$), it remains strongly negative inside the constriction ($\radeqpot \approx \mSI{-47}{\mV}$).
Finally, at high reservoir concentrations ($\cbulk \ge \mSI{0.5}{\Molar}$) the potential is close to
\SI{\approx0}{\mV} over the entire \lumen{} of the pore, with only a small negative potential remaining inside
the constriction. A summary of the most salient $\radeqpot$ values can be found in
\cref{tab:radial_potential}.

%
\begin{figure}[p]
  \centering
  
  %
  \begin{subfigure}[t]{11cm}
    \centering
    \caption{}\vspace{-5mm}\label{fig:transport_energy_profiles}
    \includegraphics[scale=1]{transport_energy_profiles}
  \end{subfigure}
  %
  \\
  %
  \begin{subfigure}[t]{9cm}
    \centering
    \caption{}\vspace{-3mm}\label{fig:transport_energy_barriers}
    \includegraphics[scale=1]{transport_energy_barriers}
  \end{subfigure}
  %

  \caption[Non-equilibrium electrostatic energy landscape for single ions]%
  %
  {%
    %
    \textbf{Non-equilibrium electrostatic energy landscape for single ions.}
    %
    (\subref{fig:transport_energy_profiles})
    %
    Radially averaged non-equilibrium electrostatic energy landscape for single ions, $\radenergy =
    \chargen_{i} \echarge \radpot$, as calculated directly from the radial electrostatic potential at $\vbias
    = \mSIlist{+150;-150}{\mV}$ for monovalent cations and anions. The gray arrows indicate the direction in
    which the ions must travel in order for them to contribute positively to the ionic current.
    %
    (\subref{fig:transport_energy_barriers})
    %
    Height of the electrostatic energy barrier ($\Delta E_{\text{B},\mathit{trans}}$) at the \transi{}
    constriction as a function of the bulk salt concentration. Note that $\Delta E_{\text{B},\mathit{trans}}$
    is much higher for negative voltages and rises logarithmically at lower concentrations. The divergence
    between \SI[explicit-sign=+]{0}{\mV} and \SI[explicit-sign=-]{0}{\mV} for $\cbulk < \mSI{0.3}{\Molar}$
    highlights the difference in barrier height when traversing the pore from \cisi{} to \transi{} or
    \textit{vice versa}.
    %
  }\label{fig:transport_energy}
\end{figure}
%

\subsubsection{Non-equilibrium electrostatic energy at $\mathbf{\pm150}$~mV.}
%

To link back the observed ionic conductance properties to the electrostatic potential, we computed
$\radenergy$, the radially averaged electrostatic energy for a monovalent ion
%
\begin{equation}\label{eq:epnp-ns_radenergy}
  \radenergy = \chargen_{i} \echarge \radpot
  \text{ ,}
\end{equation}
%
at $\vbias = \mSI{\pm150}{\mV}$ for the entire range of simulated ionic strengths
(\cref{fig:transport_energy_profiles}). The resulting plot represents the energy landscape---filled with
barriers (hills) or traps (valleys)---that a positive or negative ion must traverse in order to contribute
positively to (\ie~increase) the ionic current.

At positive bias voltages, cations traverse the pore from \transi{} to \cisi{}
(\cref{fig:transport_energy_profiles}, first plot). Upon entering the negatively charged constriction, their
electrostatic energy drops dramatically, followed by a relatively flat section with a small barrier for entry
in the \lumen{} at $z \approx \mSI{1.6}{\nm}$. At very low ionic strengths ($\cbulk < \mSI{0.05}{\Molar}$),
the energy at \transi{} is significantly lower than the energy of the cation in the \cisi{} compartment
(\eg~$\Delta\radenergy > \mSI{2}{\kT}$ at \SI{0.005}{\Molar}), forcing the ions to accumulate inside the pore.
At higher concentrations ($\cbulk > \mSI{0.05}{\Molar}$), the increased screening smooths out the potential
drop inside the pore, allowing the cations to migrate unhindered across the entire length of the pore. Anions
at $\vbias = \mSI{+150}{\mV}$ travel from \cisi{} to \transi{} (\cref{fig:transport_energy_profiles}, second
plot) and must overcome energy barriers at both sides of the pore. The \cisi{} barrier prevents anions from
entering the pore, but because its magnitude is attenuated strongly with increasing salt concentration
(\SIrange{1.7}{0.5}{\kT} when increasing the reservoir salt concentration from $\cbulk =
\mSIrange{0.005}{0.05}{\Molar}$), it is only relevant at lower ionic strengths ($\cbulk <
\mSI{0.05}{\Molar}$). Once inside the \lumen{}, anions can move relatively unencumbered to the \transi{}
constriction, where they face the second, more significant energy barrier. This prevents them from fully
translocating and causes them to accumulate inside the lumen and explains why we observe higher \Cl{}
concentrations inside the pore at positive bias voltages (\cref{fig:transport_concentration_contours}). As
with the cations, an increase in the ionic strength significantly reduces these hurdles, resulting in a much
smoother landscape for $\cbulk > \mSI{0.15}{\Molar}$.

At negative voltages, cations move through the pore from \cisi{} to \transi{}, with a slow and continuous drop
of the electrostatic energy throughout the \lumen{} of the pore up until the constriction
(\cref{fig:transport_energy_profiles}, third plot). This results in the efficient removal of cations from the
pore \lumen{}, and explains the lower \Na{} concentration observed at positive voltages
(\cref{fig:transport_concentration_average}). To fully exit from the pore, however, cations must overcome a
large energy barrier, which reduces the nanopore's ability to conduct cations compared to positive potentials
and hence contributes to the ion current rectification. The situation for anions at negative bias voltages
(\ie~traveling from \transi{} to \cisi{}) is very different (\cref{fig:transport_energy_profiles}, fourth
plot). Their ability to even enter the pore is severely hampered by an energy barrier of a few \si{\kT} at the
\transi{} constriction. Any anions that do cross this barrier, and those still present in the \lumen{} of
\gls{clya}, will rapidly move towards the \cisi{} entry and exit from the pore due to a continuous drop of
their electrostatic energy. This effectively depletes the entire \lumen{} of anions, which can be observed
from the much lower \Cl{} concentrations at negative voltages
(see~\cref{fig:transport_concentration_average}).

\subsubsection{Concentration and voltage dependencies of the energy barrier at the constriction.}
%

Many biological nanopores contain constrictions that play crucial roles in shaping their ionic conductance
properties~\cite{Maglia-2008,Franceschini-2016,Huang-2017}. The reason for this is two-fold, (1) the narrowest
part dominates the overall resistance of the pore and (2) confinement of charged residues results in much
larger electrostatic energy barriers. With its highly negatively charged \transi{} constriction, \gls{clya}'s
affinity for transport of anions is diminished and that for cations is enhanced compared to bulk, even at high
ionic strengths (\cref{fig:transport_ioncurrent_tsodium_concentration})~\cite{Soskine-2013}. To further
elucidate the significance of the \transi{} electrostatic barrier ($\deltaEt$), we quantified its height at
positive and negative voltages as a function of the salt concentration (\cref{fig:transport_energy_barriers}).

Because the application of a bias voltage effectively tilts the energy landscape, it reduces the magnitude of
the energy barriers for both positive and negative potentials, as evidenced by the lowering of the curves with
increasing bias magnitude (\cref{fig:transport_energy_barriers}, light to dark color shading). Likewise,
raising the bulk salt concentration results in a continuous decrease of $\deltaEt$ due to an increase in the
screening of the fixed charges lining the constriction. At moderate to higher reservoir concentrations
($\cbulk > \mSIrange{0.1}{0.5}{\Molar}$, depending on $\vbias$), $\deltaEt$ falls below \SI{1}{\kT} regardless
of the bias voltage, and its effect on the ion transport through the pore is significantly reduced.

The ions under the influence of a positive bias voltage (\ie~\Na{} moving from \transi{} to \cisi{} and \Cl{}
moving from \cisi{} to \transi{}, blue lines in \cref{fig:transport_energy_barriers}) experience a $\deltaEt$
roughly half that of those under a negative voltage (red lines in \cref{fig:transport_energy_barriers}). For
example, increasing the salt concentration from \SIrange{0.005}{0.15}{\Molar}, causes $\deltaEt$ to drop from
\SIrange{1.8}{0.73}{\kT} at $\vbias = \mSI{+150}{\mV}$ and from \SIrange{4.4}{1.5}{\kT} at $\vbias =
\mSI{-150}{\mV}$. These differences in barrier heights are directly reflected by \gls{clya}'s higher degree of
ion selectivity at negative compared to positive bias voltages
(\cref{fig:transport_ioncurrent_tsodium_concentration}). 



\subsection{Transport of water through {ClyA}}
%
\label{sec:eof}
%

The charged nature of the inner surface of many nanopores gives rise to a net flux of water through the pore,
called the \gls{eof}~\cite{Qiao-Aluru-2003,Thompson-2003,Mao-2014}. The \gls{eof} not only contributes
significantly to the ionic current, but the magnitude of the viscous drag force it exerts on proteins is often
of the same order as the Coulombic electrophoretic force
(EPF)~\cite{vanDorp-2009,Firnkes-2010,Willems-Ruic-Biesemans-2019}. Hence, it strongly influences the capture
and translocation of biomolecules including nucleic acids~\cite{Wong-2007,Luan-2008,Firnkes-2010},
peptides~\cite{Huang-2017,Li-2018,Huang-2019}, and proteins~\cite{Soskine-2012,Soskine-2013,VanMeervelt-2014,
Soskine-Biesemans-2015,Biesemans-2015,Wloka-2017,Galenkamp-2018,Willems-Ruic-Biesemans-2019}. Because the drag
exerted by the \gls{eof} depends primarily on the size and shape of the biomolecule of interest and not on its
charge~\cite{Willems-Ruic-Biesemans-2019}, it can be employed to capture molecules even against the electric
field~\cite{Soskine-2012}. The \gls{eof} is a consequence of interaction between the fixed charges on the
nanopore walls and mobile charges in the electrolyte and can be described by two closely related mechanisms:
(1) the excess transport of the hydration shell water molecules in one direction due to the pore's ion
selectivity, and (2) the viscous drag exerted by the unidirectional movement of the electrical double layer
inside the pore~\cite{Tagliazucchi-2015,Bonome-2017}. The first mechanism likely dominates in pores with a
diameter close to that of the hydrated ions (\SI{\le1}{\nm}) such as \gls{ahl} or
\gls{frac}~\cite{Huang-2017,Huang-2019}, whereas the second is expected to be stronger for larger pores
(\SI{>1}{\nm}), such as \gls{clya}~\cite{Soskine-2012,Willems-Ruic-Biesemans-2019} or most solid-state
nanopores~\cite{Mao-2014,Laohakunakorn-2015}. In our simulation, the \gls{eof} is generated according to the
second mechanism by coupling of the Navier-Stokes and the Poisson-Nernst-Planck equations through a volume
force ($\volumeforce$, \cref{eq:ion_force_density}). This coupling dictates that the electric field exerts a
net force on the fluid if it contains a net ionic charge density---as is the case for the electrical double
layer lining the walls of \gls{clya} (\cref{fig:transport_charge_total}).

%
\begin{figure*}[p]
  \centering
  
  \begin{minipage}[t]{6cm}
    %
    \begin{subfigure}[t]{6cm}
      \centering
      \caption{}\vspace{-3mm}\label{fig:transport_flow_contour}
      \includegraphics[scale=1]{transport_flow_contour}
    \end{subfigure}
    %
    \\
    %
    \begin{subfigure}[t]{5.5cm}
      \centering
      \caption{}\vspace{-3mm}\label{fig:transport_flow_profiles}
      \includegraphics[scale=1]{transport_flow_profiles}
    \end{subfigure}
    %
  \end{minipage}
  %
  \hspace{0.25cm}
  %
  \begin{minipage}[t]{5cm}
    %
    \begin{subfigure}[t]{5cm}
      \centering
      \caption{}\vspace{-3mm}\label{fig:transport_flow_conductance}
      \includegraphics[scale=1]{transport_flow_conductance}
    \end{subfigure}
    %
    \\
    %
    \begin{subfigure}[t]{5cm}
      \centering
      \caption{}\vspace{-3mm}\label{fig:transport_flow_rectification}
      \includegraphics[scale=1]{transport_flow_rectification}
    \end{subfigure}
    %
  \end{minipage}

  \caption[Concentration and voltage depend. of the {EOF} inside {ClyA-AS}]%
  {%
    %
    \textbf{Concentration and voltage dependency of the electro-osmotic flow inside {ClyA-AS}.}
    %
    (\subref{fig:transport_flow_contour})
    %
    Contour plot of the \gls{eof} velocity magnitude $U$ at \SI{0.5}{\Molar} and \SI{-100}{\mV} bias voltage.
    The arrows  on the streamlines indicate the direction of the flow. As observed
    experimentally~\cite{Soskine-2013} and expected from a negatively charged conical nanopore, the \gls{eof}
    follows the direction of the cation (\ie~from \cisi{} to \transi{} under negative bias voltages and
    \textit{vice versa} for positive ones). Note that in contrast to \gls{plyab-r}, whose oppositely charged
    \lumen{} and constriction induce Eddy currents within the pore~\cite{Huang-2020}, the uniform distribution
    of negative charges on the walls of \gls{clya} results in a smooth, unidirectional flow.
    %
    (\subref{fig:transport_flow_profiles})
    %
    Cross-section profiles of the absolute value of the water velocity $\left|U_z\right|$ inside \transi{}
    constriction (at $z = \mSI{-1}{\nm}$) for various salt concentrations at $\vbias = \mSI{-100}{\mV}$.
    Notice that at high salt concentrations ($\cbulk > \mSI{1}{\Molar}$), the velocity profile exhibits two
    `lobes' close to the nanopore walls and hence deviates from the parabolic shape observed at lower ionic
    strengths.
    %
    (\subref{fig:transport_flow_conductance})
    %
    Concentration dependency of the electro-osmotic conductance $\flowcond = \flowrate / \vbias$, with
    $\flowrate$ the total flow rate through the pore (\cref{eq:flowrate}). In the low concentration regime,
    $\flowcond$ increases rapidly between \SIlist{0.005;0.5}{\Molar} after which it decreases logarithmically
    for higher concentrations.
    %
    (\subref{fig:transport_flow_rectification})
    %
    The rectification of the electro-osmotic conductance ($\eor(V) = \flowcond (+V) / \flowcond (-V)$) plotted
    against the bulk salt concentration. The $\eor$ increases with bias voltage and exhibits an inversion
    point at $\cbulk \approx \mSI{0.45}{\Molar}$.
    %
  }\label{fig:transport_flow}

\end{figure*}
%

\subsubsection{Direction, magnitude and distribution of the water velocity.}
%
As expected, given \gls{clya}'s negatively charged interior surface and the resulting positively charged
electrical double layer, the direction of the net water flow inside \gls{clya} follows the electric field:
from \cisi{} to \transi{} at negative bias voltages (\cref{fig:transport_flow_contour}). This corresponds to
the observations and analysis of single-molecule protein capture~\cite{Soskine-2013} and
trapping~\cite{Soskine-Biesemans-2015,Biesemans-2015,Willems-Ruic-Biesemans-2019} experiments using the
\gls{clya-as} nanopore. Along the longitudinal axis ($z$) of the pore, the water velocity is governed by the
conservation of mass, meaning it is lowest in the wide \cisi{} \textit{lumen} and highest in the narrow
\transi{} constriction (\cref{fig:transport_flow_contour}). For example, at $\vbias = \mSI{-100}{\mV}$ and
$\cbulk = \mSI{0.5}{\Molar}$ the velocity at the center of the pore is \SI{\approx0.07}{\mps} in the
\textit{lumen} and \SI{\approx0.21}{\mps} in the constriction.

Along the radial axis ($r$), $\velocity$ has a parabolic profile with the highest value at the center of the
pore and the lowest at the wall due to the no-slip boundary condition (\cref{fig:transport_flow_profiles}).
Such a parabolic profile contrasts the expected `plug flow' for an \gls{eof}, but follows logically from the
overlap of the electrical double layer inside the pore and the resulting uniform volume force---analogous to a
gravity- or pressure-driven Stokes flow. At concentrations higher than \SI{0.5}{\Molar}, however, the
increasing degree of confinement of the double layer---and its charge---to the nanopore walls
(see~\cref{fig:transport_charge_total}, $\pav{\Qion}{s}$) results in a flattening of the central maximum and
hence a plug flow profile. Interestingly, at very high salt concentrations ($\cbulk \ge \mSI{1}{\Molar}$) the
velocity profile in the constriction exhibits a dimple at the center of the pore
(\cref{fig:transport_flow_profiles}). This is the result of a self-induced pressure gradient caused by the
expansion of the \gls{eof} as it exits the pore~\cite{Melnikov-2017}.

\subsubsection{Influence of bulk ionic strength and bias voltage on the electro-osmotic conductance.}
%
In analogy to the ionic conductance, the amount of water transported by \gls{clya} can be expressed by the
electro-osmotic conductance $\flowcond = \flowrate / \vbias$ (\cref{fig:transport_flow_conductance}). Here,
$\flowrate$ is the net volumetric flow rate of water through the pore and computed by integrating the water
velocity across the reservoir boundary $S$
%
\begin{align}\label{eq:flowrate}
  \flowrate = \int_{S}\left(\normvec\cdot\vec{\velocity}\right)dS
  \text{ .}
\end{align}
%
The strength of the \gls{eof} depends strongly and non-monotonically on the bulk ionic strength: $\flowcond$
rapidly increases with ionic strength until a peak value is reached at $\cbulk \approx \mSI{0.5}{\Molar}$,
followed by a gradual logarithmic decline (\cref{fig:transport_flow_conductance}). For example, at $\vbias =
\mSI{-150}{\mV}$, $\flowcond$ first increases from \SI{1.85}{\cnmpnspv} at \SI{0.005}{\Molar} to
\SI{11.3}{\cnmpnspv} at \SI{0.5}{\Molar}, followed by a gradual decline to \SI{4.00}{\cnmpnspv} at
\SI{5}{\Molar}. As with the ionic conductance, more details on the effect of the concentration, wall distance
and steric corrections on the electro-osmotic conductance can be found
in~\cref{sec:transport:comparison_corrections} (\cref{fig:transport_comparison_water}).

The sensitivity of the \gls{eof} to the magnitude and sign of the bias voltage is given by the electro-osmotic
conductance rectification $\eor(V) = \flowcond (+V) / \flowcond (-V)$
(\cref{fig:transport_flow_rectification}). For all voltage magnitudes, $\eor$ shows a maximum at $\cbulk
\approx \mSI{0.045}{\Molar}$, after which it falls rapidly to reach unity ($\eor = 1$) at approximately
$\cbulk \approx \mSI{0.45}{\Molar}$. A minimum is then reached at \SI{\approx1}{\Molar}, followed by a gradual
approach towards unity at $\cbulk = \mSI{5}{\Molar}$.

%
\begin{figure*}[t]
  \centering
  
  %
  \begin{subfigure}[t]{6cm}
    \centering
    \caption{}\vspace{-3mm}\label{fig:transport_pressure_contour}
    \includegraphics[scale=1]{transport_pressure_contour}
  \end{subfigure}
  %
  \hspace{-5mm}
  %
  \begin{subfigure}[t]{4.5cm}
    \centering
    \caption{}\vspace{-3mm}\label{fig:transport_pressure_profiles}
    \includegraphics[scale=1]{transport_pressure_profiles}
  \end{subfigure}
  %

  \caption[Pressure distribution inside {ClyA-AS}]
  %
  {%
    %
    \textbf{Pressure distribution inside {ClyA-AS}.}
    %
    (\subref{fig:transport_pressure_contour})
    %
    Contour map of the hydrodynamic pressure $\pressure$ at $\cbulk = \mSI{0.15}{\Molar}$ and
    $\vbias=\mSI{0}{\mV}$, showing that the large variations in \Na{} concentration along the pore wall result
    in osmotic pressure `hotspots' (\SIrange{5}{30}{\atm}) inside the confined fluid.
    %
    (\subref{fig:transport_pressure_profiles})
    %
    The axial pressure profile and averaged along the entire radius of the pore at $\vbias=\SI{0}{\mV}$.
    %
  }\label{fig:transport_pressure}

\end{figure*}
%


\subsubsection{Pressure distribution inside ClyA.}
%
The large variations of \Na{} concentration along the walls of \gls{clya}---up to several orders of magnitude
over the course of a few nanometers (see~\cref{fig:transport_concentration_contours})---induce regions of high
`osmotic' pressure with peak values up to \SI{30}{\atm} (\cref{fig:transport_pressure_contour}). The largest
`hotspots' are located at \cisi{} entry of the pore ($z = \mSI{11}{\nm}$), in the middle of the \lumen{} ($z =
\mSI{4.5}{\nm}$) and inside the entire constriction ($z = \mSI{-1}{\nm}$)
(\cref{fig:transport_pressure_profiles}), and their influence extends well towards the center of the pore. Up
until $\cbulk \approx \mSI{0.5}{\Molar}$, increasing the reservoir salt concentration does not seem to
strongly influence the overall magnitude of the pressure spots. Hence, because such large pressure differences
can  exert a significant amount of force on particles translocating through nanopores~\cite{Hoogerheide-2014},
we expect them to play an important role in the detailed trapping dynamics of proteins inside
\gls{clya}~\cite{Soskine-Biesemans-2015,Willems-Ruic-Biesemans-2019}.


\subsection{Effect of the individual corrections on the ionic and water conductances}
%
\label{sec:transport:comparison_corrections}
%

To estimate the influence to which each group of corrections (\ie~the wall distance, concentration and steric
effects) on the conductance properties of \gls{clya-as}, we performed a set of simulations for the following
conditions: \gls{epnp-ns} without wall distance corrections (`ePNP-NS no WDF'; $\diffusion_{i}^w=1$,
$\mobility_{i}^w=1$, $\viscosity^w=1$), \gls{epnp-ns} without concentration-dependent corrections (`ePNP-NS no
CDF'; $\permittivity_{r,\text{f}}^c=1$, $\diffusion_{i}^c=1$, $\mobility_{i}^c=1$, $\viscosity^c=1$,
$\density^c=1$), \gls{epnp-ns} without steric effect (`ePNP-NS no SMP'; $\vec{\beta}=0$). For comparison, we
plotted the ionic (\cref{fig:transport_comparison_ionic}) and water (\cref{fig:transport_comparison_water})
conductances of these models, normalized over the values obtained with the full \gls{epnp-ns} equations.

%
\begin{figure*}[p]
  \centering
  %
  \begin{subfigure}[t]{11.5cm}
    \centering
    \caption{}\vspace{0mm}\label{fig:transport_comparison_ionic}
    \includegraphics[scale=1]{transport_comparison_ionic}
  \end{subfigure}
  %
  \\
  %
  \begin{subfigure}[t]{11.5cm}
    \centering
    \caption{}\vspace{0mm}\label{fig:transport_comparison_water}
    \includegraphics[scale=1]{transport_comparison_water}
  \end{subfigure}
  %

  \caption[Effect of individ. corrections on the sim. ionic and water conductance]%
  {%
    %
    \textbf{Effect of individual corrections on the simulated ionic and water conductance.}
    %
    (\subref{fig:transport_comparison_ionic})
    %
    The ionic conductance $\conductance = \current / \vbias$ and
    %
    (\subref{fig:transport_comparison_water})
    %
    the water conductance $\flowcond = \flowrate / \vbias$ of {ClyA-AS} at \SI{+150}{\mV} (left) and
    \SI{-150}{\mV} (right), normalized over the values of the \gls{epnp-ns} equations. Comparison between the
    experimental data (expt.), the simple resistor model (bulk, \cref{eq:bulk_nanopore_current}), classic
    {PNP-NS} ({sim. PNP-NS}), \gls{epnp-ns} without wall distance corrections ({sim. ePNP-NS no WDF}:
    $\diffusion_{i}^w=1$, $\mobility_{i}^w=1$, $\viscosity^w=1$), \gls{epnp-ns} without
    concentration-dependent corrections ({sim. ePNP-NS no CDF}: $\permittivity_{r,\text{f}}^c=1$,
    $\diffusion_{i}^c=1$, $\mobility_{i}^c=1$, $\viscosity^c=1$, $\density^c=1$), \gls{epnp-ns} without steric
    effect ({sim. ePNP-NS no SMP}: $\vec{\beta}=0$) and full with all corrections enabled (\ie~\gls{epnp-ns}).
    %
  }\label{fig:transport_comparison}
\end{figure*}
%

%
\subsubsection{Effect of disabling the wall distance corrections.}
%

Disabling the wall distance corrections yields a higher (\SI{\approx10}{\percent} increase) ionic conductance
compared to the full \gls{epnp-ns} equations (\cref{fig:transport_comparison_ionic}, blue curve). In effect,
the implementation of the wall distance corrections results in an smaller effective pore size.  The influence
on the water flow is stronger, with the reduction of the viscosity near the wall giving rise to a
\SIrange{25}{50}{\percent} increase in the water flow (\cref{fig:transport_comparison_water}, blue curve).

%
\subsubsection{Effect of disabling the concentration corrections.}
%

The removal of the concentration-dependent corrections has a large influence on the simulated results. The
ionic conductance increases dramatically: from \SIlist{5;13;29;152}{\percent} at
\SIlist{0.005;0.05;0.5;5}{\Molar}, respectively (\cref{fig:transport_comparison_ionic}, red curve). This is
expected however, given that the large reduction of both diffusion coefficients and ionic mobilities with
increasing salt concentrations is not taken into account. The influence on the water flow shows only a limited
effect from \SI{0.005}{\Molar} (\SI{9}{\percent} decrease) to \SI{1}{\Molar} (\SI{2}{\percent} increase),
followed by a rapid increase of \SI{82}{\percent} between \SIlist{1;5}{\Molar}
(\cref{fig:transport_comparison_water}, red curve). This is a direct result of the \SI{15}{\percent} increase
of the electrolyte viscosity between \SIrange{1}{5}{\Molar}, compared to the mere \SI{5}{\percent} increase
between \SIrange{0}{1}{\Molar} (\cref{fig:epnpns_concentration_solv_eta}).

%
\subsubsection{Effect of disabling the steric corrections.}
%
The steric corrections appear to have little influence on the ionic conductance, with a maximum deviation of
at most \SI{\pm3}{\percent} over the entire concentration range (\cref{fig:transport_comparison_ionic}, brown
curve). The effect on the water flow is larger, with a decrease of \SI{10}{\percent} at \SI{0.5}{\Molar}
(\cref{fig:transport_comparison_water}, brown curve). By placing an upper limit on the total ion concentration
(\SI{\approx13.3}{\Molar} in our case), the steric corrections prevent excessive screening of fixed charges by
nonphysical ionic strengths (\ie~concentrations that would require fitting more ions into a given space than
what is physically possible). This is an essential mechanic that allows one to realistically model the
electrical double layer near surfaces with high charge densities---as is the case for most biological
nanopores.


\section{Conclusion}
%
\label{sec:transport:conclusion}
%

The use of computationally inexpensive continuum models is pervasive in the solid-state nanopore field, but
their application to the structurally more complex biological nanopores has been limited to date. We made use
of the radial symmetry of the biological nanopore \gls{clya} to create a 2D-axisymmetric model of the pore
which, in conjunction with the \gls{epnp-ns} equations, is able to accurately describe the ionic current of
\gls{clya} for a wide range of experimentally relevant ionic strengths and bias voltages. Our approach shows
that continuum modeling of biological nanopores is not only feasible, but can also be predictive. Our results
describe in great detail the properties of \gls{clya}, such as its true ion selectivity, the differences
between cation and anion concentrations inside the pore, the distribution and magnitude of the electrostatic
potential, the velocity of the electro-osmotic flow and the presence of highly localized `hotspots' of osmotic
pressure. These findings do not only provide valuable insights into physical mechanisms of nanoscale
transport, but they also reveal a great deal about the conditions within the pore itself. For example, strong
deviations from bulk values, such as the depletion of anions from the \lumen{} of \gls{clya} at negative bias
voltages, may have a profound impact on the activity and structure of trapped proteins.

The \gls{epnp-ns} framework comprises an empirical approach that significantly improves the quantitative
accuracy for continuum simulations of nanoscale transport phenomena. We expect our framework and its
principles to be transferable to other biological nanopores and nanoscale transport systems. We believe our
model constitutes a powerful and practical tool that can aid with (1) elucidating the link between ionic
current observed during a nanopore experiment and the actual physical phenomenon, (2) describing the
electrophoretic and electro-osmotic properties of any biological nanopore and (3) guiding the rational design
of new variants of existing nanopores. For example, by further expanding and automating our framework, it
could be used as a probe to gauge---and tailor---the properties of specific nanopore
mutants~\cite{Huang-2020,Cao-2019}, or to study the dynamics of translocating proteins by computing the forces
that act on model particles~\cite{Willems-Ruic-Biesemans-2019}. Both of these applications could play a
crucial role in designing the next generation of (biological) nanopore-based biosensors.


\section{Materials and methods}
%
\label{sec:transport:methods}
%
%

\subsection{{ClyA-AS} expression and purification}
%

\Gls{clya-as} monomers were expressed, purified and oligomerized using methods described in detail
elsewhere~\cite{Soskine-2012,Soskine-2013}. Briefly, \textit{E. cloni} {EXPRESS BL21} (DE3) cells (Lucigen
Corporation, Middleton, USA) were transformed with a {pT7-SC1} plasmid containing the \gls{clya-as} gene,
followed by overexpression after induction with \SI{0.5}{\mM} \gls{iptg} (Carl Roth, Karlsruhe, Germany). The
\gls{clya} monomers were purified using \ce{Ni+}-NTA affinity chromatography and oligomerized by incubation in
\SI{0.2}{\percent} \gls{ddm} (Sigma-Aldrich, Zwijndrecht, The Netherlands) for \SI{30}{\minute} at
\SI{37}{\celsius}. Pure \gls{clya-as} {type I} (12-mer) nanopores were obtained using native \gls{page} on a
\SIrange[range-phrase = --]{4}{15}{\percent} gradient gel (Bio-Rad, Veenendaal, The Netherlands) and
subsequent excision of the correct oligomer band.

\subsection{Recording of single-channel current-voltage curves}
%

Experimental current-voltage curves where measured using single-channel electrophysiology, as detailed
elsewhere~\cite{Maglia-2010,Soskine-2012,Soskine-2013}. First, a black lipid bilayer was formed inside a
\SI{\approx100}{\um} diameter aperture in a thin Teflon film separating two buffered electrolyte compartments.
This was achieved by applying a droplet of \SI{5}{\percent} hexadecane in pentane (Sigma-Aldrich, Zwijndrecht,
The Netherlands) over the aperture and leaving it to dry for \SI{1}{\minute} at \SI{25}{\celsius}. The
buffered electrolyte solution was added to both compartments, topped with \SI{10}{\uL} of
\SI{6.25}{\milli\gram\per\milli\liter} \gls{dphpc} (Avanti Polar Lipids, Alabaster, USA) in pentane. The
pentane was left to evaporate for \SI{2}{\minute} at \SI{25}{\celsius}. A lipid bilayer was formed by lowering
and raising the buffer level over the aperture. Minute amounts (\SI{\approx0.2}{\uL}) of the purified
\gls{clya-as} {type I} oligomer were then added to the grounded \cisi{} reservoir and allowed to insert into
the lipid bilayer. Single-channel current-voltage curves were recorded using a custom pulse protocol of the
\code{Clampex 10.4} (Molecular Devices, San Jose, USA) software package connected to AxoPatch 200B patch-clamp
amplifier (Molecular Devices, San Jose, USA) \textit{via} a Digidata 1440A digitizer (Molecular Devices, San
Jose, USA). Data was acquired at \SI{10}{\kHz} and filtered using a \SI{2}{\kHz} low-pass filter. Measurements
at different ionic strengths were performed at \SI{\approx25}{\celsius} in aqueous \ce{NaCl} (Carl Roth,
Karlsruhe, Germany) solutions, buffered at \pH{7.5} using \SI{10}{\mM} \gls{mops} (Carl Roth, Karlsruhe,
Germany).


%%%%%%%%%%%%%%%%%%%%%%%%%%%%%%%%%%%%%%%%%%%%%%%%%%
% Keep the following \cleardoublepage at the end of this file,
% otherwise \includeonly includes empty pages.
\cleardoublepage

% vim: tw=70 nocindent expandtab foldmethod=marker foldmarker={{{}{,}{}}}
